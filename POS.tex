\pagebreak
\thispagestyle{empty}
\movetooddpage
\chapter[Posfácio, \emph{por Vinicius Matteucci de Andrade Lopes}]{Posfácio \subtitulo{Friedrich Engels e o ponto de saída\\ da filosofia clássica alemã}}

\begin{flushright}
\textsc{vinicius matteucci de andrade lopes}
\end{flushright}

\noindent{}Em sua clássica biografia teórica sobre o jovem Marx, Lukács sugere,
para além das qualidades investigativas, determinação e rigor, a
existência na obra de Marx de uma intima coincidência, que cabe a
``poucas personalidades da história mundial'', ``entre desenvolvimento
teórico individual e desenvolvimento social geral''.\footnote{``O
  processo de superação do hegelianismo, também a ultrapassagem por
  Feuerbach, a fundação da dialética materialista, coincide, no seu
  processo formativo, com o desenvolvimento da posição da democracia
  revolucionária em direção ao socialismo consciente. Ambas as linhas
  formam uma unidade necessária, mas todo o processo transcorre --- 
  de modo algum contingente --- em um período da história alemã, no qual
  --- após a ascensão ao trono de Frederico Guilherme \textsc{iv} na
  Prússia, e a conversão reacionário"-romântica da política interna
  prussiana --- estabelece"-se uma agitação universal política e
  ideológica: o período de preparação da revolução democrático"-burguesa
  de 1848''. (\textsc{lukács}, Georg. \emph{Der junge Marx. Seine
  philosophische Entwiklung von 1840--1844}. Stuttgart. Verlag Günther
  Neske Pfullingen, 1965, p.~6)}

Como se sabe, um elemento fundamental para construção dessa ``intima
coincidência'', que se expressa em diversos textos de Marx,
principalmente os com veiculação jornalística, é que muitos eram
respostas diretas a momentos políticos, acontecimentos históricas e
debates teóricos contemporâneos. O alcance e profundidade da crítica é
traço a ser medido pela singularidade de cada texto. E aqui sem
mistificar a abstração \emph{marxismo} --- constituída principalmente a
partir do século \textsc{xix} como sintoma conjuntural, muito mais a
favor dos seus detratores do que dos seus defensores ---, termo que
Engels reputa como correto em relação ao seu próprio modo investigativo,
já que a rubrica \emph{marxismo} serve justamente para estabelecer a diferenciação frente aos
materialismos, socialismos e positivismos do final do século
\textsc{xix}, afinal, sem Marx, já morto há cerca de três anos, ``a
teoria não seria hoje, nem de longe, aquilo que ela é. Ela tem,
portanto, também com razão, seu nome''.\footnote{\emph{Ludwig Feuerbach
  e o fim da filosofia clássica alemã (parte \textsc{iv})}.}

A obra de Engels, por sua vez, também é exemplo dessa rara
\emph{confluência}. Um fator fundamental é a mesma interlocução direta,
via textos jornalísticos, com seu tempo histórico. Nesses textos, tanto
Marx como Engels enfrentaram abertamente o que na metade do século
\textsc{xix} europeu era inevitável: o impacto da expansão do mundo
burguês em \emph{todas} as suas dimensões. Mais de meio século depois da
burguesia francesa estabelecer a retórica política de legitimação de sua
estrutura de poder, a totalidade dessa expansão, já há muito ``visível''
do \emph{Standpunkt} europeu em uma de suas figuras centrais --- o avanço %MANTIVE
do poder do ``mercado mundial'' --- torna"-se inquestionável no século
\textsc{xix} com o estabelecimento da grande indústria, como o próprio
Engels escreve:

\begin{quote}
Desde a efetivação (\emph{Durchführung}) da grande indústria,
portanto, pelo menos desde a paz europeia de 1815, não era mais segredo
para homem nenhum na Inglaterra que lá toda a luta política girava em
torno das pretensões de dominação de duas classes: a aristocracia
possuidora de terras (\emph{landed aristocracy}) e a burguesia
(\emph{middle class}). Na França, a consciência do mesmo fato foi obtida
com o regresso dos Bourbon; os historiadores da época da Restauração, de
Thierry a Guizot, Mignet e Thiers, falam disso, por toda a parte, como a
chave para a compreensão da história francesa desde a Idade Média. E,
desde 1830, em ambos os países, a classe dos trabalhadores, o
proletariado, foi reconhecido como a terceira força por essa dominação.
\end{quote}

A consciência desse processo se explicita não apenas na reestruturação
dos modos e relações de produção, na transformação da relação
campo"-cidade em diversas regiões europeias, mas também nas mais
abstratas formas de consciência, condicionadas à revelia de suas mais
nobres intenções. Afinal, como também aponta Engels, ``a moral de
Feuerbach está talhada pela atual sociedade capitalista, por mais que
ele próprio não queira isso ou possa suspeitar''.\footnote{\emph{Ludwig
  Feuerbach e o fim da filosofia clássica alemã (parte \textsc{iii})}.}

A presente obra é, portanto, um exemplo paradigmático desse
enfrentamento de conjuntura. Tem sua origem justamente em um convite que
Engels recebeu dos editores da revista social"-democrata, \emph{Die
Neue Zeit}, provavelmente de Kautsky,\footnote{\emph{Apparat},
  \textsc{mega} (\textsc{i}.30, 2011), p.~738.} para realizar uma
resenha crítica da mencionada tese de doutorado, recém"-lançada em
1885, do filósofo dinamarquês Carl Nicolaj Starcke: uma exposição
filosófica do desenvolvimento teórico de Feuerbach. Essa informação
importa aqui, já que, conforme adiantado, a resenha empreendida por
Engels não corresponde a uma revisão da obra de Starcke, uma retomada
crítica de Feuerbach ou, a partir disso, uma crítica conceitual interna
dos pressupostos da filosofia de Feuerbach. Quem buscar qualquer uma
desses pontos no texto dificilmente acompanhará a sua potência crítica.
Como indicado no \emph{Apparat} da \textsc{mega} (\textsc{i}.30, 2011),
trabalhado por Renate Merkil"-Melis:

\begin{quote}
A monografia de Starcke sobre Feuerbach ofereceu a Engels a
possibilidade de confrontar de modo exemplar as novas tendências no
pensamento filosófico, recusa de Hegel, neokantismo ético, positivismo.
O livro {[}de Starcke{]} destacava"-se tanto como uma medição
intencional entre idealismo e realismo, como enquanto um direcionamento
para metafísica e ética.\footnote{\emph{Idem}.}
\end{quote}

Essa tentativa de mediação entre uma metafísica e uma ética que Starcke
vê na obra de Feuerbach é um traço central do embate das ideias na
Alemanha desde 1830, principalmente tendo em vista os efeitos práticos
da fraseologia da Revolução Francesa, ou, como coloca Starcke no prefácio
da obra, dos efeitos da ``indeterminação'' da Revolução de 1789, por
meio da qual direitos abstratos como a liberdade universal\footnote{``Na
  Revolução de 1789 tudo ainda era indeterminado; falava"-se de direitos
  humanos, mas esses eram apenas frases feitas, na medida em que não
  conseguiam oferecer para grande maioria um elemento determinado, real,
  para que os homens tivessem direitos''. \textsc{starcke}, C. N.
  \emph{Ludwig Feuerbach.} Stuttgart: \emph{Verlag von Ferdinand Enke},
  1885, p.\textsc{vii}} passam a ser considerados como inatos ao homem,
deixando em aberto questões como: qual homem? Qual liberdade? Qual a
estrutura desse caráter inato? A ética e filosofia da religião de
Feuerbach são, para Starcke, tentativas de respostas \emph{não
sistêmicas} a essas questões.

\section*{A ilusão burguesa da autonomia da vontade}

Para Engels, leitor do real via obra de Starcke, ou mesmo para Marx e
Engels, leitores do real via Feuerbach em 1844/45 nos manuscritos da
\emph{Ideologia Alemã}, qualquer formulação de resposta metafísica,
moral ou ética a essas questões conduz inevitavelmente às confusões
fundamentais do mundo burguês: entre elas, apreender o Estado moderno
enquanto ``realidade efetiva da liberdade concreta'';\footnote{\textsc{hegel},
  G. W. F. \emph{Grundlinien der Philosophie des Rechts oder Naturrecht
  und Staatswissenschaft im Grundrisse}, Frankfurt am Main, Suhrkamp
  Verlag, 1970, p.~415 (§ 270). ``Que o fim do Estado seja o interesse
  universal como tal e que, nisso, seja a conservação dos interesses
  particulares como substância destes últimos, isso é 1) sua
  realidade"-efetiva abstrata ou substancialidade; mas esta última é 2)
  sua necessidade, enquanto ela se divide nas distinções conceituais de
  sua atividade, que são, do mesmo modo, graças àquela substancialidade,
  determinações estáveis e reais, poderes; 3) porém, tal
  substancialidade é, precisamente, o espírito, que, por haver passado
  pela forma de uma tradição cultural {[}Die \emph{Form der Bildung}{]},
  sabe"-se e quer a si mesmo. O Estado sabe, por isso, o que quer, e o
  sabe em sua universalidade, como algo pensado; ele age e atua, por
  isso, segundo fins sabidos, princípios conhecidos e segundo leis que
  não são somente em si, mas para a consciência; e, do mesmo modo, na
  medida em que suas ações se atêm às circunstâncias e relações
  existentes, age e atua segundo o conhecimento determinado que tem
  delas.''} ou ainda, a dimensão interna dessa concepção, apreender os
imperativos político"-morais (o \emph{Sollen, o dever"-ser}) como
resoluções formais da relação entre vontade e devir, isto é, como
\emph{locus} concreto da resolução da tensão entre vontade subjetiva e
história.

Na semântica de uma metafísica burguesa sobre a forma da autonomia da
vontade, o problema se coloca como uma tentativa de delimitação da
diferença entre um \emph{sollen subjetivo} e um \emph{objetivo}, entre o
elemento posto objetivamente pela vontade e a indeterminação subjetiva
da constituição dessa vontade. Essa delimitação passa pela justificação
dos \emph{imperativos sociais como elementos subjetivos sempre já
superados} pelo \emph{devir}, pela \emph{``forma da formação
cultural''}, nos termos de Hegel. O Estado aparece como a conservação
(\emph{Erhaltung}) da ``substância'' do interesse particular que se
expressa enquanto interesse universal justamente por ter atravessado
\emph{uma} \emph{forma de formação cultural}. A crítica do jovem Marx a
Hegel aponta justamente para neutralização dessa relação recíproca entre
universal e particular, questionando a ``omissão das determinações
concretas''\footnote{\textsc{marx}, K. \emph{Zur Kritik der Hegelschen
  Rechtsphilosophie. Kritik des Hegelschen Staatsrechts}. \textsc{mew},
  Band 1, p.~217.} diante das determinações abstratas. Omissão esta,
imposta pelo mesmo movimento da concretude, que revela o caráter
arbitrário das abstrações, sempre a favor de \emph{uma forma} de
``formação cultural'' que \emph{aparece} como \emph{única forma
possível}. Aqui a conhecida proposição de Marx, para ficarmos ainda em
1843, por meio da qual essa distorção funciona: ``o momento filosófico
não é a lógica da determinação prática (\emph{Sache}) {[}como pretende
Hegel e as variadas formas de consciência no mundo burguês{]}, mas a
determinação pratica (\emph{Sache}) da lógica {[}como ocorre à revelia
da intenção dos filósofos{]}''.\footnote{Idem, p.~216.}

Marx e Engels, enquanto historiadores do processo de formação do mundo
burguês, questionam justamente o \emph{locus} de aparecimento da relação
(\emph{Verhältnis}) constitutiva dos imperativos sociais. A crítica de
Marx à relação"-capital (\emph{Kapitalverhältnis}) implode justamente
qualquer pretensão de localizar essa relação em uma mediação, mais ou
menos direta, entre \emph{razão objetiva} e \emph{subjetividade moral.}
Em termos kantianos, criticados via Feuerbach por Engels (parte
\textsc{iii}), a ``impotência'' do imperativo categorial, a pretensão de
se livrar das condições empíricas, escapar às limitações arbitrárias e
imposições objetivas, e alcançar uma forma universal da subjetividade,
não passa de uma potência sistêmica. Na \emph{Fundamentação da
metafísica dos costumes} (1785), Kant defende que a máxima subjetiva não
seria dada pelas condições objetivas, mas já expressaria um conteúdo
externo à lei posta --- pelo Estado ou tradição cultural --- localizada
em um espaço imaginado \emph{entre} o subjetivo e o objetivo. A
universalidade se provaria, nesse lugar, como forma ``\emph{natural}''
da vontade, como se fosse uma decantação subjetiva da realidade. Como se
o Estado, ou o próprio elemento político do mundo burguês, fosse uma
\emph{decantação natural} de uma \emph{vontade histórica} universal. O
\emph{caráter natural}\footnote{``Devido ao fato da
  universalidade da lei, segundo a qual efeitos acontecem, constituir
  aquilo a que, na realidade, chama"-se \emph{natureza} no sentido mais
  amplo da palavra (quanto à forma), isto é, em relação à existência das
  coisas, enquanto determinada por leis universais, o imperativo
  universal do dever (\emph{Pflicht --- obrigação}) poderia também
  exprimir"-se assim: \emph{age como se a máxima da tua ação devesse se
  tornar, pela tua vontade, lei universal da natureza}.''
  (\textsc{kant}, I. \emph{Grundlegung zur Metaphysik der
  Sitten.}Frankfurt am Main, Suhrkamp Verlag, 2007, p.~53)} se
expressaria como aquilo que é determinado pelo elemento universal, isto
é, localizado tanto \emph{fora da vontade subjetiva} como \emph{fora das
condições objetivas}. A \emph{vontade histórica} seria o outro lado da
mesma moeda, já que explicitaria, enquanto devir histórico, o caráter
\emph{interno} de uma suposta autorresolução da tensão entre vontade
subjetiva e condições objetivas.

É bastante ilustrativa, nesse sentido, a conhecida passagem de Marx na
\emph{Miséria da filosofia}, onde critica o fato dos economistas
burgueses dividirem as instituições em ``artificiais''
(\emph{künstliche / de l'art}) e ``naturais'' (\emph{natürliche /
de la nature}). Assim como os teólogos defendem a religião dos outros
como criação dos homens e a sua como revelação de Deus, os economistas
burgueses defendem suas instituições, ``suas relações no presente'',
como revelações de leis naturais eternas que devem governar toda
sociedade, enquanto as instituições feudais, bem como as relações de produção feudal,
seriam históricas, necessariamente transitórias: ``Havia assim 
uma história, mas agora não há mais.''\footnote{\textsc{marx}, K.
  \emph{Das Elend der Philosophie,} \textsc{mew} 6, p.~139.} Ou, de modo
mais enfático: havia uma história assim, \emph{justamente por isso}, não
há mais.

Na realidade, como não cansaram de demonstrar Marx e Engels, o Estado
moderno, o elemento político burguês, ou a forma da vontade (\emph{a
Sitllichkeit}) burguesa, nada mais são do que uma decantação
\emph{historicamente imposta} (aqui a igualação entre interesse
universal e a forma da formação cultural apontada por Hegel) de uma
vontade igualmente histórica, mas que não se constitui como caráter
\emph{interno} da autorresolução da tensão entre vontade subjetiva e
condições objetivas. Essa autorresolução não é a revelação da verdade,
ainda que contenha elementos concretos, justamente porque ela não é o
resultado automático dessa autorresolução. E ela não é automática
porque não há o \emph{fora idealizado.} Nem como transcendental, nem
como transcendente. Esse seria um elemento básico da dimensão
materialista, necessariamente histórica. Toda a crítica da
economia"-política de Marx explicita as diversas dimensões desse
\emph{fora idealizada}. Um fora que é idealizado e conjurado pelo
movimento concreto, bastando atentar, por exemplo, para a cisão entre
potencias intelectuais do processo de produção e o trabalho manual,
mediado pelo fora idealizado do trabalho abstrato, que se apresenta
historicamente quando o autômato da grande indústria se põe.\footnote{``Toda
  produção capitalista, por ser não apenas processo de trabalho, mas, ao
  mesmo tempo, processo de valorização do capital, tem em comum o fato
  de que não é o trabalhador que emprega as condições de trabalho, mas,
  ao contrário, são estas últimas que empregam o trabalhador; porém,
  apenas com a maquinaria essa inversão adquire uma realidade tangível.
  Transformado num autômato, o próprio meio de trabalho se confronta,
  durante o processo de trabalho, com o trabalhador como capital, como
  trabalho morto a dominar e sugar a força de trabalho viva. A cisão
  entre as potências intelectuais do processo de produção e o trabalho
  manual, assim como a transformação daquelas em potências do capital
  sobre o trabalho, consuma"-se, como já indicado anteriormente, na
  grande indústria, erguida sobre a base da maquinaria'' (Das Kapital,
  \emph{p.~495}). Para uma indicação clara e concisa do trabalho
  abstrato, entendido não como uma categoria lógica abstrata, mas como
  um movimento histórico que se expressa pela divisão do trabalho na
  grande industria, conferir: \textsc{versolato}, Rafael. \emph{O
  mistério do real: capital e trabalho assalariado}. 2016. Faculdade de
  Filosofia Letras e Ciências Humanas. Departamento de Filosofia.
  Universidade de São Paulo, São Paulo, 2016, p.~220 e seguintes.} A
exposição de Engels toma esse ponto como pressuposto evidente.

Essa \emph{fora idealizado,} essa ``dominação violenta estanha''
(\emph{fremde Gewalt}), indicado por Marx e Engels já em 1844/45,\footnote{``O poder (\emph{Macht}) social, isto é, a força de produção
  multiplicada que nasce da cooperação (\emph{Zusammenwirken ---
  atividades que visam um fim comum}) dos diversos indivíduos
  condicionada pela divisão do trabalho, aparece a esses indivíduos,
  porque a própria cooperação não é voluntária, mas espontânea
  (\emph{naturwüchsig}), não como seu próprio poder (\emph{Macht})
  unificado, mas como um poder"-violência (\emph{Gewalt}) externo
  (\emph{fremde}), situado fora deles, sobre o qual não sabem de onde
  vêm ou aonde vão, uma violência (\emph{Gewalt}), portanto, que não
  podem mais dominar {[}e{]} que, pelo contrário, percorre agora uma
  sequência particular de fases e etapas de desenvolvimento,
  independente do querer e da conduta do ser humano (\emph{Luafen des
  Menschen}) e que até mesmo dirige esse querer e conduta.''
  \textsc{marx} K.; \textsc{engels} F. Die deutsche Ideologie.
  \emph{Werke}. p.~29.} porém, e isso é central, não é
\emph{contingente}. Pelo contrário, sua existência é o sinal da
evidência da sua contradição: ``As relações se simplificaram tanto que
era preciso fechar os olhos propositalmente para não ver na luta dessas
três grandes classes {[}a aristocracia possuidora de terras, burguesia e
trabalhador assalariado{]} e no conflito de seus interesses, a força
impulsionadora da história moderna --- pelo menos, nos dois países mais
avançados.''\footnote{\emph{Ludwig Feuerbach e o fim da filosofia
  clássica alemã (parte \textsc{iv})}.} O desenvolvimento histórico do
mundo burguês, assim como a complexa luta de classes que lhe é imanente,
revela que a dinâmica do movimento do real, que aparece a todos como o
poder estranho concreto de uma formação cultural universal, é um
artifício \emph{sistêmico e arbitrário} de converter os interesses
particulares da \emph{forma burguesa} na \emph{forma universal do
interesse}.

A clareza com que essa distorção se expressa com a grande indústria,
como aponta Engels, ``atravessa'' (\emph{Durchführung}) a divisão social
do trabalho no século \textsc{xix}, alterando o \emph{ponto de saída da
reflexividade histórica} no mundo burguês. Tanto enquanto acúmulos de
círculos de reprodução do mais"-valor (\emph{reprodução ampliada}),
gerações de trabalhadores já nascem sob a forma da relação"-capital,
como enquanto processo de reposição, dos modos mais variados, da
expropriação das formas de consciência. Uma alteração complexa,
inquestionável, que não se reduz a uma relação imediata e causal entre
divisão social do trabalho e formas de consciência e permite questionar
tanto a ideia de \emph{uma} \emph{pragmática} \emph{da} \emph{história} ---
que somente poderia ser ``apreendida (\emph{abgefasst}) à medida que produz
(\emph{macht}) prudência (\emph{klug}), isto é, caso ensine ao mundo como
ele poderia assegurar sua melhor vantagem, pelo menos de modo tão bom quanto o
mundo precedente''\footnote{\textsc{kant}, I. \emph{Grundlegung zur
  Metaphysik der Sitten.} Frankfurt am Main, Suhrkamp Verlag, 2007, p.~47.}
--- como a ideia de uma história que entende o presente como resultado
fatalmente contingente de algo sempre já superado, isto é, como
naturalização do elemento transitório. Ambas as concepções correspondem
às duas faces da ``forma da \emph{Bildung}'' burguesa.

A necessidade de enfrentar racionalmente a contingência das contradições
do mundo burguês como \emph{totalidade em movimento} é um ponto
fundamental do texto de Engels. Uma exigência primária para enfrentar
essa totalidade é justamente conseguir visualizá"-la, ter
\emph{consciência do alcance de sua dinâmica}. A discussão sobre o modo
de investigação, sobre o método da crítica, não é, portanto, uma questão
que se coloca separada da luta política concreta. Na realidade, a
possibilidade dessa separação, que irá contaminar o marxismo do século
\textsc{xx} por meio das mais variadas formas de positivismos, começa a
se colocar como tendência a partir desse período de 1880, como o próprio
Engels indica, com a reabilitação de Kant e Hume. Mais à frente
retonaremos a isso.

\section*{Consciência histórica e totalidade}

Em relação ao modo investigativo, o texto é uma \emph{tentativa de
exposição sintética} do materialismo histórico e de como somente por
meio dele uma crítica do desenvolvimento histórico burguês é possível.
Lukács, em seu ensaio \emph{Consciência de classe}, toma a ``famosa''
exposição de Engels como ponto de partida para introduzir sua definição
de \emph{consciência de classe}, mais precisamente de \emph{consciência
classe imputada/atribuída} (\emph{zugerechnet}). Em um diálogo
subjacente com a clássica crítica de Lenin (\emph{Que fazer?}) à
social"-democracia russa (e alemã) sobre a formação de uma classe
revolucionária --- se espontânea, isto é, resultado automático do
desdobramento das contradições do mundo burguês, ou se consciente,
trazida ``de fora'' da relação patrão"-empregado por estruturas
pratico"-teóricas capazes de fazer frente à marcha inexorável do avanço
do desenvolvimento capitalista e da ideologia burguesa ---, o ponto de
partida de Lukács é a crítica à falsa objetividade concreta da
racionalidade histórica burguesa. ``Seu erro consiste em querer
encontrar esse elemento concreto no indivíduo histórico empírico (não
importa se trata de uma pessoa, uma classe, ou povo) e na consciência
empiricamente dada (isto é, dada por uma psicologia ou por uma
psicologia das massas)''.\footnote{\textsc{lukacs}, G. \emph{Geschichte
  und Klassenbewusstsein.Studien über marxistische Dialektik}. Verlag de
  Munter, Amsterdam, 1967, p. 61.} Na sequência de sua argumentação,
Lukács apresenta a retomada --- em sentido contrário ao
anti"-hegelianismo da social democracia alemã, principalmente em Kaustky
--- da categoria da \emph{totalidade concreta, da sociedade
civil"-burguesa como totalidade}. É curioso que Kautsky fora um
entusiasta desse mesmo texto de Engels, que permitiu, porém, outro
caminho de leitura a Lukács.

Um aspecto da dificuldade de uma crítica do desenvolvimento da
totalidade histórica burguesa reside justamente na particularidade do
processo de reprodução da racionalidade político"-moral no interior da
racionalidade histórica, já que na concretude da dinâmica histórica
ambas são a mesma coisa. Como vimos brevemente, elementos dessa confusão
concreta haviam sido apontados por Marx já em 1843 na crítica a Hegel.
No presente texto, Engels expõe a inevitabilidade da constituição dessa
totalidade do ponto de vista da relação entre vontade e história de modo
bastante claro:

\begin{quote}
Os homens fazem a sua história, aconteça ela como acontecer, na medida
em que cada um persegue conscientemente as finalidades que eles mesmos
querem, e a resultante destas várias vontades que atuam em direções
diversas e da sua influencia múltipla sobre o mundo exterior é
justamente a história. Depende, portanto, do que os muitos indivíduos
querem. A vontade é determinada por paixão ou reflexão. Mas as alavancas
que, por sua vez, determinam imediatamente a paixão ou reflexão, são de
tipos muito diversos. Em parte podem ser finalidades exteriores, em
parte \emph{fundamentos ideais do movimento} (\emph{ideelle
Beweggründe}), ambição, ``entusiasmo pela verdade e pela justiça'', ódio
pessoal, ou também caprichos puramente individuais de toda a espécie.
Mas, por outro lado, vimos que as várias vontades individuais ativas na
história, na maioria dos casos, produzem resultados totalmente
diferentes dos pretendidos --- \emph{muitas vezes contrapostos} --- e que,
portanto, para o resultado do todo, seus fundamentos de movimento têm um
significado subordinado. Por outro lado, é possível questionar ainda
mais: quais forças impulsionadoras estão novamente por detrás destes
fundamentos do movimento, que causas históricas transformam, na cabeça
dos agentes, esses fundamentos de movimento?
\end{quote}

A concepção materialista precisa necessariamente enfrentar esse enigma
de compreender \emph{o todo} das inter"-relações que movimentam
\emph{consciente e inconscientemente} suas ações:

\begin{quote}
Quando se trata, portanto, de investigar as potências impulsionadoras --- conscientes ou inconscientes e, de fato, frequentemente inconscientes --- que estão por detrás dos fundamentos dos movimentos dos
homens que agem historicamente, potências estas que constituem propriamente as forças 
motrizes últimas da história, não se pode levar em conta apenas os fundamentos
de movimento dos indivíduos, mesmo considerando aqueles que agem de modo eminente e põem em movimento grandes massas, povos inteiros e,
em cada povo, por sua vez, classes inteiras. Tampouco se pode considerar apenas as ações que se dão
por uma explosão momentânea passageira, fogo de palha que queima
rapidamente, mas a ação duradoura que se alastra em uma grande
transformação histórica.
\end{quote}


Os fundamentos do movimento, sejam eles ideias ou não, direcionam o
olhar necessariamente para além dos processos automáticos de
subjetivação ou repetição objetiva dos limites concretos, seja da psique
ou da classe social. Mas é justamente por isso que eles são apreensíveis
e podem ser explicados por suas causas históricas concretas, porque seu
ponto de saída não é simplesmente o amplo espectro que compõe a
subjetividade, mas justamente aquilo que escapa ao arco da vontade: ``as
várias vontades individuais ativas na história, na maioria dos casos,
produzem resultados totalmente diferentes dos pretendidos --- \emph{muitas
vezes contrapostos} --- e que, portanto, para o resultado do
\emph{todo}, seus fundamentos de movimento têm um significado
subordinado.''

A colisão das vontades, a contradição da luta em movimento, escapa ao
velho materialismo de Feuerbach, tanto quanto escapa a qualquer
racionalidade que não considera o arco de constituição dos fins da
vontade de um tempo histórico, e daquilo que é encoberto no presente,
podendo ou não se realizar no futuro. Ainda assim, algo se põe e
constitui uma estrutura interna do movimento, que não está ao alcance da
consciência, tampouco é resultado de uma autorresolução entre vontade
subjetiva e condição histórica objetiva, como indicado acima. ``Assim,
as colisões das inúmeras vontades singulares e ações singulares no
âmbito histórico proporcionam um estado que é totalmente análogo ao que
domina na natureza \emph{desprovida de consciência}.''\footnote{\emph{Ludwig
  Feuerbach e o fim da filosofia clássica alemã (parte \textsc{iv})}.} A
analogia, antes de configurar um princípio heurístico de apreensão da
realidade histórica, explicita justamente que somente há um acesso
possível e crítico da realidade histórica quando considerada a questão
básica do caráter de \emph{movimento do elemento histórico}. O interesse
subjetivo da ação sempre se põe em movimento, pois ele mesmo é resultado
ideal de um movimento anterior que se projeta objetivamente no limite
dessa idealidade no presente. Esse limite é a realidade contraditória
que forma a consciência de sua existência no presente, a qual não pode
ser buscada meramente naquilo ``que é pensado, sentido e querido
\emph{factualmente} em determinadas condições históricas em situações
determinadas de classe, etc.''.\footnote{``A inter"-relação com a
  totalidade concreta e as determinações dialéticas dela decorrentes
  ultrapassam a simples descrição e resultam na categoria da
  possibilidade objetiva. Ao relacionar a consciência com o todo da
  sociedade {[}como uma \emph{inter"-relação}{]}, são reconhecidos todos
  os pensamentos, sensações, etc., que os homens \emph{teriam} em uma
  situação determinada de vida, se eles \emph{fossem capazes de
  apreender completamente (vollkommende --- consumando até o fim)} essa
  situação, os interesses dela resultantes, tanto em relação à ação
  imediata como em relação à --- conforme tais interesses --- estrutura
  constitutiva (\emph{Aufbau}) de toda a sociedade; os pensamentos, etc.,
  isto, os que estão em conformidade com tal situação objetiva. Em
  nenhuma, o número de tais situações de vida é ilimitado. Mesmo caso se
  busque aperfeiçoar ainda mais sua tipologia por meio de pesquisas
  singulares detalhadas, resulta, porém, em alguns tipos fundamentais
  completamente afastados um do outro. Tipos fundamentais cujo elemento
  essencial (\emph{Wesenart}) é determinado por meio da tipologia da
  posição dos homens no processo de produção. Aqui a reação
  racionalmente adequada que é \emph{imputada}
  {[}\emph{zugerechnet} --- correta, justificada, adjudicada
  --- algo é lançado à adequação por uma ação
  imposta/objetiva{]} no modo de uma situação típica
  determinada no processo de produção, é a consciência de classe.''
  (\textsc{lukacs}, G.\emph{Geschichte und Klassenbewusstsein}.
  \emph{Op.\,Cit.} p.~62)} Apesar da autocrítica lukacsiana da
incapacidade prática da \emph{consciência da falsa consciência}, da
consciência de classe historicamente imputada,\footnote{\textsc{lukacs},
  G. \emph{Geschichte und Klassenbewusstsein}. \emph{Op.\,Cit. Vorwort}
  (Prefácio) 1962: ``A transformação da consciência `imputada' em
  consciência revolucionária aparece aqui, considerada objetivamente,
  como um puro milagre''.} o ponto de partida do problema continua o
mesmo apontado tanto por Engels como por Lukács: o caráter absolutamente
inevitável da relação \emph{consciente"-inconsciente} com a totalidade
histórica.

\begin{quote}
Tudo o que põe os homens em movimento tem de passar por sua cabeça; mas
que configuração toma nessa cabeça, depende muito das circunstâncias. Os
trabalhadores, sob nenhuma circunstância, reconciliaram"-se com o
maquinário fabril capitalista, mesmo que não mais tenham simplesmente
quebrado em pedaços as máquinas, como ainda em 1848 no Reno.\footnote{\emph{Ludwig Feuerbach e o fim da filosofia clássica alemã (parte \textsc{iv})}.}
\end{quote}

Aqui para muitos, como para a crítica social alemã dos anos 1920 e 1930,
salta aos olhos da concretude o erro da última passagem. A
\emph{reconciliação} entre uma posição de classe e circunstâncias
contrárias teria se mostrado como um fenômeno corriqueiro. Na realidade,
a questão dessa reconciliação, não por acaso uma dimensão subjacente às
diversas discussões sociais"-democratas do entre guerras, é expressão de
mais um sintoma da expansão da relação"-capital, que reforça os
processos de expropriação das formas de consciência\footnote{Dois
  exemplos de exposição dos mecanismos de \emph{expropriação das formas
  de consciência} podem ser encontrados na crítica da anatomia da classe
  média alemã reacionária dos anos 1920 e 1930, fração dos trabalhadores
  que irá apoiar o Nacional Socialismo, presente na análise
  literário"-sociológica de Siegfried Kracauer de 1930, \emph{Os
  empregados,} e na obra de Ernst Bloch, de 1935, \emph{A herança dessa
  época,} onde inclusive há uma brilhante definição da estrutura
  ideológica que forma essa figura da ``classe média reacionária'',
  definida por Bloch com um termo ambíguo em alemão, mas que, bem
  entendido, é quase autoexplicativo: \emph{dispersão"-distrativa
  (Zerstreuung)}.} na época do \emph{imperialismo}, momento em que na
passagem do século \textsc{xix} para o \textsc{xx} a totalidade
sistêmica do capital já posta há mais de meio século começa a se
transfigurar em outros elementos totalizantes: nacionalismo, pátria,
progresso, técnica. A ``falsa consciência'' de classe é poderosa
justamente porque ela é resultado do impulso reflexivo dessa
``reconciliação'', inatingível por meio da dimensão subjetiva da
consciência, mas que explicita o limite objetivo do movimento histórico
de constituição da divisão social do trabalho. Esse limite expressa a
impossibilidade da ``reconciliação''. A começar porque nunca houve uma
unidade anterior a ser reconciliada entre o trabalhador no mundo burguês
(entendido na dinâmica do devir da escravidão e servidão para forma
assalariada) e as circunstâncias da acumulação originária (primitiva) --- mesmo se pensarmos na ``cisão entre o trabalhador e a \emph{propriedade} das condições de efetivação de seu trabalho'' ---, já que a relação das
passagens históricas da cooperação até a grande indústria com os
impulsos violentos que geram a cisão (\emph{Trennung}), entre eles, por %MANTIVE
exemplo, a Revolução Gloriosa mencionada por Engels nesse texto, indicam
que não é possível considerar essa não"-cisão a partir de uma
\emph{conciliação anterior}. A subjetividade da consciência não alcança
os fins que coloca, pois eles nunca são apenas seus, ainda que se creia
profundamente nisso. A objetividade do desenvolvimento histórico se põe
a partir e para além da subjetividade. Para além da imposição sistêmica
do ``ter que trabalhar'', poucos trabalhadores ``se reconciliariam'' com
a vida que levam no mundo do império do livre"-arbítrio, como é \emph{a
aparência} do mundo burguês, sem um reforço sistêmico de elaboração de
\emph{convicções}, sem um reforço sistêmico de todas as ``pedagogias''
para \emph{servidão voluntária}, muito mais eficientes do que La Boétie
poderia imaginar em 1563.

\section*{O destino da filosofia alemã com\break a consolidação do mundo burguês}

As contradições do mundo burguês a partir do século \textsc{xix}
\emph{escapam} para todos os lados. É \emph{impossível não sentir os
efeitos da contradição histórica da expansão da relação"-capital na
Alemanha de 1886}. O \emph{fim} da época da filosofia clássica alemã,
nesse sentido, não é simplesmente o \emph{ponto final} da filosofia
clássica. A filosofia clássica não se esgotou com a morte de Hegel em
1831. E a referência ao neokantismo na Alemanha e à retomada de Hume na
Inglaterra pelo próprio Engels indica justamente que não se trata de um
fim da filosofia clássica alemã em si mesma, mas de um \emph{ponto de
saída} da \emph{relação} da filosofia alemã com \emph{o devir do
desenvolvimento histórico burguês}. A metade do século \textsc{xix} é o
ponto de transição, de saída, para o momento em que o mundo burguês
\emph{fica sobre os próprios} \emph{pés}, completa esse devir e revela
as contradições do seu próprio desenvolvimento, encobertas pelo seu
próprio devir. Essa filosofia torna"-se clássica justamente porque se
converte em ponto de mediação intransponível com o que vem antes dela, o
que expressa seu caráter sintomático, já que não pode ser pensada como
tendo o mesmo papel que tinha anteriormente, distorcida como uma
\emph{filosofia em si mesma}. Claro, Engels sabia há muito que não há
algo como uma \emph{filosofia em si mesma} separada da realidade:
platonismo, kantismo, hegelianismo, nem mesmo marxismo. O limite do
movimento do efetivo, do real, é a consciência (filosófica, jurídica,
política, econômica, religiosa, moral, etc.), tanto quanto a consciência
está limitada pelo efetivo, aqui colocando em outras palavras a
conhecida e mal compreendia proposição da \emph{Ideologia Alemã}: ``Não
é a consciência que determina (\emph{bestimmt --- indica o limite, o
termo}) a vida, mas a vida que determina (\emph{bestimmt --- indica o
limite, o termo}) a consciência''.

Em 1848 estávamos diante do \emph{ponto de saída da filosofia clássica
alemã}. O titulo, o \emph{fim (Ausgang}) \emph{da filosófica clássica
alemã}, que desde a tradução francesa de 1894 revisada pelo próprio
Engels --- \emph{Ludwig Feuerbach et la fin de la philosophie classique
allemande} --- tornou"-se cânone das traduções latinas, induz, portanto,
a uma limitação. Não se trata, porém, de um erro de tradução.
Normalmente \emph{Ausgang} é traduzido como ``saída'', ``ponto de
saída'', ao contrário do termo ``fim'', \emph{Ende,} mas no campo
semântico em que foi empregado, relacionado a uma época histórica, cabe
a tradução por ``fim'', desde que se tenha em vista essa dimensão de ser
\emph{um ponto de passagem ou de saída} para outra época histórica. Esse
\emph{ponto de saída} é o que define o enfrentamento de conjuntura
empreendido por Engels em 1886, já que, como apontado acima, e
certamente ficará claro ao leitor do livro, a filosofia de Feuerbach
apresentada por Starcke, sua crítica ou defesa, não é o objeto da obra.
Feuerbach é apresentado no máximo como catalisador de um \emph{ponto de
transição} que a própria realidade já evidenciava nos anos 1840 alemães.

Engels define a conjuntura 1848/1886 como resultado e explicitação do
processo de expansão da relação"-capital, apreendido pela via
\emph{prussiana,} que virá a ser tematizada por ele próprio em
1887/1888, em outro artigo a ser publicado na \emph{Die Neue Zeit}.\footnote{\textsc{engels}, F. \emph{Die Rolle der Gewalt in der
  Geschichte.} \textsc{mew}, 21. Dietz Verlag, Berlin. 5. Auflage, 1975.}
Três vias de unificação da Alemanha abriram"-se com o fracasso das
Revoluções de 1848, a primeira, a \emph{via alemã}, revolucionária que
suprimiria a autonomia político"-econômica dos pequenos estados
(\emph{Einzelstaaten}) por uma guerra de unificação conduzida pelo povo
contra Napoleão \textsc{iii} e as dinastias prussianas e austríacas, a
segunda, \emph{a via pela dominação da Áustria} e a terceira, a via da
\emph{Realpolitik} de Bismarck, \emph{a via prussiana}. Como sabemos, a
última, não por acaso, impôs"-se, já que vinha sendo gestada desde 1830
com a \emph{Zollverein} (união aduaneira), quando a estrutura de
estabilização do modo de produção e relação burguesa começa a forjar a
unidade territorial e política de parte do atual território alemão, que
terá como ponto de chegada o \textsc{ii} Reich (\emph{Deutsches
Kaiserreich}).

Partindo da via prussiana, a metáfora do \emph{ponto de saída} atravessa
toda a argumentação de Engels e se estrutura em duas dimensões
indissociáveis. A primeira refere"-se ao fato de que ``na Alemanha do
século \textsc{xix}, a revolução filosófica preparou o colapso
político''. A crítica que preparou o colapso político não foi, portanto,
empreendida por uma burguesia politicamente revolucionária e liberal,
apesar de ser possível questionar o verdadeiro papel social dos
``mandarins alemães''. De fato, na conjuntura 1830/1848 e depois em 1870
com o processo de unificação da Alemanha com Bismarck, o tempo de uma
burguesia revolucionária já havia passado. Uma imagem de unidade de
época se rompe. No plano filosófico, o sistema de Hegel é a última
tentativa de refletir uma unidade, principalmente na milenar relação
\emph{teologia/política/direito/filosofia.}

O caráter antissistêmico, em oposição à sistematicidade da filosofia
hegeliana, que agradou Starcke na obra de Feuerbach, já havia chamado
atenção de Marx e Engels há 40 anos. Na realidade, um aspecto
conjuntural fundamental das Revoluções de 1848 na Alemanha. A
\emph{unidade política alemã,} sob a égide do avanço da \emph{unidade do
mundo burguês}, traz para o chão da concretude a abstrata unidade
teórica da antiga relação política/religião. A intenção filosófica dessa
unidade, que --- \emph{no plano das ideias} --- pode saltar
arbitrariamente de Aristóteles a Hegel, antes de cair na unidade do
\emph{absoluto}, implode justamente com a consideração interna da
dinâmica de transitoriedade da racionalidade dialética. Uma questão
elementar, apontada por Marx já na tese de doutorado (1837), apreendida
como crítica ao modo de intuição filosófico idealista, é que o propósito
do filosofar, ``do espírito teórico'', não é
\emph{explicar"-compreender} o mundo, mas \emph{intervir no mundo},
mesmo quando quer apenas \emph{interpretá"-lo}:

\begin{quote}
É uma lei psicológica que o espírito teórico, que vem a ser livre em
si (\emph{in sich}) mesmo, torne"-se uma energia prática, emergindo
enquanto \emph{vontade} do reino das sombras de \emph{Amenthes},
volte"-se contra a mundana, existente (\emph{vorhanden}) sem o
espírito, realidade efetiva (\textsc{i}.1/ 67 e 68).
\end{quote}

A filosofia como vontade (``como impulso de se efetivar'') se contorce
para fora (\emph{herauskehren}) do mundo fenomênico o trazendo consigo.
A tensão contraditória entre a relação de reflexividade da ideia e
autonomia pré"-existente (\emph{vorhande}) do mundo não pode ser
superada por si só, pois o movimento da ideia consome a si mesmo ao se
efetivar no mundo e altera: \emph{a si mesmo}, \emph{ao mundo} e
\emph{à própria reflexividade}.\footnote{``Assim surge a consequência de
  que o vir a ser filosófico do mundo é ao mesmo tempo um vir a ser
  mundano da filosofia, que a sua efetivação é, ao mesmo tempo, sua
  perda (\emph{Mangel}), que, aquilo que ela combate no lado de fora é
  seu próprio defeito interno, que justamente na luta ela incorre em
  danos (\emph{Schade} --- falta de vínculos), que ela combate como
  danos e que somente os supera (\emph{aufheben}), na medida em que
  incorre nos mesmos''. (\textsc{mega} \textsc{i}.1/67 e ss).} A
pressuposição de uma autonomia pré"-existente, sempre à disposição da
força pura da ideia, sem perceber (isto é, sem considerar fundamental a
compreensão do movimento real das coisas) a \emph{influência recíproca}
concreta dessa relação \emph{contraditória} incompleta, em que tanto do
lado subjetivo da ideia, como do lado objetivo da realidade --- na busca
de uma relação proporcional (estável) capaz de expressar a medida das
coisas --- é um erro. Os excessos e elementos evasivos, danos
(\emph{Schaden}) no vocabulário da tese de 1837, são sempre inevitáveis.
Nem mesmo a sistematicidade hegeliana escaparia desse movimento, ou do
\emph{próprio movimento da história}, como Engels demonstra em alguns
aspectos. Por isso, a relação de Marx e Engels com a tradição do
idealismo alemão, passando justamente pelos jovens hegelianos e
Feuerbach, sempre teve em mente a contradição e a tensão entre ideia e
realidade, mas percebendo a possibilidade de compreender nessa tensão os
excessos e elementos evasivos, os danos que o embate entre ideia e
realidade impõe.

A filosofia hegeliana seria a expressão dos limites desse embate:

\begin{quote}
Com Hegel se encerra a filosofia em geral. Por um lado, porque ele
reuniu em seu sistema, do modo mais grandioso, todo o desenvolvimento da
filosofia; por outro, porque, ainda que inconscientemente, nos mostra o
caminho para fora desse labirinto de sistemas em direção ao conhecimento
positivo e efetivo do mundo.\footnote{\emph{Ludwig Feuerbach e o fim
  da filosofia clássica alemã (parte \textsc{i})}.}
\end{quote}

A segunda dimensão do ponto de saída corresponde à contradição imanente
da relação \emph{reação"-revolução} da expansão do mundo burguês. Entre
1789 e 1851, a revolução burguesa se resolve na indeterminação entre
\emph{a reação} feudal à revolução burguesa em 1789 e \emph{a reação}
burguesa à \emph{possibilidade da} revolução proletária, principalmente
a partir da comuna de Paris em 1870. As duas reações se fundem e se
sobrepõem sistemicamente enquanto totalização da relação"-capital entre o
bonapartismo na França em 1851 e via prussiana com Bismarck, apesar da
evidente particularidade do desenvolvimento histórico de cada processo.

Para sintetizar algo complexo que vale uma obra, questão trabalhada em
muitos aspectos por Lukács, \emph{a via prussiana é a via de saída da
filosofia clássica alemã,} restando às intenções filosóficas teóricas,
de desvelar os mistérios do real, converterem"-se, a partir da metade do
século \textsc{xix}, \emph{na forma farsesca dos seus conteúdos
trágicos}, forçando aqui a brilhante e conhecida metáfora de Marx. A
razão disso, de modo sucinto, é a modificação do papel social do saber
filosófico, em razão da, nos termos de Engels, ``simplificação'' da
estrutura social europeia: ``tanto que era preciso fechar os olhos
propositalmente para não ver na luta dessas três grandes classes {[}a
aristocracia possuidora de terras, burguesia e trabalhador
assalariado{]} e no conflito de seus interesses, a força impulsionadora
da história moderna --- pelo menos, nos dois países mais avançados.''

O lugar social do filósofo começa a ser determinado pelos problemas a
que é chamado a oferecer respostas e de onde ele formula suas respostas.
A mudança do papel das universalidades a partir da metade do século
\textsc{xix} em relação ao desenvolvimento das forças produtivas prepara
a passagem da centralidade do ``filósofo"-teólogo'' para o
``cientista"-empregado''. Essa passagem é subjacente à análise de
Engels, elaborada ainda nas ruínas do papel singular do papel do
filósofo"-teólogo na Alemanha. Essa dinâmica estabelece necessariamente
um novo tipo de ruptura com a realidade, estabelecendo um papel diverso
dos filósofos"-teólogos do mundo pré"-burguês. Uma ruptura que ocorre
tanto no nível cognitivo mais elementar, enquanto relação com a
objetividade e seus circuitos cotidianos, como também em um nível
social"-político mais amplo como autorreflexividade do mundo burguês
enquanto \emph{comunidade universal.} O caráter sistêmico da
desigualdade social e da falsa liberdade concreta do mundo burguês,
nessa passagem de uma filosofia burguesa crítica dos fundamentos do
mundo feudal para uma burguesia reacionária em crise com o mundo que se
põe, consciente ou inconscientemente, desloca o \emph{locus} do problema
e explicita justamente --- aqui o clássico problema do \emph{fetiche da
objetividade} inaugurado por Lukács\footnote{Conferir, por exemplo:
  \textsc{lukács}, G. \emph{Existentialismus oder Marxismus?} Berlin:
  Aufbau"-Verlag, 1951, p.~8 e ss.} como questão para o marxismo do
século \textsc{xx} --- um \emph{descolamento específico} em relação à
realidade efetiva, que não \emph{pode mais ser apreendida por
fundamentos universais, sem vincular a si mesma aos
deslocamentos"-dissimuladores} impostos pela universalidade e unidade
social do mundo burguês.

Engels tenta, em certo momento, delimitar a alteração de função da
\emph{intelligentsia} filosófica alemã, opondo as descobertas
científicas à antiga filosofia da natureza alemã, indicando, por
exemplo, a materialidade da ``coisa em si'' de Kant: ``as matérias
químicas produzidas em corpos vegetais e animais eram as tais ``coisas
em si'' até que a química orgânica começou a apresentá-las uma após
outra; com isso, a ``coisa em si'' se tornou uma coisa para nós''. A
questão é que Marx já havia mostrado que o enigma da objetividade da
experiência tem um \emph{locus} não enigmático. A reprodutibilidade da
relação"-capial é o que produz a simplificação da luta das três classes
apontadas por Engels, produz a divisão do acesso aos meios de
subsistência, a vinculação entre modo de produção e relação de produção.
A mobilidade dessa reprodutibilidade é inegável, a ilusão de um controle
\emph{externo} a essa reprodutibilidade é base de todos os enigmas da
forma político"-moral burguesa.

Nesse sentido, os compromissos de classes, entre Napoelão \textsc{iii} e
a burguesia francesa, Bismarck e a burguesia alemã, são resultados
sistêmicos desses deslocamentos. O irracionalismo que Lukács apontou
para a filosofia na época do imperialismo tem esse caminho. O caminho da
miséria da filosofia se abre com a complexa dinâmica de expansão do
mundo burguês. A partir de 1848 nenhum tema das sintomaticamente
denominadas \emph{Geistwissenschaften} [\emph{ciências do espírito}] escapará dessa relação, que
sempre voltará com suas mascaras. É nesse contexto que as retomadas,
tanto do agnosticismo, como do neokantismo, podem ser compreendidas:

\begin{quote}
Se, entretanto, a reabilitação da concepção kantiana é tentada na
Alemanha pelos neokantianos e a reabilitação de Hume na Inglaterra (onde
nunca morreu) pelos agnósticos, cientificamente, diante da refutação
teórica e prática há muito alcançada, isso é um retrocesso e,
praticamente, apenas um modo envergonhado de aceitar o materialismo
pelas costas e de o negar perante o mundo.\footnote{\emph{Ludwig
  Feuerbach e o fim da filosofia clássica alemã (parte \textsc{ii})}.}
\end{quote}

\section*{Engels crítico do neokantismo}

A partir de 1880 o neokantismo configura"-se como a escola predominante
nas universidades de filosofia na Alemanha, tendo efeitos diretos nas
chamadas \emph{Geistwissenschaften}, particularmente na \emph{Ciência do
Direito} (o exemplo paradigmático será a \emph{Teoria pura do Direito}
de Hans Kelsen na Áustria em 1934). Um tema central inaugurado por
Engels nesse texto: \emph{a} \emph{primeira crítica materialista do
neokantismo}. A pré"-história teórica da retomada de Kant delimita
justamente o problema da relação entre subjetividade e razão, tendo em
vista justamanete o problema do \emph{a priori}: poderia este ser
compreendido como ``um instituição orgânica inata ao gênero
(\emph{angeborene Gattungsorganisation})'' humano?\footnote{\textsc{ollig},
  Hans"-Ludwig, \emph{Der Neukantismus,} Stuttgart, Springer"-Verlag,
  GmbH, 1979, p.~1.}

A retomada de Hume, o agnosticismo inglês, não por acaso, compartilha a
mesma dimensão filosófica. Em termos teóricos, a \emph{aporia} da
experiência sensível racional, entre ceticismo e filosofia
transcendental, gira em torno, como se sabe, do desafio assumido por
Kant diante de Hume, ``o geógrafo da razão humana'', de construir um
fundamento racional para proposições como: ``toda mudança necessita de
uma causa''. A existência da causa é faticamente, empiricamente,
inegável, afinal a água não esquenta sem o fogo, e a maior prova disso é
que se colocarmos a mão na água quente iremos nos queimar. Para além do
mau ``hábito'' de queimar a mão, caberia comprovar que a água sempre
esquenta com o fogo, comprovar como posso saber que entre esses dois
entes --- água quente e fogo --- há uma ligação que não é fruto da mera
contingência empírica. Em outras palavras: como é possível antecipar
racionalmente um vínculo entre fenômenos sem precisar sempre recorrer à
má experiência? Uma questão que nesse nível de empiria, água quente e
fogo, parece banal, mas que se torna complexo quando passamos para
dimensão metafísica: como justificar racionalmente Deus como causa do
mundo? Ou ainda: como entender a relação entre ``má experiência'' e
produtividade? A resposta kantiana ao desafio cético passa pela
aceitação da inacessibilidade metafísica da dimensão fenomênica do
mundo, mas que, justamente por isso, devido à indeterminação sensível do
fenômeno, revelaria a inevitabilidade do papel atuante e determinante da
\emph{noumenon} no conhecer e pensar da razão transcendental.

A dificuldade aqui, na realidade, é situar um fundo histórico desse
debate, principalmente tendo em vista a ideia de uma ``instituição
orgânica do gênero'' humano enquanto ser racional e livre, que contém
inato, em si, a potência \emph{a priori}, ao mesmo tempo, espontânea e
reflexionada, de constituir o \emph{critério} para avaliar os
automatismos da realidade efetiva e o sentido das transformações sociais
que o século \textsc{xix} trazia. O estabelecimento de um critério
transcendental, a partir de um vínculo interno entre \emph{autonomia da
vontade} e realidade efetiva, é a questão central da \emph{Fundamentação
da metafísica dos costumes}. Como indicado no inicio dessa apresentação,
a obra de Marx e Engels questiona, como nunca antes, a possibilidade da
\emph{autonomia da vontade em termos transcendentais}, e expande o
problema para além de uma questão meramente de critica da política, da
religião ou da filosofia. O desenvolvimento e protagonismo das ciências
da natureza no século \textsc{xix}, pressuposto da ``querela do
materialismo'' (\emph{Materialismusstreit}) diante do transcendental,
como aponta Engels, não se deu apenas pela ``força do puro pensamento'',
mas também pelo ``sempre mais veloz impetuoso progresso da ciência da
natureza e da indústria''. Esse progresso impetuoso implica tanto uma
mudança ``interna'', de avanço e novas descobertas, desenvolvimento da
matemática, física, química, biologia, como uma transformação
``externa'', pelo desdobramento histórico da indústria e alteração
continua do modo de produção e divisão social do trabalho, que chega a
um ponto culminante em 1848.

O neokantismo, que direcionou as discussões das universidades alemãs por
quase um século no âmbito das ciências do espírito, é uma resposta, mais
ou menos direta, ao posicionamento"-constitutivo da totalidade
sistêmica do capital com a grande indústria. As ciências duras, da
natureza, não podem, portanto, mais ser compreendidas unicamente pelo
desenvolvimento interno de seu caráter cognitivo ou capacidade de
explicar os mistérios da natureza e da humanidade, e não ocupam mais um
espaço ao lado das discussões metafísicas, submetidas ao controle
político"-social da Igreja, inseridas, principalmente no idealismo
alemão, como o próprio Engels aponta, nas diversas ``filosofias da
natureza''. Na realidade, a querela entre o papel ativo da subjetividade
diante dos automatismos da realidade, pano de fundo do neokantismo, é um
dos complexos efeitos ideológicos do estabelecimento de uma nova divisão
do trabalho ``no interior da sociedade'',\footnote{\textsc{marx}, K. \textit{Das Kapital. Erster Band}. 39 Aufl. 2008, p.\,371.} que tem como um dos aspectos fundamentais o longo processo de reposição das ciências da natureza no processo histórico de passagem da
manufatura para grande indústria.

Um processo que parte do sistema de cooperação artesanal capitalista, passa pela
autonomização parcial do trabalhador, divisão territorial do trabalho e
concentração da produção na mão do capitalista na manufatura, até chegar
à completa autonomização abstrata do trabalho com a maquinaria. O núcleo
do argumento de Marx é a \emph{interação recíproca} entre divisão do
trabalho na manufatura capitalista, isto é, a divisão do processo de
produção ``no interior da manufatura'' e a divisão do trabalho no
``interior da sociedade civil"-burguesa''. ``Na medida em que a produção
e circulação de mercadorias é o pressuposto geral do modo de produção
capitalista, a divisão do trabalho mediada pela manufatura exige que a
divisão do trabalho tenha amadurecido até certo grau de desenvolvimento
no interior da sociedade. Às avessas, a divisão do trabalho mediada pela
manufatura desenvolve e multiplica por efeito retroativo
(\emph{rückwirkend}) aquela divisão social do trabalho''.\footnote{\textsc{marx}, K., idem.} Os
meios de subsistência somente existem enquanto meios de produção, ou em
função dos meios de produção, o que se explicita, por exemplo, com a
nova divisão entre campo e cidade que se intensifica sem precedentes a
partir do século \textsc{xix} europeu, constituindo a cidade como
\emph{telos} automático ideal de certo desenvolvimento histórico do
campo. Esse longo processo de transformação, como se sabe, acontece de
modo diverso em cada país e passa sempre pelo desdobramento das formas
de dominação precedentes.

Além de naturalizar, disciplinar e condicionar o trabalhador, como
aquele a quem não resta nada além de vender suar força de trabalho, o
devir da fábrica manufatureira em grande indústria, a divisão
``interna'' do trabalho, constitui a autoridade incondicional do
capitalista, na mesma medida em que permite a este, no âmbito da divisão
social do trabalho ``no interior da sociedade'', ``por seu efeito
retroativo'', aparecer como um produtor e vendedor de mercadoria
submetido à mesma concorrência, imposição e pressão dos interesses que
parecem \emph{externos} à divisão do trabalho fabril, como se fossem
resultados unicamente do arbítrio e contingência, em última instância,
de uma ``necessidade'' histórica decorrente da abstração da
\emph{vontade} \emph{individual}.

A autoridade incondicional que o capitalista detém, iniciada com a
cooperação quando o capitalista representa a unidade e vontade do corpo
de trabalho, intensifica"-se com a hierarquização imposta pelo trabalho
parcial na manufatura e se completa com a grande indústria, pela
instituição da maquinaria, com a cisão entre ciência e força de
trabalho, antes unidas no camponês autônomo ou trabalhador manual, tudo
e todos agora a serviço da produção de mais"-valor.\footnote{\textsc{marx}, K., op.\,cit., p.\,377} A dimensão subjetiva do trabalhador é completamente suplantada
pela existência material da maquinaria, assim como a identificação
imediata do capitalista como unidade e vontade do corpo de trabalho. A
ciência da natureza torna"-se a ``consciência'' da produção capitalista
que substitui a rotina baseada na experiência: ``Enquanto maquinaria o
meio de trabalho passa a receber um modo de existência, que condiciona a
substituição da força de trabalho humana pelas forças naturais e as
rotinas mediadas pela experiência pela aplicação consciente da ciência
da natureza''.\footnote{\textsc{marx}, K., op.\,cit., p.\,407.} Essa substituição será fundamental, pois os efeitos
indiretos, antes de qualquer salto ``filosófico'' apressado entre
\emph{noumenon e phainomenon}, implicam uma reorganização e
redistribuição do controle dos centros de produção de saber nas
universidades, delimitando a centralidade das ciências da natureza. A
oposição entre materialismo e idealismo poderá assim alcançar um núcleo
social de reflexividade. As ciências da natureza terão um papel
fundamental como um dos pontos constitutivos da ``consciência'' da
sociedade civil"-burguesa, fundamental para entender a subsunção
intelectual efetiva do trabalho ao capital, às potências de controle das
forças produtivas. Uma ``consciência'' impessoal que necessariamente irá
se formar \emph{a partir e para além} de qualquer dicotomia entre
sujeito e objeto, autonomia da vontade e autonomia da natureza,
autonomia da vontade subjetiva"-individual e autonomia da vontade
objetiva"-social.

Na sua dimensão concreta, o neokantismo pode ser, portanto, compreendido
como uma das expressões do processo de legitimação da dominação da
relação"-capital por meio do movimento de suas ``formas de
consciência''. É justamente aqui que o problema de uma ética
transcendental ganha sentido, na relação entre \emph{autonomia da
vontade} e \emph{legitimação do monopólio da dominação
racional"-violenta}, não apenas pelo Estado e pela forma da política, do
direito, ou pelas racionalidades de gerenciamento social, mas como
tentativa de tradução filosófica do enfrentamento do automatismo por
meio do qual a realidade efetiva se impõe enquanto potência estranha
(\emph{fremde Macht}), fazendo com que o mundo apareça, ao mesmo tempo,
como \emph{possível} representação universal do gênero racional humano e
como \emph{necessária} imposição objetiva de um conteúdo empírico que
escapa a qualquer representação subjetiva possível. A ilusão de
adequação a essa contradição é o lugar de conforto do
estranhamento"-de"-si da burguesia, que assume tal ilusão como seu
próprio poder, mas de aniquilação do proletariado a quem está dada, no
chão da história, a impossibilidade de adequação a ela.

\section*{Todo o efetivo é irracional,\break todo o irracional é efetivo}

A irracionalidade torna"-se estrutural na medida em que a materialidade
da história é recusada por uma subjetividade forjada que sempre já não
alcançou a concretude do movimento histórico. A expansão da
relação"-capital determina o ritmo da miséria da razão e seu novo
\emph{Standpunkt}, um movimento mais do que transparente já em 1886 na
Alemanha:

\begin{quote}
E, no âmbito das ciências históricas, incluindo a filosofia,
desapareceu, junto à filosofia clássica, com maior razão, o velho
espírito teórico"-brutal (\emph{theoretisch"-rücksichtslose}):
ecletismo desprovido de pensamento, preocupação angustiada com carreiras
e rendimentos descendo até ao arrivismo (\emph{Strebertum}) mais
ordinário, tomam seu lugar. Os representantes oficiais desta ciência
tornaram"-se ideólogos não encobertos da burguesia e do Estado
existente --- mas em um tempo em que ambos estão em oposição aberta à
classe trabalhadora.\footnote{\emph{Ludwig Feuerbach e o fim da
  filosofia clássica alemã (parte \textsc{iv})}.}
\end{quote}

A crítica brutal da realidade efetiva é ainda a tarefa. Como Marx
menciona na conhecida carta a Ruge, onde reflete a situação política na
Alemanha em 1843 e a função do crítico, este não pode retirar suas
``armas'' da história da filosofia, do direito ou da economia, mas da
``\emph{crítica brutal de toda realidade dada}''\footnote{\textsc{marx},
  K. \emph{Deutsche Französische Jahrbücher 1. Doppellieferund},
  Februar, 1844. In: \textsc{marx}, K.; \textsc{engels}, F. Werke. Band
  1. Berlin/\textsc{ddr}: Dietz Verlag, 1976, p.~344.} (\emph{die
rücksichtlose Kritik alles Bestehenden}), ou seja, da crítica que tem em
vista o todo de uma tradição que se estabilizou (\emph{Bestehende}) e
cujos elementos, justamente por isso, \emph{aparecem como necessários}
no presente. A ciência crítica da história, que opera sem restrições,
que faz a \emph{crítica brutal da realidade}, é aquela que trabalha
\emph{no interior do movimento de realização} da ``determinação ideal e
seus pressupostos efetivos'', \emph{a partir e para além da necessidade
do movimento}.

Apenas assim a crítica consegue ser ``brutal'' ou ``sem restrições''
(\emph{rücksichtlos}), como define Marx, ``tanto no sentido de não poder
temer os seus próprios resultados quando no sentido de não poder temer os
conflitos com os poderes estabelecidos''.\footnote{\textsc{marx}, K., idem.}
A filosofia alemã, que tem como auge a dialética hegeliana, antes de ser
absorvida pela consolidação do mundo burguês na Alemanha pós"-1848,
vendo ``\emph{de fora''} o avanço da relação"-capital na França e
Inglaterra, estava ainda em condições de questionar a \emph{necessidade
do movimento para si,} até o momento que esse mesmo movimento se torna
sua \emph{própria necessidade} e é incorporado pelas suas próprias
restrições.


