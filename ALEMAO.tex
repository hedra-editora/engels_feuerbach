%\begingroup\makeatletter\@openrightfalse

{\let\clearpage\relax\chapter*{Vorbemerkung}}
%\@openrighttrue\makeatother \endgroup


\noindent{}In der \emph{Vorrede von „Zur Kritik der Politische Ökonomie``},
Berlin, 1859, erzählt Karl Marx, wie wir beide 1845 in Brüssel uns
dranmanchten, „den Gegenstaz usrer Ansicht`` --- die namentlich durch
Marx herausgearbeiteten materialistischen Geschichtsauffassung --- „gegen
die idelogische der deutsche Philosophie gemeinschaftlich auszuarbeiten,
in der Tat mit unserm ehemaligen philosophischen Gewissen aubzurechnen.
Der vorsatz wurde ausgeführt in der Form einer \emph{Kritik der
nachhegelschen Philosophie}. Das Manuskript, zwei strake Oktavbände, was
längst an seinem Verlagsort in Westfalen angelangt, als wir die
Nachricht erhielten, daß veränderte Umstände den Druck nicht erlaubten.
Wir überließen das Manuskript der nagenden Kritik der Mäuse um so
williger, als wir unsern Hauptzweck erreicht hatten ---
Selbstverstädigung.``

Seitdem sind über vierzig Jahre verflossen, und Marx ist
gestorben, ohne daß sich einem von uns Gelegenheit geboten hätte, auf
den Gegenstand zurückzukommen. Über unser Verhältnis zu Hegel haben wir
uns stellenweise geäußert, doch nirgends in umfassendem Zusammenhang.
Auf Feuerbach, der doch in mancher Beziehung ein Mittelglied zwischen
der Hegelschen Philosophie und unsrer Auffassung bildet, sind wir nie
wieder zurückgekommen.

Inzwischen hat die Marxsche Weltanschauung Vertreter gefunden
weit über Deutschlands und Europas Grenzen hinaus und in allen
gebildeten Sprachen der Welt. Andrerseits erlebt die klassische deutsche
Philosophie im Ausland eine Art Wiedergeburt, namentlich in England und
Skandinavien, und selbst in Deutschland scheint man die eklektischen
Bettelsuppen satt zu bekommen, die dort an den Universitäten ausgelöffelt werden unter dem Namen Philosophie.

Unter diesen Umständen erschien mir eine kurze, zusammenhängende
Darlegung unsres Verhältnisses zur Hegelschen Philosophie, unsres
Ausgangs wie unsrer Trennung von ihr, mehr und mehr geboten. Und ebenso
erschien mir eine volle Anerkennung des Einflusses, den vor allen andern
nachhegelschen Philosophen Feuerbach, während unsrer Sturm-und
Drangperiode, auf uns hatte, als eine unabgetragene Ehrenschuld. Ich
ergriff also gern die Gelegenheit, als die Redaktion der „Neuen Zeit``
mich um eine kritische Besprechung des Starckeschen Buchs über Feuerbach
bat. Meine Arbeit wurde im 4. und 5. Heft 1886 jener Zeitschrift
veröffentlicht und erscheint hier in revidiertem Sonderabdruck.

Ehe ich diese Zeilen in die Presse schicke, habe ich das alte
Manuskript von 1845/46 nochmals herausgesucht und angesehn. Der
Abschnitt über Feuerbach ist nicht vollendet. Der fertige Teil besteht
in einer Darlegung der materialistischen Geschichtsauffassung, die nur
beweist, wie unvollständig unsre damaligen Kenntnisse der ökonomischen
Geschichte noch waren. Die Kritik der Feuerbachschen Doktrin selbst
fehlt darin; für den gegenwärtigen Zweck war es also unbrauchbar.
Dagegen habe ich in einem alten Heft von Marx die im Anhang
abgedruckten elf Thesen über Feuerbach gefunden. Es sind Notizen für
spätere Ausarbeitung, rasch hingeschrieben, absolut nicht für den Druck
bestimmt, aber unschätzbar als das erste Dokument, worin der geniale
Keim der neuen Weltanschauung niedergelegt ist.

\bigskip

\hfill{}London, 21. Februar 1888 

\quebra

\begin{flushright}
\emph{Die Neue Zeit. Jg. 4.}\\
\emph{1886. Nr, 4, April}
\end{flushright}

\vspace{2cm}

\section{I}

\noindent{}Die vorliegende
Schrift führt uns zurück zu einer
Periode, die, der Zeit nach, ein gutes Menschenalter hinter uns liegt,
die aber der jetzigen Generation in Deutschland so fremd geworden ist,
als wäre sie schon ein volles Jahrhundert alt. Und doch war sie die
Periode der Vorbereitung Deutschlands für die Revolution von 1848; und
alles, was seitdem bei uns geschehn, ist nur eine Fortsetzung von 1848,
nur Testamentsvollstreckung der Revolution.

Wie in Frankreich im achtzehnten, so leitete auch in Deutschland
im neunzehnten Jahrhundert die philosophische Revolution den politischen
Zusammenbruch ein. Aber wie verschieden sahn die beiden aus! Die
Franzosen in offnem Kampf mit der ganzen offiziellen Wissenschaft, mit
der Kirche, oft auch mit dem Staat; ihre Schriften jenseits der Grenze,
in Holland oder England gedruckt, und sie selbst oft genug drauf und
dran, in die Bastille zu wandern. Dagegen die Deutschen --- Professoren,
vom Staat eingesetzte Lehrer der Jugend, ihre Schriften anerkannte
Lehrbücher, und das abschließende System der ganzen Entwicklung, das
Hegelsche, sogar gewissermaßen zum Rang einer königlich preußischen
Staatsphilosophie erhoben! Und hinter diesen Professoren, hinter ihren
pedantisch-dunklen Worten, in ihren schwerfälligen, langweiligen
Perioden sollte sich die Revolution verstecken? Waren denn nicht grade
die Leute, die damals für die Vertreter der Revolution galten, die
Liberalen, die heftigsten Gegner dieser die Köpfe verwirrenden
Philosophie? Was aber weder die Regierungen noch die Liberalen sahen,
das sah bereits 1833 wenigstens \emph{ein} Mann, und der hieß allerdings
Heinrich Heine.



Nehmen wir ein Beispiel. Kein philosophischer Satz hat so sehr
den Dank beschränkter Regierungen und den Zorn ebenso beschränkter
Liberalen auf sich geladen wie der berühmte Satz Hegels: „Alles was
wirklich ist, ist vernünftig, und alles was vernünftig ist, ist
wirklich.``

Das war doch handgreiflich die Heiligsprechung alles
Bestehenden, die philosophische Einsegnung des Despotismus, des
Polizeistaats, der Kabinettsjustiz, der Zensur. Und so nahm es Friedrich
Wilhelm \versal{III}., so seine Untertanen. Bei Hegel aber ist keineswegs alles,
was besteht, ohne weiteres auch wirklich. Das Attribut der Wirklichkeit
kommt bei ihm nur demjenigen zu, was zugleich notwendig ist; „die
Wirklichkeit erweist sich in ihrer Entfaltung als die Notwendigkeit``;
eine beliebige Regierungsmaßregel --- Hegel führt selbst das Beispiel
„einer gewissen Steuereinrichtung`` an --- gilt ihm daher auch keineswegs
schon ohne weiteres als wirklich. Was aber notwendig ist, erweist sich
in letzter Instanz auch als vernünftig, und auf den damaligen
preußischen Staat angewandt, heißt also der Hegelsche Satz nur: Dieser
Staat ist vernünftig, der Vernunft entsprechend, soweit er notwendig
ist; und wenn er uns dennoch schlecht vorkommt, aber trotz seiner
Schlechtigkeit fortexistiert, so findet die Schlechtigkeit der Regierung
ihre Berechtigung und ihre Erklärung in der entsprechenden
Schlechtigkeit der Untertanen. Die damaligen Preußen hatten die
Regierung, die sie verdienten.

Nun ist aber die Wirklichkeit nach Hegel keineswegs ein
Attribut, das einer gegebnen gesellschaftlichen oder politischen
Sachlage unter allen Umständen und zu allen Zeiten zukommt. Im
Gegenteil. Die römische Republik war wirklich, aber das \textbar{}\,sie
verdrängende\,\textbar{} römische Kaiserreich auch. Die französische Monarchie war
\textbar{}\,1789\,\textbar{} so unwirklich geworden, d.h. so aller Notwendigkeit beraubt, so
unvernünftig, daß sie vernichtet werden mußte durch die große
Revolution, von der Hegel stets mit der höchsten Begeisterung spricht.
Hier war also die Monarchie das Unwirkliche, die Revolution das
Wirkliche. Und so wird im Lauf der Entwicklung alles früher Wirkliche
unwirklich, verliert seine Notwendigkeit, sein Existenzrecht, seine
Vernünftigkeit; an die Stelle des absterbenden Wirklichen tritt eine
neue, lebensfähige Wirklichkeit --- friedlich, wenn das Alte verständig
genug ist, ohne Sträuben mit Tode abzugehn, gewaltsam, wenn es sich
gegen diese Notwendigkeit sperrt. Und so dreht sich der Hegelsche Satz
durch die Hegelsche Dialektik selbst um in sein Gegenteil: Alles, was im
Bereich der Menschengeschichte wirklich ist, wird mit der Zeit
unvernünftig, ist also schon seiner Bestimmung nach unvernünftig, ist
von vornherein mit Unvernünftigkeit behaftet; und alles, was in den
Köpfen der Menschen vernünftig ist, ist bestimmt, wirklich zu werden,
mag es auch noch so sehr der bestehenden scheinbaren Wirklichkeit
widersprechen. Der Satz von der Vernünftigkeit alles Wirklichen löst
sich nach allen Regeln der Hegelschen Denkmethode auf in den andern:
Alles was besteht, ist wert, daß es zugrunde geht.

Dann aber grade lag die wahre Bedeutung und der revolutionäre
Charakter der Hegelschen Philosphie (auf die, als den Abschluß der
ganzen Bewegung seit Kant, wir uns hier beschränken müssen), daß sie der
Endgültigkeit aller Ergebnisse des menschlichen Denkens und Handelns ein
für allemal den Garaus machte. Die Wahrheit, die es in der Philosophie
zu erkennen galt, war bei Hegel nicht mehr eine Sammlung fertiger
dogmatischer Sätze, die, einmal gefunden, nur auswendig gelernt sein
wollen; die Wahrheit lag nun in dem Prozeß des Erkennens selbst, in der
langen geschichtlichen Entwicklung der Wissenschaft, die von niedern zu
immer höhern Stufen der Erkenntnis aufsteigt, ohne aber jemals durch
Ausfindung einer sogenannten absoluten Wahrheit zu dem Punkt zu
gelangen, wo sie nicht mehr weiter kann, wo ihr nichts mehr übrigbleibt,
als die Hände in den Schoß zu legen und die gewonnene absolute Wahrheit
anzustaunen. Und wie auf dem Gebiet der philosophischen, so auf dem
jeder andern Erkenntnis und auf dem des praktischen Handelns.
Ebensowenig wie die Erkenntnis kann die Geschichte einen vollendenden
Abschluß finden in einem vollkommnen Idealzustand der Menschheit; eine
vollkommne Gesellschaft, ein vollkommner „Staat`` sind Dinge, die nur in
der Phantasie bestehn können; im Gegenteil sind alle nacheinander
folgenden geschichtlichen Zustände nur vergängliche Stufen im endlosen
Entwicklungsgang der menschlichen Gesellschaft vom Niedern zum Höhern.
Jede Stufe ist notwendig, also berechtigt für die Zeit und die
Bedingungen, denen sie ihren Ursprung verdankt; aber sie wird hinfällig
und unberechtigt gegenüber neuen, höhern Bedingungen, die sich
allmählich in ihrem eignen Schoß entwickeln; sie muß einer höhern Stufe
Platz machen, die ihrerseits wieder an die Reihe des Verfalls und des
Untergangs kommt. Wie die Bourgeoisie durch die große Industrie, die
Konkurrenz und den Weltmarkt alle stabilen, altehrwürdigen Institutionen
praktisch auflöst, so löst diese dialektische Philosophie alle
Vorstellungen von endgültiger absoluter Wahrheit und ihr entsprechenden
absoluten Menschheitszuständen auf. Vor ihr besteht nichts Endgültiges,
Absolutes, Heiliges; sie weist von allem und an allem die
Vergänglichkeit auf, und nichts besteht vor ihr als der ununterbrochne
Prozeß des Werdens und Vergehens, des Aufsteigens ohne Ende vom Niedern
zum Höhern, dessen bloße Widerspiegelung im denkenden Hirn sie selbst
ist. Sie hat allerdings auch eine konservative Seite: Sie erkennt die
Berechtigung bestimmter Erkenntnis- und Gesellschaftsstufen für deren
Zeit und Umstände an; aber auch nur so weit. Der Konservatismus dieser
Anschauungsweise ist relativ, ihr revolutionärer Charakter ist absolut ---
das einzig Absolute, das sie gelten läßt.

Wir brauchen hier nicht auf die Frage einzugehn, ob diese
Anschauungsweise durchaus mit dem jetzigen Stand der Naturwissenschaft
stimmt, die der Existenz der Erde selbst ein mögliches, ihrer
Bewohnbarkeit aber ein ziemlich sichres Ende vorhersagt, die also auch
der Menschengeschichte nicht nur einen aufsteigenden, sondern auch einen
absteigenden Ast zuerkennt. Wir befinden uns jedenfalls noch ziemlich
weit von dem Wendepunkt, von wo an es mit der Geschichte der
Gesellschaft abwärtsgeht, und können der Hegelschen Philosophie nicht
zumuten, sich mit einem Gegenstand zu befassen, den zu ihrer Zeit die
Naturwissenschaft noch gar nicht auf die Tagesordnung gesetzt hatte.

Was aber in der Tat hier zu sagen, ist dies: Die obige
Entwicklung findet sich in dieser Schärfe nicht bei Hegel. Sie ist eine
notwendige Konsequenz seiner Methode, die er selbst aber in dieser
Ausdrücklichkeit nie gezogen hat. Und zwar aus dem einfachen Grund, weil
er genötigt war, ein System zu machen, und ein System der Philosophie
muß nach den hergebrachten Anforderungen mit irgendeiner Art von
absoluter Wahrheit abschließen. Sosehr also auch Hegel, namentlich in
der „Logik``, betont, daß diese ewige Wahrheit nichts andres ist als der
logische, resp. der geschichtliche Prozeß selbst, so sieht er sich doch
selbst gezwungen, diesem Prozeß ein Ende zu geben, weil er eben mit
seinem System irgendwo zu Ende kommen muß. In der „Logik`` kann er dies
Ende wieder zum Anfang machen, indem hier der Schlußpunkt, die absolute
Idee --- die nur insofern absolut ist, als er absolut nichts von ihr zu
sagen weiß --- sich in die Natur „entäußert``, d.h. verwandelt, und später
im Geist, d.h. im Denken und in der Geschichte, wieder zu sich selbst
kommt. Aber am Schluß der ganzen Philosophie ist ein ähnlicher
Rückschlag in den Anfang nur auf \emph{einem} Weg möglich. Nämlich indem
man das Ende der Geschichte dann setzt, daß die Menschheit zur
Erkenntnis eben dieser absoluten Idee kommt, \textbar{}\,und erklärt, daß diese
Erkenntnis der absoluten Idee in der Hegelschen Philosophie erreicht ist.\,\textbar{} Damit wird aber der ganze dogmatische Inhalt des Hegelschen Systems
für die absolute Wahrheit erklärt, im Widerspruch mit seiner
dialektischen, alles Dogmatische auflösenden Methode; damit wird die
revolutionäre Seite erstickt unter der überwuchernden konservativen. Und
was von der philosophischen Erkenntnis, gilt auch von der
geschichtlichen Praxis. Die Menschheit, die es, in der Person Hegels,
bis zur Herausarbeitung der absoluten Idee gebracht hat, muß auch
praktisch so weit gekommen sein, daß sie diese absolute Idee in der
Wirklichkeit durchführen kann. Die praktischen politischen Forderungen
der absoluten Idee an die Zeitgenossen dürfen also nicht zu hoch
gespannt sein. Und so finden wir am Schluß der „Rechtsphilosophie``, daß
die absolute Idee sich verwirklichen soll in derjenigen ständischen
Monarchie, die Friedrich Wilhelm \versal{III}. seinen Untertanen so hartnäckig
vergebens versprach, also in einer den deutschen kleinbürgerlichen
Verhältnissen von damals angemessenen, beschränkten und gemäßigten,
indirekten Herrschaft der besitzenden Klassen; wobei uns noch die
Notwendigkeit des Adels auf spekulativem Wege demonstriert wird.

Die innern Notwendigkeiten des Systems reichen also allein hin,
die Erzeugung einer sehr zahmen politischen Schlußfolgerung, vermittelst
einer durch und durch revolutionären Denkmethode, zu erklären. Die
spezifische Form dieser Schlußfolgerung rührt allerdings davon her, daß
Hegel ein Deutscher war und ihm wie seinem Zeitgenossen Goethe ein Stück
Philisterzopfs hinten hing. Goethe wie Hegel waren jeder auf seinem
Gebiet ein olympischer Zeus, aber den deutschen Philister wurden beide
nie ganz los.

Alles dies hinderte jedoch das Hegelsche System nicht, ein
unvergleichlich größeres Gebiet zu umfassen als irgendein früheres
System und auf diesem Gebiet einen Reichtum des Gedankens zu entwickeln,
der noch heute in Erstaunen setzt. Phänomenologie des Geistes (die man
eine Parallele der Embryologie und der Paläontologie des Geistes nennen
könnte, eine Entwicklung des individuellen Bewußtseins durch seine
verschiedenen Stufen, gefaßt als abgekürzte Reproduktion der Stufen, die
das Bewußtsein der Menschen geschichtlich durchgemacht), Logik,
Naturphilosophie, Philosophie des Geistes, und diese letztere wieder in
ihren einzelnen geschichtlichen Unterformen ausgearbeitet: Philosophie
der Geschichte, des Rechts, der Religion, Geschichte der Philosophie,
Ästhetik usw. --- auf allen diesen verschiednen geschichtlichen Gebieten
arbeitet Hegel daran, den durchgehenden Faden der Entwicklung
aufzufinden und nachzuweisen; und da er nicht nur ein schöpferisches
Genie war, sondern auch ein Mann von enzyklopädischer Gelehrsamkeit, so
tritt er überall epochemachend auf. Es versteht sich von selbst, daß
kraft der Notwendigkeiten des „Systems`` er hier oft genug zu jenen
gewaltsamen Konstruktionen seine Zuflucht nehmen muß, von denen seine
zwerghaften Anfeinder bis heute ein so entsetzliches Geschrei machen.
Aber diese Konstruktionen sind nur der Rahmen und das Baugerüst seines
Werks; hält man sich hierbei nicht unnötig auf, dringt man tiefer ein in
den gewaltigen Bau, so findet man ungezählte Schätze, die auch heute
noch ihren vollen Wert behaupten. Bei allen Philosophen ist grade das
„System`` das Vergängliche, und zwar grade deshalb, weil es aus einem
unvergänglichen Bedürfnis des Menschengeistes hervorgeht: dem Bedürfnis
der Überwindung aller Widersprüche. Sind aber alle Widersprüche ein für
allemal beseitigt, so sind wir bei der sogenannten absoluten Wahrheit
angelangt, die Weltgeschichte ist zu Ende, und doch soll sie fortgehn,
obwohl ihr nichts mehr zu tun übrigbleibt --- also ein neuer, unlösbarer
Widerspruch. Sobald wir einmal eingesehn haben --- und zu dieser Einsicht
hat uns schließlich niemand mehr verhelfen als Hegel selbst ---, daß die
so gestellte Aufgabe der Philosophie weiter nichts heißt als die
Aufgabe, daß ein einzelner Philosoph das leisten soll, was nur die
gesamte Menschheit in ihrer fortschreitenden Entwicklung leisten kann ---
sobald wir das einsehn, ist es auch am Ende mit der ganzen Philosophie
im bisherigen Sinn des Worts. Man läßt die auf diesem Weg und für jeden
einzelnen unerreichbare „absolute Wahrheit`` laufen und jagt dafür den
erreichbaren relativen Wahrheiten nach auf dem Weg der positiven
Wissenschaften und der Zusammenfassung ihrer Resultate vermittelst des
dialektischen Denkens. Mit Hegel schließt die Philosophie überhaupt ab;
einerseits weil er ihre ganze Entwicklung in seinem System in der
großartigsten Weise zusammenfaßt, andrerseits weil er uns, wenn auch
unbewußt, den Weg zeigt aus diesem Labyrinth der Systeme zur wirklichen
positiven Erkenntnis der Welt.

Man begreift, welch ungeheure Wirkung dies Hegelsche System in
der philosophisch gefärbten Atmosphäre Deutschlands hervorbringen mußte.
Es war ein Triumphzug, der Jahrzehnte dauerte und mit dem Tod Hegels
keineswegs zur Ruhe kam. Im Gegenteil, grade von 1830 bis 1840 herrschte
die „Hegelei`` am ausschließlichsten und hatte selbst ihre Gegner mehr
oder weniger angesteckt; grade in dieser Zeit drangen Hegelsche
Anschauungen am reichlichsten, bewußt oder unbewußt, in die
verschiedensten Wissenschaften ein und durchsäuerten auch die populäre
Literatur und die Tagespresse, aus denen das gewöhnliche „gebildete
Bewußtsein`` seinen Gedankenstoff bezieht. Aber dieser Sieg auf der
ganzen Linie war nur das Vorspiel eines innern Kampfs.

Die Gesamtlehre Hegels ließ, wie wir gesehn, reichlichen Raum
für die Unterbringung der verschiedensten praktischen
Parteianschauungen; und praktisch waren im damaligen theoretischen
Deutschland vor allem zwei Dinge: die Religion und die Politik. Wer das
Hauptgewicht auf das \emph{System} Hegels legte, konnte auf beiden
Gebieten ziemlich konservativ sein; wer in der
dialektischen \emph{Methode} die Hauptsache sah, konnte religiös wie
politisch zur äußersten Opposition gehören. Hegel selbst schien, trotz
der ziemlich häufigen revolutionären Zornesausbrüche in seinen Werken,
im ganzen mehr zur konservativen Seite zu neigen; hatte ihm doch sein
System weit mehr „saure Arbeit des Gedankens`` gekostet als seine
Methode. Gegen Ende der dreißiger Jahre trat die Spaltung in der Schule
mehr und mehr hervor. Der linke Flügel, die sogenannten Junghegelianer,
gaben im Kampf mit pietistischen Orthodoxen und feudalen Reaktionären
ein Stück nach dem andern auf von jener philosophisch-vornehmen
Zurückhaltung gegenüber den brennenden Tagesfragen, die ihrer Lehre
bisher staatliche Duldung und sogar Protektion gesichert hatte; und als
gar 1840 die orthodoxe Frömmelei und die feudal-absolutistische Reaktion
mit Friedrich Wilhelm \versal{IV}. den Thron bestiegen, wurde offne Parteinahme
unvermeidlich. Der Kampf wurde noch mit philosophischen Waffen geführt,
aber nicht mehr um abstrakt-philosophische Ziele; es handelte sich
direkt um Vernichtung der überlieferten Religion und des bestehenden
Staats. Und wenn in den „Deutschen Jahrbüchern`` die praktischen
Endzwecke noch vorwiegend in philosophischer Verkleidung auftraten, so
enthüllte sich die junghegelsche Schule in der „Rheinischen Zeitung`` von
1842 direkt als die Philosophie der aufstrebenden radikalen Bourgeoisie
und brauchte das philosophische Deckmäntelchen nur noch zur Täuschung
der Zensur.

Die Politik war aber damals ein sehr dorniges Gebiet, und so
wandte sich der Hauptkampf gegen die Religion; dies war ja, namentlich
seit 1840, indirekt auch ein politischer Kampf. Den ersten Anstoß hatte
Strauß' „Leben Jesu`` 1835 gegeben. Der hierin entwickelten Theorie der
evangelischen Mythenbildung trat später Bruno Bauer mit dem Nachweis
gegenüber, daß eine ganze Reihe evangelischer Erzählungen von den
Verfassern selbst fabriziert worden. Der Streit zwischen beiden wurde
geführt in der philosophischen Verkleidung eines Kampfes des
„Selbstbewußtseins`` gegen die „Substanz``; die Frage, ob die
evangelischen Wundergeschichten durch bewußtlos-traditionelle
Mythenbildung im Schoß der Gemeinde entstanden oder ob sie von den
Evangelisten selbst fabriziert seien, wurde aufgebauscht zu der Frage,
ob in der Weltgeschichte die „Substanz`` oder das „Selbstbewußtsein`` die
entscheidend wirkende Macht sei; und schließlich kam Stirner, der
Prophet des heutigen Anarchismus --- Bakunin hat sehr viel aus ihm
genommen --- und übergipfelte das souveräne „Selbstbewußtsein`` durch
seinen souveränen „Einzigen``.

Wir gehen auf diese Seite des Zersetzungsprozesses der Hegelschen
Schule nicht weiter ein. Wichtiger für uns ist dies: Die Masse der
entschiedensten Junghegelianer wurde durch die praktischen
Notwendigkeiten ihres Kampfs gegen die positive Religion auf den
englisch-französischen Materialismus zurückgedrängt. Und hier kamen sie
in Konflikt mit ihrem Schulsystem. Während der Materialismus die Natur
als das einzig Wirkliche auffaßt, stellt diese im Hegelschen System nur
die „Entäußerung`` der absoluten Idee vor, gleichsam eine Degradation der
Idee; unter allen Umständen ist hier das Denken und sein
Gedankenprodukt, die Idee, das Ursprüngliche, die Natur das Abgeleitete,
das nur durch die Herablassung der Idee überhaupt existiert. Und in
diesem Widerspruch trieb man sich herum, so gut und so schlecht es gehen
wollte.

Da kam Feuerbachs „Wesen des Christenthums``. Mit einem Schlag
zerstäubte es den Widerspruch, indem es den Materialismus ohne
Umschweife wieder auf den Thron erhob. Die Natur existiert unabhängig
von aller Philosophie; sie ist die Grundlage, auf der wir Menschen,
selbst Naturprodukte, erwachsen sind; außer der Natur und den Menschen
existiert nichts, und die höhern Wesen, die unsere religiöse Phantasie
erschuf, sind nur die phantastische Rückspiegelung unsers eignen Wesens.
Der Bann war gebrochen; das „System`` war gesprengt und beiseite
geworfen, der Widerspruch war, als nur in der Einbildung vorhanden,
aufgelöst. --- Man muß die befreiende Wirkung dieses Buchs selbst erlebt
haben, um sich eine Vorstellung davon zu machen. Die Begeisterung war
allgemein: Wir waren alle momentan Feuerbachianer. Wie enthusiastisch
Marx die neue Auffassung begrüßte \textbar{}\,und wie sehr er --- trotz aller
kritischen Vorbehalte --- von ihr beeinflußt wurde,\,\textbar{} kann man in
der „Heiligen Familie`` lesen.

Selbst die Fehler des Buchs trugen zu seiner augenblicklichen
Wirkung bei. Der belletristische, stellenweise sogar schwülstige Stil
sicherte ein größeres Publikum und war immerhin eine Erquickung nach den
langen Jahren abstrakter und abstruser Hegelei. Dasselbe gilt von der
überschwenglichen Vergötterung der Liebe, die gegenüber der unerträglich
gewordnen Souveränität des „reinen Denkens`` eine Entschuldigung, wenn
auch keine Berechtigung fand. Was wir aber nicht vergessen dürfen: Grade
an diese beiden Schwächen Feuerbachs knüpfte der seit 1844 sich im
„gebildeten`` Deutschland wie eine Seuche verbreitende „wahre
Sozialismus`` an, der an die Stelle der wissenschaftlichen Erkenntnis die
belletristische Phrase, an die Stelle der Emanzipation des Proletariats
durch die ökonomische Umgestaltung der Produktion die Befreiung der
Menschheit vermittelst der „Liebe`` setzte, kurz, sich in die
widerwärtige Belletristik und Liebesschwüligkeit verlief, deren Typus
Herr Karl Grün war.

Was fernerhin nicht zu vergessen: Die Hegelsche Schule war
aufgelöst, aber die Hegelsche Philosophie war nicht kritisch überwunden.
Strauß und Bauer nahmen jeder eine ihrer Seiten heraus und kehrten sie
polemisch gegen die andre. Feuerbach durchbrach das System und warf es
einfach beiseite. Aber man wird nicht mit einer Philosophie fertig
dadurch, daß man sie einfach für falsch erklärt. Und ein so gewaltiges
Werk wie die Hegelsche Philosophie, die einen so ungeheuren Einnuß auf
die geistige Entwicklung der Nation gehabt, ließ sich nicht dadurch
beseitigen, daß man sie kurzerhand ignorierte. Sie mußte in ihrem
eigenen Sinn „aufgehoben`` werden, d.h. in dem Sinn, daß ihre Form
kritisch vernichtet, der durch sie gewonnene neue Inhalt\,aber\,gerettet\,wurde.\,Wie\,dies\,geschah,\,davon\,weiter\,unten.

Einstweilen schob die Revolution von 1848 jedoch die gesamte
Philosophie ebenso ungeniert beiseite wie Feuerbach seinen Hegel. Und
damit wurde auch Feuerbach selbst in den Hintergrund gedrängt.

\quebra

\mbox{}
\vspace{2cm}

\section{II}



\noindent{}Die große Grundfrage aller, speziell neueren Philosophie ist die
nach dem Verhältnis von Denken und Sein. Seit der sehr frühen Zeit, wo
die Menschen, noch in gänzlicher Unwissenheit über ihren eigenen
Körperbau und angeregt durch Traumerscheinungen, auf die Vorstellung
kamen, ihr Denken und Empfinden sei nicht eine Tätigkeit ihres Körpers,
sondern einer besonderen, in diesem Körper wohnenden und ihn beim Tode
verlassenden Seele --- seit dieser Zeit mußten sie über das Verhältnis
dieser Seele zur äußern Welt sich Gedanken machen. Wenn sie im Tod sich
vom Körper trennte, fortlebte, so lag kein Anlaß vor, ihr noch einen
besondren Tod anzudichten; so entstand die Vorstellung von ihrer
Unsterblichkeit, die auf jener Entwicklungsstufe keineswegs als ein
Trost erscheint, sondern als ein Schicksal, wogegen man nicht ankann,
und oft genug, wie bei den Griechen, als ein positives Unglück. Nicht
das religiöse Trostbedürfnis, sondern die aus gleich allgemeiner
Beschränktheit hervorwachsende Verlegenheit, was mit der einmal
angenommenen Seele, nach dem Tod des Körpers, anzufangen, führte
allgemein zu der langweiligen Einbildung von der persönlichen
Unsterblichkeit. Auf ganz ähnlichem Weg entstanden, durch
Personifikation der Naturmächte, die ersten Götter, die in der weitern
Ausbildung der Religionen eine mehr und mehr außerweltliche Gestalt
annahmen, bis endlich durch einen im Verlauf der geistigen Entwicklung
sich naturgemäß einstellenden Abstraktions-, ich möchte fast sagen %assim mesmo esse hífen?
Destillationsprozeß aus den vielen, mehr oder minder beschränkten\est\ und
sich gegenseitig beschränkenden Göttern die Vorstellung von dem einen
ausschließlichen Gott der monotheistischen Religionen in den Köpfen der
Menschen entstand.

Die Frage nach dem Verhältnis des Denkens zum Sein, des Geistes
zur Natur, die höchste Frage der gesamten Philosophie hat also, nicht
minder als alle Religion, ihre Wurzel in den bornierten und unwissenden
Vorstellungen des Wildheitszustands. Aber in ihrer vollen Schärfe konnte
sie erst gestellt werden, ihre ganze Bedeutung konnte sie erst erlangen,
als die europäische Menschheit aus dem langen Winterschlaf des
christlichen Mittelalters erwachte. Die Frage nach der Stellung des
Denkens zum Sein, die übrigens auch in der Scholastik des Mittelalters
ihre große Rolle gespielt, die Frage: Was ist das Ursprüngliche, der
Geist oder die Natur? --- diese Frage spitzte sich, der Kirche gegenüber,
dahin zu: Hat Gott die Welt erschaffen, oder ist die Welt von Ewigkeit
da?

Je nachdem diese Frage so oder so beantwortet wurde, spalteten
sich die Philosophen in zwei große Lager. Diejenigen, die die
Ursprünglichkeit des Geistes gegenüber der Natur behaupteten, also in
letzter Instanz eine Weltschöpfung irgendeiner Art annahmen --- und diese
Schöpfung ist oft bei den Philosophen, z.B. bei Hegel, noch weit
verzwickter und unmöglicher als im Christentum ---, bildeten das Lager des
Idealismus. Die andern, die die Natur als das Ursprüngliche ansahen,
gehören zu den verschiednen Schulen des Materialismus.

Etwas andres als dies bedeuten die beiden Ausdrücke: Idealismus
und Materialismus ursprünglich nicht, und in einem andern Sinne werden
sie hier auch nicht gebraucht. Welche Verwirrung entsteht, wenn man
etwas andres in sie hineinträgt, werden wir unten sehn.

Die Frage nach dem Verhältnis von Denken und Sein hat aber noch
eine andre Seite: Wie verhalten sich unsere Gedanken über die uns
umgebende Welt zu dieser Welt selbst? Ist unser Denken imstande, die
wirkliche Welt zu erkennen, vermögen wir in unsern Vorstellungen und
Begriffen von der wirklichen Welt ein richtiges Spiegelbild der
Wirklichkeit zu erzeugen? Diese Frage heißt in der philosophischen
Sprache die Frage nach der Identität von Denken und Sein und wird von
der weitaus größten Zahl der Philosophen bejaht. Bei Hegel z.B. versteht
sich ihre Bejahung von selbst; denn das, was wir in der wirklichen Welt
erkennen, ist eben ihr gedankenmäßiger Inhalt, dasjenige, was die Welt
zu einer stufenweisen Verwirklichung der absoluten Idee macht, welche
absolute Idee von Ewigkeit her, unabhängig von der Welt und vor der
Welt, irgendwo existiert hat; daß aber das Denken einen Inhalt erkennen
kann, der schon von vornherein Gedankeninhalt ist, leuchtet ohne weitres
ein. Ebensosehr leuchtet ein, daß hier das zu Beweisende im stillen
schon in der Voraussetzung enthalten ist. Das hindert aber Hegel
keineswegs, aus seinem Beweis der Identität von Denken und Sein den
weitern Schluß zu ziehen, daß seine Philosophie, weil für sein Denken
richtig, nun auch die einzig richtige ist und daß die Identität von
Denken und Sein sich darin zu bewähren hat, daß die Menschheit sofort
seine Philosophie aus der Theorie in die Praxis übersetzt und die ganze
Welt nach Hegelschen Grundsätzen umgestaltet. Es ist dies eine Illusion,
die er so ziemlich mit allen Philosophen teilt.

Daneben gibt es aber noch eine Reihe andrer Philosophen, die die
Möglichkeit einer Erkenntnis der Welt oder doch einer erschöpfenden
Erkenntnis bestreiten. Zu ihnen gehören unter den neueren Hume und Kant,
und sie haben eine sehr bedeutende Rolle in der philosophischen
Entwicklung gespielt. Das Entscheidende zur Widerlegung dieser Ansicht
ist bereits von Hegel gesagt, soweit dies vom idealistischen Standpunkt
möglich war; was Feuerbach Materialistisches hinzugefügt, ist mehr
geistreich als tief. Die schlagendste Widerlegung dieser wie aller
andern philosophischen Schrullen ist die Praxis, nämlich\est\ das Experiment
und die Industrie. Wenn wir die Richtigkeit unsrer Auffassung eines
Naturvorgangs beweisen können, indem wir ihn selbst machen, ihn aus
seinen Bedingungen erzeugen, ihn obendrein unsern Zwecken dienstbar
werden lassen, so ist es mit dem Kantschen unfaßbaren „Ding an sich`` zu
Ende. Die im pflanzlichen und tierischen Körper erzeugten chemischen
Stoffe blieben solche „Dinge an sich``, bis die organische Chemie sie
einen nach dem andern darzustellen anfing; damit wurde das „Ding an
sich`` ein Ding für uns, wie z.B. der Farbstoff des Krapps, das Alizarin,
das wir nicht mehr auf dem Felde in den Krappwurzeln wachsen lassen,
sondern aus Kohlenteer weit wohlfeiler und einfacher herstellen. Das
kopernikanische Sonnensystem war dreihundert Jahre lang eine Hypothese,
auf die hundert, tausend, zehntausend gegen eins zu wetten war, aber
doch immer eine Hypothese; als aber Leverrier aus den durch dies System
gegebenen Daten nicht nur die Notwendigkeit der Existenz eines
unbekannten Planeten, sondern auch den Ort berechnete, wo dieser Planet
am Himmel stehn müsse, und als Galle dann diesen Planeten wirklich fand,
da war das kopernikanische System bewiesen. Wenn dennoch die Neubelebung
der Kantschen Auffassung in Deutschland durch die Neukantianer und der
Humeschen in England (wo sie nie ausgestorben) durch die Agnostiker
versucht wird, so ist das, der längst erfolgten theoretischen und
praktischen Widerlegung gegenüber, wissenschaftlich ein Rückschritt und
praktisch nur eine verschämte Weise, den Materialismus hinterrücks zu
akzeptieren und vor der Welt zu verleugnen.

Die Philosophen wurden aber in dieser langen Periode von
Descartes bis Hegel und von Hobbes bis Feuerbach keineswegs, wie sie
glaubten, allein durch die Kraft des reinen Gedankens vorangetrieben. Im
Gegenteil. Was sie in Wahrheit vorantrieb, das war namentlich der
gewaltige und immer schneller voranstürmende Fortschritt der
Naturwissenschaft \textbar{}\,und der Industrie.\,\textbar{} Bei den Materialisten zeigte
sich dies schon auf der Oberfläche, aber auch die idealistischen Systeme
erfüllten sich mehr und mehr mit materialistischem Inhalt und suchten
den Gegensatz von Geist und Materie pantheistisch zu versöhnen; so daß
schließlich das Hegelsche System nur einen nach Methode und Inhalt
idealistisch auf den Kopf gestellten Materialismus repräsentiert.

Es ist hiermit begreiflich, daß Starcke in seiner Charakteristik
Feuerbachs zunächst dessen Stellung zu dieser Grundfrage über das
Verhältnis von Denken und Sein untersucht. Nach einer kurzen Einleitung,
worin die Auffassung der frühern Philosophen, namentlich seit Kant, in
unnötig philosophisch-schwerfälliger Sprache geschildert wird und wobei
Hegel durch allzu formalistisches Festhalten an einzelnen Stellen seiner
Werke sehr zu kurz kommt, folgt eine ausführliche Darstellung des
Entwicklungsgangs der Feuerbachschen „Metaphysik`` selbst, wie er sich
aus der Reihenfolge der betreffenden Schriften dieses Philosophen
ergibt. Diese Darstellung ist fleißig und übersichtlich gearbeitet, nur
wie das ganze Buch mit einem keineswegs überall unvermeidlichen Ballast
philosophischer Ausdrucksweise beschwert, der um so störender wirkt, je
weniger sich der Verfasser an die Ausdrucksweise einer und derselben
Schule, oder auch Feuerbachs selbst hält, und je mehr er Ausdrücke der
verschiedensten, namentlich der jetzt grassierenden, sich philosophisch
nennenden Richtungen hinein mengt.

Der Entwicklungsgang Feuerbachs ist der eines --- freilich nie
ganz orthodoxen --- Hegelianers zum Materialismus hin, eine Entwicklung,
die auf einer bestimmten Stufe einen totalen Bruch mit dem
idealistischen System seines Vorgängers bedingt. Mit unwiderstehlicher
Gewalt drängt sich ihm schließlich die Einsicht auf, daß die Hegelsche
vorweltliche Existenz der „absoluten Idee``, die „Präexistenz der
logischen Kategorien``, ehe denn die Welt war, weiter nichts ist als ein
phantastischer Überrest des Glaubens an einen außerweltlichen Schöpfer;
daß die stoffliche, sinnlich wahrnehmbare Welt, zu der wir selbst
gehören, das einzig Wirkliche, und daß unser Bewußtsein und Denken, so
übersinnlich es scheint, das Erzeugnis eines stofflichen, körperlichen
Organs, des Gehirns ist. Die Materie ist nicht ein Erzeugnis des
Geistes, sondern der Geist ist selbst nur das höchste Produkt der
Materie. Dies ist natürlich reiner Materialismus. Hier angekommen,
stutzt Feuerbach. Er kann das gewohnheitsmäßige, philosophische
Vorurteil nicht überwinden, das Vorurteil nicht gegen die Sache, sondern
gegen den Namen des Materialismus. Er sagt: „Der Materialismus ist für
mich die Grundlage des Gebäudes des menschlichen Wesens und Wissens;
aber er ist für mich nicht, was er für den Physiologen, den
Naturforscher im engem Sinn, z.B. Moleschott ist, und zwar notwendig von
ihrem Standpunkt und Beruf aus ist, das Gebäude selbst. Rückwärts stimme
ich den Materialisten vollkommen bei, aber nicht vorwärts.``

Feuerbach wirft hier den Materialismus, der eine auf einer
bestimmten Auffassung des Verhältnisses von Materie und Geist beruhende
allgemeine Weltanschauung ist, zusammen mit der besondern Form, worin
diese Weltanschauung auf einer bestimmten geschichtlichen Stufe, \textbar{} nämlich
im 18. Jahrhundert,~\textbar{} zum Ausdruck kam. Noch mehr, er wirft ihn zusammen
mit der verflachten, vulgarisierten Gestalt, worin der Materialismus des
18. Jahrhunderts heute in den Köpfen von Naturforschern und Ärzten
fortexistiert und in den fünfziger Jahren von Büchner, Vogt und
Moleschott gereisepredigt wurde. Aber wie der Idealismus eine Reihe von
Entwicklungsstufe durchlief, so auch der Materialismus. Mit jeder
epochemachenden Entdeckung schon auf naturwissenschaftlichem Gebiet
mußte er seine Form ändern, und seitdem auch die Geschichte der
materialistischen Behandlung unterworfen, eröffnet sich auch hier eine
neue Bahn der Entwicklung.

Der Materialismus des vorigen Jahrhunderts war vorwiegend
mechanisch, weil von allen Naturwissenschaften damals nur die Mechanik,
und zwar auch nur die der --- himmlischen und irdischen --- festen Körper,
kurz, die Mechanik der Schwere, zu einem gewissen Abschluß gekommen war.
Die Chemie existierte nur erst in ihrer kindlichen, phlogistischen
Gestalt. Die Biologie lag noch in den Windeln; der pflanzliche und
tierische Organismus war nur im groben untersucht und wurde aus rein
mechanischen Ursachen erklärt; wie dem Descartes das Tier, war den
Materialisten des 18. Jahrhunderts der Mensch eine Maschine. Diese
ausschließliche Anwendung des Maßstabs der Mechanik auf Vorgänge, die
chemischer und organischer Natur sind und bei denen die mechanischen
Gesetze zwar auch gelten, aber von andern, höhern Gesetzen in den
Hintergrund gedrängt werden, bildet die eine spezifische, aber ihrer
Zeit unvermeidliche Beschränktheit des klassischen französischen
Materialismus.

Die zweite spezifische Beschränktheit dieses Materialismus
bestand in seiner Unfähigkeit, die Welt als einen Prozeß, als einen in
einer geschichtlichen Fortbildung begriffenen Stoff aufzufassen. Dies
entsprach dem damaligen Stand der Naturwissenschaft und der damit
zusammenhängenden metaphysischen, d.h. antidialektischen Weise des
Philosophierens. Die Natur, das wußte man, war in ewiger Bewegung
begriffen. Aber diese Bewegung drehte sich nach damaliger Vorstellung
ebenso ewig im Kreise und kam daher nie vom Fleck; sie erzeugte immer
wieder dieselben Ergebnisse. Diese Vorstellung war damals unvermeidlich.
Die Kantsche Theorie von der Entstehung des Sonnensystems war erst
soeben aufgestellt und passierte nur noch als bloßes Kuriosum. Die
Geschichte der Entwicklung der Erde, die Geologie, war noch total
unbekannt, und die Vorstellung, daß die heutigen belebten Naturwesen das
Ergebnis einer langen Entwicklungsreihe vom Einfachen zum Komplizierten
sind, konnte damals wissenschaftlich überhaupt nicht aufgestellt werden.
Die unhistorische Auffassung der Natur war also unvermeidlich. \textbar{}\,Man
kann den Philosophen des 18. Jahrhunderts daraus um so weniger einen
Vorwurf machen, als sie sich auch bei Hegel findet. Bei diesem ist die
Natur, als bloße „Entäußerung`` der Idee, keiner Entwicklung in der Zeit
fähig, sondern nur einer Ausbreitung ihrer Mannigfaltigkeit im Raum, so
daß sie alle in ihr einbegriffnen Entwicklungsstufen gleichzeitig und
nebeneinander ausstellt und zu ewiger Wiederholung stets derselben
Prozesse verdammt ist. Und diesen Widersinn einer Entwicklung im Raum,
aber außer der Zeit --- der Grundbedingung aller Entwicklung --- bürdet
Hegel der Natur auf grade zu derselben Zeit, wo die Geologie, die
Embryologie, die pflanzliche und tierische Physiologie und die
organische Chemie ausgebildet wurden und wo überall auf Grundlage dieser
neuen Wissenschaften geniale Vorahnungen der späteren
Entwicklungstheorie auftauchten (z.B. Goethe und Lamarck). Aber das
System erforderte es so, und so mußte die Methode, dem System zulieb,
sich selbst untreu werden.\,\textbar{} Dieselbe unhistorische Auffassung galt auch
auf dem Gebiet der Geschichte. Hier hielt der Kampf gegen die Reste des
Mittelalters den Blick befangen. Das Mittelalter galt als einfache
Unterbrechung der Geschichte durch tausendjährige allgemeine Barbarei;
die großen Fortschritte des Mittelalters --- die Erweiterung des
europäischen Kulturgebiets, die lebensfähigen großen Nationen, die sich
dort nebeneinander gebildet, endlich die enormen technischen
Fortschritte des 14. und 15. Jahrhunderts ---, alles das sah man nicht.
Damit war aber eine rationelle Einsicht in den großen geschichtlichen
Zusammenhang unmöglich gemacht, und die Geschichte diente höchstens als
eine Sammlung von Beispielen und Illustrationen zum Gebrauch der
Philosophen.

Die vulgarisierenden Hausierer, die in den fünfziger Jahren in
Deutschland in Materialismus machten, kamen in keiner Weise über diese
Schranke ihrer Lehrer hinaus. Alle seitdem gemachten Fortschritte der
Naturwissenschaft dienten ihnen nur als neue Beweisgründe gegen die
Existenz des Weltschöpfers; und in der Tat lag es ganz außerhalb ihres
Geschäfts, die Theorie weiterzuentwickeln. War der Idealismus am Ende
seines Lateins und durch die Revolution von 1848 auf den Tod getroffen,
so erlebte er die Genugtuung, daß der Materialismus momentan noch tiefer
heruntergekommen war. Feuerbach hatte entschieden recht, wenn er die
Verantwortung für diesen Materialismus ablehnte; nur durfte er die Lehre
der Reiseprediger nicht verwechseln mit dem Materialismus überhaupt.

Indes ist hier zweierlei zu bemerken. Erstens war auch zu
Feuerbachs Lebzeiten die Naturwissenschaft noch in jenem heftigen
Gärungsprozeß begriffen, der erst in den letzten fünfzehn Jahren einen
klärenden, relativen Abschluß erhalten hat; es wurde neuer
Erkenntnisstoff in bisher unerhörtem Maß geliefert, aber die Herstellung
des Zusammenhangs und damit der Ordnung in diesem Chaos sich
überstürzender Entdeckungen ist erst ganz neuerdings möglich geworden.
Zwar hat Feuerbach die drei entscheidenden Entdeckungen --- die der Zelle,
der Verwandlung der Energie und der nach Darwin benannten
Entwicklungstheorie --- noch alle erlebt. Aber wie sollte der einsame
Philosoph auf dem Lande die Wissenschaft hinreichend verfolgen können,
um Entdeckungen vollauf zu würdigen, die die Naturforscher selbst damals
teils noch bestritten, teils nicht hinreichend verstanden? Die Schuld
fällt hier einzig auf die erbärmlichen deutschen Zustände, kraft deren
die Lehrstühle der Philosophie von spintisierenden eklektischen
Flohknackern in Beschlag genommen wurden, während Feuerbach, der sie
alle turmhoch überragte, in einem kleinen Dorf verbauern und versauern
mußte. Es ist also nicht Feuerbachs Schuld, wenn die jetzt möglich
gewordne, alle Einseitigkeiten des französischen Materialismus
entfernende, historische Naturauffassung ihm unzugänglich blieb.

Zweitens aber hat Feuerbach darin ganz recht, daß der bloß
naturwissenschaftliche Materialismus zwar die „Grundlage des Gebäudes
des menschlichen Wissens ist, aber nicht das Gebäude selbst``. Denn wir
leben nicht nur in der Natur,\est\ sondern auch in der menschlichen
Gesellschaft, und auch diese hat ihre Entwicklungsgeschichte und ihre
Wissenschaft nicht minder als die Natur. Es handelte sich also darum,
die Wissenschaft von der Gesellschaft, d.h. den Inbegriff der
sogenannten historischen und philosophischen Wissenschaften, mit der
materialistischen Grundlage %\textbar{}281\textbar{}
in Einklang zu bringen und auf ihr zu rekonstruieren. Dies aber war
Feuerbach nicht vergönnt. Hier blieb er, trotz der „Grundlage``, in den
überkommnen idealistischen Banden befangen, und dies erkennt er an mit
den Worten: „Rückwärts stimme ich den Materialisten bei, aber nicht
vorwärts.`` Wer aber hier, auf dem gesellschaftlichen Gebiet, nicht
„vorwärts`` kam, nicht über seinen Standpunkt von 1840 oder 1844 hinaus,
das war Feuerbach selbst, und zwar wiederum hauptsächlich infolge seiner
Verödung, die ihn zwang, Gedanken aus seinem einsamen Kopf zu
produzieren --- ihn, der vor allen andern Philosophen auf geselligen
Verkehr veranlagt war --- statt im freundlichen und feindlichen
Zusammentreffen mit andern Menschen seines Kalibers. Wie sehr er auf
diesem Gebiet Idealist bleibt, werden wir später im einzelnen sehn.

Hier ist nur noch zu bemerken, daß Starcke den Idealismus
Feuerbachs am unrechten Ort sucht. „Feuerbach ist Idealist, er glaubt an
den Fortschritt der Menschheit.`` (S.\,19.) --- „Die Grundlage, der Unterbau
des Ganzen, bleibt nichtsdestoweniger der Idealismus. Der Realismus ist
für uns nichts weiter als ein Schutz gegen Irrwege, während wir unsern
idealen Strömungen folgen. Sind nicht Mitleid, Liebe und Begeisterung
für Wahrheit und Recht ideale Mächte?\,`` (\versal{S.\,VIII}.)

Erstens heißt hier Idealismus nichts andres als Verfolgung
idealer Ziele. Diese aber haben notwendig zu tun höchstens mit dem
Kantschen Idealismus und seinem „kategorischen Imperativ``; aber selbst
Kant nannte seine Philosophie „transzendentalen Idealismus``, keineswegs,
weil es sich darin auch um sittliche Ideale handelt, sondern aus ganz
andren Gründen, wie Starcke sich erinnern wird. Der Aberglaube, daß der
philosophische Idealismus sich um den Glauben an sittliche, d.h.
gesellschaftliche Ideale drehe, ist entstanden außerhalb der
Philosophie, beim deutschen Philister, der die ihm nötigen wenigen
philosophischen Bildungsbrocken in Schillers Gedichten auswendig lernt.
Niemand hat den ohnmächtigen Kantschen „kategorischen Imperativ`` ---
ohnmächtig, weil er das Unmögliche fordert, also nie zu etwas Wirklichem
kommt --- schärfer kritisiert, niemand die durch Schiller vermittelte
Philisterschwärmerei für unrealisierbare Ideale grausamer verspottet
(siehe z.B. die „Phänomenologie``) als grade der vollendete Idealist
Hegel.

Zweitens aber ist es nun einmal nicht zu vermeiden, daß alles,
was einen Menschen bewegt, den Durchgang durch seinen Kopf machen muß ---
sogar Essen und Trinken, das infolge von vermittelst des Kopfs
empfundnem  Hunger und Durst begonnen und infolge von ebenfalls
vermittelst des Kopfs empfundner Sättigung beendigt wird. Die
Einwirkungen der Außenwelt auf den Menschen drücken sich in seinem Kopf
aus, spiegeln sich darin ab als Gefühle, Gedanken, Triebe,
Willensbestimmungen, kurz, als „ideale Strömungen``, und werden in dieser
Gestalt zu „idealen Mächten``. Wenn nun der Umstand, daß dieser Mensch
überhaupt „idealen Strömungen folgt`` und „idealen Mächten`` einen
Einfluß auf sich zugesteht --- wenn dies ihn zum Idealisten macht, so ist
jeder einigermaßen normal entwickelte Mensch ein geborner Idealist, und
wie kann es da überhaupt noch Materialisten geben?

Drittens hat die Überzeugung, daß die Menschheit, augenblicklich
wenigstens, sich im ganzen und großen in fortschreitender Richtung
bewegt, absolut nichts zu tun mit dem Gegensatz von Materialismus und
Idealismus. Die französischen Materialisten hatten diese Überzeugung in
fast fanatischem Grad, nicht minder die Deisten Voltaire und Rousseau,
und brachten oft genug die größten persönlichen Opfer. Wenn irgend
jemand der „Begeisterung für Wahrheit und Recht`` --- die Phrase\est\ im guten
Sinn genommen --- das ganze Leben weihte, so war es z.B. Diderot. Wenn
also Starcke dies alles für Idealismus erklärt, so beweist dies nur, daß
das Wort Materialismus und der ganze Gegensatz beider Richtungen für ihn
hier allen Sinn verloren hat.

Die Tatsache ist, daß Starcke hier dem von der langjährigen
Pfaffenverlästerung her überkommenen Philistervorurteil gegen den Namen
Materialismus eine unverzeihliche Konzession macht --- wenn auch
vielleicht unbewußt. Der Philister versteht unter Materialismus Fressen,
Saufen, Augenlust, Fleischeslust und hoffärtiges Wesen, Geldgier, Geiz,
Habsucht, Profitmacherei und Börsenschwindel, kurz alle die schmierigen
Laster, denen er selbst im stillen frönt; und unter Idealismus den
Glauben an Tugend, allgemeine Menschenliebe und überhaupt die „bessere
Welt``, womit er vor andern renommiert, woran er selbst aber höchstens
glaubt, so lange er den auf seine gewohnheitsmäßigen „materialistischen``
Exzesse notwendig folgenden Katzenjammer oder Bankerott durchzumachen
pflegt und dazu sein Lieblingslied singt: Was ist der Mensch --- halb
Tier, halb Engel.

Im übrigen gibt sich Starcke viel Mühe, Feuerbach gegen die
Angriffe und Lehrsätze der sich heute unter dem Namen Philosophen in
Deutschland breitmachenden Dozenten zu verteidigen. Für Leute, die sich
für diese Nachgeburt der klassischen deutschen Philosophie
interessieren, ist das gewiß wichtig; für Starcke selbst mochte dies
notwendig scheinen. Wir verschonen den Leser damit.

\quebra

\begin{flushright}
\emph{Die Neue Zeit. Jg. 4.}\\
\emph{1886. Nr. 5, Mai}
\end{flushright}

\vspace{2cm}

\section{III}


\noindent{}Der wirkliche Idealismus Feuerbachs tritt zutage, sobald wir auf
seine Religionsphilosophie und Ethik kommen. Er will die Religion
keineswegs abschaffen, er will sie vollenden. Die Philosophie selbst
soll aufgehn in Religion. „Die Perioden der Menschheit unterscheiden
sich nur durch religiöse Veränderungen. Nur da geht eine geschichtliche
Bewegung auf den Grund ein, wo sie auf das Herz des Menschen eingeht.
Das Herz ist nicht eine Form der Religion, so daß sie auch im Herzen
sein sollte; es ist das Wesen der Religion.`` (Zitiert bei Starcke, S.
168.) Religion ist nach Feuerbach das Gefühlsverhältnis, das
Herzensverhältnis zwischen Mensch und Mensch, das bisher in einem
phantastischen Spiegelbild der Wirklichkeit --- in der Vermittlung durch
einen oder viele Götter, phantastische Spiegelbilder menschlicher
Eigenschaften --- seine Wahrheit suchte, jetzt aber in der Liebe zwischen
Ich und Du sie direkt und ohne Vermittlung findet. Und so wird bei
Feuerbach schließlich die Geschlechtsliebe eine der höchsten, wenn nicht
die höchste Form der Ausübung seiner neuen Religion.

Nun haben Gefühlsverhältnisse zwischen den Menschen, namentlich
auch zwischen beiden Geschlechtern bestanden, solange es Menschen gibt.
Die Geschlechtsliebe speziell hat in den letzten achthundert Jahren eine
Ausbildung erhalten und eine Stellung erobert, die sie während dieser
Zeit zum obligatorischen Drehzapfen aller Poesie gemacht hat. Die
bestehenden positiven Religionen haben sich darauf beschränkt, der
staatlichen Regelung der Geschlechtsliebe, d.h. der Ehegesetzgebung, die
höhere Weihe zu geben, und können morgen sämtlich verschwinden, ohne daß
an der Praxis von Liebe und Freundschaft das Geringste geändert wird.
Wie die christliche Religion denn auch in Frankreich von 1793 bis 1798
faktisch so sehr verschwunden war, daß selbst Napoleon sie nicht ohne
Widerstreben und Schwierigkeit wieder einführen konnte, ohne daß jedoch
während des Zwischenraums das Bedürfnis nach einem Ersatz im Sinn
Feuerbachs hervortrat.

Der Idealismus besteht hier bei Feuerbach darin, daß er die auf
gegenseitiger Neigung beruhenden Verhältnisse der Menschen zueinander,
Geschlechtsliebe, Freundschaft, Mitleid, Aufopferung usw., nicht einfach
als das gelten läßt, was sie ohne Rückerinnerung an eine, auch für ihn
der Vergangenheit angehörige, besondre Religion aus sich selbst sind,
sondern behauptet, sie kämen erst zu ihrer vollen Geltung, sobald man
ihnen eine höhere Weihe gibt durch den Namen Religion. Die Hauptsache
für ihn ist nicht, daß diese rein menschlichen Beziehungen existieren,
sondern daß sie als die neue, wahre Religion aufgefaßt werden. Sie
sollen für voll gelten, erst wenn sie religiös abgestempelt sind.
Religion kommt her von religare und heißt ursprünglich Verbindung. Also
ist jede Verbindung zweier Menschen eine Religion. Solche etymologische
Kunststücke bilden das letzte Auskunftsmittel der idealistischen
Philosophie. Nicht was das Wort nach der geschichtlichen Entwicklung
seines wirklichen Gebrauchs bedeutet, sondern was es der Abstammung nach
bedeuten sollte, das soll gelten. Und so wird die Geschlechtsliebe und
die geschlechtliche Verbindung in eine „Religion`` verhimmelt, damit nur
ja nicht das der idealistischen Erinnerung teure Wort Religion aus der
Sprache verschwinde. Grade so sprachen in den vierziger Jahren die
Pariser Reformisten der Louis Blancschen Richtung, die sich ebenfalls
einen Menschen ohne Religion nur als ein Monstrum vorstellen konnten und
uns sagten: Donc, l'athéisme c'est votre religion! \textbar{}\,Also der
Atheismus ist eure Religion!\,\textbar{} Wenn Feuerbach die wahre Religion
auf Grundlage einer wesentlich materialistischen Naturanschauung
herstellen will, so heißt das soviel, wie die moderne Chemie als die
wahre Alchimie auffassen. Wenn die Religion ohne ihren Gott bestehen
kann, dann auch die Alchimie ohne ihren Stein der Weisen. Es besteht
übrigens ein sehr enges Band zwischen Alchimie und Religion. Der Stein
der Weisen hat viele gottähnliche Eigenschaften, und die
ägyptisch-griechischen Alchimisten der ersten beiden Jahrhunderte
unserer Zeitrechnung haben bei der Ausbildung der christlichen Doktrin
ihr Händchen mit im Spiel gehabt, wie die bei Kopp und Berthelot
gegebenen Daten beweisen.

Entschieden falsch ist Feuerbachs Behauptung, daß die „Perioden
der Menschheit sich nur durch religiöse Veränderungen unterscheiden``. \textbar{}\,Große geschichtliche Wendepunkte sind von religiösen
Veränderungen \emph{begleitet} worden, nur soweit die drei
Weltreligionen in Betracht kommen, die bisher bestanden haben:
Buddhismus, Christentum, Islam.\,\textbar{} Die alten naturwüchsig entstandnen
Stammes- und Nationalreligionen waren \textbar{}\,nicht propagandistisch und
verloren\,\textbar{} alle Widerstandskraft, sobald die Selbständigkeit der Stämme
und Völker gebrochen war; bei den Germanen genügte sogar die einfache
Berührung mit dem verfallenden römischen Weltreich und der von ihm
soeben aufgenommenen, seinem ökonomischen, politischen und ideellen
Zustand angemeßnen christlichen Weltreligion. Erst bei diesen mehr oder
weniger künstlich entstandnen Weltreligionen, namentlich beim
Christentum und Islam, finden wir, daß allgemeinere geschichtliche
Bewegungen ein religiöses Gepräge annehmen, und \textbar{}\,selbst auf dem Gebiet
des Christentums\,\textbar{} ist das religiöse Gepräge, für Revolutionen von
wirklich universeller Bedeutung, beschränkt auf die ersten Stufen des
Emanzipationskampfs der Bourgeoisie, vom dreizehnten bis zum siebzehnten
Jahrhundert, und erklärt sich nicht, wie Feuerbach meint, aus dem Herzen
des Menschen und seinem Religionsbedürfnis, sondern aus der ganzen
mittelalterlichen Vorgeschichte, die keine andere Form der Ideologie
kannte als eben die Religion und Theologie. Als aber die Bourgeoisie im
18. Jahrhundert hinreichend erstarkt war, um auch ihre eigne, ihrem
Klassenstandpunkt angemeßne Ideologie zu haben, da machte sie ihre große
und endgültige Revolution, die französische, unter dem ausschließlichen
Appell an juristische und politische Ideen durch und kümmerte sich um
die Religion nur so weit, als diese ihr im Wege stand; es fiel ihr aber
nicht ein, eine neue Religion an die Stelle der alten zu setzen; \textbar{}\,man
weiß, wie Robespierre damit scheiterte.\,\textbar{}

Die Möglichkeit rein menschlicher Empfindung im Verkehr mit
andern Menschen wird uns heutzutage schon genug verkümmert durch die auf
Klassengegensatz und Klassenherrschaft gegründete Gesellschaft, in der
wir uns bewegen müssen: Wir haben keinen Grund, sie uns selbst noch mehr
zu verkümmern, indem wir diese Empfindungen in eine Religion verhimmeln.
Und ebenso wird das Verständnis der geschichtlichen großen Klassenkämpfe
von der landläufigen Geschichtschreibung, namentlich in Deutschland,
schon hinreichend verdunkelt, auch ohne daß wir nötig hätten, es durch
Verwandlung dieser Kampfesgeschichte in einen bloßen Anhang der
Kirchengeschichte uns vollends unmöglich zu machen. Schon hier zeigt
sich, wie weit wir uns heute von Feuerbach entfernt haben. Seine
„schönsten Stellen``, zur Feier dieser neuen Liebesreligion, sind heute
gar nicht mehr lesbar.

Die einzige Religion, die Feuerbach ernstlich untersucht, ist
das Christentum, die Weltreligion des Abendlands, die auf den
Monotheismus gegründet ist. Er weist nach, daß der christliche Gott nur
der phantastische Reflex, das Spiegelbild des Menschen ist. Nun aber ist
dieser Gott selbst das Produkt eines langwierigen Abstraktionsprozesses,
die konzentrierte Quintessenz der früheren vielen Stammes- und
Nationalgötter. Und dementsprechend ist auch der Mensch, dessen Abbild
jener Gott ist, nicht ein wirklicher Mensch, sondern ebenfalls die
Quintessenz der vielen wirklichen Menschen, der abstrakte Mensch, also
selbst wieder ein Gedankenbild. Derselbe Feuerbach, der auf jeder Seite
Sinnlichkeit, Versenkung ins Konkrete, in die Wirklichkeit predigt, er
wird durch und durch abstrakt, sowie er auf einen weiteren als den bloß
geschlechtlichen Verkehr zwischen den Menschen zu sprechen kommt.

Dieser Verkehr bietet ihm nur eine Seite: die Moral. Und hier
frappiert uns wieder die erstaunliche Armut Feuerbachs verglichen mit
Hegel. Dessen Ethik oder Lehre von der Sittlichkeit ist die
Rechtsphilosophie und umfaßt: 1. das abstrakte Recht, 2. die Moralität,
3. die Sittlichkeit, unter welcher wieder zusammengefaßt sind: die
Familie, die bürgerliche Gesellschaft, der Staat. So idealistisch die
Form, so realistisch ist hier der Inhalt. Das ganze Gebiet des Rechts,
der Ökonomie, der Politik ist neben der Moral hier mit einbegriffen. Bei
Feuerbach grade umgekehrt. Er ist der Form nach realistisch, er geht vom
Menschen aus; aber von der Welt, worin dieser Mensch lebt, ist absolut
nicht die Rede, und so bleibt dieser Mensch stets derselbe abstrakte
Mensch, der in der Religionsphilosophie das Wort führte. Dieser Mensch
ist eben nicht aus dem Mutterleib geboren, er hat sich aus dem Gott der
monotheistischen Religionen entpuppt, er lebt daher auch nicht in einer
wirklichen, geschichtlich entstandenen und geschichtlich bestimmten
Welt; er verkehrt zwar mit andern Menschen, aber jeder andere ist ebenso
abstrakt wie er selbst. In der Religionsphilosophie hatten wir doch noch
Mann und Weib, aber in der Ethik verschwindet auch dieser letzte
Unterschied. Allerdings kommen bei Feuerbach in weiten Zwischenräumen
Sätze vor wie: „In einem Palast denkt man anders als in einer Hütte.`` ---
„Wo du vor Hunger, vor Elend keinen Stoff im Leibe hast, da hast du auch
in deinem Kopfe, in deinem Sinne und Herzen keinen Stoff zur Moral.`` ---
„Die Politik muß unsere Religion werden``\est\ usw.Aber mit diesen Sätzen weiß
Feuerbach absolut nichts anzufangen, sie bleiben pure Redensarten, und
selbst Starcke muß eingestehn, daß die Politik für Feuerbach eine
unpassierbare Grenze war und die „Gesellschaftslehre, die Soziologie für
ihn eine terra incognita \textbar{}\,ein unbekanntes Land\,\textbar{}``.

Ebenso flach erscheint er gegenüber Hegel in der Behandlung des
Gegensatzes von Gut und Böse. „Man glaubt etwas sehr Großes zu sagen ---
heißt es bei Hegel --- wenn man sagt: Der Mensch ist von Natur gut; aber
man vergißt, daß man etwas weit Größeres sagt mit den Worten: Der Mensch
ist von Natur böse.`` Bei Hegel ist das Böse die Form, worin die
Triebkraft der geschichtlichen Entwicklung sich darstellt. Und zwar
liegt hierin der doppelte Sinn, daß einerseits jeder neue Fortschritt
notwendig auftritt als Frevel gegen ein Heiliges, als Rebellion gegen
die alten, absterbenden, aber durch die Gewohnheit geheiligten Zustände,
und andrerseits, daß seit dem Aufkommen der Klassengegensätze es grade
die schlechten Leidenschaften der Menschen sind, Habgier und
Herrschsucht, die zu Hebeln der geschichtlichen Entwicklung werden,
wovon z.B. die Geschichte des Feudalismus und der Bourgeoisie ein
einziger fortlaufender Beweis ist. Aber die historische Rolle des
moralisch Bösen zu untersuchen, fällt Feuerbach nicht ein. Die
Geschichte ist ihm überhaupt ein ungemütliches, unheimliches Feld. Sogar
sein Ausspruch: „Der Mensch, der ursprünglich aus der Natur entsprang,
war auch nur ein reines Naturwesen, kein Mensch. Der Mensch ist ein
Produkt des Menschen, der Kultur, der Geschichte`` -selbst dieser
Ausspruch bleibt bei ihm durchaus unfruchtbar.

Was uns Feuerbach über Moral mitteilt, kann hiernach nur äußerst
mager sein. Der Glückseligkeitstrieb ist dem Menschen eingeboren und muß
daher die Grundlage aller Moral bilden. Aber der Glückseligkeitstrieb
erfährt eine doppelte Korrektur. Erstens durch die natürlichen Folgen
unsrer Handlungen: Auf den Rausch folgt der Katzenjammer, auf den
gewohnheitsmäßigen Exzeß die Krankheit. Zweitens durch ihre
gesellschaftlichen Folgen: Respektieren wir nicht den gleichen
Glückseligkeitstrieb der andern, so wehren sie sich und stören unsern
eignen Glückseligkeitstrieb. Hieraus folgt, daß wir, um unsern Trieb zu
befriedigen, die Folgen unsrer Handlungen richtig abzuschätzen imstande
sein und andrerseits die Gleichberechtigung des entsprechenden Triebs
bei andern gelten lassen müssen. Rationelle Selbstbeschränkung in
Beziehung auf uns selbst und Liebe --- immer wieder Liebe! --- im Verkehr
mit andern sind also die Grundregeln der Feuerbachschen Moral, aus denen
alle andern sich ableiten. Und weder die geistvollsten Ausführungen
Feuerbachs noch die stärksten Lobsprüche Starckes können die Dünnheit
und Plattheit dieser paar Sätze verdecken.

Der Glückseligkeitstrieb befriedigt sich nur sehr ausnahmsweise
und keineswegs zu seinem und andrer Leute Vorteil durch die
Beschäftigung eines Menschen mit ihm selbst. Sondern er erfordert
Beschäftigung mit der Außenwelt, Mittel der Befriedigung, also Nahrung,
ein Individuum des andern Geschlechts, Bücher, Unterhaltung, Debatte,
Tätigkeit, Gegenstände der Vernutzung und Verarbeitung. Die
Feuerbachsche Moral setzt entweder voraus, daß diese Mittel und
Gegenstände der Befriedigung jedem Menschen ohne weiteres gegeben sind,
oder aber sie gibt ihm nur unanwendbare gute Lehren, ist also keinen
Schuß Pulver wert für die Leute, denen diese Mittel fehlen. Und das
erklärt Feuerbach selbst in dürren Worten: „In einem Palast denkt man
anders als in einer Hütte.`` „Wo du vor Hunger, vor Elend keinen Stoff im
Leibe hast, da hast du auch in deinem Kopfe, in deinem Sinne und Herzen
keinen Stoff zur Moral.``

Steht es etwa besser mit der Gleichberechtigung des
Glückseligkeitstriebs andrer? Feuerbach stellt diese Forderung absolut
hin, als gültig für alle Zeiten und Umstände. Aber seit wann gilt sie?
War im Altertum zwischen Sklaven und Herren, im Mittelalter zwischen
Leibeignen und Baronen\est\ je die Rede von Gleichberechtigung des
Glückseligkeitstriebs? Wurde nicht der Glückseligkeitstrieb der
unterdrückten Klasse rücksichtslos und „von Rechts wegen`` dem der
herrschenden zum Opfer gebracht? --- Ja, das war auch unmoralisch, jetzt
aber ist die Gleichberechtigung anerkannt. --- Anerkannt in der Phrase,
seitdem und sintemal die Bourgeoisie in ihrem Kampf gegen die Feudalität
und in der Ausbildung der kapitalistischen Produktion gezwungen war,
alle ständischen, d.h. persönlichen Privilegien abzuschaffen und zuerst
die privatrechtliche, dann auch allmählich die staatsrechtliche,
juristische Gleichberechtigung der Person einzuführen. Aber der
Glückseligkeitstrieb lebt nur zum geringsten Teil von ideellen Rechten
und zum allergrößten von materiellen Mitteln, und da sorgt die
kapitalistische Produktion dafür, daß der großen Mehrzahl der
gleichberechtigten Personen nur das zum knappen Leben Notwendige
zufällt, respektiert also die Gleichberechtigung des
Glückseligkeitstriebs der Mehrzahl kaum, wenn überhaupt, besser, als die
Sklaverei oder die Leibeigenschaft dies tat. Und steht es besser in
betreff der geistigen Mittel der Glückseligkeit, der Bildungsmittel? Ist
nicht selbst „der Schulmeister von Sadowa`` eine mythische Person?

Noch mehr. Nach der Feuerbachschen Moraltheorie ist die
Fondsbörse der höchste Tempel der Sittlichkeit --- vorausgesetzt nur, daß
man stets richtig spekuliert. Wenn mein Glückseligkeitstrieb mich auf
die Börse führt und ich dort die Folgen meiner Handlungen so richtig
erwäge, daß sie mir  nur Annehmlichkeit und keinen Nachteil bringen,
d.h. daß ich stets gewinne, so ist Feuerbachs Vorschrift erfüllt. Auch
greife ich dadurch nicht in den gleichen Glückseligkeitstrieb eines
andern ein, denn der andre ist ebenso freiwillig an die Börse gegangen
wie ich, ist beim Abschluß des Spekulationsgeschäfts mit mir ebensogut
seinem Glückseligkeitstrieb gefolgt wie ich dem meinigen. Und verliert
er sein Geld, so beweist sich eben dadurch seine Handlung, weil schlecht
berechnet, als unsittlich, und indem ich an ihm die verdiente Strafe
vollstrecke, kann ich mich sogar als moderner Rhadamanthus stolz in die
Brust werfen. Auch die Liebe herrscht an der Börse, insoweit sie nicht
bloß sentimentale Phrase ist, denn jeder findet im andern die
Befriedigung seines Glückseligkeitstriebs, und das ist ja, was die Liebe
leisten soll und worin sie praktisch sich betätigt. Und wenn ich da in
richtiger Voraussicht der Folgen meiner Operationen, also mit Erfolg
spiele, so erfülle ich alle die strengsten Forderungen der
Feuerbachschen Moral und werde ein reicher Mann obendrein. Mit andern
Worten, Feuerbachs Moral ist auf die heutige kapitalistische
Gesellschaft zugeschnitten, so wenig er selbst das wollen oder ahnen
mag.

Aber die Liebe! --- Ja, die Liebe ist überall und immer der
Zaubergott, der bei Feuerbach über alle Schwierigkeiten des praktischen
Lebens hinweghelfen soll --- und das in einer Gesellschaft, die in Klassen
mit diametral entgegengesetzten Interessen gespalten ist. Damit ist denn
der letzte Rest ihres revolutionären Charakters aus der Philosophie
verschwunden, und es bleibt nur die alte Leier: Liebet euch
untereinander, fallt euch in die Arme ohne Unterschied des Geschlechts
und des Standes --- allgemeiner Versöhnungsdusel!

Kurz und gut. Es geht der Feuerbachschen Moraltheorie wie allen
ihren Vorgängerinnen. Sie ist auf alle Zeiten, alle Völker, alle
Zustände zugeschnitten, und eben deswegen ist sie nie und nirgends
anwendbar und bleibt der wirklichen Welt gegenüber ebenso ohnmächtig wie
Kants kategorischer Imperativ. In Wirklichkeit hat jede Klasse, sogar
jede Berufsart ihre eigne Moral und bricht auch diese, wo sie es
ungestraft tun kann, und die Liebe, die alles einen soll, kommt zu Tag
in Kriegen, Streitigkeiten, Prozessen, häuslichem Krakeel, Ehescheidung
und möglichster Ausbeutung der einen durch die andern.\est\

Wie aber war es möglich, daß der gewaltige, durch Feuerbach
gegebene Anstoß für ihn selbst so unfruchtbar auslief? Einfach dadurch,
daß Feuerbach aus dem ihm selbst tödlich verhaßten Reich der
Abstraktionen den Weg nicht finden kann zur lebendigen Wirklichkeit. Er
klammert sich gewaltsam an die Natur und den Menschen; aber Natur und
Mensch bleiben bei ihm bloß Worte. Weder von der wirklichen Natur noch
von den wirklichen Menschen weiß er uns etwas Bestimmtes zu sagen. Vom
Feuerbachschen abstrakten Menschen kommt man aber nur zu den wirklichen
lebendigen Menschen, wenn man sie in der Geschichte handelnd betrachtet.
Und dagegen sträubte sich Feuerbach, und daher bedeutete das Jahr 1848,
das er nicht begriff, für ihn nur den endgültigen Bruch mit der
wirklichen Welt, den Rückzug in die Einsamkeit. Die Schuld hieran tragen
wiederum hauptsächlich die deutschen Verhältnisse, die ihn elend
verkommen ließen.

Aber der Schritt, den Feuerbach nicht tat, mußte dennoch getan
werden; der Kultus des abstrakten Menschen, der den Kern der
Feuerbachschen neuen Religion bildete, mußte ersetzt werden durch die
Wissenschaft von den wirklichen Menschen und ihrer geschichtlichen
Entwicklung. Diese Fortentwicklung des Feuerbachschen Standpunkts über
Feuerbach hinaus wurde eröffnet 1845 durch Marx in der „Heiligen
Familie``.

\quebra

\mbox{}
\vspace{2cm}

\section{IV}


\noindent{}Strauß, Bauer, Stirner, Feuerbach, das waren die Ausläufer der
Hegelschen Philosophie, soweit sie den philosophischen Boden nicht
verließen. Strauß hat, nach dem „Leben Jesu`` \textbar{}\,und der „Dogmatik``,\,\textbar{} nur
noch philosophische und kirchengeschichtliche Belletristik à la Renan
getrieben; Bauer hat nur auf dem Gebiet der Entstehungsgeschichte des
Christentums etwas geleistet, aber hier auch Bedeutendes; Stirner blieb
ein Kuriosum, selbst nachdem Bakunin ihn mit Proudhon verquickt und
diese Verquickung „Anarchismus`` getauft hatte; Feuerbach allein war
bedeutend als Philosoph. Aber nicht nur blieb die Philosophie, die
angeblich über allen besondern Wissenschaften schwebende, sie
zusammenfassende wissenschaftswissenschaft, für ihn eine
unüberschreitbare Schranke, ein unantastbar Heiliges; er blieb auch als
Philosoph auf halbem Wege stehen, war unten Materialist, oben Idealist;
er wurde mit Hegel nicht kritisch fertig, sondern warf ihn als
unbrauchbar einfach beiseite, während er selbst, gegenüber dem
enzyklopädischen Reichtum des Hegelschen Systems, nichts Positives
fertigbrachte als eine schwülstige Liebesreligion und eine magere,
ohnmächtige Moral.

Aus der Auflösung der Hegelschen Schule ging aber noch eine
andere Richtung hervor, die einzige, die wirklich Früchte getragen hat,
und diese Richtung knüpft sich wesentlich an den Namen
Marx \textsuperscript{(1)}.

Die Trennung von der Hegelschen Philosophie erfolgte auch hier
durch die Rückkehr zum materialistischen Standpunkt. Das heißt, man
entschloß sich, die wirkliche Welt --- Natur und Geschichte --- so
aufzufassen, wie sie sich selbst einem jeden gibt, der ohne vorgefaßte
idealistische Schrullen an sie herantritt; man entschloß sich, jede
idealistische Schrulle unbarmherzig zum Opfer zu bringen, die sich mit
den in ihrem eignen Zusammenhang, und in keinem phantastischen,
aufgefaßten Tatsachen nicht in Einklang bringen ließ. Und weiter heißt
Materialismus überhaupt nichts. Nur daß hier zum erstenmal mit der
materialistischen Weltanschauung wirklich Ernst gemacht, daß sie auf
allen in Frage kommenden Gebieten des Wissens --- wenigstens in den
Grundzügen --- konsequent durchgeführt wurde.

Hegel wurde nicht einfach abseits gelegt; man knüpfte im
Gegenteil an an seine oben entwickelte revolutionäre Seite, an die
dialektische Methode. Aber diese Methode war in ihrer Hegelschen Form
unbrauchbar. Bei Hegel ist die Dialektik die Selbstentwicklung des
Begriffs. Der absolute Begriff ist nicht nur von Ewigkeit --- unbekannt
wo? --- vorhanden, er ist auch die eigentliche lebendige Seele der ganzen
bestehenden Welt. Er entwickelt sich zu sich selbst durch alle die
Vorstufen, die in der „Logik`` des breiteren abgehandelt und die alle in
ihm eingeschlossen sind; dann „entäußert`` er sich, indem er sich in die
Natur verwandelt, wo er ohne Bewußtsein seiner selbst, verkleidet als
Naturnotwendigkeit eine neue Entwicklung durchmacht und zuletzt im
Menschen wieder zum Selbstbewußtsein kommt; dies Selbstbewußtsein
arbeitet sich nun in der Geschichte wieder aus dem Rohen heraus, bis
endlich der absolute Begriff wieder vollständig zu sich selbst kommt in
der Hegelschen Philosophie. Bei Hegel ist also die in der Natur und
Geschichte zutage tretende dialektische Entwicklung, d.h. der
ursächliche Zusammenhang des, durch alle Zickzackbewegungen und
momentanen Rückschritte hindurch, sich durchsetzenden Fortschreitens vom
Niedern zum Höhern, nur der Abklatsch der von Ewigkeit her, man weiß
nicht wo, aber jedenfalls unabhängig von jedem denkenden Menschenhirn
vor sich gehenden Selbstbewegung des Begriffs. Diese ideologische
Verkehrung galt es zu beseitigen. Wir faßten die Begriffe unsres Kopfs
wieder materialistisch als die
Abbilder der wirklichen Dinge, statt die wirklichen Dinge als Abbilder
dieser oder jener Stufe des absoluten Begriffs. Damit reduzierte sich
die Dialektik auf die Wissenschaft von den allgemeinen Gesetzen der
Bewegung, sowohl der äußern Welt wie des menschlichen Denkens --- zwei
Reihen von Gesetzen, die der Sache nach identisch, dem Ausdruck nach
aber insofern verschieden sind, als der menschliche Kopf sie mit
Bewußtsein anwenden kann, während sie in der Natur und bis jetzt auch
großenteils in der Menschengeschichte sich in unbewußter Weise, in der
Form der äußern Notwendigkeit, inmitten einer endlosen Reihe scheinbarer
Zufälligkeiten durchsetzen. Damit aber wurde die Begriffsdialektik
selbst nur der bewußte Reflex der dialektischen Bewegung der wirklichen
Welt, und damit wurde die Hegelsche Dialektik auf den Kopf, oder
vielmehr vom Kopf, auf dem sie stand, wieder auf die Füße gestellt. Und
diese materialistische Dialektik, die seit Jahren unser bestes
Arbeitsmittel und unsere schärfste Waffe war, wurde merkwürdigerweise
nicht nur von uns, sondern außerdem noch, unabhängig von uns und selbst
von Hegel, wieder entdeckt von einem deutschen Arbeiter, Josef
Dietzgen.

Hiermit war aber die revolutionäre Seite der Hegelschen
Philosophie wieder aufgenommen und gleichzeitig von den idealistischen
Verbrämungen befreit, die bei Hegel ihre konsequente Durchführung
verhindert hatten. Der große Grundgedanke, daß die Welt nicht als ein
Komplex von fertigen Dingen zu fassen ist, sondern als ein Komplex
von \emph{Prozessen}, worin die scheinbar stabilen Dinge nicht minder
wie ihre Gedankenabbilder in unserm Kopf, die Begriffe, eine
ununterbrochene Veränderung des Werdens und Vergehens durchmachen, in
der bei aller scheinbaren Zufälligkeit und trotz aller momentanen
Rückläufigkeit schließlich eine fortschreitende Entwicklung sich
durchsetzt --- dieser große Grundgedanke ist, namentlich seit Hegel, so
sehr in das gewöhnliche Bewußtsein übergegangen, daß er in dieser
Allgemeinheit wohl kaum noch Widerspruch findet. Aber ihn in der Phrase
anerkennen und ihn in der Wirklichkeit im einzelnen auf jedem zur
Untersuchung kommenden Gebiet durchführen, ist zweierlei. Geht man aber
bei der Untersuchung stets von diesem Gesichtspunkt aus, so hört die
Forderung endgültiger Lösungen und ewiger Wahrheiten ein für allemal
auf; man ist sich der notwendigen Beschränktheit aller gewonnenen
Erkenntnis stets bewußt, ihrer Bedingtheit durch die Umstände, unter
denen sie gewonnen wurde; aber man läßt sich auch nicht mehr imponieren
durch die der noch stets landläufigen alten Metaphysik unüberwindlichen
Gegensätze von Wahr und Falsch,
Gut und Schlecht, Identisch und Verschieden, Notwendig und Zufällig; man
weiß, daß diese Gegensätze nur relative Gültigkeit haben, daß das jetzt
für wahr Erkannte seine verborgene, später hervortretende falsche Seite
ebensogut hat wie das jetzt als falsch Erkannte seine wahre Seite, kraft
deren es früher für wahr gelten konnte; daß das behauptete Notwendige
sich aus lauter Zufälligkeiten zusammensetzt und das angeblich Zufällige
die Form ist, hinter der die Notwendigkeit sich birgt --- und so weiter.

Die alte Untersuchungs- und Denkmethode, die Hegel die
„metaphysische`` nennt, die sich vorzugsweise mit Untersuchung der Dinge
als gegebener fester Bestände beschäftigte und deren Reste noch stark in
den Köpfen spuken, hatte ihrerzeit eine große geschichtliche
Berechtigung. Die Dinge mußten erst untersucht werden, ehe die Prozesse
untersucht werden konnten. Man mußte erst wissen, was ein beliebiges
Ding war, ehe man die an ihm vorgehenden Veränderungen wahrnehmen
konnte. Und so war es in der Naturwissenschaft. Die alte Metaphysik, die
die Dinge als fertige hinnahm, entstand aus einer Naturwissenschaft, die
die toten und lebendigen Dinge\est\ als fertige untersuchte. Als aber diese
Untersuchung so weit gediehen war, daß der entscheidende Fortschritt
möglich wurde, der Übergang zur systematischen Untersuchung der mit
diesen Dingen in der Natur selbst vorgehenden Veränderungen, da schlug
auch auf philosophischem Gebiet die Sterbestunde der alten Metaphysik.
Und in der Tat, wenn die Naturwissenschaft bis Ende des letzten
Jahrhunderts vorwiegend \emph{sammelnde }Wissenschaft, Wissenschaft von
fertigen Dingen war, so ist sie in unserm Jahrhundert
wesentlich \emph{ordnende} Wissenschaft,Wissenschaft von den Vorgängen,
vom Ursprung und der Entwicklung dieser Dinge und vom Zusammenhang, der
diese Naturvorgänge zu einem großen Ganzen verknüpft. Die Physiologie,
die die Vorgänge im pflanzlichen und tierischen Organismus untersucht,
die Embryologie, die die Entwicklung des einzelnen Organismus vom Keim
bis zur Reife behandelt, die Geologie, die die allmähliche Bildung der
Erdoberfläche verfolgt, sie alle sind Kinder unseres Jahrhunderts.


Vor allem sind es aber drei große Entdeckungen, die unsere
Kenntnis vom Zusammenhang der Naturprozesse mit Riesenschritten
vorangetrieben haben: Erstens die Entdeckung der Zelle als der Einheit,
aus deren Vervielfältigung \textbar{}\,und Differenzierung\,\textbar{} der ganze pflanzliche
und tierische Körper sich entwickelt, so daß nicht nur die Entwicklung
und das Wachstum aller höheren Organismen als nach einem einzigen
allgemeinen Gesetz vor sich gehend erkannt, \textbar{}\,sondern auch in der
Veränderungsfähigkeit der Zelle
der Weg gezeigt ist, auf dem
Organismen ihre Art verändern und damit eine mehr als individuelle
Entwicklung durchmachen können. --- Zweitens\,\textbar{} die Verwandlung der
Energie, die uns alle zunächst in der anorganischen Natur wirksamen
sogenannten Kräfte, die mechanische Kraft und ihre Ergänzung, die
sogenannte potentielle Energie, Wärme, Strahlung (Licht, resp.
strahlende Wärme), Elektrizität, Magnetismus, chemische Energie, als
verschiedene Erscheinungsformen der universellen Bewegung nachgewiesen
hat, die in bestimmten Maßverhältnissen die eine in die andere übergehn,
so daß für die Menge der einen, die verschwindet, eine bestimmte Menge
einer andern wiedererscheint und so daß die ganze Bewegung der Natur
sich auf diesen unaufhörlichen Prozeß der Verwandlung aus einer Form in
die andre reduziert. --- Endlich der zuerst von Darwin im Zusammenhang
entwickelte Nachweis, daß der heute uns umgebende Bestand organischer
Naturprodukte, die Menschen eingeschlossen, das Erzeugnis eines langen
Entwicklungsprozesses aus wenigen ursprünglich einzelligen Keimen ist
und diese wieder aus, auf chemischem Weg entstandenem, Protoplasma oder
Eiweiß hervorgegangen sind.

Dank diesen drei großen Entdeckungen und den übrigen gewaltigen
Fortschritten der Naturwissenschaft sind wir jetzt so weit, den
Zusammenhang zwischen den Vorgängen in der Natur nicht nur auf den
einzelnen Gebieten, sondern auch den der einzelnen Gebiete unter sich im
ganzen und großen nachweisen und so ein übersichtliches Bild des
Naturzusammenhangs in annähernd systematischer Form, vermittelst der
durch die empirische Naturwissenschaft selbst gelieferten Tatsachen
darstellen zu können. Dies Gesamtbild zu liefern, war früher die Aufgabe
der sogenannten Naturphilosophie. Sie konnte dies nur, indem sie die
noch unbekannten wirklichen Zusammenhänge durch ideelle, phantastische
ersetzte, die fehlenden Tatsachen durch Gedankenbilder ergänzte, die
wirklichen Lücken in der bloßen Einbildung ausfüllte. Sie hat bei diesem
Verfahren manche geniale Gedanken gehabt, manche spätern Entdeckungen
vorausgeahnt, aber auch beträchtlichen Unsinn zutage gefördert, wie das
nicht anders möglich war. Heute, wo man die Resultate der Naturforschung
nur dialektisch, d.h. im Sinn ihres eignen Zusammenhangs aufzufassen
braucht, um zu einem für unsere Zeit genügenden „System der Natur`` zu
kommen, wo der dialektische Charakter dieses Zusammenhangs sich sogar
den metaphysisch geschulten Köpfen der Naturforscher gegen ihren Willen
aufzwingt, heute ist die Naturphilosophie endgültig beseitigt. Jeder
Versuch ihrer Wiederbelebung wäre nicht nur überflüssig, \emph{er wäre
ein Rückschritt}.

Was aber von der Natur gilt, die hiermit auch als ein
geschichtlicher Entwicklungsprozeß
erkannt ist, das gilt auch von der Geschichte der Gesellschaft in allen
ihren Zweigen und von der Gesamtheit aller der Wissenschaften, die sich
mit menschlichen (und göttlichen) Dingen beschäftigen. Auch hier hat die
Philosophie der Geschichte, des Rechts, der Religion usw. darin
bestanden, daß an die Stelle des in den Ereignissen nachzuweisenden
wirklichen Zusammenhangs ein im Kopf des Philosophen gemachter gesetzt
wurde, daß die Geschichte im ganzen wie in ihren einzelnen Teilstücken
gefaßt wurde als die allmähliche Verwirklichung von Ideen, und zwar
natürlich immer nur der Lieblingsideen des Philosophen selbst. Die
Geschichte arbeitete hiernach unbewußt, aber mit Notwendigkeit, auf ein
gewisses, von vornherein feststehendes ideelles Ziel los, wie z.B. bei
Hegel auf die Verwirklichung seiner absoluten Idee, und die
unverrückbare Richtung auf diese absolute Idee bildete den Innern
Zusammenhang in den geschichtlichen Ereignissen. An die Stelle des
wirklichen, noch unbekannten Zusammenhangs setzte man somit eine neue ---
unbewußte oder allmählich zum Bewußtsein kommende --- mysteriöse
Vorsehung. Hier galt es also, ganz wie auf dem Gebiet der Natur, diese
gemachten künstlichen Zusammenhänge zu beseitigen durch die Auffindung
der wirklichen; eine Aufgabe, die schließlich darauf hinausläuft, die
allgemeinen Bewegungsgesetze zu entdecken, die sich in der Geschichte
der menschlichen Gesellschaft als herrschende durchsetzen.

Nun aber erweist sich die Entwicklungsgeschichte der
Gesellschaft in einem Punkt als wesentlich verschiedenartig von der der
Natur. In der Natur sind es --- soweit wir die Rückwirkung der Menschen
auf die Natur außer acht lassen --- lauter bewußtlose blinde Agenzien, die
aufeinander einwirken und in deren Wechselspiel das allgemeine Gesetz
zur Geltung kommt. Von allem, was geschieht --- weder von den zahllosen
scheinbaren Zufälligkeiten, die auf der Oberfläche sichtbar werden, noch
von den schließlichen, die Gesetzmäßigkeit innerhalb dieser
Zufälligkeiten bewährenden Resultaten ---, geschieht nichts als gewollter
bewußter Zweck. Dagegen in der Geschichte der Gesellschaft sind die
Handelnden lauter mit Bewußtsein begabte, mit Überlegung oder
Leidenschaft handelnde, auf bestimmte Zwecke hinarbeitende Menschen;
nichts geschieht ohne bewußte Absicht, ohne gewolltes Ziel. Aber dieser
Unterschied, so wichtig er für die geschichtliche Untersuchung
namentlich einzelner Epochen und Begebenheiten ist, kann nichts ändern
an der Tatsache, daß der Lauf der Geschichte durch innere allgemeine
Gesetze beherrscht wird. Denn auch hier herrscht auf der Oberfläche,
trotz der bewußt gewollten Ziele aller einzelnen, im ganzen und großen
scheinbar der Zufall. Nur selten geschieht das Gewollte, in
den \textbar{}\,meisten Fällen
durchkreuzen und widerstreiten sich die vielen gewollten Zwecke oder
sind diese Zwecke selbst von vornherein undurchführbar oder die Mittel
unzureichend. So führen die Zusammenstöße der zahllosen Einzelwillen und
Einzelhandlungen auf geschichtlichem Gebiet einen Zustand herbei, der
ganz dem in der bewußtlosen Natur herrschenden analog ist. Die Zwecke
der Handlungen sind gewollt, aber die Resultate, die wirklich aus den
Handlungen folgen, sind nicht gewollt, oder soweit sie dem gewollten
Zweck zunächst doch zu entsprechen scheinen, haben sie schließlich ganz
andre als die gewollten Folgen. Die geschichtlichen Ereignisse
erscheinen so im ganzen und großen ebenfalls als von der Zufälligkeit
beherrscht. Wo aber auf der Oberfläche der Zufall sein Spiel treibt, da
wird er stets durch innre verborgne Gesetze beherrscht, und es kommt nur
darauf an, diese Gesetze zu entdecken.\est\

Die Menschen machen ihre Geschichte, wie diese auch immer
ausfalle, indem jeder seine eignen, bewußt gewollten Zwecke verfolgt,
und die Resultante dieser vielen in verschiedenen Richtungen agierenden
Willen und ihrer mannigfachen Einwirkung auf die Außenwelt ist eben die
Geschichte. Es kommt also auch darauf an, was die vielen einzelnen
wollen. Der Wille wird bestimmt durch Leidenschaft oder Überlegung. Aber
die Hebel, die wieder die Leidenschaft oder die Überlegung unmittelbar
bestimmen, sind sehr verschiedener Art. Teils können es äußere
Gegenstände sein, teils ideelle Beweggründe, Ehrgeiz, „Begeisterung für
Wahrheit und Recht``, persönlicher Haß oder auch rein individuelle
Schrullen aller Art. Aber einerseits haben wir gesehn, daß die in der
Geschichte tätigen vielen Einzelwillen meist ganz andre als die
gewollten --- oft geradezu die entgegengesetzten --- Resultate
hervorbringen, ihre Beweggründe also ebenfalls für das Gesamtergebnis
nur von untergeordneter Bedeutung sind. Andrerseits fragt es sich
weiter, welche treibenden Kräfte wieder hinter diesen Beweggründen
stehn, welche geschichtlichen Ursachen es sind, die sich in den Köpfen
der Handelnden zu solchen Beweggründen umformen?

Diese Frage hat sich der alte Materialismus nie vorgelegt. Seine
Geschichtsauffassung, soweit er überhaupt eine hat, ist daher auch
wesentlich pragmatisch, beurteilt alles nach den Motiven der Handlung,
teilt die geschichtlich handelnden Menschen in edle und unedle und
findet dann in der Regel, daß die edlen die Geprellten und die unedlen
die Sieger sind, woraus dann folgt für den alten Materialismus, daß beim
Geschichtsstudium nicht viel Erbauliches herauskommt, und für uns, daß
auf dem geschichtlichen Gebiet der alte Materialismus sich selbst untreu
wird, weil er die dort wirksamen ideellen Triebkräfte als letzte
Ursachen hinnimmt, statt zu
unter- suchen, was denn hinter
ihnen steht, was die Triebkräfte dieser Triebkräfte sind. Nicht darin
liegt die Inkonsequenz, daß \emph{ideelle} Triebkräfte\est\ anerkannt werden,
sondern darin, daß von diesen nicht weiter zurückgegangen wird auf ihre
bewegenden Ursachen. Die Geschichtsphilosophie dagegen, wie sie
namentlich durch Hegel vertreten wird, erkennt an, daß die ostensiblen
und auch die wirklich tätigen Beweggründe der geschichtlich handelnden
Menschen keineswegs die letzten Ursachen der geschichtlichen Ereignisse
sind, daß hinter diesen Beweggründen andere bewegende Mächte stehn, die
es zu erforschen gilt; aber sie sucht diese Mächte nicht in der
Geschichte selbst auf, sie importiert sie vielmehr von außen, aus der
philosophischen Ideologie, in die Geschichte hinein. Statt die
Geschichte des alten Griechenlands aus ihrem eignen, innern Zusammenhang
zu erklären, behauptet Hegel z.B. einfach, sie sei weiter nichts als die
Herausarbeitung der „Gestaltungen der schönen Individualität``, die
Realisation des „Kunstwerks`` als solches. Er sagt viel Schönes und
Tiefes bei dieser Gelegenheit über die alten Griechen, aber das hindert
nicht, daß wir uns heute nicht mehr abspeisen lassen mit einer solchen
Erklärung, die eine bloße Redensart ist.

Wenn es also darauf ankommt, die treibenden Mächte zu
erforschen, die --- bewußt oder unbewußt, und zwar sehr häufig unbewußt ---
hinter den Beweggründen der geschichtlich handelnden Menschen stehn und
die eigentlichen letzten Triebkräfte der Geschichte ausmachen, so kann
es sich nicht so sehr um die Beweggründe bei einzelnen, wenn auch noch
so hervorragenden Menschen handeln, als um diejenigen, welche große
Massen, ganze Völker und in jedem Volk wieder ganze Volksklassen in
Bewegung setzen; und auch dies nicht momentan zu einem vorübergehenden
Aufschnellen und rasch verlodernden Strohfeuer, sondern zu dauernder, in
einer großen geschichtlichen Veränderung auslaufender Aktion. Die
treibenden Ursachen zu ergründen, die sich hier in den Köpfen der
handelnden Massen und ihrer Führer --- der sogenannten großen\est\ Männer --- als
bewußte Beweggründe klar oder unklar, unmittelbar oder in ideologischer,
selbst in verhimmelter Form widerspiegeln --- das ist der einzige Weg, der
uns auf die Spur der die Geschichte im ganzen und großen wie in den
einzelnen Perioden und Ländern beherrschenden Gesetze führen kann.
Alles, was die Menschen in Bewegung setzt, muß durch ihren Kopf
hindurch; aber welche Gestalt es in diesem Kopf annimmt, hängt sehr von
den Umständen ab. Die Arbeiter haben sich keineswegs mit dem
kapitalistischen Maschinenbetrieb versöhnt, seitdem sie die Maschinen
nicht mehr, wie noch 1848 am Rhein, einfach in Stücke schlagen.

Während aber in allen früheren Perioden die Erforschung dieser
treibenden Ursachen der
Geschichte fast unmöglich war --- wegen der verwickelten und verdeckten
Zusammenhänge mit ihren Wirkungen ---, hat unsre gegenwärtige Periode
diese Zusammenhänge so weit vereinfacht, daß das Rätsel gelöst werden
konnte. Seit der Durchführung der großen Industrie, also mindestens seit
dem europäischen Frieden von 1815, war es keinem Menschen in England ein
Geheimnis mehr, daß dort der ganze politische Kampf sich drehte um die
Herrschaftsansprüche zweier Klassen, der grundbesitzenden Aristokratie
(landed aristocracy) und der Bourgeoisie (middle class). In Frankreich
kam mit der Rückkehr der Bourbonen dieselbe Tatsache zum Bewußtsein; die
Geschichtsschreiber der Restaurationszeit von Thierry bis Guizot, Mignet
und Thiers sprechen sie überall aus als den Schlüssel zum Verständnis
der französischen Geschichte seit dem Mittelalter. Und seit 1830 wurde
als dritter Kämpfer um die Herrschaft in beiden Ländern die
Arbeiterklasse, das Proletariat, anerkannt. Die Verhältnisse hatten sich
so vereinfacht, daß man die Augen absichtlich verschließen mußte, um
nicht im Kampf dieser drei großen Klassen und im Widerstreit ihrer
Interessen die treibende Kraft der modernen Geschichte zu sehn ---
wenigstens in den beiden fortgeschrittensten Ländern.

Wie aber waren diese Klassen entstanden? Konnte man auf den
ersten Blick dem großen, ehmals feudalen Grundbesitz noch einen Ursprung
aus --- wenigstens zunächst --- politischen Ursachen, aus gewaltsamer
Besitzergreifung zuschreiben, so ging das bei der Bourgeoisie und dem
Proletariat nicht mehr an. Hier lag der Ursprung und die Entwicklung
zweier großer Klassen aus rein ökonomischen Ursachen klar und
handgreiflich zutage. Und ebenso klar war es, daß in dem Kampf zwischen
Grundbesitz und Bourgeoisie, nicht minder als in dem zwischen
Bourgeoisie und Proletariat, es sich in erster Linie um ökonomische
Interessen handelte, zu deren Durchführung die politische Macht als
bloßes Mittel dienen sollte. Bourgeoisie und Proletariat waren beide
entstanden infolge einer Veränderung der ökonomischen Verhältnisse,
genauer gesprochen der Produktionsweise. Der Übergang zuerst vom
zünftigen Handwerk zur Manufaktur, dann von der Manufaktur zur großen
Industrie mit Dampf- und Maschinenbetrieb, hatte diese beiden Klassen
entwickelt. Auf einer gewissen Stufe wurden die von der Bourgeoisie in
Bewegung gesetzten neuen Produktionskräfte --- zunächst die Teilung der
Arbeit und die Vereinigung vieler Teilarbeiter in einer Gesamtmanufaktur
--- und die durch sie entwickelten Austauschbedingungen und
Austauschbedürfnisse unverträglich mit der bestehenden, geschichtlich
überlieferten und durch Gesetz geheiligten Produktionsordnung, d.h. den
zünftigen und den zahllosen andern persönlichen und lokalen
Privilegien (die für die
nichtprivilegierten Stände ebenso viele Fesseln waren) der feudalen
Gesellschaftsverfassung. Die Produktionskräfte, vertreten durch die
Bourgeoisie, rebellierten gegen die Produktionsordnung, vertreten durch
die feudalen Grundbesitzer und die Zunftmeister; das Ergebnis ist
bekannt, die feudalen Fesseln wurden zerschlagen, in England allmählich,
in Frankreich mit einem Schlag, in Deutschland ist man noch nicht damit
fertig. Wie aber die Manufaktur auf einer bestimmten Entwicklungsstufe
in Konflikt kam mit der feudalen, so ist jetzt schon die große Industrie
in Konflikt geraten mit der an ihre Stelle gesetzten bürgerlichen
Produktionsordnung. Gebunden durch diese Ordnung, durch die engen
Schranken der kapitalistischen Produktionsweise, produziert sie
einerseits eine sich immer steigernde Proletarisierung der gesamten
großen Volksmasse, andrerseits eine immer größere Masse unabsetzbarer
Produkte. Überproduktion und Massenelend, jedes die Ursache des andern,
das ist der absurde Widerspruch, worin sie ausläuft und der eine
Entfesselung der Produktivkräfte durch Änderung der Produktionsweise mit
Notwendigkeit fordert.

In der modernen Geschichte wenigstens ist also bewiesen, daß
alle politischen Kämpfe Klassenkämpfe, und alle Emanzipationskämpfe von
Klassen, trotz ihrer notwendig politischen Form --- denn jeder
Klassenkampf ist ein politischer Kampf --- sich schließlich
um \emph{ökonomische} Emanzipation drehen. Hier wenigstens ist also der
Staat, die politische Ordnung, das Untergeordnete, die bürgerliche
Gesellschaft, das Reich der ökonomischen Beziehungen, das entscheidende
Element. Die althergebrachte Anschauung, der auch Hegel huldigt, sah im
Staat das bestimmende, in der bürgerlichen Gesellschaft das durch ihn
bestimmte Element. Der Schein entspricht dem. Wie beim einzelnen
Menschen alle Triebkräfte seiner Handlungen durch seinen Kopf
hindurchgehn, sich in Beweggründe seines Willens verwandeln müssen, um
ihn zum Handeln zu bringen, so müssen auch alle Bedürfnisse der
bürgerlichen Gesellschaft --- gleichviel, welche Klasse grade herrscht ---
durch den Staatswillen hindurchgehn, um allgemeine Geltung in Form von
Gesetzen zu erhalten. Das ist die formelle Seite der Sache, die sich von
selbst versteht; es fragt sich nur, welchen Inhalt dieser nur formelle
Wille --- des einzelnen wie des Staats --- hat, und woher dieser Inhalt
kommt, warum gerade dies und nichts andres gewollt wird. Und wenn wir
hiernach fragen, so finden wir, daß in der modernen Geschichte der
Staatswille im ganzen und großen bestimmt wird durch die wechselnden
Bedürfnisse der bürgerlichen Gesellschaft, durch die Übermacht dieser
oder jener Klasse, in letzter Instanz durch die Entwicklung der
Produktivkräfte und der Austauschverhältnisse.

Wenn aber schon in unsrer modernen Zeit mit ihren riesigen
Produktions- und Verkehrsmitteln der Staat nicht ein selbständiges
Gebiet mit selbständiger Entwicklung ist, sondern sein Bestand wie seine
Entwicklung in letzter Instanz zu erklären ist aus den ökonomischen
Lebensbedingungen der Gesellschaft, so muß dies noch viel mehr gelten
für alle früheren Zeiten, wo die Produktion des materiellen Lebens der
Menschen noch nicht mit diesen reichen Hülfsmitteln betrieben wurde, wo
also die Notwendigkeit dieser Produktion eine noch größere Herrschaft
über die Menschen ausüben mußte. Ist der Staat noch heute, zur Zeit der
großen Industrie und der Eisenbahnen, im ganzen und großen nur der
Reflex, in zusammenfassender Form, der ökonomischen Bedürfnisse der die
Produktion beherrschenden Klasse, so mußte er dies noch viel mehr sein
zu einer Epoche, wo eine Menschengeneration einen weit größeren Teil
ihrer Gesamtlebenszeit auf die Befriedigung ihrer materiellen
Bedürfnisse verwenden mußte, also weit abhängiger von ihnen war, als wir
heute sind. Die Untersuchung der Geschichte früherer Epochen, sobald sie
ernstlich auf diese Seite eingeht, bestätigt dies im reichlichsten Maße;
hier kann dies aber selbstredend nicht verhandelt werden.

Wird der Staat und das Staatsrecht durch die ökonomischen
Verhältnisse bestimmt, so selbstverständlich auch das Privatrecht, das
ja wesentlich nur die bestehenden, unter den gegebnen Umständen normalen
ökonomischen Beziehungen zwischen den einzelnen sanktioniert. Die Form,
in der dies geschieht, kann aber sehr verschieden sein. Man kann, wie in
England im Einklang mit der ganzen nationalen Entwicklung geschah, die
Formen des alten feudalen Rechts großenteils beibehalten und ihnen einen
bürgerlichen Inhalt geben, ja, dem feudalen Namen direkt einen
bürgerlichen Sinn unterschieben; man kann aber auch, wie im
kontinentalen Westeuropa, das erste Weltrecht einer Waren produzierenden
Gesellschaft, das römische, mit seiner unübertrefflich scharfen
Ausarbeitung aller wesentlichen Rechtsbeziehungen einfacher
Warenbesitzer (Käufer und Verkäufer, Gläubiger und Schuldner, Vertrag,
Obligation usw.) zugrunde legen. Wobei man es zu Nutz und Frommen einer
noch kleinbürgerlichen und halbfeudalen Gesellschaft entweder einfach
durch die gerichtliche Praxis auf den Stand dieser Gesellschaft
herunterbringen kann (gemeines Recht), oder aber mit Hülfe angeblich
aufgeklärter, moralisierender Juristen es in ein, diesem
gesellschaftlichen Stand entsprechendes, apartes Gesetzbuch verarbeiten
kann, welches unter diesen Umständen auch juristisch schlecht sein wird
(preußisches Landrecht); wobei man aber auch, nach einer großen
bürgerlichen Revolution, auf Grundlage eben dieses römischen Rechtes,
ein so klassisches Gesetzbuch
der Bourgeoisgesellschaft herausarbeiten kann wie der französische Code
civil. Wenn also die bürgerlichen Rechtsbestimmungen nur die
ökonomischen Lebensbedingungen der Gesellschaft in Rechtsform
ausdrücken, so kann dies je nach Umständen gut oder schlecht geschehen.

Im Staate stellt sich uns die erste ideologische Macht über den
Menschen dar. Die Gesellschaft schafft sich ein Organ zur Wahrung ihrer
gemeinsamen Interessen gegenüber inneren und äußeren Angriffen. Dies
Organ ist die Staatsgewalt. Kaum entstanden, verselbständigt sich dies
Organ gegenüber der Gesellschaft, und zwar um so mehr, je mehr es Organ
einer bestimmten Klasse wird, die Herrschaft dieser Klasse direkt zur
Geltung bringt. Der Kampf der unterdrückten gegen die herrschende Klasse
wird notwendig ein politischer, ein Kampf zunächst gegen die politische
Herrschaft dieser Klasse; das Bewußtsein des Zusammenhangs dieses
politischen Kampfes mit seiner ökonomischen Unterlage wird dumpfer und
kann ganz verlorengehen. Wo dies auch nicht bei den Beteiligten
vollständig der Fall ist, geschieht es fast immer bei den
Geschichtschreibern. Von den alten Quellen über die Kämpfe innerhalb der
römischen Republik sagt uns nur Appian klar und deutlich, um was es sich
schließlich handelte --- nämlich um das Grundeigentum.

Der Staat aber, einmal eine selbständige Macht geworden
gegenüber der Gesellschaft, erzeugt alsbald eine weitere Ideologie. Bei
den Politikern von Profession, bei den Theoretikern des Staatsrechts und
den Juristen des Privatrechts nämlich geht der Zusammenhang mit den
ökonomischen Tatsachen erst recht verloren. Weil in jedem einzelnen
Falle die ökonomischen Tatsachen die Form juristischer Motive annehmen
müssen, um in Gesetzesform sanktioniert zu werden, und weil dabei auch
selbstverständlich Rücksicht zu nehmen ist auf das ganze schon geltende
Rechtssystem, deswegen soll nun die juristische Form alles sein und der
ökonomische Inhalt nichts. Staatsrecht und Privatrecht werden als
selbständige Gebiete behandelt, die ihre unabhängige geschichtliche
Entwicklung haben, die in sich selbst einer systematischen Darstellung
fähig sind und ihrer bedürfen durch konsequente Ausrottung aller inneren
Widersprüche.

Noch höhere, d.h. noch mehr von der materiellen, ökonomischen
Grundlage sich entfernende Ideologien nehmen die Form der Philosophie
und der Religion an. Hier wird der Zusammenhang der Vorstellungen mit
ihren materiellen Daseinsbedingungen immer verwickelter, immer mehr
durch Zwischenglieder verdunkelt. Aber er existiert. Wie die ganze
Renaissancezeit, seit Mitte des 15. Jahrhunderts, ein wesentliches
Produkt der Städte, also des Bürgertums war, so auch die seitdem
neuerwachte Philosophie;
ihr Inhalt war wesentlich nur der
philosophische Ausdruck der der Entwicklung des Klein- und
Mittelbürgertums zur großen Bourgeoisie entsprechenden Gedanken. Bei den
Engländern und Franzosen des vorigen Jahrhunderts, die vielfach
ebensowohl politische Ökonomen wie Philosophen waren, tritt dies klar
hervor, und bei der Hegelschen Schule haben wir es oben nachgewiesen.

Gehen wir indes nur noch kurz auf die Religion ein, weil diese
dem materiellen Leben am fernsten steht und am fremdesten zu sein
scheint. Die Religion ist entstanden zu einer sehr waldursprünglichen
Zeit aus mißverständlichen, waldursprünglichen Vorstellungen der
Menschen über ihre eigne und die sie umgebende äußere Natur. Jede
Ideologie entwickelt sich aber, sobald sie einmal vorhanden, im Anschluß
an den gegebenen Vorstellungsstoff, bildet ihn weiter aus; sie wäre
sonst keine Ideologie, d.h. Beschäftigung mit Gedanken als mit
selbständigen, sich unabhängig entwickelnden, nur ihren eignen Gesetzen
unterworfnen Wesenheiten. Daß die materiellen Lebensbedingungen der
Menschen, in deren Köpfen dieser Gedankenprozeß vor sich geht, den
Verlauf dieses Prozesses schließlich bestimmen, bleibt diesen Menschen
notwendig unbewußt, denn sonst wäre es mit der ganzen Ideologie am Ende.
Diese ursprünglichen religiösen Vorstellungen also, die meist für jede
verwandte Völkergruppe gemeinsam sind, entwickeln sich, nach der
Trennung der Gruppe, bei jedem Volk eigentümlich, je nach den ihm
beschiednen Lebensbedingungen, und dieser Prozeß ist für eine Reihe von
Völkergruppen, namentlich für die arische (sog. indoeuropäische) im
einzelnen nachgewiesen durch die vergleichende Mythologie. Die so bei
jedem Volk herausgearbeiteten Götter waren Nationalgötter, deren Reich
nicht weiter ging als das von ihnen zu schützende nationale Gebiet,
jenseits dessen Grenzen andre Götter unbestritten das große Wort
führten. Sie konnten nur in der Vorstellung fortleben, solange die
Nation bestand; sie fielen mit deren Untergang. Diesen Untergang der
alten Nationalitäten brachte das römische Weltreich, dessen ökonomische
Entstehungsbedingungen wir hier nicht zu untersuchen haben. Die alten
Nationalgötter kamen in Verfall,\est\ selbst die römischen, die eben auch nur
auf den engen Kreis der Stadt Rom zugeschnitten waren; das Bedürfnis,
das Weltreich zu ergänzen durch eine Weltreligion, tritt klar hervor in
den Versuchen, allen irgendwie respektablen fremden Göttern neben den
einheimischen in Rom Anerkennung und Altäre zu schaffen. Aber eine neue
Weltreligion macht sich nicht in dieser Art durch kaiserliche Dekrete.
Die neue Weltreligion, das Christentum, war im stillen bereits
entstanden aus einer Mischung verallgemeinerter orientalischer,
namentlich jüdischer Theologie
und vulgarisierter griechischer, namentlich stoischer Philosophie. Wie
es ursprünglich aussah, müssen wir erst wieder mühsam erforschen, da
seine uns überlieferte offizielle Gestalt nur diejenige ist, in der es
Staatsreligion und diesem Zweck durch das Nicänische Konzil angepaßt
wurde. Genug, die Tatsache, daß es schon nach 250 Jahren Staatsreligion
wurde, beweist, daß es die den Zeitumständen entsprechende Religion war.
Im Mittelalter bildete es sich genau im Maß, wie der Feudalismus sich
entwickelte, zu der diesem entsprechenden Religion aus, mit
entsprechender feudaler Hierarchie. Und als das Bürgertum aufkam,
entwickelte sich im Gegensatz zum feudalen Katholizismus die
protestantische Ketzerei, zuerst in Südfrankreich bei den Albigensern,
zur Zeit der höchsten Blüte der dortigen Städte. Das Mittelalter hatte
alle übrigen Formen der Ideologie: Philosophie, Politik, Jurisprudenz,
an die Theologie annektiert, zu Unterabteilungen der Theologie gemacht.
Es zwang damit jede gesellschaftliche und politische Bewegung, eine
theologische Form anzunehmen; den ausschließlich mit Religion
gefütterten Gemütern der Massen mußten ihre eignen Interessen in
religiöser Verkleidung vorgeführt werden, um einen großen Sturm zu
erzeugen. \textbar{}\,Und wie das Bürgertum von Anfang an einen Anhang von
besitzlosen, keinem anerkannten Stand angehörigen städtischen Plebejern,
Tagelöhnern und Dienstleuten aller Art erzeugte, Vorläufern des spätem
Proletariats,\,\textbar{} so teilt sich auch die Ketzerei schon früh in eine
bürgerlich-gemäßigte und eine plebejisch-revolutionäre, auch von den
bürgerlichen Ketzern verabscheute.

Die Unvertilgbarkeit der protestantischen Ketzerei entsprach der
Unbesiegbarkeit des aufkommenden Bürgertums; als dies Bürgertum
hinreichend erstarkt war, begann sein bisher vorwiegend lokaler Kampf
mit dem Feudaladel nationale Dimensionen anzunehmen. Die erste große
Aktion fand in Deutschland statt --- die sogenannte Reformation. Das
Bürgertum war weder stark noch entwickelt genug, um die übrigen
rebellischen Stände --- die Plebejer der Städte, den niederen Adel und die
Bauern auf dem Lande --- unter seiner Fahne vereinigen zu können. Der Adel
wurde zuerst geschlagen; die Bauern erhoben sich zu einem Aufstand, der
den Gipfelpunkt dieser ganzen revolutionären Bewegung bildet; die Städte
ließen sie im Stich, und so erlag die Revolution den Heeren der
Landesfürsten, die den ganzen Gewinn einstrichen. Von da an verschwindet
Deutschland auf drei Jahrhunderte aus der Reihe der selbständig in die
Geschichte eingreifenden Länder. Aber neben dem Deutschen Luther hatte
der Franzose Calvin gestanden; mit echt französischer Schärfe stellte er
den bürgerlichen Charakter der Reformation in den Vordergrund,
republikanisierte und
demokratisierte die Kirche.
Während die lutherische Reformation in Deutschland versumpfte und
Deutschland zugrunde richtete, diente die calvinische den Republikanern
in Genf, in Holland, in Schottland als Fahne, machte Holland von Spanien
und vom Deutschen Reiche frei und lieferte das ideologische Kostüm zum
zweiten Akt der bürgerlichen Revolution, der in England vor sich ging.
Hier bewährte sich der Calvinismus als die echte religiöse Verkleidung
der Interessen des damaligen Bürgertums und kam deshalb auch nicht zu
voller Anerkennung, als die Revolution 1689 durch einen Kompromiß eines
Teils des Adels mit den Bürgern vollendet wurde. Die englische
Staatskirche wurde wiederhergestellt, aber nicht in ihrer frühem
Gestalt, als Katholizismus mit dem König zum Papst, sondern stark
calvinisiert. Die alte Staatskirche hatte den lustigen katholischen
Sonntag gefeiert und den langweiligen calvinistischen bekämpft, die neue
verbürgerte führte diesen ein, und er verschönert England noch jetzt.

In Frankreich wurde die calvinistische Minorität 1685
unterdrückt, katholisiert oder weggejagt; aber was half's? Schon damals
war der Freigeist Pierre Bayle mitten in der Arbeit, und 1694 wurde
Voltaire geboren. Die Gewaltmaßregel Ludwigs \versal{XIV}. erleichterte nur dem
französischen Bürgertum, daß es seine Revolution in der, der
entwickelten Bourgeoisie allein angemessenen irreligiösen,
ausschließlich politischen Form machen konnte. Statt Protestanten saßen
Freigeister in den Nationalversammlungen. Dadurch war das Christentum in
sein letztes Stadium getreten. Es war unfähig geworden, irgendeiner
progressiven Klasse fernerhin als ideologische Verkleidung ihrer
Strebungen zu dienen; es wurde mehr und mehr Alleinbesitz der
herrschenden Klassen, und diese wenden es an als bloßes
Regierungsmittel, womit die untern Klassen in Schranken gehalten werden.
Wobei dann jede der verschiednen Klassen ihre eigne entsprechende
Religion benutzt: die grundbesitzenden Junker die katholische Jesuiterei
oder protestantische Orthodoxie, die liberalen und radikalen Bourgeois
den Rationalismus; und wobei es keinen Unterschied macht, ob die Herren
an ihre respektiven Religionen selbst glauben oder auch nicht.

Wir sehn also: Die Religion, einmal gebildet, enthält stets
einen überlieferten Stoff, wie denn auf allen ideologischen Gebieten die
Tradition eine große konservative Macht ist. Aber die Veränderungen, die
mit diesem Stoff vorgehn, entspringen aus den Klassenverhältnissen, also
aus den ökonomischen Verhältnissen der Menschen, die diese Veränderungen
vornehmen. Und das ist hier hinreichend. %---

Es kann sich im Vorstehenden nur um einen allgemeinen Umriß der
Marxschen Geschichtsauffassung handeln, höchstens noch um einige
Illustrationen. Der Beweis ist an
der Geschichte selbst zu liefern, und da darf ich wohl sagen, daß er in
andern Schriften bereits hinreichend geliefert ist. Diese Auffassung
macht aber der Philosophie auf dem Gebiet der Geschichte ebenso ein
Ende, wie die dialektische Auffassung der Natur alle Naturphilosophie
ebenso unnötig wie unmöglich macht. Es kommt überall nicht mehr darauf
an, Zusammenhänge im Kopf auszudenken, sondern sie in den Tatsachen zu
entdecken. Für die aus Natur und Geschichte vertriebne Philosophie
bleibt dann nur noch das Reich des reinen Gedankens, soweit es noch
übrig: die Lehre von den Gesetzen des Denkprozesses selbst, die Logik
und Dialektik.

\asterisc

Mit der Revolution von 1848 erteilte das „gebildete`` Deutschland
der Theorie den Absagebrief und ging über auf den Boden der Praxis. Das
auf der Handarbeit beruhende Kleingewerbe und die Manufaktur wurden
ersetzt durch eine wirkliche große Industrie; Deutschland erschien
wieder auf dem Weltmarkt; das neue kleindeutsche Reich beseitigte
wenigstens die schreiendsten Mißstände, die die Kleinstaaterei, die
Reste des Feudalismus und die bürokratische Wirtschaft dieser
Entwicklung in den Weg gelegt hatten. Aber in demselben Maß, wie die
Spekulation aus der philosophischen Studierstube auszog, um ihren Tempel
zu errichten auf der Fondsbörse, in demselben Maß ging auch dem
gebildeten Deutschland jener große theoretische Sinn verloren, der der
Ruhm Deutschlands während der Zeit seiner tiefsten politischen
Erniedrigung gewesen war --- der Sinn für rein wissenschaftliche
Forschung, gleichviel, ob das erreichte Resultat praktisch verwertbar
war oder nicht, polizeiwidrig oder nicht. Zwar hielt sich die deutsche
offizielle Naturwissenschaft, namentlich auf dem Gebiet der
Einzelforschung, auf der Höhe der Zeit, aber schon das amerikanische
Journal „Science`` bemerkt mit Recht, daß die entscheidenden Fortschritte
auf dem Gebiet der großen Zusammenhänge zwischen den Einzeltatsachen,
ihre Verallgemeinerung zu Gesetzen, jetzt weit mehr in England gemacht
werden, statt wie früher in Deutschland. Und auf dem Gebiet der
historischen Wissenschaften, die Philosophie eingeschlossen, ist mit der
klassischen Philosophie der alte theoretisch-rücksichtslose Geist erst
recht verschwunden; gedankenloser Eklektizismus, ängstliche Rücksicht
auf Karriere und Einkommen bis herab zum ordinärsten Strebertum sind an
seine Stelle getreten. Die offiziellen Vertreter dieser Wissenschaft
sind die unverhüllten Ideologen der Bourgeoisie und des bestehenden
Staats geworden --- aber zu einer Zeit, wo beide im offnen Gegensatz stehn
zur Arbeiterklasse.

Und nur bei der Arbeiterklasse besteht der deutsche theoretische
Sinn unverkümmert fort. Hier ist er nicht auszurotten; hier finden keine
Rücksichten statt auf Karriere, auf Profitmacherei, auf gnädige
Protektion von oben; im Gegenteil, je rücksichtsloser und unbefangener
die Wissenschaft vorgeht, desto mehr befindet sie sich im Einklang mit
den Interessen und Strebungen der Arbeiter. Die neue Richtung, die in
der Entwicklungsgeschichte der Arbeit den Schlüssel erkannte zum
Verständnis der gesamten Geschichte der Gesellschaft, wandte sich von
vornherein vorzugsweise an die Arbeiterklasse und fand hier die
Empfänglichkeit, die sie bei der offiziellen Wissenschaft weder suchte
noch erwartete. Die deutsche Arbeiterbewegung ist die Erbin der
deutschen klassischen Philosophie.

\quebra

