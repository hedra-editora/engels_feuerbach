\documentclass[semcabeco,showtrims,trimframe,11pt,spreadimages,openany]{memoir}

\usepackage[largepost]{hedraoptions} %% << %%%%%%%%%%%%%%%%
\usepackage[baruch]{hedrastyles}
\usepackage[xetex,chicagofootnotes]{tipografia}
\usepackage[standart,compontinhos]{toc}
\usepackage{hedraextra}
\usepackage{penalidades}
\usepackage{graficos}
\usepackage{hedralogo}
\usepackage{hifensextras}
\usepackage{fichatecnica}
\usepackage[standart]{aparatos}
\usepackage{tabelas}
\usepackage{versos}
\usepackage{gitrevisioninfo}
\usepackage{parallel}


\newcommand{\forceindent}{\leavevmode{\parindent=1,4em\indent}}

\usepackage{endnotes}
\renewcommand{\notesname}{Notas}

\newcommand{\paragraphbr}[1]{\vfill\pagebreak\paragraph{#1}}
\renewcommand\theparagraph{\Roman{paragraph}}
\newcommand{\quebra}{\vfil\pagebreak}
\newcommand{\est}{\vspace{10cm}}
\newenvironment{mesma}%
   {\par\samepage}%
   {\par\pagebreak[0]}

%\counterwithin*{endnote}{part}
%\counterwithin*{endnote}{chapter}

\hyphenation{es-pe-cial-men-te}
\hyphenation{Thierry}
\hyphenation{con-di-cio-na-das}

\let\latexchapter\chapter
\makeatletter
\renewcommand\enoteheading{%
  \setcounter{secnumdepth}{-2}
  \latexchapter*{\notesname\markboth{NOTAS}{}}
  \mbox{}\par\vskip-\baselineskip
  \let\@afterindentfalse\@afterindenttrue
}
\makeatother

\usepackage{footmisc}

\renewcommand*\footnoterule{}% tira a barrinha da footnote
%\fancyhf[RO]{\cnvt{\thepage} -- \thepage}
%\fancyfoot{}
%\renewcommand{\headrulewidth}{0pt}
%\renewcommand{\footrulewidth}{0pt}}

\usepackage{fontspec}

\newfontfamily\Formular{Formular-Regular}[
BoldFont = Formular-Bold.otf,
ItalicFont = Formular-Italic.otf]
%\newcommand{\Formular}[1]{#1}
\newfontfamily\Brabo{FS Brabo Pro Regular}

%--------------------------------------------ALTERAR DISTÃNCIA ENTRE TÍTULO DO SUMÁRIO E CAPÍTULOS
%\addtocontents{toc}{\vskip-15pt}
%--------------------------------------------
\usepackage{afterpage}

\newcommand\blankpage{%
    \null
    \thispagestyle{empty}%
    \addtocounter{page}{0}%
    \newpage}

%\usepackage{imakeidx} 
%\makeindex[program=xindy, options=-C utf8 -L portuguese]
%\newcommand\gobbleone[1]{}
%\newcommand*{\seeonly}[2]{\ (\emph{\seename} #1)}
%\newcommand*{\also}[2]{\emph{cf.} #1}
%\newcommand{\Also}[2]{\emph{See also} #1}
%\renewcommand\indexname{Índice onomástico}
%\makeindex[intoc]

\setcounter{tocdepth}{0}
\setcounter{secnumdepth}{-2} 
\usepackage{commands}

\usepackage{setspace}

\makeatletter
\newenvironment{Parskip}{%
   \par
   \parskip=0.3\baselineskip \advance\parskip by 0pt plus 2pt
   \parindent=\z@
   \def\@listI{\leftmargin\leftmargini
      \topsep\z@ \parsep\parskip \itemsep\z@}
   \let\@listi\@listI
   \@listi
   \def\@listii{\leftmargin\leftmarginii
      \labelwidth\leftmarginii\advance\labelwidth-\labelsep
      \topsep\z@ \parsep\parskip \itemsep\z@}
   \def\@listiii{\leftmargin\leftmarginiii
       \labelwidth\leftmarginiii\advance\labelwidth-\labelsep
       \topsep\z@ \parsep\parskip \itemsep\z@}
   \partopsep=\z@
}{\par}
\makeatother

\makeatletter
\newenvironment{myParskip}{%
   \par
   \parskip=0.2\baselineskip \advance\parskip by 0pt plus 2pt
   \parindent=\z@
   \def\@listI{\leftmargin\leftmargini
      \topsep\z@ \parsep\parskip \itemsep\z@}
   \let\@listi\@listI
   \@listi
   \def\@listii{\leftmargin\leftmarginii
      \labelwidth\leftmarginii\advance\labelwidth-\labelsep
      \topsep\z@ \parsep\parskip \itemsep\z@}
   \def\@listiii{\leftmargin\leftmarginiii
       \labelwidth\leftmarginiii\advance\labelwidth-\labelsep
       \topsep\z@ \parsep\parskip \itemsep\z@}
   \partopsep=\z@
}{\par}
\makeatother

\newcommand{\mystar}{{\fontfamily{lmr}\selectfont$\star$}}

\makeatletter

\def\@xfootnote[#1]{%
  \protected@xdef\@thefnmark{#1}%
  \@footnotemark\@footnotetext}
%\def\footnoterule{\kern-8\p@
   %\hrule \@width 2in \kern 7.6\p@} % the \hrule is .4pt high
\skip\footins=10mm\@plus3mm\@minus5mm
\makeatother

\hyphenation{Feuer-bach}
\hyphenation{rea-cio-ná-rios}

\begin{document}

\pagebreak
\thispagestyle{empty}
\movetooddpage
\chapter{Apresentação}

%\begin{flushright}
%\textsc{vinicius matteucci de andrade lopes}
%\end{flushright}

\noindent{}\emph{Ludwig Feuerbach e o fim da filosofia clássica alemã} (1886) teve
certamente uma influência fundamental na construção do marxismo do
século \textsc{xx}. Basta dizer que é um texto com o qual Lukács e Lenin
estabeleceram mais de uma vez diálogo. Considerada pelo próprio autor um
aprofundamento da crítica que desenvolveu com Marx, desde a juventude, à
filosofia pós"-hegeliana, a obra tornou"-se um exemplo de como realizar
uma investigação marxista de importantes vertentes do pensamento
filosófico ocidental. %investigação marxista do pensamento filosófico ocidental

Por intermédio de Bernstein e Kautsky, o texto aparece para o público
pela primeira vez nos volumes 4 e 5 da revista \emph{Die Neue Zeit}, em
1886. Uma versão estendida é publicada em 1888 (Dietz, Stuttgart),
juntamente com o anexo da primeira aparição das \emph{Teses sobre %confuso
Feuerbach}, de 1845. O contexto de redação remete à publicação da tese de doutorado do
dinamarquês Carl Nicolaj Starcke acerca da obra de Ludwig Feuerbach,
filósofo que impactou profundamente a formação intelectual de Marx e
Engels, sobretudo no que se refere à crítica de ambos à filosofia
hegeliana. Mas o leitor atento observará, rapidamente, que o texto
engelsiano não se limita a apresentar uma resenha crítica aos aportes de
Starcke sobre Feuerbach. Ao contrário, Engels aproveita a resposta a
Starcke para elaborar um amplo balanço de toda a experiência filosófica na
Alemanha, sem perder de vista as imbricações entre a particularidade do
desenvolvimento capitalista alemão e suas repercussões na formação do
pensamento filosófico após o fracasso das revoluções de 1848.

Dividido em quatro partes, o texto é um aceno à
crítica de Marx e Engels à \emph{ideologia alemã} dos anos 1840 ---
lembrando que os manuscritos desse período (\emph{A ideologia
alemã}) somente serão publicados em 1932 --- e, ao mesmo tempo, uma crítica aos aspectos
fundamentais da consciência filosófica dos anos 1886, cujas raízes remetem ao contexto histórico que se segue às
revoluções de 1848, ao golpe de Napoleão \textsc{iii}, à Comuna de Paris e
ao desfecho da \emph{via prussiana} com a unificação alemã, em 1871. Mais especificamente, Engels observa que, passados quarenta anos dos manuscritos da \emph{Ideologia alemã}, o
papel da filosofia e, principalmente, da ciência havia mudado. O
\emph{ponto de saída} da expansão da relação"-capital pós"-1848 decreta o
fim do agora ``clássico'' período da filosofia alemã, abrindo espaço para
o primeiro grande momento de autorreflexividade do mundo burguês, marcado pela contradição entre a 
\emph{aparência da razão} burguesa e seu \emph{conteúdo irracional}. Ao
conectar, ainda que indiretamente, o \emph{ponto de
expansão da relação"-capital} na Alemanha pós"-1848 e o \emph{ponto de
transição do período clássico da filosofia burguesa}, Engels sugere,
pela primeira vez, um novo momento de consolidação da \emph{consciência
histórica burguesa} que iria desaguar --- seguindo um dos principais
marxistas influenciados por esse texto, Georg Lukács --- no irracionalismo
do período do \textsc{ii} Reich alemão.

Diante de tal conjuntura era preciso defender a potência crítica do
\emph{materialismo"-histórico marxista}, em direta oposição aos outros
materialismos --- principalmente ao anglo"-francês, bem como o de Feuerbach ---, ao
idealismo ``clássico'' --- na figura central de Hegel --- e às metafísicas
positivistas do final do século \textsc{xix}, posteriores ao momento clássico,
como o \emph{neokantismo} na Alemanha. A reflexão de Engels em \emph{Ludwig Feuerbach e o fim da filosofia clássica alemã} pode ser
considerada, nesse sentido, como a primeira aproximação materialista
\emph{crítica do neokantismo}.

Em termos estruturais, é possível afirmar que o texto começa, justamente,
com uma discussão conceitual sobre materialismo e idealismo mediada
pelo duplo sentido da dialética hegeliana (seu lado revolucionário
e reacionário); passa criticamente, então, por Feuerbach e pelo papel dos
progressos científicos diante da clássica relação alemã
metafísica/filosofia da natureza; e encerra com a crítica de Marx
e a consciência de classe revolucionária do trabalhador, essa o
verdadeiro \emph{ponto de saída}, e não apenas da filosofia alemã.

O leitor que conhece as obras de Engels certamente poderá,
também, estabelecer uma aproximação com outros textos escritos ao longo
da década de 1870 e 1880, como o \emph{Anti"-Dühring} (1878) e a
\emph{Dialética da natureza} (1886), principalmente no que se refere ao
problema do método dialético. Uma inflexão, porém, que poderia limitar a
intenção fundamental do próprio texto: ser, ao mesmo tempo,
um acerto de contas com o materialismo histórico iniciado por Engels e
Marx em 1845, e um enfrentamento crítico à conjuntura alemã de 1886. O
que costura esse duplo aspecto, mesmo ao levar"-se em conta as discussões
conceituais esboçadas, é o passo e descompasso específico do
desenvolvimento histórico alemão da década de 1840 até o início da década de 1880.

Engels é um autor que nunca se esquivou da tentativa de desdobrar temas
abstratos ou conceituais na sua vinculação com a concretude histórica.
Uma vinculação que se torna, a partir de então, pressuposto de qualquer
leitura materialista, muito mais fácil de ser prometida do que cumprida
enquanto investigação.

\pagebreak
\section*{Nota da tradução}

Como mencionado na Apresentação, o texto de Engels tem duas versões. A primeira, de 1886, publicada na revista \emph{Die Neue Zeit} em dois momentos (abril e maio), e uma
versão brevemente estendida publicada em 1888 pela \emph{Dietz Verlag}
como livro, juntamente com a primeira aparição das 11 teses de Marx
sobre Feuerbach de 1845 (\emph{Karl Marx über Feuerbach vom Jahre
1845}). A presente tradução, embora baseada na versão da
\textsc{mega},\footnote[*]{\textsc{engels}, F. \emph{Werke. Artikel. Entwürfe.
Oktober 1886 bis Februar} 1891, Band 31, \textsc{i}. \emph{bearbeitet
von Renate Merkel"-Melis.} Akademie Verlag GmbH, Berlin, 2002.} que se
vale da primeira edição em revista, apresenta as partes estendidas que compõe o
livro de 1888, assim como a \emph{nota prévia} não presente em 1886. Ao
longo do texto indicamos com barras verticais {[}\textbar\textbar{]}, tanto na versão alemã como na
tradução, as partes acrescentadas por Engels em 1888. As notas explicativas foram reunidas ao final do livro, quando indicadas por {[}\versal{N.\,A.}{]} referem"-se a comentários do próprio Engels, quando assinaladas por  {[}\versal{N.\,T.}{]} são de autoria do tradutor. Mantivemos o título original das traduções latinas que
seguem a versão francesa de 1894 (\emph{Ludwig Feuerbach et la fin}
{[}\emph{Ausgang}{]} \emph{de la philosophie classique allemande}). A
rigor, o termo exato, mais literal e que abrange o múltiplo sentido da
leitura de Engels seria \emph{o ponto de saída} (\emph{Ausgang})
\emph{da filosofia clássica alemã}. A tradução italiana de Palmiro
Togliatti --- \emph{Ludwig Feuerbach e il punto d'approdo della filosofia
classica tedesca} (1976) --- também questiona, de certa forma, essa
limitação do \emph{Ausgang} como um mero \emph{fim}, em que um movimento
se encerraria e deixaria de existir em si mesmo. Trata"-se, antes, de um
\emph{ponto de chegada} (\emph{punto d'approdo}), que acumula todo um
processo precedente e abre um novo caminho \emph{a partir} \emph{desse}
\emph{acumulo}. Importante aqui é considerar o sentido de movimento, de
transição, que o ano de 1848 representava como ponto, ao mesmo tempo,
de chegada e saída, pelo qual as portas para a outra época da
racionalidade burguesa se abrem.


%\chapter*{}
%\addcontentsline{toc}{part}{Ludwig Feuerbach e o fim da filosofia clássica alemã}
%\begin{center}
%\begin{vplace}[0.3]
%\Large
%Ludwig Feuerbach e o fim\\ da filosofia clássica alemã\\
%\emph{Ludwig Feuerbach und der Ausgang\\ der klassischen deutschen Philosophie}
%\end{vplace}
%\end{center}
%\thispagestyle{empty}
\part[Ludwig Feuerbach e o fim da filosofia clássica alemã]{Ludwig Feuerbach e o fim\\ da filosofia clássica alemã\\ \emph{Ludwig Feuerbach und der Ausgang\\ der klassischen deutschen Philosophie}}

\begin{Parallel}[p]{}{} 
\ParallelLText{\selectlanguage{ngerman} {\let\clearpage\relax\chapter*{Vorbemerkung}}

In der \emph{Vorrede von „Zur Kritik der Politische Ökonomie``},
Berlin, 1859, erzählt Karl Marx, wie wir beide 1845 in Brüssel uns
dranmanchten, „den Gegenstaz usrer Ansicht`` --- die namentlich durch
Marx herausgearbeiteten materialistischen Geschichtsauffassung --- „gegen
die idelogische der deutsche Philosophie gemeinschaftlich auszuarbeiten,
in der Tat mit unserm ehemaligen philosophischen Gewissen aubzurechnen.
Der vorsatz wurde ausgeführt in der Form einer \emph{Kritik der
nachhegelschen Philosophie}. Das Manuskript, zwei strake Oktavbände, was
längst an seinem Verlagsort in Westfalen angelangt, als wir die
Nachricht erhielten, daß veränderte Umstände den Druck nicht erlaubten.
Wir überließen das Manuskript der nagenden Kritik der Mäuse um so
williger, als wir unsern Hauptzweck erreicht hatten ---
Selbstverstädigung``

Seitdem sind über vierzig Jahre verflossen, und Marx ist
gestorben, ohne daß sich einem von uns Gelegenheit geboten hätte, auf
den Gegenstand zurückzukommen. Über unser Verhältnis zu Hegel haben wir
uns stellenweise geäußert, doch nirgends in umfassendem Zusammenhang.
Auf Feuerbach, der doch in mancher Beziehung ein Mittelglied zwischen
der Hegelschen Philosophie und unsrer Auffassung bildet, sind wir nie
wieder zurückgekommen.

Inzwischen hat die Marxsche Weltanschauung Vertreter gefunden
weit über Deutschlands und Europas Grenzen hinaus und in allen
gebildeten Sprachen der Welt. Andrerseits erlebt die klassische deutsche
Philosophie im Ausland eine Art Wiedergeburt, namentlich in England und
Skandinavien, und selbst in Deutschland scheint man die eklektischen
Bettelsuppen satt zu bekommen, die dort an den Universitäten ausgelöffelt werden unter dem Namen Philosophie.

Unter diesen Umständen erschien mir eine kurze, zusammenhängende
Darlegung unsres Verhältnisses zur Hegelschen Philosophie, unsres
Ausgangs wie unsrer Trennung von ihr, mehr und mehr geboten. Und ebenso
erschien mir eine volle Anerkennung des Einflusses, den vor allen andern
nachhegelschen Philosophen Feuerbach, während unsrer Sturm-und
Drangperiode, auf uns hatte, als eine unabgetragene Ehrenschuld. Ich
ergriff also gern die Gelegenheit, als die Redaktion der "Neuen Zeit"
mich um eine kritische Besprechung des Starckeschen Buchs über Feuerbach
bat. Meine Arbeit wurde im 4. und 5. Heft 1886 jener Zeitschrift
veröffentlicht und erscheint hier in revidiertem Sonderabdruck.

Ehe ich diese Zeilen in die Presse schicke, habe ich das alte
Manuskript von 1845/46 nochmals herausgesucht und angesehn. Der
Abschnitt über Feuerbach ist nicht vollendet. Der fertige Teil besteht
in einer Darlegung der materialistischen Geschichtsauffassung, die nur
beweist, wie unvollständig unsre damaligen Kenntnisse der ökonomischen
Geschichte noch waren. Die Kritik der Feuerbachschen Doktrin selbst
fehlt darin; für den gegenwärtigen Zweck war es also unbrauchbar.
Dagegen habe ich in einem alten Heft von Marx die im Anhang
abgedruckten elf Thesen über Feuerbach gefunden. Es sind Notizen für
spätere Ausarbeitung, rasch hingeschrieben, absolut nicht für den Druck
bestimmt, aber unschätzbar als das erste Dokument, worin der geniale
Keim der neuen Weltanschauung niedergelegt ist.

\bigskip

\hfill{}London, 21. Februar 1888

\quebra

\hfill{}\emph{Die Neue Zeit. Jg. 4.}

\hfill{}\emph{1886. Nr, 4, April}

\section{I}

Die vorliegende
Schrift führt uns zurück zu einer
Periode, die, der Zeit nach, ein gutes Menschenalter hinter uns liegt,
die aber der jetzigen Generation in Deutschland so fremd geworden ist,
als wäre sie schon ein volles Jahrhundert alt. Und doch war sie die
Periode der Vorbereitung Deutschlands für die Revolution von 1848; und
alles, was seitdem bei uns geschehn, ist nur eine Fortsetzung von 1848,
nur Testamentsvollstreckung der Revolution.

Wie in Frankreich im achtzehnten, so leitete auch in Deutschland
im neunzehnten Jahrhundert die philosophische Revolution den politischen
Zusammenbruch ein. Aber wie verschieden sahn die beiden aus! Die
Franzosen in offnem Kampf mit der ganzen offiziellen Wissenschaft, mit
der Kirche, oft auch mit dem Staat; ihre Schriften jenseits der Grenze,
in Holland oder England gedruckt, und sie selbst oft genug drauf und
dran, in die Bastille zu wandern. Dagegen die Deutschen - Professoren,
vom Staat eingesetzte Lehrer der Jugend, ihre Schriften anerkannte
Lehrbücher, und das abschließende System der ganzen Entwicklung, das
Hegelsche, sogar gewissermaßen zum Rang einer königlich preußischen
Staatsphilosophie erhoben! Und hinter diesen Professoren, hinter ihren
pedantisch-dunklen Worten, in ihren schwerfälligen, langweiligen
Perioden sollte sich die Revolution verstecken? Waren denn nicht grade
die Leute, die damals für die Vertreter der Revolution galten, die
Liberalen, die heftigsten Gegner dieser die Köpfe verwirrenden
Philosophie? Was aber weder die Regierungen noch die Liberalen sahen,
das sah bereits 1833 wenigstens \emph{ein} Mann, und der hieß allerdings
Heinrich Heine.



Nehmen wir ein Beispiel. Kein philosophischer Satz hat so sehr
den Dank beschränkter Regierungen und den Zorn ebenso beschränkter
Liberalen auf sich geladen wie der berühmte Satz Hegels: "Alles was
wirklich ist, ist vernünftig, und alles was vernünftig ist, ist
wirklich."

Das war doch handgreiflich die Heiligsprechung alles
Bestehenden, die philosophische Einsegnung des Despotismus, des
Polizeistaats, der Kabinettsjustiz, der Zensur. Und so nahm es Friedrich
Wilhelm III., so seine Untertanen. Bei Hegel aber ist keineswegs alles,
was besteht, ohne weiteres auch wirklich. Das Attribut der Wirklichkeit
kommt bei ihm nur demjenigen zu, was zugleich notwendig ist;"die
Wirklichkeit erweist sich in ihrer Entfaltung als die Notwendigkeit";
eine beliebige Regierungsmaßregel - Hegel führt selbst das Beispiel
"einer gewissen Steuereinrichtung" an - gilt ihm daher auch keineswegs
schon ohne weiteres als wirklich. Was aber notwendig ist, erweist sich
in letzter Instanz auch als vernünftig, und auf den damaligen
preußischen Staat angewandt, heißt also der Hegelsche Satz nur: Dieser
Staat ist vernünftig, der Vernunft entsprechend, soweit er notwendig
ist; und wenn er uns dennoch schlecht vorkommt, aber trotz seiner
Schlechtigkeit fortexistiert, so findet die Schlechtigkeit der Regierung
ihre Berechtigung und ihre Erklärung in der entsprechenden
Schlechtigkeit der Untertanen. Die damaligen Preußen hatten die
Regierung, die sie verdienten.

Nun ist aber die Wirklichkeit nach Hegel keineswegs ein
Attribut, das einer gegebnen gesellschaftlichen oder politischen
Sachlage unter allen Umständen und zu allen Zeiten zukommt. Im
Gegenteil. Die römische Republik war wirklich, aber das ǁ sie
verdrängende ǁ römische Kaiserreich auch. Die französische Monarchie war
ǁ 1789 ǁ so unwirklich geworden, d.h. so aller Notwendigkeit beraubt, so
unvernünftig, daß sie vernichtet werden mußte durch die große
Revolution, von der Hegel stets mit der höchsten Begeisterung spricht.
Hier war also die Monarchie das Unwirkliche, die Revolution das
Wirkliche. Und so wird im Lauf der Entwicklung alles früher Wirkliche
unwirklich, verliert seine Notwendigkeit, sein Existenzrecht, seine
Vernünftigkeit; an die Stelle des absterbenden Wirklichen tritt eine
neue, lebensfähige Wirklichkeit - friedlich, wenn das Alte verständig
genug ist, ohne Sträuben mit Tode abzugehn, gewaltsam, wenn es sich
gegen diese Notwendigkeit sperrt. Und so dreht sich der Hegelsche Satz
durch die Hegelsche Dialektik selbst um in sein Gegenteil: Alles, was im
Bereich der Menschengeschichte wirklich ist, wird mit der Zeit
unvernünftig, ist also schon seiner Bestimmung nach unvernünftig, ist
von vornherein mit Unvernünftigkeit behaftet; und alles, was in den
Köpfen der Menschen vernünftig ist, ist bestimmt, wirklich zu werden,
mag es auch noch so sehr der bestehenden scheinbaren Wirklichkeit
widersprechen. Der Satz von der Vernünftigkeit alles Wirklichen löst
sich nach allen Regeln der Hegelschen Denkmethode auf in den andern:
Alles was besteht, ist wert, daß es zugrunde geht.

Dann aber grade lag die wahre Bedeutung und der revolutionäre
Charakter der Hegelschen Philosphie (auf die, als den Abschluß der
ganzen Bewegung seit Kant, wir uns hier beschränken müssen), daß sie der
Endgültigkeit aller Ergebnisse des menschlichen Denkens und Handelns ein
für allemal den Garaus machte. Die Wahrheit, die es in der Philosophie
zu erkennen galt, war bei Hegel nicht mehr eine Sammlung fertiger
dogmatischer Sätze, die, einmal gefunden, nur auswendig gelernt sein
wollen; die Wahrheit lag nun in dem Prozeß des Erkennens selbst, in der
langen geschichtlichen Entwicklung der Wissenschaft, die von niedern zu
immer höhern Stufen der Erkenntnis aufsteigt, ohne aber jemals durch
Ausfindung einer sogenannten absoluten Wahrheit zu dem Punkt zu
gelangen, wo sie nicht mehr weiter kann, wo ihr nichts mehr übrigbleibt,
als die Hände in den Schoß zu legen und die gewonnene absolute Wahrheit
anzustaunen. Und wie auf dem Gebiet der philosophischen, so auf dem
jeder andern Erkenntnis und auf dem des praktischen Handelns.
Ebensowenig wie die Erkenntnis kann die Geschichte einen vollendenden
Abschluß finden in einem vollkommnen Idealzustand der Menschheit; eine
vollkommne Gesellschaft, ein vollkommner "Staat" sind Dinge, die nur in
der Phantasie bestehn können; im Gegenteil sind alle nacheinander
folgenden geschichtlichen Zustände nur vergängliche Stufen im endlosen
Entwicklungsgang der menschlichen Gesellschaft vom Niedern zum Höhern.
Jede Stufe ist notwendig, also berechtigt für die Zeit und die
Bedingungen, denen sie ihren Ursprung verdankt; aber sie wird hinfällig
und unberechtigt gegenüber neuen, höhern Bedingungen, die sich
allmählich in ihrem eignen Schoß entwickeln; sie muß einer höhern Stufe
Platz machen, die ihrerseits wieder an die Reihe des Verfalls und des
Untergangs kommt. Wie die Bourgeoisie durch die große Industrie, die
Konkurrenz und den Weltmarkt alle stabilen, altehrwürdigen Institutionen
praktisch auflöst, so löst diese dialektische Philosophie alle
Vorstellungen von endgültiger absoluter Wahrheit und ihr entsprechenden
absoluten Menschheitszuständen auf. Vor ihr besteht nichts Endgültiges,
Absolutes, Heiliges; sie weist von allem und an allem die
Vergänglichkeit auf, und nichts besteht vor ihr als der ununterbrochne
Prozeß des Werdens und Vergehens, des Aufsteigens ohne Ende vom Niedern
zum Höhern, dessen bloße Widerspiegelung im denkenden Hirn sie selbst
ist. Sie hat allerdings auch eine konservative Seite: Sie erkennt die
Berechtigung bestimmter Erkenntnis- und Gesellschaftsstufen für deren
Zeit und Umstände an; aber auch nur so weit. Der Konservatismus dieser
Anschauungsweise ist relativ, ihr revolutionärer Charakter ist absolut ---
das einzig Absolute, das sie gelten läßt.

Wir brauchen hier nicht auf die Frage einzugehn, ob diese
Anschauungsweise durchaus mit dem jetzigen Stand der Naturwissenschaft
stimmt, die der Existenz der Erde selbst ein mögliches, ihrer
Bewohnbarkeit aber ein ziemlich sichres Ende vorhersagt, die also auch
der Menschengeschichte nicht nur einen aufsteigenden, sondern auch einen
absteigenden Ast zuerkennt. Wir befinden uns jedenfalls noch ziemlich
weit von dem Wendepunkt, von wo an es mit der Geschichte der
Gesellschaft abwärtsgeht, und können der Hegelschen Philosophie nicht
zumuten, sich mit einem Gegenstand zu befassen, den zu ihrer Zeit die
Naturwissenschaft noch gar nicht auf die Tagesordnung gesetzt hatte.

Was aber in der Tat hier zu sagen, ist dies: Die obige
Entwicklung findet sich in dieser Schärfe nicht bei Hegel. Sie ist eine
notwendige Konsequenz seiner Methode, die er selbst aber in dieser
Ausdrücklichkeit nie gezogen hat. Und zwar aus dem einfachen Grund, weil
er genötigt war, ein System zu machen, und ein System der Philosophie
muß nach den hergebrachten Anforderungen mit irgendeiner Art von
absoluter Wahrheit abschließen. Sosehr also auch Hegel, namentlich in
der "Logik", betont, daß diese ewige Wahrheit nichts andres ist als der
logische, resp. der geschichtliche Prozeß selbst, so sieht er sich doch
selbst gezwungen, diesem Prozeß ein Ende zu geben, weil er eben mit
seinem System irgendwo zu Ende kommen muß. In der "Logik" kann er dies
Ende wieder zum Anfang machen, indem hier der Schlußpunkt, die absolute
Idee - die nur insofern absolut ist, als er absolut nichts von ihr zu
sagen weiß - sich in die Natur "entäußert", d.h. verwandelt, und später
im Geist, d.h. im Denken und in der Geschichte, wieder zu sich selbst
kommt. Aber am Schluß der ganzen Philosophie ist ein ähnlicher
Rückschlag in den Anfang nur auf \emph{einem} Weg möglich. Nämlich indem
man das Ende der Geschichte dann setzt, daß die Menschheit zur
Erkenntnis eben dieser absoluten Idee kommt, ǁ und erklärt, daß diese
Erkenntnis der absoluten Idee in der Hegelschen Philosophie erreicht ist
ǁ. Damit wird aber der ganze dogmatische Inhalt des Hegelschen Systems
für die absolute Wahrheit erklärt, im Widerspruch mit seiner
dialektischen, alles Dogmatische auflösenden Methode; damit wird die
revolutionäre Seite erstickt unter der überwuchernden konservativen. Und
was von der philosophischen Erkenntnis, gilt auch von der
geschichtlichen Praxis. Die Menschheit, die es, in der Person Hegels,
bis zur Herausarbeitung der absoluten Idee gebracht hat, muß auch
praktisch so weit gekommen sein, daß sie diese absolute Idee in der
Wirklichkeit durchführen kann. Die praktischen politischen Forderungen
der absoluten Idee an die Zeitgenossen dürfen also nicht zu hoch
gespannt sein. Und so finden wir am Schluß der "Rechtsphilosophie", daß
die absolute Idee sich verwirklichen soll in derjenigen ständischen
Monarchie, die Friedrich Wilhelm III. seinen Untertanen so hartnäckig
vergebens versprach, also in einer den deutschen kleinbürgerlichen
Verhältnissen von damals angemessenen, beschränkten und gemäßigten,
indirekten Herrschaft der besitzenden Klassen; wobei uns noch die
Notwendigkeit des Adels auf spekulativem Wege demonstriert wird.

Die innern Notwendigkeiten des Systems reichen also allein hin,
die Erzeugung einer sehr zahmen politischen Schlußfolgerung, vermittelst
einer durch und durch revolutionären Denkmethode, zu erklären. Die
spezifische Form dieser Schlußfolgerung rührt allerdings davon her, daß
Hegel ein Deutscher war und ihm wie seinem Zeitgenossen Goethe ein Stück
Philisterzopfs hinten hing. Goethe wie Hegel waren jeder auf seinem
Gebiet ein olympischer Zeus, aber den deutschen Philister wurden beide
nie ganz los.

Alles dies hinderte jedoch das Hegelsche System nicht, ein
unvergleichlich größeres Gebiet zu umfassen als irgendein früheres
System und auf diesem Gebiet einen Reichtum des Gedankens zu entwickeln,
der noch heute in Erstaunen setzt. Phänomenologie des Geistes (die man
eine Parallele der Embryologie und der Paläontologie des Geistes nennen
könnte, eine Entwicklung des individuellen Bewußtseins durch seine
verschiedenen Stufen, gefaßt als abgekürzte Reproduktion der Stufen, die
das Bewußtsein der Menschen geschichtlich durchgemacht), Logik,
Naturphilosophie, Philosophie des Geistes, und diese letztere wieder in
ihren einzelnen geschichtlichen Unterformen ausgearbeitet: Philosophie
der Geschichte, des Rechts, der Religion, Geschichte der Philosophie,
Ästhetik usw. - auf allen diesen verschiednen geschichtlichen Gebieten
arbeitet Hegel daran, den durchgehenden Faden der Entwicklung
aufzufinden und nachzuweisen; und da er nicht nur ein schöpferisches
Genie war, sondern auch ein Mann von enzyklopädischer Gelehrsamkeit, so
tritt er überall epochemachend auf. Es versteht sich von selbst, daß
kraft der Notwendigkeiten des "Systems" er hier oft genug zu jenen
gewaltsamen Konstruktionen seine Zuflucht nehmen muß, von denen seine
zwerghaften Anfeinder bis heute ein so entsetzliches Geschrei machen.
Aber diese Konstruktionen sind nur der Rahmen und das Baugerüst seines
Werks; hält man sich hierbei nicht unnötig auf, dringt man tiefer ein in
den gewaltigen Bau, so findet man ungezählte Schätze, die auch heute
noch ihren vollen Wert behaupten. Bei allen Philosophen ist grade das
"System" das Vergängliche, und zwar grade deshalb, weil es aus einem
unvergänglichen Bedürfnis des Menschengeistes hervorgeht: dem Bedürfnis
der Überwindung aller Widersprüche. Sind aber alle Widersprüche ein für
allemal beseitigt, so sind wir bei der sogenannten absoluten Wahrheit
angelangt, die Weltgeschichte ist zu Ende, und doch soll sie fortgehn,
obwohl ihr nichts mehr zu tun übrigbleibt - also ein neuer, unlösbarer
Widerspruch. Sobald wir einmal eingesehn haben - und zu dieser Einsicht
hat uns schließlich niemand mehr verhelfen als Hegel selbst ---, daß die
so gestellte Aufgabe der Philosophie weiter nichts heißt als die
Aufgabe, daß ein einzelner Philosoph das leisten soll, was nur die
gesamte Menschheit in ihrer fortschreitenden Entwicklung leisten kann ---
sobald wir das einsehn, ist es auch am Ende mit der ganzen Philosophie
im bisherigen Sinn des Worts. Man läßt die auf diesem Weg und für jeden
einzelnen unerreichbare "absolute Wahrheit" laufen und jagt dafür den
erreichbaren relativen Wahrheiten nach auf dem Weg der positiven
Wissenschaften und der Zusammenfassung ihrer Resultate vermittelst des
dialektischen Denkens. Mit Hegel schließt die Philosophie überhaupt ab;
einerseits weil er ihre ganze Entwicklung in seinem System in der
großartigsten Weise zusammenfaßt, andrerseits weil er uns, wenn auch
unbewußt, den Weg zeigt aus diesem Labyrinth der Systeme zur wirklichen
positiven Erkenntnis der Welt.

Man begreift, welch ungeheure Wirkung dies Hegelsche System in
der philosophisch gefärbten Atmosphäre Deutschlands hervorbringen mußte.
Es war ein Triumphzug, der Jahrzehnte dauerte und mit dem Tod Hegels
keineswegs zur Ruhe kam. Im Gegenteil, grade von 1830 bis 1840 herrschte
die "Hegelei" am ausschließlichsten und hatte selbst ihre Gegner mehr
oder weniger angesteckt; grade in dieser Zeit drangen Hegelsche
Anschauungen am reichlichsten, bewußt oder unbewußt, in die
verschiedensten Wissenschaften ein und durchsäuerten auch die populäre
Literatur und die Tagespresse, aus denen das gewöhnliche "gebildete
Bewußtsein" seinen Gedankenstoff bezieht. Aber dieser Sieg auf der
ganzen Linie war nur das Vorspiel eines innern Kampfs.

Die Gesamtlehre Hegels ließ, wie wir gesehn, reichlichen Raum
für die Unterbringung der verschiedensten praktischen
Parteianschauungen; und praktisch waren im damaligen theoretischen
Deutschland vor allem zwei Dinge: die Religion und die Politik. Wer das
Hauptgewicht auf das \emph{System} Hegels legte, konnte auf beiden
Gebieten ziemlich konservativ sein; wer in der
dialektischen \emph{Methode} die Hauptsache sah, konnte religiös wie
politisch zur äußersten Opposition gehören. Hegel selbst schien, trotz
der ziemlich häufigen revolutionären Zornesausbrüche in seinen Werken,
im ganzen mehr zur konservativen Seite zu neigen; hatte ihm doch sein
System weit mehr "saure Arbeit des Gedankens" gekostet als seine
Methode. Gegen Ende der dreißiger Jahre trat die Spaltung in der Schule
mehr und mehr hervor. Der linke Flügel, die sogenannten Junghegelianer,
gaben im Kampf mit pietistischen Orthodoxen und feudalen Reaktionären
ein Stück nach dem andern auf von jener philosophisch-vornehmen
Zurückhaltung gegenüber den brennenden Tagesfragen, die ihrer Lehre
bisher staatliche Duldung und sogar Protektion gesichert hatte; und als
gar 1840 die orthodoxe Frömmelei und die feudal-absolutistische Reaktion
mit Friedrich Wilhelm IV. den Thron bestiegen, wurde offne Parteinahme
unvermeidlich. Der Kampf wurde noch mit philosophischen Waffen geführt,
aber nicht mehr um abstrakt-philosophische Ziele; es handelte sich
direkt um Vernichtung der überlieferten Religion und des bestehenden
Staats. Und wenn in den "Deutschen Jahrbüchern" die praktischen
Endzwecke noch vorwiegend in philosophischer Verkleidung auftraten, so
enthüllte sich die junghegelsche Schule in der "Rheinischen Zeitung" von
1842 direkt als die Philosophie der aufstrebenden radikalen Bourgeoisie
und brauchte das philosophische Deckmäntelchen nur noch zur Täuschung
der Zensur.

Die Politik war aber damals ein sehr dorniges Gebiet, und so
wandte sich der Hauptkampf gegen die Religion; dies war ja, namentlich
seit 1840, indirekt auch ein politischer Kampf. Den ersten Anstoß hatte
Strauß' "Leben Jesu" 1835 gegeben. Der hierin entwickelten Theorie der
evangelischen Mythenbildung trat später Bruno Bauer mit dem Nachweis
gegenüber, daß eine ganze Reihe evangelischer Erzählungen von den
Verfassern selbst fabriziert worden. Der Streit zwischen beiden wurde
geführt in der philosophischen Verkleidung eines Kampfes des
"Selbstbewußtseins" gegen die "Substanz"; die Frage, ob die
evangelischen Wundergeschichten durch bewußtlos-traditionelle
Mythenbildung im Schoß der Gemeinde entstanden oder ob sie von den
Evangelisten selbst fabriziert seien, wurde aufgebauscht zu der Frage,
ob in der Weltgeschichte die "Substanz" oder das "Selbstbewußtsein" die
entscheidend wirkende Macht sei; und schließlich kam Stirner, der
Prophet des heutigen Anarchismus - Bakunin hat sehr viel aus ihm
genommen - und übergipfelte das souveräne "Selbstbewußtsein" durch
seinen souveränen "Einzigen".

Wir gehn auf diese Seite des Zersetzungsprozesses der Hegelschen
Schule nicht weiter ein. Wichtiger für uns ist dies: Die Masse der
entschiedensten Junghegelianer wurde durch die praktischen
Notwendigkeiten ihres Kampfs gegen die positive Religion auf den
englisch-französischen Materialismus zurückgedrängt. Und hier kamen sie
in Konflikt mit ihrem Schulsystem. Während der Materialismus die Natur
als das einzig Wirkliche auffaßt, stellt diese im Hegelschen System nur
die "Entäußerung" der absoluten Idee vor, gleichsam eine Degradation der
Idee; unter allen Umständen ist hier das Denken und sein
Gedankenprodukt, die Idee, das Ursprüngliche, die Natur das Abgeleitete,
das nur durch die Herablassung der Idee überhaupt existiert. Und in
diesem Widerspruch trieb man sich herum, so gut und so schlecht es gehn
wollte.

Da kam Feuerbachs "Wesen des Christenthums". Mit einem Schlag
zerstäubte es den Widerspruch, indem es den Materialismus ohne
Umschweife wieder auf den Thron erhob. Die Natur existiert unabhängig
von aller Philosophie; sie ist die Grundlage, auf der wir Menschen,
selbst Naturprodukte, erwachsen sind; außer der Natur und den Menschen
existiert nichts, und die höhern Wesen, die unsere religiöse Phantasie
erschuf, sind nur die phantastische Rückspiegelung unsers eignen Wesens.
Der Bann war gebrochen; das "System" war gesprengt und beiseite
geworfen, der Widerspruch war, als nur in der Einbildung vorhanden,
aufgelöst. - Man muß die befreiende Wirkung dieses Buchs selbst erlebt
haben, um sich eine Vorstellung davon zu machen. Die Begeisterung war
allgemein: Wir waren alle momentan Feuerbachianer. Wie enthusiastisch
Marx die neue Auffassung begrüßte ǁ und wie sehr er - trotz aller
kritischen Vorbehalte - von ihr beeinflußt wurde, ǁ kann man in
der "Heiligen Familie" lesen.

Selbst die Fehler des Buchs trugen zu seiner augenblicklichen
Wirkung bei. Der belletristische, stellenweise sogar schwülstige Stil
sicherte ein größeres Publikum und war immerhin eine Erquickung nach den
langen Jahren abstrakter und abstruser Hegelei. Dasselbe gilt von der
überschwenglichen Vergötterung der Liebe, die gegenüber der unerträglich
gewordnen Souveränität des "reinen Denkens" eine Entschuldigung, wenn
auch keine Berechtigung fand. Was wir aber nicht vergessen dürfen: Grade
an diese beiden Schwächen Feuerbachs knüpfte der seit 1844 sich im
"gebildeten" Deutschland wie eine Seuche verbreitende "wahre
Sozialismus" an, der an die Stelle der wissenschaftlichen Erkenntnis die
belletristische Phrase, an die Stelle der Emanzipation des Proletariats
durch die ökonomische Umgestaltung der Produktion die Befreiung der
Menschheit vermittelst der "Liebe" setzte, kurz, sich in die
widerwärtige Belletristik und Liebesschwüligkeit verlief, deren Typus
Herr Karl Grün war.

Was fernerhin nicht zu vergessen: Die Hegelsche Schule war
aufgelöst, aber die Hegelsche Philosophie war nicht kritisch überwunden.
Strauß und Bauer nahmen jeder eine ihrer Seiten heraus und kehrten sie
polemisch gegen die andre. Feuerbach durchbrach das System und warf es
einfach beiseite. Aber man wird nicht mit einer Philosophie fertig
dadurch, daß man sie einfach für falsch erklärt. Und ein so gewaltiges
Werk wie die Hegelsche Philosophie, die einen so ungeheuren Einnuß auf
die geistige Entwicklung der Nation gehabt, ließ sich nicht dadurch
beseitigen, daß man sie kurzerhand ignorierte. Sie mußte in ihrem
eigenen Sinn "aufgehoben" werden, d.h. in dem Sinn, daß ihre Form
kritisch vernichtet, der durch sie gewonnene neue Inhalt aber gerettet
wurde. Wie dies geschah, davon weiter unten.

Einstweilen schob die Revolution von 1848 jedoch die gesamte
Philosophie ebenso ungeniert beiseite wie Feuerbach seinen Hegel. Und
damit wurde auch Feuerbach selbst in den Hintergrund gedrängt.

\pagebreak

\section{II}



Die große Grundfrage aller, speziell neueren Philosophie ist die
nach dem Verhältnis von Denken und Sein. Seit der sehr frühen Zeit, wo
die Menschen, noch in gänzlicher Unwissenheit über ihren eigenen
Körperbau und angeregt durch Traumerscheinungen, auf die Vorstellung
kamen, ihr Denken und Empfinden sei nicht eine Tätigkeit ihres Körpers,
sondern einer besonderen, in diesem Körper wohnenden und ihn beim Tode
verlassenden Seele - seit dieser Zeit mußten sie über das Verhältnis
dieser Seele zur äußern Welt sich Gedanken machen. Wenn sie im Tod sich
vom Körper trennte, fortlebte, so lag kein Anlaß vor, ihr noch einen
besondren Tod anzudichten; so entstand die Vorstellung von ihrer
Unsterblichkeit, die auf jener Entwicklungsstufe keineswegs als ein
Trost erscheint, sondern als ein Schicksal, wogegen man nicht ankann,
und oft genug, wie bei den Griechen, als ein positives Unglück. Nicht
das religiöse Trostbedürfnis, sondern die aus gleich allgemeiner
Beschränktheit hervorwachsende Verlegenheit, was mit der einmal
angenommenen Seele, nach dem Tod des Körpers, anzufangen, führte
allgemein zu der langweiligen Einbildung von der persönlichen
Unsterblichkeit. Auf ganz ähnlichem Weg entstanden, durch
Personifikation der Naturmächte, die ersten Götter, die in der weitern
Ausbildung der Religionen eine mehr und mehr außerweltliche Gestalt
annahmen, bis endlich durch einen im Verlauf der geistigen Entwicklung
sich naturgemäß einstellenden Abstraktions-, ich möchte fast sagen %assim mesmo esse hífen?
Destillationsprozeß aus den vielen, mehr oder minder beschränkten und
sich gegenseitig beschränkenden Göttern die Vorstellung von dem einen
ausschließlichen Gott der monotheistischen Religionen in den Köpfen der
Menschen entstand.

Die Frage nach dem Verhältnis des Denkens zum Sein, des Geistes
zur Natur, die höchste Frage der gesamten Philosophie hat also, nicht
minder als alle Religion, ihre Wurzel in den bornierten und unwissenden
Vorstellungen des Wildheitszustands. Aber in ihrer vollen Schärfe konnte
sie erst gestellt werden, ihre ganze Bedeutung konnte sie erst erlangen,
als die europäische Menschheit aus dem langen Winterschlaf des
christlichen Mittelalters erwachte. Die Frage nach der Stellung des
Denkens zum Sein, die übrigens auch in der Scholastik des Mittelalters
ihre große Rolle gespielt, die Frage: Was ist das Ursprüngliche, der
Geist oder die Natur? - diese Frage spitzte sich, der Kirche gegenüber,
dahin zu: Hat Gott die Welt erschaffen, oder ist die Welt von Ewigkeit
da?

Je nachdem diese Frage so oder so beantwortet wurde, spalteten
sich die Philosophen in zwei große Lager. Diejenigen, die die
Ursprünglichkeit des Geistes gegenüber der Natur behaupteten, also in
letzter Instanz eine Weltschöpfung irgendeiner Art annahmen - und diese
Schöpfung ist oft bei den Philosophen, z.B. bei Hegel, noch weit
verzwickter und unmöglicher als im Christentum ---, bildeten das Lager des
Idealismus. Die andern, die die Natur als das Ursprüngliche ansahen,
gehören zu den verschiednen Schulen des Materialismus.

Etwas andres als dies bedeuten die beiden Ausdrücke: Idealismus
und Materialismus ursprünglich nicht, und in einem andern Sinne werden
sie hier auch nicht gebraucht. Welche Verwirrung entsteht, wenn man
etwas andres in sie hineinträgt, werden wir unten sehn.

Die Frage nach dem Verhältnis von Denken und Sein hat aber noch
eine andre Seite: Wie verhalten sich unsere Gedanken über die uns
umgebende Welt zu dieser Welt selbst? Ist unser Denken imstande, die
wirkliche Welt zu erkennen, vermögen wir in unsern Vorstellungen und
Begriffen von der wirklichen Welt ein richtiges Spiegelbild der
Wirklichkeit zu erzeugen? Diese Frage heißt in der philosophischen
Sprache die Frage nach der Identität von Denken und Sein und wird von
der weitaus größten Zahl der Philosophen bejaht. Bei Hegel z.B. versteht
sich ihre Bejahung von selbst; denn das, was wir in der wirklichen Welt
erkennen, ist eben ihr gedankenmäßiger Inhalt, dasjenige, was die Welt
zu einer stufenweisen Verwirklichung der absoluten Idee macht, welche
absolute Idee von Ewigkeit her, unabhängig von der Welt und vor der
Welt, irgendwo existiert hat; daß aber das Denken einen Inhalt erkennen
kann, der schon von vornherein Gedankeninhalt ist, leuchtet ohne weitres
ein. Ebensosehr leuchtet ein, daß hier das zu Beweisende im stillen
schon in der Voraussetzung enthalten ist. Das hindert aber Hegel
keineswegs, aus seinem Beweis der Identität von Denken und Sein den
weitern Schluß zu ziehen, daß seine Philosophie, weil für sein Denken
richtig, nun auch die einzig richtige ist und daß die Identität von
Denken und Sein sich darin zu bewähren hat, daß die Menschheit sofort
seine Philosophie aus der Theorie in die Praxis übersetzt und die ganze
Welt nach Hegelschen Grundsätzen umgestaltet. Es ist dies eine Illusion,
die er so ziemlich mit allen Philosophen teilt.

Daneben gibt es aber noch eine Reihe andrer Philosophen, die die
Möglichkeit einer Erkenntnis der Welt oder doch einer erschöpfenden
Erkenntnis bestreiten. Zu ihnen gehören unter den neueren Hume und Kant,
und sie haben eine sehr bedeutende Rolle in der philosophischen
Entwicklung gespielt. Das Entscheidende zur Widerlegung dieser Ansicht
ist bereits von Hegel gesagt, soweit dies vom idealistischen Standpunkt
möglich war; was Feuerbach Materialistisches hinzugefügt, ist mehr
geistreich als tief. Die schlagendste Widerlegung dieser wie aller
andern philosophischen Schrullen ist die Praxis, nämlich das Experiment
und die Industrie. Wenn wir die Richtigkeit unsrer Auffassung eines
Naturvorgangs beweisen können, indem wir ihn selbst machen, ihn aus
seinen Bedingungen erzeugen, ihn obendrein unsern Zwecken dienstbar
werden lassen, so ist es mit dem Kantschen unfaßbaren "Ding an sich" zu
Ende. Die im pflanzlichen und tierischen Körper erzeugten chemischen
Stoffe blieben solche "Dinge an sich", bis die organische Chemie sie
einen nach dem andern darzustellen anfing; damit wurde das "Ding an
sich" ein Ding für uns, wie z.B. der Farbstoff des Krapps, das Alizarin,
das wir nicht mehr auf dem Felde in den Krappwurzeln wachsen lassen,
sondern aus Kohlenteer weit wohlfeiler und einfacher herstellen. Das
kopernikanische Sonnensystem war dreihundert Jahre lang eine Hypothese,
auf die hundert, tausend, zehntausend gegen eins zu wetten war, aber
doch immer eine Hypothese; als aber Leverrier aus den durch dies System
gegebenen Daten nicht nur die Notwendigkeit der Existenz eines
unbekannten Planeten, sondern auch den Ort berechnete, wo dieser Planet
am Himmel stehn müsse, und als Galle dann diesen Planeten wirklich fand,
da war das kopernikanische System bewiesen. Wenn dennoch die Neubelebung
der Kantschen Auffassung in Deutschland durch die Neukantianer und der
Humeschen in England (wo sie nie ausgestorben) durch die Agnostiker
versucht wird, so ist das, der längst erfolgten theoretischen und
praktischen Widerlegung gegenüber, wissenschaftlich ein Rückschritt und
praktisch nur eine verschämte Weise, den Materialismus hinterrücks zu
akzeptieren und vor der Welt zu verleugnen.

Die Philosophen wurden aber in dieser langen Periode von
Descartes bis Hegel und von Hobbes bis Feuerbach keineswegs, wie sie
glaubten, allein durch die Kraft des reinen Gedankens vorangetrieben. Im
Gegenteil. Was sie in Wahrheit vorantrieb, das war namentlich der
gewaltige und immer schneller voranstürmende Fortschritt der
Naturwissenschaft ǁ und der Industrie ǁ. Bei den Materialisten zeigte
sich dies schon auf der Oberfläche, aber auch die idealistischen Systeme
erfüllten sich mehr und mehr mit materialistischem Inhalt und suchten
den Gegensatz von Geist und Materie pantheistisch zu versöhnen; so daß
schließlich das Hegelsche System nur einen nach Methode und Inhalt
idealistisch auf den Kopf gestellten Materialismus repräsentiert.

Es ist hiermit begreiflich, daß Starcke in seiner Charakteristik
Feuerbachs zunächst dessen Stellung zu dieser Grundfrage über das
Verhältnis von Denken und Sein untersucht. Nach einer kurzen Einleitung,
worin die Auffassung der frühern Philosophen, namentlich seit Kant, in
unnötig philosophisch-schwerfälliger Sprache geschildert wird und wobei
Hegel durch allzu formalistisches Festhalten an einzelnen Stellen seiner
Werke sehr zu kurz kommt, folgt eine ausführliche Darstellung des
Entwicklungsgangs der Feuerbachschen "Metaphysik" selbst, wie er sich
aus der Reihenfolge der betreffenden Schriften dieses Philosophen
ergibt. Diese Darstellung ist fleißig und übersichtlich gearbeitet, nur
wie das ganze Buch mit einem keineswegs überall unvermeidlichen Ballast
philosophischer Ausdrucksweise beschwert, der um so störender wirkt, je
weniger sich der Verfasser an die Ausdrucksweise einer und derselben
Schule, oder auch Feuerbachs selbst hält, und je mehr er Ausdrücke der
verschiedensten, namentlich der jetzt grassierenden, sich philosophisch
nennenden Richtungen hinein mengt.

Der Entwicklungsgang Feuerbachs ist der eines - freilich nie
ganz orthodoxen - Hegelianers zum Materialismus hin, eine Entwicklung,
die auf einer bestimmten Stufe einen totalen Bruch mit dem
idealistischen System seines Vorgängers bedingt. Mit unwiderstehlicher
Gewalt drängt sich ihm schließlich die Einsicht auf, daß die Hegelsche
vorweltliche Existenz der "absoluten Idee", die "Präexistenz der
logischen Kategorien", ehe denn die Welt war, weiter nichts ist als ein
phantastischer Überrest des Glaubens an einen außerweltlichen Schöpfer;
daß die stoffliche, sinnlich wahrnehmbare Welt, zu der wir selbst
gehören, das einzig Wirkliche, und daß unser Bewußtsein und Denken, so
übersinnlich es scheint, das Erzeugnis eines stofflichen, körperlichen
Organs, des Gehirns ist. Die Materie ist nicht ein Erzeugnis des
Geistes, sondern der Geist ist selbst nur das höchste Produkt der
Materie. Dies ist natürlich reiner Materialismus. Hier angekommen,
stutzt Feuerbach. Er kann das gewohnheitsmäßige, philosophische
Vorurteil nicht überwinden, das Vorurteil nicht gegen die Sache, sondern
gegen den Namen des Materialismus. Er sagt:"Der Materialismus ist für
mich die Grundlage des Gebäudes des menschlichen Wesens und Wissens;
aber er ist für mich nicht, was er für den Physiologen, den
Naturforscher im engem Sinn, z.B. Moleschott ist, und zwar notwendig von
ihrem Standpunkt und Beruf aus ist, das Gebäude selbst. Rückwärts stimme
ich den Materialisten vollkommen bei, aber nicht vorwärts."

Feuerbach wirft hier den Materialismus, der eine auf einer
bestimmten Auffassung des Verhältnisses von Materie und Geist beruhende
allgemeine Weltanschauung ist, zusammen mit der besondern Form, worin
diese Weltanschauung auf einer bestimmten geschichtlichen Stufe, nämlich
im 18. Jahrhundert, zum Ausdruck kam. Noch mehr, er wirft ihn zusammen
mit der verflachten, vulgarisierten Gestalt, worin der Materialismus des
18. Jahrhunderts heute in den Köpfen von Naturforschern und Ärzten
fortexistiert und in den fünfziger Jahren von Büchner, Vogt und
Moleschott gereisepredigt wurde. Aber wie der Idealismus eine Reihe von
Entwicklungsstufe durchlief, so auch der Materialismus. Mit jeder
epochemachenden Entdeckung schon auf naturwissenschaftlichem Gebiet
mußte er seine Form ändern, und seitdem auch die Geschichte der
materialistischen Behandlung unterworfen, eröffnet sich auch hier eine
neue Bahn der Entwicklung.

Der Materialismus des vorigen Jahrhunderts war vorwiegend
mechanisch, weil von allen Naturwissenschaften damals nur die Mechanik,
und zwar auch nur die der - himmlischen und irdischen - festen Körper,
kurz, die Mechanik der Schwere, zu einem gewissen Abschluß gekommen war.
Die Chemie existierte nur erst in ihrer kindlichen, phlogistischen
Gestalt. Die Biologie lag noch in den Windeln; der pflanzliche und
tierische Organismus war nur im groben untersucht und wurde aus rein
mechanischen Ursachen erklärt; wie dem Descartes das Tier, war den
Materialisten des 18. Jahrhunderts der Mensch eine Maschine. Diese
ausschließliche Anwendung des Maßstabs der Mechanik auf Vorgänge, die
chemischer und organischer Natur sind und bei denen die mechanischen
Gesetze zwar auch gelten, aber von andern, höhern Gesetzen in den
Hintergrund gedrängt werden, bildet die eine spezifische, aber ihrer
Zeit unvermeidliche Beschränktheit des klassischen französischen
Materialismus.

Die zweite spezifische Beschränktheit dieses Materialismus
bestand in seiner Unfähigkeit, die Welt als einen Prozeß, als einen in
einer geschichtlichen Fortbildung begriffenen Stoff aufzufassen. Dies
entsprach dem damaligen Stand der Naturwissenschaft und der damit
zusammenhängenden metaphysischen, d.h. antidialektischen Weise des
Philosophierens. Die Natur, das wußte man, war in ewiger Bewegung
begriffen. Aber diese Bewegung drehte sich nach damaliger Vorstellung
ebenso ewig im Kreise und kam daher nie vom Fleck; sie erzeugte immer
wieder dieselben Ergebnisse. Diese Vorstellung war damals unvermeidlich.
Die Kantsche Theorie von der Entstehung des Sonnensystems war erst
soeben aufgestellt und passierte nur noch als bloßes Kuriosum. Die
Geschichte der Entwicklung der Erde, die Geologie, war noch total
unbekannt, und die Vorstellung, daß die heutigen belebten Naturwesen das
Ergebnis einer langen Entwicklungsreihe vom Einfachen zum Komplizierten
sind, konnte damals wissenschaftlich überhaupt nicht aufgestellt werden.
Die unhistorische Auffassung der Natur war also unvermeidlich. ǁ Man
kann den Philosophen des 18. Jahrhunderts daraus um so weniger einen
Vorwurf machen, als sie sich auch bei Hegel findet. Bei diesem ist die
Natur, als bloße "Entäußerung" der Idee, keiner Entwicklung in der Zeit
fähig, sondern nur einer Ausbreitung ihrer Mannigfaltigkeit im Raum, so
daß sie alle in ihr einbegriffnen Entwicklungsstufen gleichzeitig und
nebeneinander ausstellt und zu ewiger Wiederholung stets derselben
Prozesse verdammt ist. Und diesen Widersinn einer Entwicklung im Raum,
aber außer der Zeit - der Grundbedingung aller Entwicklung - bürdet
Hegel der Natur auf grade zu derselben Zeit, wo die Geologie, die
Embryologie, die pflanzliche und tierische Physiologie und die
organische Chemie ausgebildet wurden und wo überall auf Grundlage dieser
neuen Wissenschaften geniale Vorahnungen der späteren
Entwicklungstheorie auftauchten (z.B. Goethe und Lamarck). Aber das
System erforderte es so, und so mußte die Methode, dem System zulieb,
sich selbst untreu werden. ǁ Dieselbe unhistorische Auffassung galt auch
auf dem Gebiet der Geschichte. Hier hielt der Kampf gegen die Reste des
Mittelalters den Blick befangen. Das Mittelalter galt als einfache
Unterbrechung der Geschichte durch tausendjährige allgemeine Barbarei;
die großen Fortschritte des Mittelalters - die Erweiterung des
europäischen Kulturgebiets, die lebensfähigen großen Nationen, die sich
dort nebeneinander gebildet, endlich die enormen technischen
Fortschritte des 14. und 15. Jahrhunderts ---, alles das sah man nicht.
Damit war aber eine rationelle Einsicht in den großen geschichtlichen
Zusammenhang unmöglich gemacht, und die Geschichte diente höchstens als
eine Sammlung von Beispielen und Illustrationen zum Gebrauch der
Philosophen.

Die vulgarisierenden Hausierer, die in den fünfziger Jahren in
Deutschland in Materialismus machten, kamen in keiner Weise über diese
Schranke ihrer Lehrer hinaus. Alle seitdem gemachten Fortschritte der
Naturwissenschaft dienten ihnen nur als neue Beweisgründe gegen die
Existenz des Weltschöpfers; und in der Tat lag es ganz außerhalb ihres
Geschäfts, die Theorie weiterzuentwickeln. War der Idealismus am Ende
seines Lateins und durch die Revolution von 1848 auf den Tod getroffen,
so erlebte er die Genugtuung, daß der Materialismus momentan noch tiefer
heruntergekommen war. Feuerbach hatte entschieden recht, wenn er die
Verantwortung für diesen Materialismus ablehnte; nur durfte er die Lehre
der Reiseprediger nicht verwechseln mit dem Materialismus überhaupt.

Indes ist hier zweierlei zu bemerken. Erstens war auch zu
Feuerbachs Lebzeiten die Naturwissenschaft noch in jenem heftigen
Gärungsprozeß begriffen, der erst in den letzten fünfzehn Jahren einen
klärenden, relativen Abschluß erhalten hat; es wurde neuer
Erkenntnisstoff in bisher unerhörtem Maß geliefert, aber die Herstellung
des Zusammenhangs und damit der Ordnung in diesem Chaos sich
überstürzender Entdeckungen ist erst ganz neuerdings möglich geworden.
Zwar hat Feuerbach die drei entscheidenden Entdeckungen - die der Zelle,
der Verwandlung der Energie und der nach Darwin benannten
Entwicklungstheorie - noch alle erlebt. Aber wie sollte der einsame
Philosoph auf dem Lande die Wissenschaft hinreichend verfolgen können,
um Entdeckungen vollauf zu würdigen, die die Naturforscher selbst damals
teils noch bestritten, teils nicht hinreichend verstanden? Die Schuld
fällt hier einzig auf die erbärmlichen deutschen Zustände, kraft deren
die Lehrstühle der Philosophie von spintisierenden eklektischen
Flohknackern in Beschlag genommen wurden, während Feuerbach, der sie
alle turmhoch überragte, in einem kleinen Dorf verbauern und versauern
mußte. Es ist also nicht Feuerbachs Schuld, wenn die jetzt möglich
gewordne, alle Einseitigkeiten des französischen Materialismus
entfernende, historische Naturauffassung ihm unzugänglich blieb.

Zweitens aber hat Feuerbach darin ganz recht, daß der bloß
naturwissenschaftliche Materialismus zwar die "Grundlage des Gebäudes
des menschlichen Wissens ist, aber nicht das Gebäude selbst". Denn wir
leben nicht nur in der Natur, sondern auch in der menschlichen
Gesellschaft, und auch diese hat ihre Entwicklungsgeschichte und ihre
Wissenschaft nicht minder als die Natur. Es handelte sich also darum,
die Wissenschaft von der Gesellschaft, d.h. den Inbegriff der
sogenannten historischen und philosophischen Wissenschaften, mit der
materialistischen \textbar{}281\textbar{} Grundlage
in Einklang zu bringen und auf ihr zu rekonstruieren. Dies aber war
Feuerbach nicht vergönnt. Hier blieb er, trotz der "Grundlage", in den
überkommnen idealistischen Banden befangen, und dies erkennt er an mit
den Worten: "Rückwärts stimme ich den Materialisten bei, aber nicht
vorwärts." Wer aber hier, auf dem gesellschaftlichen Gebiet, nicht
"vorwärts "kam, nicht über seinen Standpunkt von 1840 oder 1844 hinaus,
das war Feuerbach selbst, und zwar wiederum hauptsächlich infolge seiner
Verödung, die ihn zwang, Gedanken aus seinem einsamen Kopf zu
produzieren - ihn, der vor allen andern Philosophen auf geselligen
Verkehr veranlagt war - statt im freundlichen und feindlichen
Zusammentreffen mit andern Menschen seines Kalibers. Wie sehr er auf
diesem Gebiet Idealist bleibt, werden wir später im einzelnen sehn.

Hier ist nur noch zu bemerken, daß Starcke den Idealismus
Feuerbachs am unrechten Ort sucht."Feuerbach ist Idealist, er glaubt an
den Fortschritt der Menschheit." (S. 19.) - "Die Grundlage, der Unterbau
des Ganzen, bleibt nichtsdestoweniger der Idealismus. Der Realismus ist
für uns nichts weiter als ein Schutz gegen Irrwege, während wir unsern
idealen Strömungen folgen. Sind nicht Mitleid, Liebe und Begeisterung
für Wahrheit und Recht ideale Mächte?" (S. VIII.)

Erstens heißt hier Idealismus nichts andres als Verfolgung
idealer Ziele. Diese aber haben notwendig zu tun höchstens mit dem
Kantschen Idealismus und seinem "kategorischen Imperativ"; aber selbst
Kant nannte seine Philosophie "transzendentalen Idealismus", keineswegs,
weil es sich darin auch um sittliche Ideale handelt, sondern aus ganz
andren Gründen, wie Starcke sich erinnern wird. Der Aberglaube, daß der
philosophische Idealismus sich um den Glauben an sittliche, d.h.
gesellschaftliche Ideale drehe, ist entstanden außerhalb der
Philosophie, beim deutschen Philister, der die ihm nötigen wenigen
philosophischen Bildungsbrocken in Schillers Gedichten auswendig lernt.
Niemand hat den ohnmächtigen Kantschen "kategorischen Imperativ" ---
ohnmächtig, weil er das Unmögliche fordert, also nie zu etwas Wirklichem
kommt - schärfer kritisiert, niemand die durch Schiller vermittelte
Philisterschwärmerei für unrealisierbare Ideale grausamer verspottet
(siehe z.B. die "Phänomenologie") als grade der vollendete Idealist
Hegel.

Zweitens aber ist es nun einmal nicht zu vermeiden, daß alles,
was einen Menschen bewegt, den Durchgang durch seinen Kopf machen muß ---
sogar Essen und Trinken, das infolge von vermittelst des Kopfs
empfundnem  Hunger und Durst begonnen und infolge von ebenfalls
vermittelst des Kopfs empfundner Sättigung beendigt wird. Die
Einwirkungen der Außenwelt auf den Menschen drücken sich in seinem Kopf
aus, spiegeln sich darin ab als Gefühle, Gedanken, Triebe,
Willensbestimmungen, kurz, als "ideale Strömungen", und werden in dieser
Gestalt zu "idealen Mächten". Wenn nun der Umstand, daß dieser Mensch
überhaupt " idealen Strömungen folgt" und "idealen Mächten" einen
Einfluß auf sich zugesteht - wenn dies ihn zum Idealisten macht, so ist
jeder einigermaßen normal entwickelte Mensch ein geborner Idealist, und
wie kann es da überhaupt noch Materialisten geben?

Drittens hat die Überzeugung, daß die Menschheit, augenblicklich
wenigstens, sich im ganzen und großen in fortschreitender Richtung
bewegt, absolut nichts zu tun mit dem Gegensatz von Materialismus und
Idealismus. Die französischen Materialisten hatten diese Überzeugung in
fast fanatischem Grad, nicht minder die Deisten Voltaire und Rousseau,
und brachten oft genug die größten persönlichen Opfer. Wenn irgend
jemand der "Begeisterung für Wahrheit und Recht" - die Phrase im guten
Sinn genommen - das ganze Leben weihte, so war es z.B. Diderot. Wenn
also Starcke dies alles für Idealismus erklärt, so beweist dies nur, daß
das Wort Materialismus und der ganze Gegensatz beider Richtungen für ihn
hier allen Sinn verloren hat.

Die Tatsache ist, daß Starcke hier dem von der langjährigen
Pfaffenverlästerung her überkommenen Philistervorurteil gegen den Namen
Materialismus eine unverzeihliche Konzession macht - wenn auch
vielleicht unbewußt. Der Philister versteht unter Materialismus Fressen,
Saufen, Augenlust, Fleischeslust und hoffärtiges Wesen, Geldgier, Geiz,
Habsucht, Profitmacherei und Börsenschwindel, kurz alle die schmierigen
Laster, denen er selbst im stillen frönt; und unter Idealismus den
Glauben an Tugend, allgemeine Menschenliebe und überhaupt die "bessere
Welt", womit er vor andern renommiert, woran er selbst aber höchstens
glaubt, so lange er den auf seine gewohnheitsmäßigen "materialistischen"
Exzesse notwendig folgenden Katzenjammer oder Bankerott durchzumachen
pflegt und dazu sein Lieblingslied singt: Was ist der Mensch - halb
Tier, halb Engel.

Im übrigen gibt sich Starcke viel Mühe, Feuerbach gegen die
Angriffe und Lehrsätze der sich heute unter dem Namen Philosophen in
Deutschland breitmachenden Dozenten zu verteidigen. Für Leute, die sich
für diese Nachgeburt der klassischen deutschen Philosophie
interessieren, ist das gewiß wichtig; für Starcke selbst mochte dies
notwendig scheinen. Wir verschonen den Leser damit.

\pagebreak

\hfill\emph{Die Neue Zeit. Jg. 4.}

\hfill\emph{1886. Nr. 5, Mai}

\section{III}


Der wirkliche Idealismus Feuerbachs tritt zutage, sobald wir auf
seine Religionsphilosophie und Ethik kommen. Er will die Religion
keineswegs abschaffen, er will sie vollenden. Die Philosophie selbst
soll aufgehn in Religion. "Die Perioden der Menschheit unterscheiden
sich nur durch religiöse Veränderungen. Nur da geht eine geschichtliche
Bewegung auf den Grund ein, wo sie auf das Herz des Menschen eingeht.
Das Herz ist nicht eine Form der Religion, so daß sie auch im Herzen
sein sollte; es ist das Wesen der Religion." (Zitiert bei Starcke, S.
168.) Religion ist nach Feuerbach das Gefühlsverhältnis, das
Herzensverhältnis zwischen Mensch und Mensch, das bisher in einem
phantastischen Spiegelbild der Wirklichkeit - in der Vermittlung durch
einen oder viele Götter, phantastische Spiegelbilder menschlicher
Eigenschaften - seine Wahrheit suchte, jetzt aber in der Liebe zwischen
Ich und Du sie direkt und ohne Vermittlung findet. Und so wird bei
Feuerbach schließlich die Geschlechtsliebe eine der höchsten, wenn nicht
die höchste Form der Ausübung seiner neuen Religion.

Nun haben Gefühlsverhältnisse zwischen den Menschen, namentlich
auch zwischen beiden Geschlechtern bestanden, solange es Menschen gibt.
Die Geschlechtsliebe speziell hat in den letzten achthundert Jahren eine
Ausbildung erhalten und eine Stellung erobert, die sie während dieser
Zeit zum obligatorischen Drehzapfen aller Poesie gemacht hat. Die
bestehenden positiven Religionen haben sich darauf beschränkt, der
staatlichen Regelung der Geschlechtsliebe, d.h. der Ehegesetzgebung, die
höhere Weihe zu geben, und können morgen sämtlich verschwinden, ohne daß
an der Praxis von Liebe und Freundschaft das Geringste geändert wird.
Wie die christliche Religion denn auch in Frankreich von 1793 bis 1798
faktisch so sehr verschwunden war, daß selbst Napoleon sie nicht ohne
Widerstreben und Schwierigkeit wieder einführen konnte, ohne daß jedoch
während des Zwischenraums das Bedürfnis nach einem Ersatz im Sinn
Feuerbachs hervortrat.

Der Idealismus besteht hier bei Feuerbach darin, daß er die auf
gegenseitiger Neigung beruhenden Verhältnisse der Menschen zueinander,
Geschlechtsliebe, Freundschaft, Mitleid, Aufopferung usw., nicht einfach
als das gelten läßt, was sie ohne Rückerinnerung an eine, auch für ihn
der Vergangenheit angehörige, besondre Religion aus sich selbst sind,
sondern behauptet, sie kämen erst zu ihrer vollen Geltung, sobald man
ihnen eine höhere Weihe gibt durch den Namen Religion. Die Hauptsache
für ihn ist nicht, daß diese rein menschlichen Beziehungen existieren,
sondern daß sie als die neue, wahre Religion aufgefaßt werden. Sie
sollen für voll gelten, erst wenn sie religiös abgestempelt sind.
Religion kommt her von religare und heißt ursprünglich Verbindung. Also
ist jede Verbindung zweier Menschen eine Religion. Solche etymologische
Kunststücke bilden das letzte Auskunftsmittel der idealistischen
Philosophie. Nicht was das Wort nach der geschichtlichen Entwicklung
seines wirklichen Gebrauchs bedeutet, sondern was es der Abstammung nach
bedeuten sollte, das soll gelten. Und so wird die Geschlechtsliebe und
die geschlechtliche Verbindung in eine "Religion" verhimmelt, damit nur
ja nicht das der idealistischen Erinnerung teure Wort Religion aus der
Sprache verschwinde. Grade so sprachen in den vierziger Jahren die
Pariser Reformisten der Louis Blancschen Richtung, die sich ebenfalls
einen Menschen ohne Religion nur als ein Monstrum vorstellen konnten und
uns sagten: Donc, l'athéisme c'est votre religion! \textbar{}Also der
Atheismus ist eure Religion!\textbar{} Wenn Feuerbach die wahre Religion
auf Grundlage einer wesentlich materialistischen Naturanschauung
herstellen will, so heißt das soviel, wie die moderne Chemie als die
wahre Alchimie auffassen. Wenn die Religion ohne ihren Gott bestehen
kann, dann auch die Alchimie ohne ihren Stein der Weisen. Es besteht
übrigens ein sehr enges Band zwischen Alchimie und Religion. Der Stein
der Weisen hat viele gottähnliche Eigenschaften, und die
ägyptisch-griechischen Alchimisten der ersten beiden Jahrhunderte
unserer Zeitrechnung haben bei der Ausbildung der christlichen Doktrin
ihr Händchen mit im Spiel gehabt, wie die bei Kopp und Berthelot
gegebenen Daten beweisen.

Entschieden falsch ist Feuerbachs Behauptung, daß die "Perioden
der Menschheit sich nur durch religiöse Veränderungen unterscheiden". ǁ
Große geschichtliche Wendepunkte sind von religiösen
Veränderungen \emph{begleitet} worden, nur soweit die drei
Weltreligionen in Betracht kommen, die bisher bestanden haben:
Buddhismus, Christentum, Islam.ǁ Die alten naturwüchsig entstandnen
Stammes- und Nationalreligionen waren ǁ nicht propagandistisch und
verloren ǁ alle Widerstandskraft, sobald die Selbständigkeit der Stämme
und Völker gebrochen war; bei den Germanen genügte sogar die einfache
Berührung mit dem verfallenden römischen Weltreich und der von ihm
soeben aufgenommenen, seinem ökonomischen, politischen und ideellen
Zustand angemeßnen christlichen Weltreligion. Erst bei diesen mehr oder
weniger künstlich entstandnen Weltreligionen, namentlich beim
Christentum und Islam, finden wir, daß allgemeinere geschichtliche
Bewegungen ein religiöses Gepräge annehmen, und ǁ selbst auf dem Gebiet
des Christentums ǁ ist das religiöse Gepräge, für Revolutionen von
wirklich universeller Bedeutung, beschränkt auf die ersten Stufen des
Emanzipationskampfs der Bourgeoisie, vom dreizehnten bis zum siebzehnten
Jahrhundert, und erklärt sich nicht, wie Feuerbach meint, aus dem Herzen
des Menschen und seinem Religionsbedürfnis, sondern aus der ganzen
mittelalterlichen Vorgeschichte, die keine andere Form der Ideologie
kannte als eben die Religion und Theologie. Als aber die Bourgeoisie im
18. Jahrhundert hinreichend erstarkt war, um auch ihre eigne, ihrem
Klassenstandpunkt angemeßne Ideologie zu haben, da machte sie ihre große
und endgültige Revolution, die französische, unter dem ausschließlichen
Appell an juristische und politische Ideen durch und kümmerte sich um
die Religion nur so weit, als diese ihr im Wege stand; es fiel ihr aber
nicht ein, eine neue Religion an die Stelle der alten zu setzen; ǁ man
weiß, wie Robespierre damit scheiterte.ǁ

Die Möglichkeit rein menschlicher Empfindung im Verkehr mit
andern Menschen wird uns heutzutage schon genug verkümmert durch die auf
Klassengegensatz und Klassenherrschaft gegründete Gesellschaft, in der
wir uns bewegen müssen: Wir haben keinen Grund, sie uns selbst noch mehr
zu verkümmern, indem wir diese Empfindungen in eine Religion verhimmeln.
Und ebenso wird das Verständnis der geschichtlichen großen Klassenkämpfe
von der landläufigen Geschichtschreibung, namentlich in Deutschland,
schon hinreichend verdunkelt, auch ohne daß wir nötig hätten, es durch
Verwandlung dieser Kampfesgeschichte in einen bloßen Anhang der
Kirchengeschichte uns vollends unmöglich zu machen. Schon hier zeigt
sich, wie weit wir uns heute von Feuerbach entfernt haben. Seine
"schönsten Stellen", zur Feier dieser neuen Liebesreligion, sind heute
gar nicht mehr lesbar.

Die einzige Religion, die Feuerbach ernstlich untersucht, ist
das Christentum, die Weltreligion des Abendlands, die auf den
Monotheismus gegründet ist. Er weist nach, daß der christliche Gott nur
der phantastische Reflex, das Spiegelbild des Menschen ist. Nun aber ist
dieser Gott selbst das Produkt eines langwierigen Abstraktionsprozesses,
die konzentrierte Quintessenz der früheren vielen Stammes- und
Nationalgötter. Und dementsprechend ist auch der Mensch, dessen Abbild
jener Gott ist, nicht ein wirklicher Mensch, sondern ebenfalls die
Quintessenz der vielen wirklichen Menschen, der abstrakte Mensch, also
selbst wieder ein Gedankenbild. Derselbe Feuerbach, der auf jeder Seite
Sinnlichkeit, Versenkung ins Konkrete, in die Wirklichkeit predigt, er
wird durch und durch abstrakt, sowie er auf einen weiteren als den bloß
geschlechtlichen Verkehr zwischen den Menschen zu sprechen kommt.

Dieser Verkehr bietet ihm nur eine Seite: die Moral. Und hier
frappiert uns wieder die erstaunliche Armut Feuerbachs verglichen mit
Hegel. Dessen Ethik oder Lehre von der Sittlichkeit ist die
Rechtsphilosophie und umfaßt: 1. das abstrakte Recht, 2. die Moralität,
3. die Sittlichkeit, unter welcher wieder zusammengefaßt sind: die
Familie, die bürgerliche Gesellschaft, der Staat. So idealistisch die
Form, so realistisch ist hier der Inhalt. Das ganze Gebiet des Rechts,
der Ökonomie, der Politik ist neben der Moral hier mit einbegriffen. Bei
Feuerbach grade umgekehrt. Er ist der Form nach realistisch, er geht vom
Menschen aus; aber von der Welt, worin dieser Mensch lebt, ist absolut
nicht die Rede, und so bleibt dieser Mensch stets derselbe abstrakte
Mensch, der in der Religionsphilosophie das Wort führte. Dieser Mensch
ist eben nicht aus dem Mutterleib geboren, er hat sich aus dem Gott der
monotheistischen Religionen entpuppt, er lebt daher auch nicht in einer
wirklichen, geschichtlich entstandenen und geschichtlich bestimmten
Welt; er verkehrt zwar mit andern Menschen, aber jeder andere ist ebenso
abstrakt wie er selbst. In der Religionsphilosophie hatten wir doch noch
Mann und Weib, aber in der Ethik verschwindet auch dieser letzte
Unterschied. Allerdings kommen bei Feuerbach in weiten Zwischenräumen
Sätze vor wie:"In einem Palast denkt man anders als in einer Hütte. " ---
"Wo du vor Hunger, vor Elend keinen Stoff im Leibe hast, da hast du auch
in deinem Kopfe, in deinem Sinne und Herzen keinen Stoff zur Moral." ---
"Die Politik muß unsere Religion werden" usw.Aber mit diesen Sätzen weiß
Feuerbach absolut nichts anzufangen, sie bleiben pure Redensarten, und
selbst Starcke muß eingestehn, daß die Politik für Feuerbach eine
unpassierbare Grenze war und die"Gesellschaftslehre, die Soziologie für
ihn eine terra incognita \textbar{}ein unbekanntes Land\textbar{}".

Ebenso flach erscheint er gegenüber Hegel in der Behandlung des
Gegensatzes von Gut und Böse."Man glaubt etwas sehr Großes zu sagen ---
heißt es bei Hegel - wenn man sagt: Der Mensch ist von Natur gut; aber
man vergißt, daß man etwas weit Größeres sagt mit den Worten: Der Mensch
ist von Natur böse." Bei Hegel ist das Böse die Form, worin die
Triebkraft der geschichtlichen Entwicklung sich darstellt. Und zwar
liegt hierin der doppelte Sinn, daß einerseits jeder neue Fortschritt
notwendig auftritt als Frevel gegen ein Heiliges, als Rebellion gegen
die alten, absterbenden, aber durch die Gewohnheit geheiligten Zustände,
und andrerseits, daß seit dem Aufkommen der Klassengegensätze es grade
die schlechten Leidenschaften der Menschen sind, Habgier und
Herrschsucht, die zu Hebeln der geschichtlichen Entwicklung werden,
wovon z.B. die Geschichte des Feudalismus und der Bourgeoisie ein
einziger fortlaufender Beweis ist. Aber die historische Rolle des
moralisch Bösen zu untersuchen, fällt Feuerbach nicht ein. Die
Geschichte ist ihm überhaupt ein ungemütliches, unheimliches Feld. Sogar
sein Ausspruch: "Der Mensch, der ursprünglich aus der Natur entsprang,
war auch nur ein reines Naturwesen, kein Mensch. Der Mensch ist ein
Produkt des Menschen, der Kultur, der Geschichte" -selbst dieser
Ausspruch bleibt bei ihm durchaus unfruchtbar.

Was uns Feuerbach über Moral mitteilt, kann hiernach nur äußerst
mager sein. Der Glückseligkeitstrieb ist dem Menschen eingeboren und muß
daher die Grundlage aller Moral bilden. Aber der Glückseligkeitstrieb
erfährt eine doppelte Korrektur. Erstens durch die natürlichen Folgen
unsrer Handlungen: Auf den Rausch folgt der Katzenjammer, auf den
gewohnheitsmäßigen Exzeß die Krankheit. Zweitens durch ihre
gesellschaftlichen Folgen: Respektieren wir nicht den gleichen
Glückseligkeitstrieb der andern, so wehren sie sich und stören unsern
eignen Glückseligkeitstrieb. Hieraus folgt, daß wir, um unsern Trieb zu
befriedigen, die Folgen unsrer Handlungen richtig abzuschätzen imstande
sein und andrerseits die Gleichberechtigung des entsprechenden Triebs
bei andern gelten lassen müssen. Rationelle Selbstbeschränkung in
Beziehung auf uns selbst und Liebe - immer wieder Liebe! - im Verkehr
mit andern sind also die Grundregeln der Feuerbachschen Moral, aus denen
alle andern sich ableiten. Und weder die geistvollsten Ausführungen
Feuerbachs noch die stärksten Lobsprüche Starckes können die Dünnheit
und Plattheit dieser paar Sätze verdecken.

Der Glückseligkeitstrieb befriedigt sich nur sehr ausnahmsweise
und keineswegs zu seinem und andrer Leute Vorteil durch die
Beschäftigung eines Menschen mit ihm selbst. Sondern er erfordert
Beschäftigung mit der Außenwelt, Mittel der Befriedigung, also Nahrung,
ein Individuum des andern Geschlechts, Bücher, Unterhaltung, Debatte,
Tätigkeit, Gegenstände der Vernutzung und Verarbeitung. Die
Feuerbachsche Moral setzt entweder voraus, daß diese Mittel und
Gegenstände der Befriedigung jedem Menschen ohne weiteres gegeben sind,
oder aber sie gibt ihm nur unanwendbare gute Lehren, ist also keinen
Schuß Pulver wert für die Leute, denen diese Mittel fehlen. Und das
erklärt Feuerbach selbst in dürren Worten: "In einem Palast denkt man
anders als in einer Hütte." "Wo du vor Hunger, vor Elend keinen Stoff im
Leibe hast, da hast du auch in deinem Kopfe, in deinem Sinne und Herzen
keinen Stoff zur Moral."

Steht es etwa besser mit der Gleichberechtigung des
Glückseligkeitstriebs andrer? Feuerbach stellt diese Forderung absolut
hin, als gültig für alle Zeiten und Umstände. Aber seit wann gilt sie?
War im Altertum zwischen Sklaven und Herren, im Mittelalter zwischen
Leibeignen und Baronen je die Rede von Gleichberechtigung des
Glückseligkeitstriebs? Wurde nicht der Glückseligkeitstrieb der
unterdrückten Klasse rücksichtslos und "von Rechts wegen" dem der
herrschenden zum Opfer gebracht? - Ja, das war auch unmoralisch, jetzt
aber ist die Gleichberechtigung anerkannt. - Anerkannt in der Phrase,
seitdem und sintemal die Bourgeoisie in ihrem Kampf gegen die Feudalität
und in der Ausbildung der kapitalistischen Produktion gezwungen war,
alle ständischen, d.h. persönlichen Privilegien abzuschaffen und zuerst
die privatrechtliche, dann auch allmählich die staatsrechtliche,
juristische Gleichberechtigung der Person einzuführen. Aber der
Glückseligkeitstrieb lebt nur zum geringsten Teil von ideellen Rechten
und zum allergrößten von materiellen Mitteln, und da sorgt die
kapitalistische Produktion dafür, daß der großen Mehrzahl der
gleichberechtigten Personen nur das zum knappen Leben Notwendige
zufällt, respektiert also die Gleichberechtigung des
Glückseligkeitstriebs der Mehrzahl kaum, wenn überhaupt, besser, als die
Sklaverei oder die Leibeigenschaft dies tat. Und steht es besser in
betreff der geistigen Mittel der Glückseligkeit, der Bildungsmittel? Ist
nicht selbst "der Schulmeister von Sadowa" eine mythische Person?

Noch mehr. Nach der Feuerbachschen Moraltheorie ist die
Fondsbörse der höchste Tempel der Sittlichkeit - vorausgesetzt nur, daß
man stets richtig spekuliert. Wenn mein Glückseligkeitstrieb mich auf
die Börse führt und ich dort die Folgen meiner Handlungen so richtig
erwäge, daß sie mir  nur Annehmlichkeit und keinen Nachteil bringen,
d.h. daß ich stets gewinne, so ist Feuerbachs Vorschrift erfüllt. Auch
greife ich dadurch nicht in den gleichen Glückseligkeitstrieb eines
andern ein, denn der andre ist ebenso freiwillig an die Börse gegangen
wie ich, ist beim Abschluß des Spekulationsgeschäfts mit mir ebensogut
seinem Glückseligkeitstrieb gefolgt wie ich dem meinigen. Und verliert
er sein Geld, so beweist sich eben dadurch seine Handlung, weil schlecht
berechnet, als unsittlich, und indem ich an ihm die verdiente Strafe
vollstrecke, kann ich mich sogar als moderner Rhadamanthus stolz in die
Brust werfen. Auch die Liebe herrscht an der Börse, insoweit sie nicht
bloß sentimentale Phrase ist, denn jeder findet im andern die
Befriedigung seines Glückseligkeitstriebs, und das ist ja, was die Liebe
leisten soll und worin sie praktisch sich betätigt. Und wenn ich da in
richtiger Voraussicht der Folgen meiner Operationen, also mit Erfolg
spiele, so erfülle ich alle die strengsten Forderungen der
Feuerbachschen Moral und werde ein reicher Mann obendrein. Mit andern
Worten, Feuerbachs Moral ist auf die heutige kapitalistische
Gesellschaft zugeschnitten, so wenig er selbst das wollen oder ahnen
mag.

Aber die Liebe! - Ja, die Liebe ist überall und immer der
Zaubergott, der bei Feuerbach über alle Schwierigkeiten des praktischen
Lebens hinweghelfen soll - und das in einer Gesellschaft, die in Klassen
mit diametral entgegengesetzten Interessen gespalten ist. Damit ist denn
der letzte Rest ihres revolutionären Charakters aus der Philosophie
verschwunden, und es bleibt nur die alte Leier: Liebet euch
untereinander, fallt euch in die Arme ohne Unterschied des Geschlechts
und des Standes - allgemeiner Versöhnungsdusel!

Kurz und gut. Es geht der Feuerbachschen Moraltheorie wie allen
ihren Vorgängerinnen. Sie ist auf alle Zeiten, alle Völker, alle
Zustände zugeschnitten, und eben deswegen ist sie nie und nirgends
anwendbar und bleibt der wirklichen Welt gegenüber ebenso ohnmächtig wie
Kants kategorischer Imperativ. In Wirklichkeit hat jede Klasse, sogar
jede Berufsart ihre eigne Moral und bricht auch diese, wo sie es
ungestraft tun kann, und die Liebe, die alles einen soll, kommt zu Tag
in Kriegen, Streitigkeiten, Prozessen, häuslichem Krakeel, Ehescheidung
und möglichster Ausbeutung der einen durch die andern.

Wie aber war es möglich, daß der gewaltige, durch Feuerbach
gegebene Anstoß für ihn selbst so unfruchtbar auslief? Einfach dadurch,
daß Feuerbach aus dem ihm selbst tödlich verhaßten Reich der
Abstraktionen den Weg nicht finden kann zur lebendigen Wirklichkeit. Er
klammert sich gewaltsam an die Natur und den Menschen; aber Natur und
Mensch bleiben bei ihm bloß Worte. Weder von der wirklichen Natur noch
von den wirklichen Menschen weiß er uns etwas Bestimmtes zu sagen. Vom
Feuerbachschen abstrakten Menschen kommt man aber nur zu den wirklichen
lebendigen Menschen, wenn man sie in der Geschichte handelnd betrachtet.
Und dagegen sträubte sich Feuerbach, und daher bedeutete das Jahr 1848,
das er nicht begriff, für ihn nur den endgültigen Bruch mit der
wirklichen Welt, den Rückzug in die Einsamkeit. Die Schuld hieran tragen
wiederum hauptsächlich die deutschen Verhältnisse, die ihn elend
verkommen ließen.

Aber der Schritt, den Feuerbach nicht tat, mußte dennoch getan
werden; der Kultus des abstrakten Menschen, der den Kern der
Feuerbachschen neuen Religion bildete, mußte ersetzt werden durch die
Wissenschaft von den wirklichen Menschen und ihrer geschichtlichen
Entwicklung. Diese Fortentwicklung des Feuerbachschen Standpunkts über
Feuerbach hinaus wurde eröffnet 1845 durch Marx in der "Heiligen
Familie".

\pagebreak

\section{IV}


Strauß, Bauer, Stirner, Feuerbach, das waren die Ausläufer der
Hegelschen Philosophie, soweit sie den philosophischen Boden nicht
verließen. Strauß hat, nach dem "Leben Jesu" ǁ und der "Dogmatik" ǁ, nur
noch philosophische und kirchengeschichtliche Belletristik à la Renan
getrieben; Bauer hat nur auf dem Gebiet der Entstehungsgeschichte des
Christentums etwas geleistet, aber hier auch Bedeutendes; Stirner blieb
ein Kuriosum, selbst nachdem Bakunin ihn mit Proudhon verquickt und
diese Verquickung "Anarchismus" getauft hatte; Feuerbach allein war
bedeutend als Philosoph. Aber nicht nur blieb die Philosophie, die
angeblich über allen besondern Wissenschaften schwebende, sie
zusammenfassende wissenschaftswissenschaft, für ihn eine
unüberschreitbare Schranke, ein unantastbar Heiliges; er blieb auch als
Philosoph auf halbem Wege stehen, war unten Materialist, oben Idealist;
er wurde mit Hegel nicht kritisch fertig, sondern warf ihn als
unbrauchbar einfach beiseite, während er selbst, gegenüber dem
enzyklopädischen Reichtum des Hegelschen Systems, nichts Positives
fertigbrachte als eine schwülstige Liebesreligion und eine magere,
ohnmächtige Moral.

Aus der Auflösung der Hegelschen Schule ging aber noch eine
andere Richtung hervor, die einzige, die wirklich Früchte getragen hat,
und diese Richtung knüpft sich wesentlich an den Namen
Marx \textsuperscript{(1)}.

Die Trennung von der Hegelschen Philosophie erfolgte auch hier
durch die Rückkehr zum materialistischen Standpunkt. Das heißt, man
entschloß sich, die wirkliche Welt - Natur und Geschichte - so
aufzufassen, wie sie sich selbst einem jeden gibt, der ohne vorgefaßte
idealistische Schrullen an sie herantritt; man entschloß sich, jede
idealistische Schrulle unbarmherzig zum Opfer zu bringen, die sich mit
den in ihrem eignen Zusammenhang, und in keinem phantastischen,
aufgefaßten Tatsachen nicht in Einklang bringen ließ. Und weiter heißt
Materialismus überhaupt nichts. Nur daß hier zum erstenmal mit der
materialistischen Weltanschauung wirklich Ernst gemacht, daß sie auf
allen in Frage kommenden Gebieten des Wissens - wenigstens in den
Grundzügen - konsequent durchgeführt wurde.

Hegel wurde nicht einfach abseits gelegt; man knüpfte im
Gegenteil an an seine oben entwickelte revolutionäre Seite, an die
dialektische Methode. Aber diese Methode war in ihrer Hegelschen Form
unbrauchbar. Bei Hegel ist die Dialektik die Selbstentwicklung des
Begriffs. Der absolute Begriff ist nicht nur von Ewigkeit - unbekannt
wo? - vorhanden, er ist auch die eigentliche lebendige Seele der ganzen
bestehenden Welt. Er entwickelt sich zu sich selbst durch alle die
Vorstufen, die in der "Logik" des breiteren abgehandelt und die alle in
ihm eingeschlossen sind; dann "entäußert" er sich, indem er sich in die
Natur verwandelt, wo er ohne Bewußtsein seiner selbst, verkleidet als
Naturnotwendigkeit eine neue Entwicklung durchmacht und zuletzt im
Menschen wieder zum Selbstbewußtsein kommt; dies Selbstbewußtsein
arbeitet sich nun in der Geschichte wieder aus dem Rohen heraus, bis
endlich der absolute Begriff wieder vollständig zu sich selbst kommt in
der Hegelschen Philosophie. Bei Hegel ist also die in der Natur und
Geschichte zutage tretende dialektische Entwicklung, d.h. der
ursächliche Zusammenhang des, durch alle Zickzackbewegungen und
momentanen Rückschritte hindurch, sich durchsetzenden Fortschreitens vom
Niedern zum Höhern, nur der Abklatsch der von Ewigkeit her, man weiß
nicht wo, aber jedenfalls unabhängig von jedem denkenden Menschenhirn
vor sich gehenden Selbstbewegung des Begriffs. Diese ideologische
Verkehrung galt es zu beseitigen. Wir faßten die Begriffe unsres Kopfs
wieder materialistisch als die
Abbilder der wirklichen Dinge, statt die wirklichen Dinge als Abbilder
dieser oder jener Stufe des absoluten Begriffs. Damit reduzierte sich
die Dialektik auf die Wissenschaft von den allgemeinen Gesetzen der
Bewegung, sowohl der äußern Welt wie des menschlichen Denkens - zwei
Reihen von Gesetzen, die der Sache nach identisch, dem Ausdruck nach
aber insofern verschieden sind, als der menschliche Kopf sie mit
Bewußtsein anwenden kann, während sie in der Natur und bis jetzt auch
großenteils in der Menschengeschichte sich in unbewußter Weise, in der
Form der äußern Notwendigkeit, inmitten einer endlosen Reihe scheinbarer
Zufälligkeiten durchsetzen. Damit aber wurde die Begriffsdialektik
selbst nur der bewußte Reflex der dialektischen Bewegung der wirklichen
Welt, und damit wurde die Hegelsche Dialektik auf den Kopf, oder
vielmehr vom Kopf, auf dem sie stand, wieder auf die Füße gestellt. Und
diese materialistische Dialektik, die seit Jahren unser bestes
Arbeitsmittel und unsere schärfste Waffe war, wurde merkwürdigerweise
nicht nur von uns, sondern außerdem noch, unabhängig von uns und selbst
von Hegel, wieder entdeckt von einem deutschen Arbeiter, Josef
Dietzgen.

Hiermit war aber die revolutionäre Seite der Hegelschen
Philosophie wieder aufgenommen und gleichzeitig von den idealistischen
Verbrämungen befreit, die bei Hegel ihre konsequente Durchführung
verhindert hatten. Der große Grundgedanke, daß die Welt nicht als ein
Komplex von fertigen Dingen zu fassen ist, sondern als ein Komplex
von \emph{Prozessen}, worin die scheinbar stabilen Dinge nicht minder
wie ihre Gedankenabbilder in unserm Kopf, die Begriffe, eine
ununterbrochene Veränderung des Werdens und Vergehens durchmachen, in
der bei aller scheinbaren Zufälligkeit und trotz aller momentanen
Rückläufigkeit schließlich eine fortschreitende Entwicklung sich
durchsetzt - dieser große Grundgedanke ist, namentlich seit Hegel, so
sehr in das gewöhnliche Bewußtsein übergegangen, daß er in dieser
Allgemeinheit wohl kaum noch Widerspruch findet. Aber ihn in der Phrase
anerkennen und ihn in der Wirklichkeit im einzelnen auf jedem zur
Untersuchung kommenden Gebiet durchführen, ist zweierlei. Geht man aber
bei der Untersuchung stets von diesem Gesichtspunkt aus, so hört die
Forderung endgültiger Lösungen und ewiger Wahrheiten ein für allemal
auf; man ist sich der notwendigen Beschränktheit aller gewonnenen
Erkenntnis stets bewußt, ihrer Bedingtheit durch die Umstände, unter
denen sie gewonnen wurde; aber man läßt sich auch nicht mehr imponieren
durch die der noch stets landläufigen alten Metaphysik unüberwindlichen
Gegensätze von Wahr und Falsch,
Gut und Schlecht, Identisch und Verschieden, Notwendig und Zufällig; man
weiß, daß diese Gegensätze nur relative Gültigkeit haben, daß das jetzt
für wahr Erkannte seine verborgene, später hervortretende falsche Seite
ebensogut hat wie das jetzt als falsch Erkannte seine wahre Seite, kraft
deren es früher für wahr gelten konnte; daß das behauptete Notwendige
sich aus lauter Zufälligkeiten zusammensetzt und das angeblich Zufällige
die Form ist, hinter der die Notwendigkeit sich birgt - und so weiter.

Die alte Untersuchungs- und Denkmethode, die Hegel die
"metaphysische" nennt, die sich vorzugsweise mit Untersuchung der Dinge
als gegebener fester Bestände beschäftigte und deren Reste noch stark in
den Köpfen spuken, hatte ihrerzeit eine große geschichtliche
Berechtigung. Die Dinge mußten erst untersucht werden, ehe die Prozesse
untersucht werden konnten. Man mußte erst wissen, was ein beliebiges
Ding war, ehe man die an ihm vorgehenden Veränderungen wahrnehmen
konnte. Und so war es in der Naturwissenschaft. Die alte Metaphysik, die
die Dinge als fertige hinnahm, entstand aus einer Naturwissenschaft, die
die toten und lebendigen Dinge als fertige untersuchte. Als aber diese
Untersuchung so weit gediehen war, daß der entscheidende Fortschritt
möglich wurde, der Übergang zur systematischen Untersuchung der mit
diesen Dingen in der Natur selbst vorgehenden Veränderungen, da schlug
auch auf philosophischem Gebiet die Sterbestunde der alten Metaphysik.
Und in der Tat, wenn die Naturwissenschaft bis Ende des letzten
Jahrhunderts vorwiegend \emph{sammelnde }Wissenschaft, Wissenschaft von
fertigen Dingen war, so ist sie in unserm Jahrhundert
wesentlich \emph{ordnende} Wissenschaft,Wissenschaft von den Vorgängen,
vom Ursprung und der Entwicklung dieser Dinge und vom Zusammenhang, der
diese Naturvorgänge zu einem großen Ganzen verknüpft. Die Physiologie,
die die Vorgänge im pflanzlichen und tierischen Organismus untersucht,
die Embryologie, die die Entwicklung des einzelnen Organismus vom Keim
bis zur Reife behandelt, die Geologie, die die allmähliche Bildung der
Erdoberfläche verfolgt, sie alle sind Kinder unseres Jahrhunderts.

Vor allem sind es aber drei große Entdeckungen, die unsere
Kenntnis vom Zusammenhang der Naturprozesse mit Riesenschritten
vorangetrieben haben: Erstens die Entdeckung der Zelle als der Einheit,
aus deren Vervielfältigung ǁ und Differenzierung ǁ der ganze pflanzliche
und tierische Körper sich entwickelt, so daß nicht nur die Entwicklung
und das Wachstum aller höheren Organismen als nach einem einzigen
allgemeinen Gesetz vor sich gehend erkannt, ǁ sondern auch in der
Veränderungsfähigkeit der Zelle
der Weg gezeigt ist, auf dem
Organismen ihre Art verändern und damit eine mehr als individuelle
Entwicklung durchmachen können. - Zweitens ǁ die Verwandlung der
Energie, die uns alle zunächst in der anorganischen Natur wirksamen
sogenannten Kräfte, die mechanische Kraft und ihre Ergänzung, die
sogenannte potentielle Energie, Wärme, Strahlung (Licht, resp.
strahlende Wärme), Elektrizität, Magnetismus, chemische Energie, als
verschiedene Erscheinungsformen der universellen Bewegung nachgewiesen
hat, die in bestimmten Maßverhältnissen die eine in die andere übergehn,
so daß für die Menge der einen, die verschwindet, eine bestimmte Menge
einer andern wiedererscheint und so daß die ganze Bewegung der Natur
sich auf diesen unaufhörlichen Prozeß der Verwandlung aus einer Form in
die andre reduziert. - Endlich der zuerst von Darwin im Zusammenhang
entwickelte Nachweis, daß der heute uns umgebende Bestand organischer
Naturprodukte, die Menschen eingeschlossen, das Erzeugnis eines langen
Entwicklungsprozesses aus wenigen ursprünglich einzelligen Keimen ist
und diese wieder aus, auf chemischem Weg entstandenem, Protoplasma oder
Eiweiß hervorgegangen sind.

Dank diesen drei großen Entdeckungen und den übrigen gewaltigen
Fortschritten der Naturwissenschaft sind wir jetzt so weit, den
Zusammenhang zwischen den Vorgängen in der Natur nicht nur auf den
einzelnen Gebieten, sondern auch den der einzelnen Gebiete unter sich im
ganzen und großen nachweisen und so ein übersichtliches Bild des
Naturzusammenhangs in annähernd systematischer Form, vermittelst der
durch die empirische Naturwissenschaft selbst gelieferten Tatsachen
darstellen zu können. Dies Gesamtbild zu liefern, war früher die Aufgabe
der sogenannten Naturphilosophie. Sie konnte dies nur, indem sie die
noch unbekannten wirklichen Zusammenhänge durch ideelle, phantastische
ersetzte, die fehlenden Tatsachen durch Gedankenbilder ergänzte, die
wirklichen Lücken in der bloßen Einbildung ausfüllte. Sie hat bei diesem
Verfahren manche geniale Gedanken gehabt, manche spätern Entdeckungen
vorausgeahnt, aber auch beträchtlichen Unsinn zutage gefördert, wie das
nicht anders möglich war. Heute, wo man die Resultate der Naturforschung
nur dialektisch, d.h. im Sinn ihres eignen Zusammenhangs aufzufassen
braucht, um zu einem für unsere Zeit genügenden "System der Natur" zu
kommen, wo der dialektische Charakter dieses Zusammenhangs sich sogar
den metaphysisch geschulten Köpfen der Naturforscher gegen ihren Willen
aufzwingt, heute ist die Naturphilosophie endgültig beseitigt. Jeder
Versuch ihrer Wiederbelebung wäre nicht nur überflüssig, \emph{er wäre
ein Rückschritt}.

Was aber von der Natur gilt, die hiermit auch als ein
geschichtlicher Entwicklungsprozeß
erkannt ist, das gilt auch von der Geschichte der Gesellschaft in allen
ihren Zweigen und von der Gesamtheit aller der Wissenschaften, die sich
mit menschlichen (und göttlichen) Dingen beschäftigen. Auch hier hat die
Philosophie der Geschichte, des Rechts, der Religion usw. darin
bestanden, daß an die Stelle des in den Ereignissen nachzuweisenden
wirklichen Zusammenhangs ein im Kopf des Philosophen gemachter gesetzt
wurde, daß die Geschichte im ganzen wie in ihren einzelnen Teilstücken
gefaßt wurde als die allmähliche Verwirklichung von Ideen, und zwar
natürlich immer nur der Lieblingsideen des Philosophen selbst. Die
Geschichte arbeitete hiernach unbewußt, aber mit Notwendigkeit, auf ein
gewisses, von vornherein feststehendes ideelles Ziel los, wie z.B. bei
Hegel auf die Verwirklichung seiner absoluten Idee, und die
unverrückbare Richtung auf diese absolute Idee bildete den Innern
Zusammenhang in den geschichtlichen Ereignissen. An die Stelle des
wirklichen, noch unbekannten Zusammenhangs setzte man somit eine neue ---
unbewußte oder allmählich zum Bewußtsein kommende - mysteriöse
Vorsehung. Hier galt es also, ganz wie auf dem Gebiet der Natur, diese
gemachten künstlichen Zusammenhänge zu beseitigen durch die Auffindung
der wirklichen; eine Aufgabe, die schließlich darauf hinausläuft, die
allgemeinen Bewegungsgesetze zu entdecken, die sich in der Geschichte
der menschlichen Gesellschaft als herrschende durchsetzen.

Nun aber erweist sich die Entwicklungsgeschichte der
Gesellschaft in einem Punkt als wesentlich verschiedenartig von der der
Natur. In der Natur sind es - soweit wir die Rückwirkung der Menschen
auf die Natur außer acht lassen - lauter bewußtlose blinde Agenzien, die
aufeinander einwirken und in deren Wechselspiel das allgemeine Gesetz
zur Geltung kommt. Von allem, was geschieht - weder von den zahllosen
scheinbaren Zufälligkeiten, die auf der Oberfläche sichtbar werden, noch
von den schließlichen, die Gesetzmäßigkeit innerhalb dieser
Zufälligkeiten bewährenden Resultaten ---, geschieht nichts als gewollter
bewußter Zweck. Dagegen in der Geschichte der Gesellschaft sind die
Handelnden lauter mit Bewußtsein begabte, mit Überlegung oder
Leidenschaft handelnde, auf bestimmte Zwecke hinarbeitende Menschen;
nichts geschieht ohne bewußte Absicht, ohne gewolltes Ziel. Aber dieser
Unterschied, so wichtig er für die geschichtliche Untersuchung
namentlich einzelner Epochen und Begebenheiten ist, kann nichts ändern
an der Tatsache, daß der Lauf der Geschichte durch innere allgemeine
Gesetze beherrscht wird. Denn auch hier herrscht auf der Oberfläche,
trotz der bewußt gewollten Ziele aller einzelnen, im ganzen und großen
scheinbar der Zufall. Nur selten geschieht das Gewollte, in
den \textbar{}meisten Fällen
durchkreuzen und widerstreiten sich die vielen gewollten Zwecke oder
sind diese Zwecke selbst von vornherein undurchführbar oder die Mittel
unzureichend. So führen die Zusammenstöße der zahllosen Einzelwillen und
Einzelhandlungen auf geschichtlichem Gebiet einen Zustand herbei, der
ganz dem in der bewußtlosen Natur herrschenden analog ist. Die Zwecke
der Handlungen sind gewollt, aber die Resultate, die wirklich aus den
Handlungen folgen, sind nicht gewollt, oder soweit sie dem gewollten
Zweck zunächst doch zu entsprechen scheinen, haben sie schließlich ganz
andre als die gewollten Folgen. Die geschichtlichen Ereignisse
erscheinen so im ganzen und großen ebenfalls als von der Zufälligkeit
beherrscht. Wo aber auf der Oberfläche der Zufall sein Spiel treibt, da
wird er stets durch innre verborgne Gesetze beherrscht, und es kommt nur
darauf an, diese Gesetze zu entdecken.

Die Menschen machen ihre Geschichte, wie diese auch immer
ausfalle, indem jeder seine eignen, bewußt gewollten Zwecke verfolgt,
und die Resultante dieser vielen in verschiedenen Richtungen agierenden
Willen und ihrer mannigfachen Einwirkung auf die Außenwelt ist eben die
Geschichte. Es kommt also auch darauf an, was die vielen einzelnen
wollen. Der Wille wird bestimmt durch Leidenschaft oder Überlegung. Aber
die Hebel, die wieder die Leidenschaft oder die Überlegung unmittelbar
bestimmen, sind sehr verschiedener Art. Teils können es äußere
Gegenstände sein, teils ideelle Beweggründe, Ehrgeiz, "Begeisterung für
Wahrheit und Recht", persönlicher Haß oder auch rein individuelle
Schrullen aller Art. Aber einerseits haben wir gesehn, daß die in der
Geschichte tätigen vielen Einzelwillen meist ganz andre als die
gewollten - oft geradezu die entgegengesetzten - Resultate
hervorbringen, ihre Beweggründe also ebenfalls für das Gesamtergebnis
nur von untergeordneter Bedeutung sind. Andrerseits fragt es sich
weiter, welche treibenden Kräfte wieder hinter diesen Beweggründen
stehn, welche geschichtlichen Ursachen es sind, die sich in den Köpfen
der Handelnden zu solchen Beweggründen umformen?

Diese Frage hat sich der alte Materialismus nie vorgelegt. Seine
Geschichtsauffassung, soweit er überhaupt eine hat, ist daher auch
wesentlich pragmatisch, beurteilt alles nach den Motiven der Handlung,
teilt die geschichtlich handelnden Menschen in edle und unedle und
findet dann in der Regel, daß die edlen die Geprellten und die unedlen
die Sieger sind, woraus dann folgt für den alten Materialismus, daß beim
Geschichtsstudium nicht viel Erbauliches herauskommt, und für uns, daß
auf dem geschichtlichen Gebiet der alte Materialismus sich selbst untreu
wird, weil er die dort wirksamen ideellen Triebkräfte als letzte
Ursachen hinnimmt, statt zu
unter- suchen, was denn hinter
ihnen steht, was die Triebkräfte dieser Triebkräfte sind. Nicht darin
liegt die Inkonsequenz, daß \emph{ideelle} Triebkräfte anerkannt werden,
sondern darin, daß von diesen nicht weiter zurückgegangen wird auf ihre
bewegenden Ursachen. Die Geschichtsphilosophie dagegen, wie sie
namentlich durch Hegel vertreten wird, erkennt an, daß die ostensiblen
und auch die wirklich tätigen Beweggründe der geschichtlich handelnden
Menschen keineswegs die letzten Ursachen der geschichtlichen Ereignisse
sind, daß hinter diesen Beweggründen andere bewegende Mächte stehn, die
es zu erforschen gilt; aber sie sucht diese Mächte nicht in der
Geschichte selbst auf, sie importiert sie vielmehr von außen, aus der
philosophischen Ideologie, in die Geschichte hinein. Statt die
Geschichte des alten Griechenlands aus ihrem eignen, innern Zusammenhang
zu erklären, behauptet Hegel z.B. einfach, sie sei weiter nichts als die
Herausarbeitung der "Gestaltungen der schönen Individualität", die
Realisation des " Kunstwerks" als solches. Er sagt viel Schönes und
Tiefes bei dieser Gelegenheit über die alten Griechen, aber das hindert
nicht, daß wir uns heute nicht mehr abspeisen lassen mit einer solchen
Erklärung, die eine bloße Redensart ist.

Wenn es also darauf ankommt, die treibenden Mächte zu
erforschen, die - bewußt oder unbewußt, und zwar sehr häufig unbewußt ---
hinter den Beweggründen der geschichtlich handelnden Menschen stehn und
die eigentlichen letzten Triebkräfte der Geschichte ausmachen, so kann
es sich nicht so sehr um die Beweggründe bei einzelnen, wenn auch noch
so hervorragenden Menschen handeln, als um diejenigen, welche große
Massen, ganze Völker und in jedem Volk wieder ganze Volksklassen in
Bewegung setzen; und auch dies nicht momentan zu einem vorübergehenden
Aufschnellen und rasch verlodernden Strohfeuer, sondern zu dauernder, in
einer großen geschichtlichen Veränderung auslaufender Aktion. Die
treibenden Ursachen zu ergründen, die sich hier in den Köpfen der
handelnden Massen und ihrer Führer - der sogenannten großen Männer - als
bewußte Beweggründe klar oder unklar, unmittelbar oder in ideologischer,
selbst in verhimmelter Form widerspiegeln - das ist der einzige Weg, der
uns auf die Spur der die Geschichte im ganzen und großen wie in den
einzelnen Perioden und Ländern beherrschenden Gesetze führen kann.
Alles, was die Menschen in Bewegung setzt, muß durch ihren Kopf
hindurch; aber welche Gestalt es in diesem Kopf annimmt, hängt sehr von
den Umständen ab. Die Arbeiter haben sich keineswegs mit dem
kapitalistischen Maschinenbetrieb versöhnt, seitdem sie die Maschinen
nicht mehr, wie noch 1848 am Rhein, einfach in Stücke schlagen.

Während aber in allen früheren Perioden die Erforschung dieser
treibenden Ursachen der
Geschichte fast unmöglich war - wegen der verwickelten und verdeckten
Zusammenhänge mit ihren Wirkungen ---, hat unsre gegenwärtige Periode
diese Zusammenhänge so weit vereinfacht, daß das Rätsel gelöst werden
konnte. Seit der Durchführung der großen Industrie, also mindestens seit
dem europäischen Frieden von 1815, war es keinem Menschen in England ein
Geheimnis mehr, daß dort der ganze politische Kampf sich drehte um die
Herrschaftsansprüche zweier Klassen, der grundbesitzenden Aristokratie
(landed aristocracy) und der Bourgeoisie (middle class). In Frankreich
kam mit der Rückkehr der Bourbonen dieselbe Tatsache zum Bewußtsein; die
Geschichtsschreiber der Restaurationszeit von Thierry bis Guizot, Mignet
und Thiers sprechen sie überall aus als den Schlüssel zum Verständnis
der französischen Geschichte seit dem Mittelalter. Und seit 1830 wurde
als dritter Kämpfer um die Herrschaft in beiden Ländern die
Arbeiterklasse, das Proletariat, anerkannt. Die Verhältnisse hatten sich
so vereinfacht, daß man die Augen absichtlich verschließen mußte, um
nicht im Kampf dieser drei großen Klassen und im Widerstreit ihrer
Interessen die treibende Kraft der modernen Geschichte zu sehn ---
wenigstens in den beiden fortgeschrittensten Ländern.

Wie aber waren diese Klassen entstanden? Konnte man auf den
ersten Blick dem großen, ehmals feudalen Grundbesitz noch einen Ursprung
aus - wenigstens zunächst - politischen Ursachen, aus gewaltsamer
Besitzergreifung zuschreiben, so ging das bei der Bourgeoisie und dem
Proletariat nicht mehr an. Hier lag der Ursprung und die Entwicklung
zweier großer Klassen aus rein ökonomischen Ursachen klar und
handgreiflich zutage. Und ebenso klar war es, daß in dem Kampf zwischen
Grundbesitz und Bourgeoisie, nicht minder als in dem zwischen
Bourgeoisie und Proletariat, es sich in erster Linie um ökonomische
Interessen handelte, zu deren Durchführung die politische Macht als
bloßes Mittel dienen sollte. Bourgeoisie und Proletariat waren beide
entstanden infolge einer Veränderung der ökonomischen Verhältnisse,
genauer gesprochen der Produktionsweise. Der Übergang zuerst vom
zünftigen Handwerk zur Manufaktur, dann von der Manufaktur zur großen
Industrie mit Dampf- und Maschinenbetrieb, hatte diese beiden Klassen
entwickelt. Auf einer gewissen Stufe wurden die von der Bourgeoisie in
Bewegung gesetzten neuen Produktionskräfte - zunächst die Teilung der
Arbeit und die Vereinigung vieler Teilarbeiter in einer Gesamtmanufaktur
--- und die durch sie entwickelten Austauschbedingungen und
Austauschbedürfnisse unverträglich mit der bestehenden, geschichtlich
überlieferten und durch Gesetz geheiligten Produktionsordnung, d.h. den
zünftigen und den zahllosen andern persönlichen und lokalen
Privilegien (die für die
nichtprivilegierten Stände ebenso viele Fesseln waren) der feudalen
Gesellschaftsverfassung. Die Produktionskräfte, vertreten durch die
Bourgeoisie, rebellierten gegen die Produktionsordnung, vertreten durch
die feudalen Grundbesitzer und die Zunftmeister; das Ergebnis ist
bekannt, die feudalen Fesseln wurden zerschlagen, in England allmählich,
in Frankreich mit einem Schlag, in Deutschland ist man noch nicht damit
fertig. Wie aber die Manufaktur auf einer bestimmten Entwicklungsstufe
in Konflikt kam mit der feudalen, so ist jetzt schon die große Industrie
in Konflikt geraten mit der an ihre Stelle gesetzten bürgerlichen
Produktionsordnung. Gebunden durch diese Ordnung, durch die engen
Schranken der kapitalistischen Produktionsweise, produziert sie
einerseits eine sich immer steigernde Proletarisierung der gesamten
großen Volksmasse, andrerseits eine immer größere Masse unabsetzbarer
Produkte. Überproduktion und Massenelend, jedes die Ursache des andern,
das ist der absurde Widerspruch, worin sie ausläuft und der eine
Entfesselung der Produktivkräfte durch Änderung der Produktionsweise mit
Notwendigkeit fordert.

In der modernen Geschichte wenigstens ist also bewiesen, daß
alle politischen Kämpfe Klassenkämpfe, und alle Emanzipationskämpfe von
Klassen, trotz ihrer notwendig politischen Form - denn jeder
Klassenkampf ist ein politischer Kampf - sich schließlich
um \emph{ökonomische} Emanzipation drehen. Hier wenigstens ist also der
Staat, die politische Ordnung, das Untergeordnete, die bürgerliche
Gesellschaft, das Reich der ökonomischen Beziehungen, das entscheidende
Element. Die althergebrachte Anschauung, der auch Hegel huldigt, sah im
Staat das bestimmende, in der bürgerlichen Gesellschaft das durch ihn
bestimmte Element. Der Schein entspricht dem. Wie beim einzelnen
Menschen alle Triebkräfte seiner Handlungen durch seinen Kopf
hindurchgehn, sich in Beweggründe seines Willens verwandeln müssen, um
ihn zum Handeln zu bringen, so müssen auch alle Bedürfnisse der
bürgerlichen Gesellschaft - gleichviel, welche Klasse grade herrscht ---
durch den Staatswillen hindurchgehn, um allgemeine Geltung in Form von
Gesetzen zu erhalten. Das ist die formelle Seite der Sache, die sich von
selbst versteht; es fragt sich nur, welchen Inhalt dieser nur formelle
Wille - des einzelnen wie des Staats - hat, und woher dieser Inhalt
kommt, warum gerade dies und nichts andres gewollt wird. Und wenn wir
hiernach fragen, so finden wir, daß in der modernen Geschichte der
Staatswille im ganzen und großen bestimmt wird durch die wechselnden
Bedürfnisse der bürgerlichen Gesellschaft, durch die Übermacht dieser
oder jener Klasse, in letzter Instanz durch die Entwicklung der
Produktivkräfte und der Austauschverhältnisse.

Wenn aber schon in unsrer modernen Zeit mit ihren riesigen
Produktions- und Verkehrsmitteln der Staat nicht ein selbständiges
Gebiet mit selbständiger Entwicklung ist, sondern sein Bestand wie seine
Entwicklung in letzter Instanz zu erklären ist aus den ökonomischen
Lebensbedingungen der Gesellschaft, so muß dies noch viel mehr gelten
für alle früheren Zeiten, wo die Produktion des materiellen Lebens der
Menschen noch nicht mit diesen reichen Hülfsmitteln betrieben wurde, wo
also die Notwendigkeit dieser Produktion eine noch größere Herrschaft
über die Menschen ausüben mußte. Ist der Staat noch heute, zur Zeit der
großen Industrie und der Eisenbahnen, im ganzen und großen nur der
Reflex, in zusammenfassender Form, der ökonomischen Bedürfnisse der die
Produktion beherrschenden Klasse, so mußte er dies noch viel mehr sein
zu einer Epoche, wo eine Menschengeneration einen weit größeren Teil
ihrer Gesamtlebenszeit auf die Befriedigung ihrer materiellen
Bedürfnisse verwenden mußte, also weit abhängiger von ihnen war, als wir
heute sind. Die Untersuchung der Geschichte früherer Epochen, sobald sie
ernstlich auf diese Seite eingeht, bestätigt dies im reichlichsten Maße;
hier kann dies aber selbstredend nicht verhandelt werden.

Wird der Staat und das Staatsrecht durch die ökonomischen
Verhältnisse bestimmt, so selbstverständlich auch das Privatrecht, das
ja wesentlich nur die bestehenden, unter den gegebnen Umständen normalen
ökonomischen Beziehungen zwischen den einzelnen sanktioniert. Die Form,
in der dies geschieht, kann aber sehr verschieden sein. Man kann, wie in
England im Einklang mit der ganzen nationalen Entwicklung geschah, die
Formen des alten feudalen Rechts großenteils beibehalten und ihnen einen
bürgerlichen Inhalt geben, ja, dem feudalen Namen direkt einen
bürgerlichen Sinn unterschieben; man kann aber auch, wie im
kontinentalen Westeuropa, das erste Weltrecht einer Waren produzierenden
Gesellschaft, das römische, mit seiner unübertrefflich scharfen
Ausarbeitung aller wesentlichen Rechtsbeziehungen einfacher
Warenbesitzer (Käufer und Verkäufer, Gläubiger und Schuldner, Vertrag,
Obligation usw.) zugrunde legen. Wobei man es zu Nutz und Frommen einer
noch kleinbürgerlichen und halbfeudalen Gesellschaft entweder einfach
durch die gerichtliche Praxis auf den Stand dieser Gesellschaft
herunterbringen kann (gemeines Recht), oder aber mit Hülfe angeblich
aufgeklärter, moralisierender Juristen es in ein, diesem
gesellschaftlichen Stand entsprechendes, apartes Gesetzbuch verarbeiten
kann, welches unter diesen Umständen auch juristisch schlecht sein wird
(preußisches Landrecht); wobei man aber auch, nach einer großen
bürgerlichen Revolution, auf Grundlage eben dieses römischen Rechtes,
ein so klassisches Gesetzbuch
der Bourgeoisgesellschaft herausarbeiten kann wie der französische Code
civil. Wenn also die bürgerlichen Rechtsbestimmungen nur die
ökonomischen Lebensbedingungen der Gesellschaft in Rechtsform
ausdrücken, so kann dies je nach Umständen gut oder schlecht geschehen.

Im Staate stellt sich uns die erste ideologische Macht über den
Menschen dar. Die Gesellschaft schafft sich ein Organ zur Wahrung ihrer
gemeinsamen Interessen gegenüber inneren und äußeren Angriffen. Dies
Organ ist die Staatsgewalt. Kaum entstanden, verselbständigt sich dies
Organ gegenüber der Gesellschaft, und zwar um so mehr, je mehr es Organ
einer bestimmten Klasse wird, die Herrschaft dieser Klasse direkt zur
Geltung bringt. Der Kampf der unterdrückten gegen die herrschende Klasse
wird notwendig ein politischer, ein Kampf zunächst gegen die politische
Herrschaft dieser Klasse; das Bewußtsein des Zusammenhangs dieses
politischen Kampfes mit seiner ökonomischen Unterlage wird dumpfer und
kann ganz verlorengehen. Wo dies auch nicht bei den Beteiligten
vollständig der Fall ist, geschieht es fast immer bei den
Geschichtschreibern. Von den alten Quellen über die Kämpfe innerhalb der
römischen Republik sagt uns nur Appian klar und deutlich, um was es sich
schließlich handelte - nämlich um das Grundeigentum.

Der Staat aber, einmal eine selbständige Macht geworden
gegenüber der Gesellschaft, erzeugt alsbald eine weitere Ideologie. Bei
den Politikern von Profession, bei den Theoretikern des Staatsrechts und
den Juristen des Privatrechts nämlich geht der Zusammenhang mit den
ökonomischen Tatsachen erst recht verloren. Weil in jedem einzelnen
Falle die ökonomischen Tatsachen die Form juristischer Motive annehmen
müssen, um in Gesetzesform sanktioniert zu werden, und weil dabei auch
selbstverständlich Rücksicht zu nehmen ist auf das ganze schon geltende
Rechtssystem, deswegen soll nun die juristische Form alles sein und der
ökonomische Inhalt nichts. Staatsrecht und Privatrecht werden als
selbständige Gebiete behandelt, die ihre unabhängige geschichtliche
Entwicklung haben, die in sich selbst einer systematischen Darstellung
fähig sind und ihrer bedürfen durch konsequente Ausrottung aller inneren
Widersprüche.

Noch höhere, d.h. noch mehr von der materiellen, ökonomischen
Grundlage sich entfernende Ideologien nehmen die Form der Philosophie
und der Religion an. Hier wird der Zusammenhang der Vorstellungen mit
ihren materiellen Daseinsbedingungen immer verwickelter, immer mehr
durch Zwischenglieder verdunkelt. Aber er existiert. Wie die ganze
Renaissancezeit, seit Mitte des 15. Jahrhunderts, ein wesentliches
Produkt der Städte, also des Bürgertums war, so auch die seitdem
neuerwachte Philosophie;
ihr Inhalt war wesentlich nur der
philosophische Ausdruck der der Entwicklung des Klein- und
Mittelbürgertums zur großen Bourgeoisie entsprechenden Gedanken. Bei den
Engländern und Franzosen des vorigen Jahrhunderts, die vielfach
ebensowohl politische Ökonomen wie Philosophen waren, tritt dies klar
hervor, und bei der Hegelschen Schule haben wir es oben nachgewiesen.

Gehn wir indes nur noch kurz auf die Religion ein, weil diese
dem materiellen Leben am fernsten steht und am fremdesten zu sein
scheint. Die Religion ist entstanden zu einer sehr waldursprünglichen
Zeit aus mißverständlichen, waldursprünglichen Vorstellungen der
Menschen über ihre eigne und die sie umgebende äußere Natur. Jede
Ideologie entwickelt sich aber, sobald sie einmal vorhanden, im Anschluß
an den gegebenen Vorstellungsstoff, bildet ihn weiter aus; sie wäre
sonst keine Ideologie, d.h. Beschäftigung mit Gedanken als mit
selbständigen, sich unabhängig entwickelnden, nur ihren eignen Gesetzen
unterworfnen Wesenheiten. Daß die materiellen Lebensbedingungen der
Menschen, in deren Köpfen dieser Gedankenprozeß vor sich geht, den
Verlauf dieses Prozesses schließlich bestimmen, bleibt diesen Menschen
notwendig unbewußt, denn sonst wäre es mit der ganzen Ideologie am Ende.
Diese ursprünglichen religiösen Vorstellungen also, die meist für jede
verwandte Völkergruppe gemeinsam sind, entwickeln sich, nach der
Trennung der Gruppe, bei jedem Volk eigentümlich, je nach den ihm
beschiednen Lebensbedingungen, und dieser Prozeß ist für eine Reihe von
Völkergruppen, namentlich für die arische (sog. indoeuropäische) im
einzelnen nachgewiesen durch die vergleichende Mythologie. Die so bei
jedem Volk herausgearbeiteten Götter waren Nationalgötter, deren Reich
nicht weiter ging als das von ihnen zu schützende nationale Gebiet,
jenseits dessen Grenzen andre Götter unbestritten das große Wort
führten. Sie konnten nur in der Vorstellung fortleben, solange die
Nation bestand; sie fielen mit deren Untergang. Diesen Untergang der
alten Nationalitäten brachte das römische Weltreich, dessen ökonomische
Entstehungsbedingungen wir hier nicht zu untersuchen haben. Die alten
Nationalgötter kamen in Verfall, selbst die römischen, die eben auch nur
auf den engen Kreis der Stadt Rom zugeschnitten waren; das Bedürfnis,
das Weltreich zu ergänzen durch eine Weltreligion, tritt klar hervor in
den Versuchen, allen irgendwie respektablen fremden Göttern neben den
einheimischen in Rom Anerkennung und Altäre zu schaffen. Aber eine neue
Weltreligion macht sich nicht in dieser Art durch kaiserliche Dekrete.
Die neue Weltreligion, das Christentum, war im stillen bereits
entstanden aus einer Mischung verallgemeinerter orientalischer,
namentlich jüdischer Theologie
und vulgarisierter griechischer, namentlich stoischer Philosophie. Wie
es ursprünglich aussah, müssen wir erst wieder mühsam erforschen, da
seine uns überlieferte offizielle Gestalt nur diejenige ist, in der es
Staatsreligion und diesem Zweck durch das Nicänische Konzil angepaßt
wurde. Genug, die Tatsache, daß es schon nach 250 Jahren Staatsreligion
wurde, beweist, daß es die den Zeitumständen entsprechende Religion war.
Im Mittelalter bildete es sich genau im Maß, wie der Feudalismus sich
entwickelte, zu der diesem entsprechenden Religion aus, mit
entsprechender feudaler Hierarchie. Und als das Bürgertum aufkam,
entwickelte sich im Gegensatz zum feudalen Katholizismus die
protestantische Ketzerei, zuerst in Südfrankreich bei den Albigensern,
zur Zeit der höchsten Blüte der dortigen Städte. Das Mittelalter hatte
alle übrigen Formen der Ideologie: Philosophie, Politik, Jurisprudenz,
an die Theologie annektiert, zu Unterabteilungen der Theologie gemacht.
Es zwang damit jede gesellschaftliche und politische Bewegung, eine
theologische Form anzunehmen; den ausschließlich mit Religion
gefütterten Gemütern der Massen mußten ihre eignen Interessen in
religiöser Verkleidung vorgeführt werden, um einen großen Sturm zu
erzeugen. ǁ Und wie das Bürgertum von Anfang an einen Anhang von
besitzlosen, keinem anerkannten Stand angehörigen städtischen Plebejern,
Tagelöhnern und Dienstleuten aller Art erzeugte, Vorläufern des spätem
Proletariats, ǁ so teilt sich auch die Ketzerei schon früh in eine
bürgerlich-gemäßigte und eine plebejisch-revolutionäre, auch von den
bürgerlichen Ketzern verabscheute.

Die Unvertilgbarkeit der protestantischen Ketzerei entsprach der
Unbesiegbarkeit des aufkommenden Bürgertums; als dies Bürgertum
hinreichend erstarkt war, begann sein bisher vorwiegend lokaler Kampf
mit dem Feudaladel nationale Dimensionen anzunehmen. Die erste große
Aktion fand in Deutschland statt - die sogenannte Reformation. Das
Bürgertum war weder stark noch entwickelt genug, um die übrigen
rebellischen Stände - die Plebejer der Städte, den niederen Adel und die
Bauern auf dem Lande - unter seiner Fahne vereinigen zu können. Der Adel
wurde zuerst geschlagen; die Bauern erhoben sich zu einem Aufstand, der
den Gipfelpunkt dieser ganzen revolutionären Bewegung bildet; die Städte
ließen sie im Stich, und so erlag die Revolution den Heeren der
Landesfürsten, die den ganzen Gewinn einstrichen. Von da an verschwindet
Deutschland auf drei Jahrhunderte aus der Reihe der selbständig in die
Geschichte eingreifenden Länder. Aber neben dem Deutschen Luther hatte
der Franzose Calvin gestanden; mit echt französischer Schärfe stellte er
den bürgerlichen Charakter der Reformation in den Vordergrund,
republikanisierte und
demokratisierte die Kirche.
Während die lutherische Reformation in Deutschland versumpfte und
Deutschland zugrunde richtete, diente die calvinische den Republikanern
in Genf, in Holland, in Schottland als Fahne, machte Holland von Spanien
und vom Deutschen Reiche frei und lieferte das ideologische Kostüm zum
zweiten Akt der bürgerlichen Revolution, der in England vor sich ging.
Hier bewährte sich der Calvinismus als die echte religiöse Verkleidung
der Interessen des damaligen Bürgertums und kam deshalb auch nicht zu
voller Anerkennung, als die Revolution 1689 durch einen Kompromiß eines
Teils des Adels mit den Bürgern vollendet wurde. Die englische
Staatskirche wurde wiederhergestellt, aber nicht in ihrer frühem
Gestalt, als Katholizismus mit dem König zum Papst, sondern stark
calvinisiert. Die alte Staatskirche hatte den lustigen katholischen
Sonntag gefeiert und den langweiligen calvinistischen bekämpft, die neue
verbürgerte führte diesen ein, und er verschönert England noch jetzt.

In Frankreich wurde die calvinistische Minorität 1685
unterdrückt, katholisiert oder weggejagt; aber was half's? Schon damals
war der Freigeist Pierre Bayle mitten in der Arbeit, und 1694 wurde
Voltaire geboren. Die Gewaltmaßregel Ludwigs XIV. erleichterte nur dem
französischen Bürgertum, daß es seine Revolution in der, der
entwickelten Bourgeoisie allein angemessenen irreligiösen,
ausschließlich politischen Form machen konnte. Statt Protestanten saßen
Freigeister in den Nationalversammlungen. Dadurch war das Christentum in
sein letztes Stadium getreten. Es war unfähig geworden, irgendeiner
progressiven Klasse fernerhin als ideologische Verkleidung ihrer
Strebungen zu dienen; es wurde mehr und mehr Alleinbesitz der
herrschenden Klassen, und diese wenden es an als bloßes
Regierungsmittel, womit die untern Klassen in Schranken gehalten werden.
Wobei dann jede der verschiednen Klassen ihre eigne entsprechende
Religion benutzt: die grundbesitzenden Junker die katholische Jesuiterei
oder protestantische Orthodoxie, die liberalen und radikalen Bourgeois
den Rationalismus; und wobei es keinen Unterschied macht, ob die Herren
an ihre respektiven Religionen selbst glauben oder auch nicht.

Wir sehn also: Die Religion, einmal gebildet, enthält stets
einen überlieferten Stoff, wie denn auf allen ideologischen Gebieten die
Tradition eine große konservative Macht ist. Aber die Veränderungen, die
mit diesem Stoff vorgehn, entspringen aus den Klassenverhältnissen, also
aus den ökonomischen Verhältnissen der Menschen, die diese Veränderungen
vornehmen. Und das ist hier hinreichend. ---

Es kann sich im Vorstehenden nur um einen allgemeinen Umriß der
Marxschen Geschichtsauffassung handeln, höchstens noch um einige
Illustrationen. Der Beweis ist an
der Geschichte selbst zu liefern, und da darf ich wohl sagen, daß er in
andern Schriften bereits hinreichend geliefert ist. Diese Auffassung
macht aber der Philosophie auf dem Gebiet der Geschichte ebenso ein
Ende, wie die dialektische Auffassung der Natur alle Naturphilosophie
ebenso unnötig wie unmöglich macht. Es kommt überall nicht mehr darauf
an, Zusammenhänge im Kopf auszudenken, sondern sie in den Tatsachen zu
entdecken. Für die aus Natur und Geschichte vertriebne Philosophie
bleibt dann nur noch das Reich des reinen Gedankens, soweit es noch
übrig: die Lehre von den Gesetzen des Denkprozesses selbst, die Logik
und Dialektik.

\asterisc

Mit der Revolution von 1848 erteilte das "gebildete" Deutschland
der Theorie den Absagebrief und ging über auf den Boden der Praxis. Das
auf der Handarbeit beruhende Kleingewerbe und die Manufaktur wurden
ersetzt durch eine wirkliche große Industrie; Deutschland erschien
wieder auf dem Weltmarkt; das neue kleindeutsche Reich beseitigte
wenigstens die schreiendsten Mißstände, die die Kleinstaaterei, die
Reste des Feudalismus und die bürokratische Wirtschaft dieser
Entwicklung in den Weg gelegt hatten. Aber in demselben Maß, wie die
Spekulation aus der philosophischen Studierstube auszog, um ihren Tempel
zu errichten auf der Fondsbörse, in demselben Maß ging auch dem
gebildeten Deutschland jener große theoretische Sinn verloren, der der
Ruhm Deutschlands während der Zeit seiner tiefsten politischen
Erniedrigung gewesen war - der Sinn für rein wissenschaftliche
Forschung, gleichviel, ob das erreichte Resultat praktisch verwertbar
war oder nicht, polizeiwidrig oder nicht. Zwar hielt sich die deutsche
offizielle Naturwissenschaft, namentlich auf dem Gebiet der
Einzelforschung, auf der Höhe der Zeit, aber schon das amerikanische
Journal "Science" bemerkt mit Recht, daß die entscheidenden Fortschritte
auf dem Gebiet der großen Zusammenhänge zwischen den Einzeltatsachen,
ihre Verallgemeinerung zu Gesetzen, jetzt weit mehr in England gemacht
werden, statt wie früher in Deutschland. Und auf dem Gebiet der
historischen Wissenschaften, die Philosophie eingeschlossen, ist mit der
klassischen Philosophie der alte theoretisch-rücksichtslose Geist erst
recht verschwunden; gedankenloser Eklektizismus, ängstliche Rücksicht
auf Karriere und Einkommen bis herab zum ordinärsten Strebertum sind an
seine Stelle getreten. Die offiziellen Vertreter dieser Wissenschaft
sind die unverhüllten Ideologen der Bourgeoisie und des bestehenden
Staats geworden - aber zu einer Zeit, wo beide im offnen Gegensatz stehn
zur Arbeiterklasse.

Und nur bei der Arbeiterklasse besteht der deutsche theoretische
Sinn unverkümmert fort. Hier ist er nicht auszurotten; hier finden keine
Rücksichten statt auf Karriere, auf Profitmacherei, auf gnädige
Protektion von oben; im Gegenteil, je rücksichtsloser und unbefangener
die Wissenschaft vorgeht, desto mehr befindet sie sich im Einklang mit
den Interessen und Strebungen der Arbeiter. Die neue Richtung, die in
der Entwicklungsgeschichte der Arbeit den Schlüssel erkannte zum
Verständnis der gesamten Geschichte der Gesellschaft, wandte sich von
vornherein vorzugsweise an die Arbeiterklasse und fand hier die
Empfänglichkeit, die sie bei der offiziellen Wissenschaft weder suchte
noch erwartete. Die deutsche Arbeiterbewegung ist die Erbin der
deutschen klassischen Philosophie.

} 
\ParallelRText{ {\let\clearpage\relax\chapter[Nota prévia]{Nota prévia\protect\endnote{\versal{ENGELS}, F. \emph{Werke.
  Artikel. Entwürfe. Oktober 1886 bis Februar} 1891. (\versal{MEGA}, 31, \versal{I}).
  Bearbeitet von Renate Merkel"-Melis. Akademie Verlag GmbH, Berlin,
  2002. A \emph{nota prévia} não aparece na primeira versão publicada na
  revista \emph{Die Neue Zeit} em 1886 em dois cadernos, de abril e
  maio, antes da versão publicada em livro no ano de 1888: \emph{Ludwig
  Feuerbach und der Ausgang der klassischen deutschen Philosophie.} Mit
  Anhang: \emph{Karl Marx über Feuerbach vom Jahre 1845}. Stuttgart:
  Verlag von J.H.W. Dietz, 1888. {[}\versal{N.\,T.}{]}}}}

%\endnotetext

\noindent{}No prefácio da \emph{Crítica da economia política}, Berlim, 1859,
Karl Marx conta como nós em 1845, em Bruxelas, começamos ``a realizar em
conjunto o contraste de nossa visão'' --- a concepção materialista da
história elaborada por Marx --- ``em oposição à visão ideológica da 
filosofia alemã, visando, de fato, acertar contas com nossa consciência
filosófica da época. O propósito foi realizado na \emph{forma} \emph{da
filosofia pós"-hegeliana}. O manuscrito {[}\emph{A ideologia alemã}{]},
dois grossos volumes dividido em oitavos, há muito já estava na editora
em Westfalen quando recebemos a mensagem de que circunstâncias
alteradas não permitiriam a publicação. Deixamos o manuscrito para a
crítica roedora dos ratos, tanto mais de boa vontade na medida em que já
havíamos alcançado nosso objetivo principal: autocompreensão''.

Desde então se passaram mais de 40 anos, e Marx morreu sem que nos
tivesse sido oferecida a oportunidade de retornar ao objeto. Sobre nossa
relação com Hegel nos pronunciamos em partes separadas, porém nunca em
um contexto amplo. A Feuerbach, que, em mais de um aspecto, estabelece
um meio de ligação entre a filosofia de Hegel e nossa concepção, nunca
retornamos.

Entretanto, a visão de mundo de Marx encontrou representantes muito além
da Alemanha e Europa e em todas as línguas cultas do mundo. Por outro
lado, a filosofia clássica alemã vivencia uma espécie de renascimento no
exterior, particularmente na Inglaterra e Escandinávia,\est\ e mesmo na
Alemanha parece estar farta de receber as sopas ecléticas como esmolas
servidas nas universidades de lá sob o nome de filosofia.

Sob essas circunstâncias, uma breve e coerente exposição de nossa
relação com a filosofia hegeliana, nosso ponto de partida e nossa
ruptura com ela, parecia cada vez mais necessária. E também me pareceu
necessário um reconhecimento completo da influência que, acima de todos
os outros filósofos pós"-hegelianos, Feuerbach teve sobre nós durante
nosso período de \emph{Strum und Drang}, como uma dívida de honra não
quitada. Por isso, aproveitei de bom grado a oportunidade quando os
editores da \emph{Neue Zeit} solicitaram"-me uma revisão crítica do
livro de Starcke sobre Feuerbach. Meu trabalho foi publicado no quarto e
quinto cadernos dessa revista em 1886, e aparece aqui em uma edição
especial revisada.

Antes de enviar essas linhas para a imprensa, olhei novamente para o antigo
manuscrito de 1845/46. A seção sobre Feuerbach não está
concluída. A parte terminada consiste em uma exposição da concepção
materialista da história que apenas demonstra o quanto, naquela época,
os nossos conhecimentos da história econômica eram ainda incompletos.
Falta a crítica à própria doutrina de Feuerbach; não era, portanto,
aproveitável para o objetivo atual. Por outro lado, em um antigo caderno
de Marx, encontrei as onze teses sobre Feuerbach reproduzidas em
apêndice.\endnote{A edição alemã de 1888 contém em anexo as
  onze teses de Feuerbach, publicadas com o título ``Karl Marx sobre
  Feuerbach do ano de 1845'' (\emph{Karl Marx über Feuerbach vom Jahre
  1845}). {[}\versal{N.\,T.}{]}} São notas para elaboração posterior, escritas rapidamente,
absolutamente não destinadas à impressão, mas inestimáveis ​​como o
primeiro documento no qual é estabelecido o genial embrião da nova
visão de mundo.

\bigskip

\hfill{}Londres, 21 de fevereiro de 1888. 

\quebra

\begin{flushright}
\emph{Die Neue Zeit. Ano 4.}\\
\emph{Caderno 4, abril de 1886.}
\end{flushright}

\vspace{2cm}


\addcontentsline{toc}{chapter}{Primeira parte}
\section{I}

\noindent{}O presente escrito\endnote{Ludwig Feuerbach, de C.
  N. Starcke, Dr. Phil. Stuttgart, Ferd. Encke, 1885. {[}\versal{N.\,E.}{]}}
nos conduz novamente a um período que, de acordo com a época, já ficou
uma boa geração para trás, mas que já se tornou tão estranho 
para a atual geração na Alemanha como se tivesse
completado um século inteiro de idade. E foi, no entanto, o período de
preparação da Alemanha para a Revolução de 1848; e tudo o que desde
então aconteceu conosco é apenas uma avanço de 1848, apenas execução do
testamento da revolução.

Assim como na França do século \versal{XVIII}, também na Alemanha do século \versal{XIX}
a revolução filosófica preparou o colapso político. Mas quão diversas
ambas parecem! Os franceses em luta aberta contra todo o saber
oficial, contra a Igreja, frequentemente também contra o
Estado; os seus escritos impressos além das fronteiras, na Holanda ou na
Inglaterra, e eles próprios, com frequência, prontos a marchar para
a Bastilha. Os alemães em contrapartida: professores universitários,
doutrinadores da juventude estabelecidos pelo Estado; seus escritos,
manuais doutrinários reconhecidos, e o sistema que encerra todo o desenvolvimento, o sistema hegeliano, que foi, inclusive, elevado ao nível de uma filosofia de Estado da realeza prussiana! E quem diria
que a revolução se esconderia por detrás desses professores, por detrás
das suas palavras pedantemente obscuras, em seus prolixos e maçantes
períodos? Afinal, não eram justamente os liberais as pessoas
consideradas, naquela época, os representantes da revolução, os
adversários mais duros dessa filosofia que confunde as cabeças? O que,
porém, nem os governos nem os liberais viram já foi visto, em 1833, pelo
menos por \emph{um} homem, e ele se chamava Heinrich Heine.\endnote{Referência ao livro de Heine, \emph{Zur
  Geschichte der Religion und Philosophie in Deutschland,} 1833 {[}Sobre
  a história da religião e filosofia na Alemanha{]}, onde trabalha a
  relação entre reforma protestante, revolução francesa e o papel da
  ``revolução'' filosófica do idealismo alemão. Heine é o epicentro da
  expressão crítica que ficou conhecida como miséria alemã,
  \emph{Sonderweg}, ou ainda, via prussiana. A problemática da
  expansão"-estabilização do mundo burguês, de totalização histórica e
  sistêmica da relação"-capital, é o processo que define o século \versal{XIX}
  europeu. Tema desenvolvido, principalmente, por Lenin, Trotski e
  Lukács. A reverberação política e social do conhecido descompasso em
  relação ao desenvolvimento histórico da Inglaterra (revolução
  econômica --- século \versal{XVII} e \versal{XVIII}) e da França (revolução política --- século \versal{XVIII}) é o que define a miséria alemã do século \versal{XIX}. O pano de fundo histórico é complexo e atravessa muitos séculos de conflitos. A
  disputa pela região da Alsácia"-Lorena, parte integrante do Sacro
  Império Romano"-Germânico em 1648, anexada à França de Luis \versal{XIV} depois
  da Paz de Vestfália, é um dos mais antigos embates envolvendo o povo
  francês e o alemão. Heinrich Heine também é aludido indiretamente por
  Marx em um texto de 1842 (\emph{Manifesto filosófico da Escola
  Histórica do Direito}) ao citar ironicamente a ``má'' influencia do
  jurista Gustav von Hugo --- ao defender a irracionalidade da
  exclusividade do impulso sexual do homem no matrimonio --- para os
  \emph{jovens alemães}, movimento literário 
  alemão da primeira metade do século \versal{XIX} com inspiração em Heine e que
  dentre suas principais revindicações exigia o amor livre. {[}\versal{N.\,T.}{]}}

Tomemos um exemplo. Nenhuma proposição filosófica carregou consigo
igualmente a gratidão de governos limitadores e a cólera de liberais
como a famosa proposição
de Hegel:
``Tudo o que é efetivo é racional, e tudo o que é racional é
efetivo.''\endnote{Passagem presente em: \emph{Vorrede,}
   \emph{Grundlinien der Philosophie des Rechts}, oder Naturrecht und
  Staatswissenschaft. Hrsg. von Eduard Gans. 2. Aufl. Berlin, 1840 e
  retomada em  \emph{Enzyklopädie der philosophischen Wissenschaften \versal{I}}
  (Dritte Ausgabe, 1830), § 6. {[}\versal{N.\,T.}{]}}

Isso era, porém, a evidente santificação de todo o elemento
existente, a consagração filosófica do despotismo, do Estado policial,
da justiça de gabinete, da censura. E tal
como Frederico
Guilherme \versal{III} assim entendeu, assim entenderam os seus súditos. Mas,
em Hegel,
de modo nenhum tudo aquilo que existe é também automaticamente efetivo. Para ele, o atributo da realidade 
efetiva cabe apenas àquilo que, ao mesmo tempo, é necessário: ``a
realidade efetiva mostra"-se em seu desdobramento como a
necessidade''.\endnote{\emph{Enzyklopädie
  der philosophischen Wissenschaften \versal{I}} (1830), § 143,
  \emph{Zusatz}: ``Se isso é possível ou impossível, depende do
  conteúdo, isto é, da totalidade dos momentos da realidade efetiva,
  \emph{que se mostra em seu desdobramento como a necessidade}''.
  (grifei) {[}\versal{N.\,T.}{]}} Um regulamento governamental arbitrário --- o próprio Hegel
remete ao exemplo ``de certa instituição
fiscal''\endnote{\emph{Enzyklopädie der philosophischen Wissenschaften \versal{I},} § 142, \emph{Zusatz}. Ao tratar da \emph{realidade efetiva} como unidade
  entre essência e existência, compreendendo o elemento efetivo como o
  ser"-posto que delimita um modo de relação que, apesar de racional ---
  adequada à ideia --- poder não ser efetivo, Hegel critica justamente a
  suposta cisão entre ideia e realidade efetiva. Ao tomar a ideia como
  representação subjetiva e a realidade como exterioridade sensível, a
  dinâmica de necessidade que vincula uma a outra se perderia em
  estruturas lógico"-formais abstratas: ou como uma dinâmica de
  necessidade manifesta de modo meramente ideal, ou como uma dinâmica de
  necessidade efetiva, expressa por automatismos irracionais. O exemplo
  que ilustra essa situação mencionada por Engels é justamente o ``plano, ou assim chamada ideia, de certa instituição fiscal'', 
  que pode ser ``em si boa ou adequada aos fins, mas 
  que não se encontra, do mesmo modo, na assim chamada realidade efetiva
  e não pode ser executada sob as relações dadas.'' (§ 142,
  \emph{Zusatz}) {[}\versal{N.\,T.}{]}} --- de modo algum tem, para ele, validade automática
como algo efetivo. O que é necessário, porém, comprova"-se, em última
instância, também como racional e, aplicada ao Estado prussiano daquela
época, a proposição
de Hegel quer
dizer apenas: este Estado é racional, correspondente à razão, na mesma
medida em que é necessário; e se ele, porém, apresenta"-se para nós como
perverso, mas apesar da sua perversidade continua a 
existir, a perversidade do governo encontra a sua justificação e a sua
explicação na correspondente perversidade dos súditos. Os prussianos
daquela época tinham o governo que mereciam.

Afinal,
segundo Hegel,
a realidade efetiva não é de modo algum um atributo que condiz com um
estado de coisas social ou político dado em todas as circunstâncias e em
todos os tempos. Pelo contrário. A República Romana era efetiva, mas o
Império Romano \textbar{} que a suplantou \textbar{} também. A Monarquia Francesa, \textbar{} em
1789 \textbar{}, tinha"-se tornado tão inefetiva, isto é, tão desprovida de toda a
necessidade, tão irracional, que tinha de ser aniquilada pela grande
revolução sobre a qual
Hegel sempre
fala com o maior entusiasmo. Aqui, portanto, a monarquia era o elemento
inefetivo, e a revolução o efetivo. E, no curso do desenvolvimento, todo o
elemento anteriormente efetivo se torna inefetivo, perde a sua
necessidade, o seu direito de existência, seu caráter racional; toma o
lugar do efetivo que padece uma nova e viável realidade efetiva:
pacificamente, se a antiga é suficientemente sensata para marchar para
morte sem resistência; pela força da violência, caso se oponha a essa
necessidade. E, assim, a proposição
de Hegel, por meio da própria dialética
hegeliana, inverte"-se no seu contrário: tudo o que no âmbito da história humana é
efetivo torna"-se, com o tempo, irracional, é, portanto, já segundo sua
determinação, irracional, está desde o princípio afetado com a
irracionalidade; e tudo o que na cabeça dos homens é racional está
determinado a se tornar efetivo, caso esse também ainda possa
contradizer a aparente realidade efetiva existente. A proposição da
racionalidade de todo o elemento efetivo real dissolve"-se, segundo todas
as regras do método de pensar
de Hegel,
nesta outra: tudo o que existe é digno de perecer.

Mas a verdadeira significação e o caráter revolucionário da filosofia
de Hegel (na
qual temos que nos limitar aqui como o desfecho de todo o movimento
desde Kant)
consistia justamente no fato que ele, de uma vez por todas, acabou com o
caráter definitivo de todos os resultados do pensar e do agir humanos. A
verdade, que valia conhecer na filosofia, não era mais
para Hegel uma
coleção de proposições dogmáticas prontas que, uma vez encontradas,
apenas se buscava decorar; a verdade consistia agora no processo do
próprio conhecer, no longo desenvolvimento histórico do saber,
que se eleva de estágios inferiores do 
conhecimento para estágios sempre superiores sem jamais, porém,
alcançar, por meio do processo de localização de uma, assim chamada, verdade absoluta, o ponto em que não pode mais avançar, em que não lhe resta
mais nada além de ficar de braços cruzados e admirar imóvel a verdade
absoluta obtida. E isso tanto no âmbito do conhecimento filosófico como no
de qualquer outro conhecimento e ação prática. Tanto quanto o
conhecimento, também a história não pode encontrar um desfecho pleno
em um estado ideal perfeito de 
humanidade; uma sociedade perfeita, um ``Estado'' perfeito, são coisas
que só podem existir na fantasia; pelo contrário, todos os estados
históricos que se seguem uns aos outros são apenas estados transitórios
no curso de desenvolvimento sem fim da sociedade humana, 
do inferior para o superior. Portanto cada estágio é necessário, está
justificado para época e condições às quais deve a sua origem; mas cada estágio se %MANTIVE
torna caduco e injustificado diante das novas condições superiores que
gradualmente se desenvolvem no seu próprio âmago; precisam dar lugar a
um estágio superior que ingressa, por sua vez, novamente na série de
declínio e decadência. Assim como a burguesia, através da grande
indústria, da concorrência e do mercado mundial, dissolve na prática
todas as instituições estáveis e veneráveis pela longevidade, essa
filosofia dialética também dissolve todas as representações da verdade
absoluta definitiva e os correspondentes estados absolutos da
humanidade. Diante dela não existe nada de definitivo, de absoluto, de
sagrado; ela mostra a transitoriedade de tudo e em tudo, e nada subsiste
diante dela a não ser o ininterrupto processo do devir e perecer, da
ascensão sem fim do inferior ao superior, da qual ela própria é mero
reflexo no cérebro pensante. Ela também tem, certamente, um lado
conservador: reconhece a justificação de determinados estados do
conhecimento e da sociedade para a sua época e circunstâncias; mas
apenas até aqui. O conservadorismo desse modo de intuição
é relativo, seu caráter revolucionário é 
absoluto --- é o único elemento absoluto que ela pode admitir como válido.\est\

Não precisamos entrar aqui na questão se este modo de intuição está de
acordo com o atual estado da ciência da natureza, a qual prevê para a
existência da Terra um fim possível~--- para o seu caráter
habitável, porém, um fim bastante certo ---, que, portanto, atribui também
à história humana não só um ramo ascendente como também um descendente.
Estamos, de qualquer forma, ainda bastante distantes do ponto de mudança
a partir do qual a história da sociedade entra em declínio, e não podemos
exigir da filosofia
de Hegel que
se ocupe de um objeto que, no tempo dela, a ciência da natureza ainda
não tinha posto na ordem do dia.

Mas o que, de fato, podemos dizer aqui é: o desenvolvimento acima
referido não se encontra com essa precisão
em Hegel.
Tal desenvolvimento é uma consequência necessária do seu método,
consequência esta, porém, que ele próprio nunca tencionou com tal
expressividade. E isso, sem dúvida, pela simples razão de que estava
obrigado a empreender um sistema, e um sistema filosófico, segundo as
exigências tradicionais, tem de finalizar com algum tipo de verdade
absoluta. Portanto, por mais que Hegel também acentue, nomeadamente
na \emph{Lógica}, que esta verdade eterna
nada mais é do que o próprio processo lógico correspondente ao
histórico, ele próprio se vê compelido a dar um fim a esse processo,
porque necessita de algum modo, precisamente, chegar ao fim com o seu
sistema. Na \emph{Lógica} ele pode voltar a tornar esse fim um início,
na medida em que aí o ponto final, a ideia absoluta --- que só é absoluta por não saber dizer absolutamente nada acerca dela ---
``exterioriza"-se'', isto é, transforma"-se na 
natureza e, mais tarde, regressa a si própria no espírito, ou seja, no
pensar e na história. Mas, na conclusão de toda filosofia, tal regresso
ao início somente é possível por \emph{um} caminho. Ou seja, ao
estabelecer o fim da história de tal modo que a humanidade chega
justamente ao conhecimento dessa ideia absoluta \textbar{} e esclarecer que esse
conhecimento da ideia absoluta foi alcançado na filosofia hegeliana. \textbar{}
Com isso, porém, todo o conteúdo dogmático do sistema
de Hegel é
qualificado como verdade absoluta, em contradição com o seu método
dialético que dissolve todo elemento dogmático; o lado revolucionário é,
assim, abafado pelo lado conservador que o encobre completamente. E o
que vale para o conhecimento filosófico vale também para a
\emph{práxis} histórica. A humanidade que, na pessoa
de Hegel,
conduziu até a elaboração da ideia absoluta precisa, também na prática,
ter chegado ao ponto de poder levar a cabo essa ideia absoluta na
realidade efetiva. As reivindicações políticas práticas da ideia
absoluta, em relação aos contemporâneos, não podem, portanto, ser
demasiadamente ambiciosas. E, assim, encontramos na conclusão
da \emph{Filosofia do direito} que a ideia absoluta deve efetivar"-se
naquela monarquia de estamentos 
que Frederico Guilherme \versal{III}, tão obstinadamente em vão, prometeu aos seus súditos,
portanto, em uma dominação indireta das classes de possuidores,
limitada, adaptada e mediada pelas relações da pequena burguesia alemã
daquela época; com isso nos é demonstrada, pela via especulativa, ainda
a necessidade da nobreza.

Portanto, as necessidades internas do sistema somente são suficientes
para, por intermédio de um método de pensar revolucionário de ponta a
ponta, esclarecer a produção de uma conclusão política
bastante dócil. A forma específica dessa conclusão 
resulta, porém, do fato de que Hegel era alemão e de que, tal como do seu
contemporâneo Goethe, pendia"-lhe uma pedaço de trança de filisteu. Goethe, assim como Hegel, eram, cada um no seu âmbito, um Zeus do Olimpo, mas ambos nunca se
libertaram completamente do filisteu alemão.

Isso tudo não impediu, contudo, o sistema
de Hegel de abarcar um âmbito incomparavelmente maior do que qualquer sistema
anterior e de desenvolver nele uma riqueza de pensamento que
ainda hoje causa espanto. \emph{Fenomenologia do espírito} (que %MANTIVE
poderíamos denominar como um paralelo\est\ entre a embriologia e a
paleontologia do espírito, um desenvolvimento da consciência individual
por meio dos seus diversos estágios, apreendido como reprodução %MANTIVE
encurtada dos estágios pelos quais a consciência dos homens passa
historicamente), \emph{Lógica}, \emph{Filosofia da natureza},
\emph{Filosofia do espírito}, e esta última, novamente, elaborada em suas
subdivisões históricas isoladas: Filosofia da História, do Direito, da
Religião, História da Filosofia, Estética, etc.
Hegel trabalha
para encontrar e demonstrar, em todos esses diversos domínios
históricos, o fio do desenvolvimento que os perpassa; e nesse processo
ele não foi apenas um gênio criador, mas também um homem de erudição
enciclopédica, fazendo assim época em todos os domínios. E evidente que,
em virtude das necessidades do ``sistema'', com muita frequência ele
teve de refugiar"-se em construções forçadas, acerca das quais
os seus inimigos, apegados a questões menores, até hoje fazem uma
gritaria tão descomunal. Mas estas construções são apenas a armação e o
andaime da sua obra; se não nos retemos aí inutilmente, se penetramos
mais profundamente no poderoso edifício, inúmeros tesouros que ainda
hoje conservam o seu pleno valor serão encontrados. Para todos os
filósofos é precisamente o ``sistema'' o elemento perecível, e isto
justamente por decorrer de uma necessidade 
imperecível do espírito humano: a necessidade de superação
de todas as contradições. Mas, se todas as 
contradições são eliminadas de uma vez por todas, atracamos na assim
chamada verdade absoluta: a história mundial está no fim e, no entanto,
deve continuar, embora não lhe reste mais nada para fazer --- portanto,
uma nova contradição, insolúvel. Assim que compreendermos --- e
definitivamente ninguém nos ajudou mais nessa intelecção do que o
próprio Hegel ---
que a tarefa da filosofia, assim estabelecida, não significa outra coisa
além do fato de que um filósofo singular deve realizar aquilo que só a
humanidade inteira no seu desenvolvimento progressivo\est\ pode realizar ---
assim que compreendermos isso, estará também no fim toda a filosofia no
sentido em que a palavra é conhecida até hoje. Abandona"-se a ``verdade
absoluta'', inalcançável por esta via e por cada um individualmente e,
em troca, perseguimos as verdades relativas alcançáveis pela via das
ciências positivas e da conexão dos seus resultados por meio do pensar
dialético.
Com Hegel
encerra"-se a filosofia em geral. Por um lado, porque ele reuniu em seu
sistema, do modo mais grandioso, todo o desenvolvimento da filosofia;
por outro, porque, ainda que inconscientemente, mostra"-nos o caminho
para fora desse labirinto de sistemas em direção ao conhecimento
positivo e efetivo do mundo.

Concebemos qual efeito monstruoso este sistema hegeliano teve de
produzir na atmosfera filosoficamente tingida da Alemanha. Foi uma
marcha triunfal que durou décadas e que de modo nenhum parou com a morte
de Hegel.
Pelo contrário, precisamente de 1830 a 1840 o ``hegelianismo
charlatão'' dominou do modo mais exclusivo possível e 
havia contagiado, mais ou menos, até mesmo seus adversários;
precisamente nesse tempo as concepções
de Hegel penetraram
com a maior abundância, consciente ou inconscientemente, nas mais
variadas ciências e azedaram completamente também a literatura popular e
a imprensa diária, nas quais a ``consciência culta'' habitual adquire a
matéria de seu pensamento. Mas esse triunfo em todas as linhas era
apenas o prelúdio de uma luta interna.

A doutrina de Hegel como
um todo, vimos, deixava um amplo espaço para alocar as mais diversas
intuições tendenciosas da prática; e, na
prática, na Alemanha teórica daquele tempo, tais intuições eram, antes
de tudo, a religião e a política. Quem colocasse ênfase
no \emph{sistema} de Hegel
podia ser bastante conservador em ambos os domínios; quem visse o
principal no \emph{método} dialético podia, tanto religiosa como
politicamente, pertencer à oposição mais extrema. O
próprio Hegel,
apesar dos ataques de ira revolucionários bastante frequentes em suas
obras, parecia, no conjunto, inclinar"-se mais para o lado conservador;
se o seu sistema não lhe tivesse custado muito mais ``trabalho amargo do
pensamento'' do que o seu método. Perto do fim dos anos trinta a tensão
na escola se evidenciou cada vez mais. A ala da esquerda, os chamados
jovens hegelianos, na luta com ortodoxos pietistas e feudais
reacionários, desistiu, pedaço por pedaço, daquela reserva
filosoficamente distinta diante das questões ardentes do dia a dia que,
até então, haviam assegurado à sua doutrina tolerância estatal e
inclusive proteção; e quando, em 1840, a hipocrisia 
ortodoxa e a reação feudal"-absolutista subiram ao trono com Frederico
Guilherme \versal{IV}, uma aberta tomada de partido tornou"-se inevitável. A luta
seria travada ainda com armas filosóficas, mas não mais por fins
abstratamente filosóficos; tratava"-se do aniquilamento
da religião tradicional e do Estado existentes. E se 
nos \emph{Anais alemães}
os fins últimos práticos ainda se mostravam 
preponderantemente sob disfarce filosófico, a escola jovem"-hegeliana
revelou"-se na \emph{Gazeta renana }de 1842 diretamente como a filosofia
da burguesia com aspirações radicais e se valeu do pretexto filosófico
apenas para ainda enganar a censura.

A política era, nessa altura, um âmbito muito espinhoso e, por isso, a
luta principal voltou"-se contra a religião; esta era, certamente desde
1840, indiretamente também uma luta política.
A \emph{Vida de Jesus} de Strauss,
em 1835, tinha dado o primeiro impulso. À teoria da formação evangélica
dos mitos aí desenvolvida,
Bruno Bauer opôs"-se mais tarde ao demonstrar que toda uma série de narrativas
evangélicas haviam sido fabricadas pelos próprios autores. A disputa
entre ambos foi conduzida sob o disfarce filosófico de uma luta da
``consciência"-de"-si'' contra a ``substância''; a questão se as histórias
dos milagres evangélicos surgiram no seio do elemento comunitário
por meio da formação mitológica\est\ inconscientemente 
tradicional, ou se seriam fabricadas pelos próprios evangelistas, foi
exagerada na questão se na história mundial a ``substância'' ou a
``consciência"-de"-si'' seria a potência decisivamente 
ativa; e, por fim, veio Stirner,
o profeta do anarquismo atual --- Bakunin tomou muito dele ---, e ultrapassou
o ponto culminante da soberana ``consciência"-de"-si'' com o seu soberano
``Único''.\endnote{Max Stirner.
  \emph{Der Einzige und sein Eigenthum }(\emph{O único e sua
  propriedade}), Leipzig 1845. {[}\versal{N.\,T.}{]}}

Não aprofundaremos mais esse lado do processo de decomposição da escola
hegeliana. Para nós o mais importante é a massa dos jovens hegelianos
mais decisivos que, pelas necessidades práticas da sua luta contra a
religião positiva, retrocedeu em direção ao materialismo anglo"-francês.
E aí entrou em conflito com o sistema de sua escola. Enquanto o
materialismo apreendia a natureza como o único elemento efetivo, esta
representava, no sistema
de Hegel, apenas a ``exteriorização alienante'' da ideia 
absoluta, por assim dizer, uma degradação da ideia; em todas as
circunstâncias, o pensar e o produto de seu pensamento, a ideia, são %MANTIVE
aqui o elemento originário, a natureza, o elemento derivado que, em
geral, apenas existe por meio da condescendência da ideia. E era em torno
dessa contradição que se vagueava, tão bem ou mal quanto se queria
prosseguir.

Aí surgiu a \emph{Essência do cristianismo} de Feuerbach.
Com um só golpe pulverizou a contradição ao colocar, sem rodeios,
o materialismo novamente no trono. A natureza existe independentemente
de qualquer filosofia; ela é o fundamento sobre o qual nós, seres
humanos, crescemos, nós mesmos produtos da natureza; fora da natureza e
dos homens não existe nada, e as essências superiores criadas por nossa
fantasia religiosa são apenas o reflexo fantástico da nossa própria
essência. O encanto estava quebrado; o ``sistema'' foi pelos ares e
jogado para o lado, a contradição, enquanto elemento existente apenas na
imaginação, foi dissolvida. É preciso ter vivido o
efeito libertador desse livro para ter uma noção disso. O entusiasmo\est\ foi
geral: éramos, momentaneamente, todos feuerbachianos. O quanto Marx
saudou com entusiasmo a nova concepção \textbar{} e o quanto ele --- apesar de
todas as reservas críticas --- foi por ela influenciado, \textbar{} pode se ler
na \emph{Sagrada família}.

Mesmo os erros do livro contribuíram para o seu efeito momentâneo. O
estilo beletrista, por vezes também pomposo, assegurou"-lhe um público
grande e foi, ainda assim, um alívio após longos anos de hegelianismo
charlatão abstrato e abstruso. O mesmo vale para a efusiva divinização
do amor que, perante a soberania do ``pensar puro'' que se tornou
insuportável, encontrou uma desculpa, até mesmo uma justificação. Mas o
que não podemos esquecer: justamente a ambas essas fraquezas
de Feuerbach
atrelou"-se o ``socialismo verdadeiro'', que desde 1844 se espalhava pela
Alemanha ``culta'' como uma praga, colocando, no lugar do conhecimento
científico, a fraseologia beletrista, no lugar da emancipação do
proletariado pela transformação econômica da 
produção, a libertação da humanidade por meio do ``amor'', em suma,
perdeu"-se no fastidioso estilo beletrista e no caráter asfixiantemente
amoroso do tipo do senhor Karl Grün.\endnote{Marx e Engels criticam o ``socialismo
  verdadeiro'' de Karl Grün em um dos manuscritos da \emph{Ideologia
  Alemã: ``V. Karl Grün:Die soziale Bewegung in Frankreich und Belgien``
  (Darmstadt 1845) oder Geschichtschreibung des wahren Sozialismus}''
  (Karl Grün: o movimento social na França e Belgica ou a
  historiografica do verdadeiro socialismo). {[}\versal{N.\,T.}{]}}

O que, além disso, não se deve esquecer: a escola
hegeliana estava dissolvida, mas a filosofia hegeliana não havia sido criticamente
superada. Strauss e Bauer extraíram
algo, cada um do seu lado, e viraram a filosofia hegeliana polemicamente
um contra o
outro. Feuerbach quebrou
o sistema e simplesmente o jogou para o lado. Mas não se extermina uma
filosofia simplesmente a qualificando de falsa. E uma obra tão poderosa
como a filosofia
de Hegel,
que teve uma influência tão grande sobre o desenvolvimento espiritual da
nação, não se permitiu ser deixada de lado pelo fato de ser ignorada sem
rodeios. Ela tinha de ser ``superada'' no seu sentido 
próprio, isto é, no sentido em que a sua forma fosse criticamente
aniquilada, porém o novo conteúdo obtido fosse salvo.
Veremos abaixo como isso se deu.

Nesse momento, porém, a Revolução de 1848 colocou de lado, sem % MANTIVE Revolução
cerimônias, toda filosofia, assim como Feuerbach fizera com seu Hegel. E assim também o próprio Feuerbach foi impelido para o pano de fundo.

\quebra

\mbox{}
\vspace{2cm}

\addcontentsline{toc}{chapter}{Segunda parte}
\section{II}

\noindent{}A grande questão fundamental de toda a filosofia, especialmente da % MANTIVE grande e fundamental
moderna, é a relação entre pensar e ser. Desde tempos muito
remotos, em que os homens, ainda em total ignorância sobre a sua própria
constituição corporal e incitados por aparições em sonhos,\endnote{Ainda hoje, entre
selvagens e bárbaros inferiores, é a representação
  universal que as figuras humanas que aparecem em sonhos seriam almas
  que abandonam temporariamente os corpos; o homem efetivo é, portanto,
  considerado também responsável pelas ações que a sua aparição em sonho
  comete diante daquele que sonha. Em \emph{Thurn} {[}Everard Ferdinand
  im Thurn: \emph{Among the Indians of Guiana being sketches chiefly
  antropologic from the interior of British Guiana}{]}, por exemplo,
  isso se encontrou, em 1884, entre os índios na Guiana. {[}\versal{N.\,E.}{]}}
chegaram à representação de que o seu pensar e sentir não seriam uma
atividade do seu corpo, mas de uma alma particular que habita esse
corpo e o abandona com a morte. Desde esses tempos, os seres humanos
tinham de criar pensamentos sobre a relação dessa alma com o mundo
exterior. Se na morte a alma se separava do corpo e continuava a viver,
não havia nenhum motivo para lhe imputar ainda uma morte particular;
assim surgiu a representação de sua imortalidade que, naquele estágio de
desenvolvimento, de modo algum aparece como um consolo, mas como um
destino contra o qual nada se pode e, de modo bastante frequente, como
entre os gregos, como uma positiva infelicidade. Não foi a necessidade
religiosa de consolação, mas a aporia proveniente 
da limitação, igualmente universal, do que começar a fazer com a suposta
alma depois da morte do corpo que levou, de modo geral, à fastidiosa
imaginação da imortalidade pessoal. Por um caminho semelhante surgiram,
através da personificação dos poderes da natureza, os primeiros Deuses
que, na ulterior elaboração das religiões, supõem cada vez mais uma
configuração extramundana, até finalmente surgir --- por meio de um processo
de abstração que se orienta naturalmente pelo curso do desenvolvimento
espiritual, diria quase que se trata de um processo de destilação ---,
na cabeça dos seres humanos, a partir dos muitos Deuses mais ou menos
limitados e que se limitam reciprocamente, a representação de um único e
exclusivo Deus das religiões monoteístas. %MANTIVE

A questão da relação entre pensar e ser, espírito e natureza, a questão
suprema da filosofia como um todo tem a sua raiz, portanto, não menos
do que todas as religiões, nas representações estreitas e ignorantes do
estado de selvageria. Mas ela somente podia ser posta em sua plena
clareza, somente podia alcançar toda a sua significação, quando a
humanidade europeia despertasse da longa hibernação da Idade Média
cristã. A questão da posição do pensar em relação ao ser que, de
qualquer forma, também desempenhou o seu grande papel na escolástica da
Idade Média. A questão é: qual é o elemento originário, o espírito ou a
natureza? Esta questão aguçou"-se diante da Igreja da seguinte forma:
criou Deus o mundo ou o mundo está aí desde a eternidade?

Na medida em que essa questão era respondida de um modo ou de outro, os
filósofos se dividiram em dois grandes campos. Aqueles que afirmavam a
originariedade do espírito diante da natureza, que em última instância admitiam, portanto, uma criação do mundo, de qualquer tipo que fosse ---
e essa criação frequentemente é entre os filósofos, por exemplo
em Hegel,
ainda muito mais complicada e impossível do que no cristianismo ---,
formavam o campo do idealismo. Os outros, que viam a natureza como o
elemento originário, pertenciam às diversas escolas do materialismo.

As duas expressões significam algo diverso: idealismo e materialismo não
são utilizados aqui em sentido original, e tampouco em outros sentidos.
Veremos abaixo qual confusão surge quando se acrescenta algo mais nelas.

Mas a questão da relação entre pensar e ser tem ainda outro lado: como
se relacionam os nossos pensamentos sobre o mundo que nos circunda com
esse mesmo mundo? O nosso pensar está em condições de conhecer o mundo
efetivo, de produzir nas nossas representações e concepções do mundo
efetivo uma imagem especular correta da realidade 
efetiva? Na linguagem filosófica, tal questão corresponde à da
identidade entre pensar e ser, e é respondida afirmativamente pela
grande maioria dos filósofos.
Em Hegel,
por exemplo, a sua resposta afirmativa compreende"-se por si mesma. Afinal, aquilo que nós conhecemos no mundo efetivo é precisamente o seu
conteúdo que está conforme o pensamento, aquilo que torna o mundo uma
efetivação gradual da ideia absoluta. Tal ideia absoluta existiu em
alguns lugares desde a eternidade, independente e anteriormente ao
mundo; mas de tal modo que parece evidente ao pensamento a capacidade de reconhecer um conteúdo que, desde o início, já é um conteúdo do pensamento. É
igualmente evidente que o elemento a ser comprovado aqui já está contido
no pressuposto. Mas isso de forma alguma impede Hegel, a partir de sua
comprovação da identidade entre pensamento e ser, de concluir que sua
filosofia, por ser adequada ao seu pensamento, é também agora a única
adequada; e que a identidade entre pensar e ser pode se comprovar pelo %MANTIVE
fato de que a humanidade traduz imediatamente sua filosofia em
\emph{práxis} a partir da teoria e transforma o mundo inteiro segundo as
proposições fundamentais hegelianas. Essa é uma ilusão que ele
compartilha com praticamente todos os filósofos.

Ao lado desses ainda há, porém, uma série de outros filósofos que
contestam a possibilidade de um conhecimento do mundo ou ainda de um
conhecimento exaustivo. Há entre eles os
modernos, Hume e Kant,
que desempenharam um papel muito significativo no desenvolvimento
filosófico. O elemento decisivo para a refutação dessa perspectiva já foi
dito por Hegel, tanto quanto isso o era possível a partir da posição
idealista; o elemento materialista
que Feuerbach acrescenta
é mais intelectualmente estimulante do que profundo. A refutação mais
convincente desse elemento materialista, como de todas as outras ideias fixas
da filosofia, é a \emph{práxis}, a saber, o 
experimento e a indústria. Se nós podemos demonstrar a correção da nossa
concepção de um processo natural, na medida em que nós mesmos ao
empreendê"-lo o engendramos a partir das suas condições, podemos, acima
de tudo, torná"-lo utilizável para nossos objetivos, estabelece"-se o fim
da ``coisa em si'' inapreensível
de Kant.
As matérias químicas produzidas em corpos vegetais e animais eram as
tais ``coisas em si'', até a química orgânica as ter começado a
apresentar uma após a outra; com isso, a ``coisa em si'' se tornou uma
coisa para nós, como, por exemplo, a matéria corante {[}da planta{]}
ruiva dos tintureiros, a alizarina, que já não permitimos que cresça em
campos nas raízes de ruiva dos tintureiros, mas a produzimos de modo
muito mais barato e simplesmente a partir do alcatrão do carvão mineral.
O sistema solar copernicano foi durante trezentos anos uma hipótese, em
que se podia apostar cem, mil, dez mil contra um, mas, ainda assim, uma
hipótese; quando Le Verrier, porém, a partir dos dados fornecidos por
este sistema, calculou não só a necessidade da existência de um planeta
desconhecido, como também o lugar desse planeta no céu, e quando então
Galle efetivamente encontrou esse planeta,
o sistema copernicano foi, naquele momento, comprovado.\endnote{Referência ao astrônomo Johann Gottfried Galle, o primeiro a
  visualizar Netuno em 23 de setembro de 1846, a partir, como indicado
  por Engels, dos cálculos do astrônomo e matemático francês Urbain Le
  Verrier do mesmo ano. {[}\versal{N.\,T.}{]}} Se, entretanto, a reabilitação da concepção
kantiana é tentada na Alemanha pelos neokantianos\endnote{O
  neokantismo tem início nos anos 1850 e 1860. Sua influência se estende
  até a segunda guerra mundial. Enquanto orientação filosófico"-acadêmica
  a intenção era um retorno à Kant por meio de uma mediação crítica com
  o idealismo hegeliano e diversos materialismos do século \versal{XIX}. A partir
  de 1880 configura"-se como a escola predominante nas universidades de
  filosofia na Alemanha, tendo efeitos diretos em todas as chamadas
  \emph{Geistwissenschaften}. A sua pré"-história teórica delimita
  justamente o problema da relação entre subjetividade e razão, tendo em
  vista o problema do \emph{a priori} kantiano: poderia este ser
  compreendido como ``um instituição orgânica inata ao gênero
  humano'' (Ollig, Hans"-Ludwig, 
  \emph{Der Neukantismus}, p.\,1)? Esse entendimento simplista de um
  ``neokantismo fisiológico'' foi reelaborado e criticado pelas escolas
  neokantianas seguintes. O centro da abordagem passou a ser a filosofia
  prática e metafísica no interior de um sistema transcendental que
  compreende a unidade das três grandes críticas, principalmente a
  partir de 1890. O texto de Engels dialoga justamente com o período de
  prelúdio dessa problemática sistêmica. Os anos 1870 e 1880
  correspondem à fase dos comentadores da obra kantiana, que serviu de
  impulso para fase de discussão sobre a legitimidade de um sistema
  transcendental (\emph{Der Neukantismus}, p. 2), tendo como principais
  representes, com suas diferenças de abordagem, a Escola de Marburg,
  Herman Cohen, Paul Nartop e Ernst Cassier e a Escola do sudoeste
  alemão, como Wilhelm Windelband, Heinrich Rickert e Emil Lask. É
  também determinante a influência do neokantismo no revisionismo da
  social"-democracia alemã, de Kaustky e Bernstein, chegando até o
  marxismo austríaco e ao jovem Lukács, aspecto que reflete a dimensão
  da \emph{distorção sistêmica} \emph{da autonomia da vontade} diante da
  questão da organização revolucionária, isto é, de constituição e
  execução de uma \emph{práxis revolucionária}, de desenvolvimento
  concreto de uma \emph{vontade revolucionária} em interação recíproca
  com as formas políticas burguesas. {[}\versal{N.\,T.}{]}} e a reabilitação de Hume na
Inglaterra (onde nunca morreu) pelos agnósticos,\endnote{A
  alusão aos agnósticos remete certamente a obra do biólogo inglês,
  amigo de Darwin e defensor de sua teoria da evolução, Thomas Henry
  Huxley, que teria cunhado em 1869, sob influência da própria teoria da
  evolução e do ceticismo de Hume, o sentido moderno do termo
  ``agnóstico''. O pano de fundo teórico entre o ceticismo de Hume e
  \emph{a aporia} da ``coisa em si'' kantiana é reflexo de um longo
  processo de alteração do lugar de legitimação teórica e social das
  ciências da natureza no século \versal{XIX}. {[}\versal{N.\,T.}{]}} isso é, cientificamente, um retrocesso diante da
refutação teórica e prática há muito alcançada e,
na prática, apenas um modo envergonhado de aceitar o materialismo
pelas costas e de o negar perante o mundo.

Os filósofos, porém, nesse longo período
de Descartes a Hegel e de Hobbes a Feuerbach,
de modo algum foram impelidos a avançar, como acreditavam, apenas pela
força do puro pensamento. Pelo contrário. O que, na verdade, os impeliu
a avançar foi, nomeadamente, o poderoso e sempre mais veloz
progresso impetuoso da ciência da natureza \textbar{} e da indústria \textbar{}. Nos materialistas
isso já se mostrava na superfície, mas também os sistemas idealistas se
completaram cada vez mais com um conteúdo materialista e procuraram
conciliar a oposição entre espírito e matéria panteisticamente; de tal
modo que, no final, o sistema de Hegel representa apenas um materialismo, segundo método e conteúdo, idealisticamente posto de cabeça para baixo. 

Por isso é concebível que Starcke, em sua caracterização
de Feuerbach,
investigue, antes de tudo, a posição dele no que concerne à questão
fundamental da relação entre pensar e ser. Após uma curta
introdução na qual, em linguagem desnecessária e filosoficamente
prolixa, é exposta a concepção dos filósofos precedentes, isto é,
desde Kant, %MANTIVE
e na qual
Hegel,
por uma retenção demasiadamente formalista a passagens isoladas de suas
obras, é minimizado, segue"-se uma exposição pormenorizada do curso do
desenvolvimento da própria ``metafísica''
de Feuerbach,
tal como resulta da sequência dos escritos deste filósofo.
Essa exposição é trabalhada de modo fluido e claro, apenas
sobrecarregado, como todo o livro, por um lastro, não de
todo inevitável, de modos de expressão filosóficos, que é tanto mais
perturbador quanto menos o autor se atém ao modo de expressão de uma só
e mesma escola --- ou então do
próprio Feuerbach ---
e quanto mais ele mistura expressões no interior das mais diversas
orientações, justamente das que agora se espalham e denominam a si
mesmas de filosóficas.

O curso do desenvolvimento
de Feuerbach é
o de um hegeliano --- nunca, porém, totalmente ortodoxo --- em direção ao
materialismo, um desenvolvimento que, em um determinado estágio,
condiciona uma ruptura total com o sistema idealista de seu predecessor.
No fim, com uma força irresistível, impõe"-se"-lhe a intelecção de que a
existência pré"-mundana da ``ideia absoluta''
de Hegel,
a ``pré"-existência das categorias lógicas'', antes, portanto, de haver
mundo, nada mais é do que um resto fantástico da crença em um criador
extramundano; já que o mundo material, sensivelmente perceptível, ao
qual\est\ nós mesmos pertencemos, é o único elemento efetivo e que a nossa
consciência e o nosso pensamento, por mais que pareçam suprassensíveis, são o
produto de um órgão material, corpóreo, o cérebro. A matéria não é um
produto do espírito, mas o próprio espírito é apenas 
o produto supremo da matéria. Isso é, naturalmente, 
puro materialismo. Ao chegar aqui, Feuerbach empaca. Ele não pode
superar o preconceito filosófico habitual, não contra a
coisa em questão, mas contra o nome materialismo. Ele diz: ``O
materialismo é para mim a base do edifício do ser e saber humanos; mas,
para mim, ele não é nada do que é para o fisiólogo, para o naturalista em
sentido estrito, por exemplo Moleschott,
para o qual, devido a sua posição e profissão, o materialismo é necessariamente o próprio
edifício. No que precede concordo inteiramente com os materialistas, mas
não no que procede.''\endnote{\emph{Ludwig Feuerbach in
  seinem Briefwechsel und Nachlass 1850--1872.} Leipzig, Heidelberg:
  Winter Verlag, Bd. 2, 1874. Citado a partir de Starcke: Ludwig
  Feuerbach, p. 166. {[}\versal{N.\,T.}{]}}

Aqui
Feuerbach coloca
no mesmo saco o materialismo, que é uma visão geral do mundo que repousa
sobre uma determinada concepção da relação entre matéria e espírito,
juntamente com a forma particular pela qual essa visão do mundo se
explicitou em um estágio histórico determinado \textbar{} no século \versal{XVIII} \textbar{}. %COLOCAR textbar no alemão
Mais ainda, coloca"-o junto com a configuração vulgar, superficial, na qual
o materialismo do século \versal{XVIII} ainda continua a existir na cabeça
de naturalistas e médicos e que, nos anos cinquenta, foi pregado por
todos os cantos
por Büchner, Vogt e Moleschott. Porém,
assim como o idealismo passou por uma série de estágios de
desenvolvimento, o materialismo também passou. Juntamente com toda a descoberta
que faz época mesmo no domínio da ciência da natureza, ele tem que mudar
a sua forma; e na medida em que também a história está submetida ao
tratamento materialista, abre"-se aqui um novo traço de
desenvolvimento.

O materialismo do século passado era, sobretudo, mecânico, já que, de
todas as ciências da natureza daquele tempo, apenas a mecânica, e
justamente apenas a mecânica dos corpos\est\ sólidos --- celestes e terrestres ---,
em suma, a mecânica dos corpos pesados, tinha chegado a certa conclusão.
A química somente existia em sua configuração infantil, flogística.
A biologia ainda
usava fraldas; apenas de modo grosseiro o organismo vegetal e animal
era investigado e explicado por causas puramente mecânicas; assim
como
para Descartes o
animal era uma máquina, o homem o era para os materialistas do século
\versal{XVIII}. Essa aplicação exclusiva do padrão da mecânica a processos que
são de natureza química e orgânica e para os quais as leis mecânicas
certamente também se aplicam, apesar de impelidas para o segundo plano
por leis superiores, forma a primeira limitação específica do
materialismo francês clássico, ainda que inevitável para seu tempo.

A segunda limitação específica desse materialismo consistiu na sua
incapacidade de apreender o mundo como um processo, como uma matéria
concebida em uma formação historicamente contínua. Isso correspondia ao
estado da ciência da natureza daquela época e ao modo
metafísico --- isto é, antidialético --- do filosofar a ela vinculado. A natureza era
concebida, isso era consciente, como um movimento eterno. Mas esse
movimento, segundo a representação daquela época, girava em um círculo
eterno e, portanto, nunca saía do lugar; produzia novamente os
mesmos resultados. Essa representação era inevitável naquela época. A
teoria
de Kant sobre
a gênese do sistema solar mal havia se estabelecido e ainda não passava
de mera curiosidade. A história do desenvolvimento da Terra, a geologia,
era ainda totalmente desconhecida, e a representação de que os atuais seres
vivos naturais são o resultado de uma longa série de
desenvolvimentos do simples ao complexo não podia, naquela época, ser,
em geral, cientificamente estabelecida. A concepção a"-histórica
da natureza era, portanto, inevitável. 
 \textbar{}\,Tão pouco é possível censurar os filósofos do século \versal{XVIII} por\est\ isso,
como também é pouco possível censurar Hegel. Para ele, a natureza, como mera
``exteriorização alienante'' da ideia, não é capaz 
de nenhum desenvolvimento no tempo, mas apenas de uma extensão de sua
multiplicidade no espaço, de tal modo que expõe simultânea e
sucessivamente todos os estágios de desenvolvimento nela compreendidos e
está condenada à repetição eterna do mesmo processo. E nesse
absurdo de um desenvolvimento no espaço, porém fora do tempo --- a
condição fundamental de todo o desenvolvimento ---,
Hegel coloca
um grande peso na natureza, justamente ao mesmo tempo em que a geologia,
a embriologia, a fisiologia vegetal e animal e a química orgânica foram
desenvolvidas e em que, por toda a parte, na fundamentação dessas novas
ciências, emergiam pressentimentos geniais da posterior teoria da
evolução (por exemplo Goethe e Lamarck).
Mas o sistema assim exigia, e o método precisava, por amor ao sistema,
ser infiel consigo mesmo.\,\textbar{} A mesma concepção a"-histórica
vigorava no âmbito da história. Aqui, a luta contra os restos da Idade
Média tornava a visão parcial. A Idade Média era considerada como
simples interrupção da história por uma barbárie 
universal de mil anos; os grandes progressos da Idade Média não eram vistos --- a expansão
do âmbito cultural europeu, as grandes nações que sobreviveram até hoje,
que ali se formaram uma ao lado da outra, por fim, os enormes progressos
técnicos dos séculos \versal{XIV} e \versal{XV}. Uma
intelecção
racional do grande nexo histórico tornou"-se assim
impossível, e a história servia, no máximo, como uma coleção de exemplos
e ilustrações para uso dos filósofos.

Os vendedores ambulantes vulgarizadores que nos anos cinquenta na
Alemanha se fizeram no materialismo, de modo algum ultrapassaram essa
barreira de seus mestres. Todos os progressos da ciência da natureza
feitos desde então lhes serviam apenas como novos fundamentos de
comprovação contra a existência do criador do mundo; e, de fato, estava
totalmente fora de questão continuar a desenvolver a teoria. Se o
idealismo tinha esgotado o seu latim e tinha se deparado com a morte por
meio da Revolução de 1848, vivenciou assim a satisfação de ver que o
materialismo momentaneamente ainda tinha caído mais
baixo. Feuerbach tinha
decididamente razão quando declinava a responsabilidade por esse
materialismo; apenas não podia confundir a doutrina dos pregadores
ambulantes com o materialismo em geral.

Entretanto, é preciso aqui observar duas coisas. Em primeiro lugar, ao
longo da vida
de Feuerbach,
a ciência da natureza era ainda compreendida naquele intenso processo de
fermentação que só nos últimos quinze anos recebeu um desfecho relativo,
esclarecedor; um novo material de conhecimento foi fornecido em uma
medida até aqui sem precedentes, mas o estabelecimento
da conexão e, assim, da ordem nesse caos de 
descobertas precipitadas, só se tornou possível muito recentemente. De
fato,
 Feuerbach ainda
vivenciou as três descobertas decisivas --- a da célula, a da
transformação da energia e da, denominada
por Darwin,
teoria da evolução. Mas como o solitário
filósofo poderia, no campo, perseguir suficientemente a ciência para
apreciar plenamente descobertas que os próprios naturalistas daquele
tempo, em parte ainda contestavam, em parte não compreendiam
suficientemente? A culpa reside aqui unicamente nas miseráveis condições
alemãs, em virtude das quais as cátedras da filosofia eram monopolizadas
por espirituosos e ecléticos esmagadores de pulgas,
enquanto Feuerbach,
que os superava do alto de uma torre, tinha que se tornar um camponês e
se atrofiar em um pequeno \emph{Dorf}. Não é, portanto, culpa
de Feuerbach
se a, agora tornada possível, concepção histórica da natureza que põe de
lado \textbar{} afasta \textbar{} todas as parcialidades do materialismo francês
permanecesse inacessível para ele.

Em segundo lugar,
porém, Feuerbach tem
toda a razão, já que o materialismo meramente científico"-natural é ``o
fundamento do edifício do saber humano, mas não o próprio edifício''.
Afinal, não vivemos apenas na natureza, mas também na sociedade humana,
e essa tem também a sua história de desenvolvimento e a sua ciência,
tanto quanto a natureza. Tratava"-se, portanto, de estabelecer uma
harmonia entre a ciência da sociedade, isto é, o complexo interno
das assim chamadas ciências históricas e filosóficas, 
e a fundamentação materialista e de reconstruí"-las sobre esse
fundamento. Isto, porém, não foi concedido
a Feuerbach.
Ele permaneceu aqui, apesar da ``fundamentação'', preso aos laços
idealistas tradicionais, e ele reconheceu isso nominalmente:
``Retrospectivamente, concordo com os materialistas, mas não
progressivamente.'' Mas quem aqui, no domínio social, não avançou
``progressivamente'', não ultrapassou a sua posição de 1840 ou de 1844,
foi o
próprio Feuerbach e,
de fato, uma vez mais, principalmente na sequência do seu isolamento,
que o impeliu a produzir pensamentos a partir da sua 
cabeça solitária --- ele que, mais do que todos os outros filósofos, era
predisposto ao intercâmbio sociável ---, em vez de os produzir no encontro
amigável e hostil com outros homens de seu calibre. O quanto, neste
âmbito, ele permaneceu idealista, veremos mais a frente em detalhes.

Aqui é preciso apenas observar que Starcke procura o idealismo
de Feuerbach no
lugar errado.
``Feuerbach é
idealista, ele acredita no progresso da humanidade.'' (p. 19) --- ``A
fundamentação, a infraestrutura do todo permanece, 
não obstante, o idealismo. O realismo não é para nós senão uma proteção
contra caminhos equivocados, enquanto seguimos as nossas correntes
ideais. Não são compaixão, amor e entusiasmo pela verdade e pela
justiça, forças ideais?'' (p. \versal{VIII}.)

Em primeiro lugar, idealismo aqui nada mais é do que persecução de
finalidades ideais. Tais finalidades, porém, remetem necessariamente, no
máximo, ao idealismo
de Kant e
o seu ``imperativo categórico''; porém,
mesmo Kant 
denominou sua filosofia de ``idealismo transcendental'', de modo algum,
por tratar de ideais éticos, mas por razões totalmente diferentes,
como Starcke recordará. A superstição, segundo a qual o idealismo
filosófico giraria em torno da crença em ideais éticos, isto é, sociais,
surgiu do lado de fora da filosofia, entre filisteus alemães que
aprenderam a decorar nos poemas
de Schiller alguns
pedaços da formação cultural filosófica que lhes era necessária. Ninguém
criticou mais agudamente o impotente ``imperativo categórico''
de Kant ---
impotente, porque ele exige o elemento impossível, portanto, que nunca
chega a ser algo efetivo ---, ninguém zombou de modo mais agudo do
entusiasmo filisteu por ideais irrealizáveis, transmitidos
por Schiller,
do que justamento o idealista perfeito, Hegel (ver, por exemplo,
a ``\emph{Fenomenologia}'').\endnote{A crítica ao imperativo
  categórico kantiano na \emph{Fenomenologia do Espírito} é
  desenvolvida, principalmente, na formulação da \emph{Verstellung}
  (deslocamento dissimulador). Uma passagem fundamental da recepção
  crítica da \emph{Razão pratica} de Kant, ponto em que Hegel expõe a
  fragilidade conceitual que julga ver no dualismo kantiano: a resolução
  da antinomia entre natureza e moralidade, entre a estrutura
  normativa"-causal da natureza e a estrutura normativa da moralidade,
  via postulado de um ``\emph{sumo bem originário''} (\emph{KpV}, A 226
  ). O centro da crítica de Hegel é justamente ao apriorismo da solução
  kantiana, à fragilidade dessa indeterminação (aqui a distorção e
  dissimulação) do lugar (do \emph{Stellung}) da consciência moral, que
  justificada como \emph{a priori}, ignora o caráter ativo da
  consciência e se torna uma mera projeção ideal que pode assumir
  qualquer conteúdo. É provavelmente a esse aspecto da crítica que
  Engels se refere. {[}\versal{N.\,T.}{]}}

Em segundo lugar, não é possível evitar que tudo o que move o homem
tenha que passar por sua cabeça --- até mesmo comer e beber, que começa
como resultado da fome e sede sentidas pela cabeça e também termina como
resultado da satisfação por meio da cabeça. Os efeitos do mundo exterior
sobre o homem se expressam na sua cabeça, refletem"-se no interior dela
como sentimentos, pensamentos, impulsos, determinações da vontade, em
suma, como ``correntes ideais'' e se tornam, nessa cofiguração,
``poderes ideais''. Aqui, se a circunstância segundo a qual esse homem,
em geral, ``segue correntes ideais'' e concede aos ``poderes ideais''
uma influência sobre si --- se isto faz dele um idealista, então todo o
homem, de algum modo desenvolvido, é um idealista nato, e como é
possível, em geral, ainda existirem aí materialistas?

Em terceiro lugar, a convicção de que a humanidade, pelo menos
momentaneamente, move"-se, de maneira geral, em direção progressiva não
tem absolutamente nada a ver com a oposição entre materialismo e
idealismo. Os materialistas franceses tinham esta convicção em grau
quase fanático, não menos do que os
deístas Voltaire e Rousseau,
e fizeram, de modo suficientemente frequente, os maiores sacrifícios
pessoais a ela. Se alguma vez alguém consagrou a vida toda ao
``entusiasmo pela verdade e pela justiça'' --- tomando a fraseologia no
bom sentido ---, foi, por
exemplo, Diderot. Se Starcke esclarece
assim tudo isso como idealismo, demonstra apenas que a palavra
materialismo, e toda oposição em ambas as direções, perdeu aqui para ele
todo o sentido.

O fato é que Starcke faz aqui uma concessão imperdoável, ainda que
talvez inconscientemente, ao preconceito filisteu contra o nome
materialismo resultante de longos anos de abuso sacerdotal. O filisteu
entende por materialismo voracidade, bebedeira, cobiça do olhar, prazer
carnal e ambição, avidez monetária, avareza, ânsia de posse, maquinação
do lucro e agiotagem, em suma, todos os vícios engordurados, aos quais
ele próprio se entrega em segredo; e por idealismo, a crença na virtude,
no amor humano universal e, em geral, em um ``mundo melhor'', com o que
se exibe diante de outros, mas nos quais ele próprio acredita, no
máximo, enquanto se preocupa em através sar a ressaca moral 
ou a bancarrota que necessariamente se seguem aos
seus habituais excessos ``materialistas'' e assim canta a sua cantiga
predileta: que é o homem --- meio animal, meio anjo.

De resto, Starcke esforça"-se muito para
defender Feuerbach dos
ataques e teoremas dos professores de segundo escalão
que hoje se propagam na Alemanha sob o nome de filósofos. Para pessoas
que se interessam por essa placenta da filosofia clássica alemã,
certamente isso é importante; para o próprio Starcke, isso pode até
parecer necessário. Nós pouparemos os leitores disso.

\quebra

\begin{flushright}
\emph{Die Neue Zeit. Ano 4.}\\
\emph{1886. Caderno 5, Maio}
\end{flushright}

\vspace{2cm}

\addcontentsline{toc}{chapter}{Terceira parte}
\section{III}

\noindent{}O idealismo efetivo
de Feuerbach se
explicita assim que chegamos à sua filosofia da religião e ética. Ele
não quer de modo algum abolir a religião, ele quer realizá"-la. A própria
filosofia deve se transformar em religião. ``Os 
períodos da humanidade se diferenciam apenas pelas transformações
religiosas. Um movimento histórico somente aceita o fundamento, quando
aceita o coração do homem. O coração não é uma forma da religião, como
se ela também devesse estar no coração; o coração é a essência da
religião.'' (Citado por Starcke, p. 168.) A religião é,
segundo Feuerbach,
a relação de sentimentos, a relação de corações entre os homens, que até
então procurava a sua verdade em uma imagem especular fantástica da
realidade efetiva --- na mediação por meio de um ou de muitos Deuses,
imagens especulares fantásticas de qualidades humanas ---, mas agora a
encontra diretamente e sem mediação no amor entre o Eu e o Tu.
Oara Feuerbach,
o amor sexual torna"-se, assim, uma das mais elevadas, se não a mais
elevada, forma de exercício de sua nova religião.

Afinal, as relações de sentimentos entre os homens, isto é, entre os
dois sexos, têm existido desde que há seres humanos. Particularmente o
amor sexual conheceu um desenvolvimento nos últimos oitocentos anos e
conquistou uma posição que, durante este tempo, o fizeram eixo
obrigatório de toda poesia. As religiões positivas existentes se
limitaram\est\ a consagrar de modo elevado a regulação estatal do amor
sexual, isto é, a legislação do matrimônio, e amanhã podem desaparecer
conjuntamente sem que na prática do amor e da amizade o mínimo tenha
sido alterado. Assim como a religião cristã na França também desapareceu
factualmente de 1793 a 1798, de tal modo que nem o próprio Napoleão pôde
reintroduzi"-la sem relutância e dificuldade, sem que, porém, tenha
surgido, durante esse intervalo, a necessidade de uma substituição, no
sentido
de Feuerbach.

Para Feuerbach, o idealismo consiste aqui no fato de simplesmente
permitir considerar as relações dos homens com base na inclinação mútua
entre si, amor sexual, amizade, compaixão, sacrifício etc., relações
estas que não são recordadas a partir de si mesmas sem que se recorde de
uma religião particular, mesmo as que para ele pertencem ao passado,
afirma, pelo contrário, que elas só alcançam sua validade plena assim
que recebem uma consagração superior sob o nome religião. A questão
principal para ele não é que essas inter"-relações puramente humanas
existam, mas que elas sejam apreendidas como a nova, verdadeira,
religião. Elas só devem ter validade plena se receberam o selo
religioso. Religião vem
de \emph{religare} e, originariamente,
significa ligação. Toda a ligação entre dois homens é, portanto, uma
religião. Tais artifícios etimológicos formam o último meio de
transmissão da filosofia idealista. O que deve valer não é o que a
palavra significa segundo o desenvolvimento histórico do seu uso
efetivo, mas o que deveria significar segundo sua descendência. E assim
o amor sexual e o vínculo sexual são elevados ao céu de uma
``religião'', para que a palavra religião, cara à lembrança idealista,
não desapareça da linguagem. É justamente assim que, nos anos quarenta,
falavam os reformistas de Paris da orientação de Louis Blanc, os quais,
igualmente, só podiam imaginar um homem sem religião como um monstro e
nos diziam: \emph{Donc, l'athéisme c'est votre
religion}!\endnote{``Portanto,
  o ateísmo é vossa religião!''. {[}\versal{N.\,T.}{]}}\est\
Feuerbach querer
estabelecer a verdadeira religião tendo como fundamento uma visão da
natureza essencialmente materialista, significa o mesmo que apreender a
química moderna como a verdadeira alquimia. Se a religião pode existir
sem o seu Deus, então a alquimia também pode sem a sua pedra filosofal.
Existe, aliás, uma conexão muito estreita entre alquimia e religião. A
pedra filosofal tem muitas propriedade semelhantes às divinas, e os
alquimistas greco"-egípcios dos dois primeiros séculos da nossa Era
tiveram um influencia na formação da doutrina cristã, como os dados
fornecidos por Kopp e Berthelot comprovam.\endnote{A relação
  entre alquimia e religião aparece em Hermann Kopp, \emph{Geschichte
  der Chemie} {[}História da química{]}, 1843 e em Marcelin Berthelot,
  ``\emph{Les origines de l´alchimie}'', 1885. {[}\versal{N.\,T.}{]}}

É absolutamente falsa a afirmação
de Feuerbach
de que os ``períodos da humanidade se diferenciam apenas por
transformações religiosas''. \textbar{} Grandes pontos de mudança histórica
foram \emph{acompanhados }por transformações religiosas, na medida em
que sejam consideradas apenas as três religiões mundiais que até agora
existiram: budismo, cristianismo, islamismo \textbar{}. As velhas religiões
tribais e nacionais, que surgiram de modo espontâneo"-natural, \textbar{} não
faziam propagandas e perderam \textbar{} todo o poder de resistência logo que a
autonomia das tribos e povos foi rompida; entre os germanos, bastou
inclusive o simples contato com o império mundial romano em decadência e
com a religião mundial cristã por ele recentemente adotada e conveniente
a seu estado econômico, político e ideal. Somente nessas religiões
mundiais, que surgem de modo mais ou menos artificial, particularmente
no cristianismo e islamismo, encontramos movimentos históricos mais
gerais que adotam um caráter religioso e, \textbar{} mesmo no âmbito do
cristianismo \textbar{}, o caráter religioso limita"-se, nas revoluções com
significado efetivamente universal, aos primeiros estágios da luta de
emancipação da burguesia, do século \versal{XIII} ao século \versal{XVII}, e não se
explica, como pensa
Feuerbach,
pelo coração do homem\est\ e por sua carência religiosa, mas por toda a
pré"-história medieval, que não conhecia outra forma de ideologia além da
religião e a teologia. Quando, porém, no século \versal{XVIII}, a burguesia se
fortaleceu o suficiente para ter a sua própria ideologia, adequada ao
sua posição de classe, fez, então, a sua grande e definitiva revolução,
a francesa, sob o apelo exclusivo a ideias jurídicas e políticas e só se
preocupou com a religião na medida em que ela se colocava no caminho;
mas não lhe ocorreu estabelecer uma nova religião no lugar da antiga; \textbar{}
sabemos como Robespierre\endnote{Referência ao \emph{Culte de l'Être suprême} (Culto do Ser supremo) desenvolvido por Robespierre que, inspirado pelo deísmo de Voltaire e
  pelo teísmo cristão de Rousseau, visava se opor ao ateísmo radical do
  \emph{Culta da Razão} de Joseph Fouché. {[}\versal{N.\,T.}{]}} fracassou nisso. \textbar{}

Hoje em dia a possibilidade de um sentimento puramente humano no
intercâmbio com outros homens já se atrofiou para nós o suficiente
devido à sociedade, na qual temos que nos movimentar, fundada na
oposição de classes e na dominação de classe: não há razão para deixar
ela nos atrofiar ainda mais elevando aos céus esses sentimentos em uma
religião. E, do mesmo modo, a compreensão das grandes lutas de classes
históricas torna"-se para nós já suficientemente obscurecida pela
historiografia corrente, especialmente na Alemanha, sem que nós, pela
transformação dessa história de lutas em um mero apêndice da história da
Igreja, também tenhamos necessidade de torná"-la completamente impossível
para nós. Já aqui fica claro o quanto hoje estamos distantes
de Feuerbach.
As suas ``mais belas passagens'', de celebração dessa nova religião do
amor, são absolutamente inteligíveis hoje.

A única religião
que Feuerbach investiga
seriamente é o cristianismo, a religião mundial do ocidente, fundada no
monoteísmo. Ele demonstra que o Deus cristão é apenas o reflexo
fantástico, a imagem especular do homem. Afinal, esse mesmo Deus é,
porém, o produto de um longo processo de abstração, a quintessência
concentrada dos muitos Deuses anteriores, de tribos e nações. E,
correspondente, o homem cuja imagem\est\ é esse Deus também não é um homem %se com sentido de advérbio, "correspondentemente"
efetivo, mas igualmente a quintessência dos muitos homens efetivos, o
homem abstrato, portanto ele próprio novamente uma imagem do
pensamento. O
mesmo Feuerbach que
a cada página prega a sensibilidade, o mergulho no elemento concreto, na
realidade efetiva, torna"-se, por todos os lados, abstrato, na medida em
que fala de um intercâmbio entre homens mais amplo do que o mero
intercâmbio sexual.

Esse intercâmbio só lhe oferece um lado: a moral. E aqui nos
surpreendemos novamente com a espantosa pobreza
de Feuerbach comparado
com Hegel.
A ética ou doutrina da eticidade de Hegel é a filosofia do direito,
abarcando: 1. o direito abstrato, 2. a moralidade, 3. a eticidade, sob
a qual, por sua vez, estão reunidos: a família, a sociedade
civil"-burguesa, o Estado. A forma é tão idealista quanto o conteúdo é
aqui realista. Todo o domínio do direito, da economia, da política, é
apreendido conjuntamente com a moral.
Em Feuerbach,
ocorre justamente o contrário. Ele é realista segundo a forma, ele parte %segundo a forma, e parte
do homem; mas não se fala absolutamente nada do mundo onde esse homem
vive e, assim, esse homem permanece sempre o mesmo homem abstrato que na
filosofia da religião detinha a palavra. Esse homem não nasceu %Esse homem não nasceu do corpo da mãe, mas sim revelou"-se (...), e, consequentemente, (...)
justamente do corpo da mãe, revelou"-se do Deus das religiões
monoteístas, consequentemente, também não vive em um mundo efetivo que
surgiu historicamente e foi determinado historicamente; de fato, ele
entra em intercâmbio com outros homens, mas cada um dos outros são tão
abstratos quanto ele. Na filosofia da religião, temos ainda homem e
mulher, mas na ética essa última diferença também desaparece.
Em Feuerbach
aparecem, de fato, longos intervalos, proposições como: ``Em um palácio
pensa"-se de modo diferente do que em uma cabana.'' --- ``Onde, diante da
fome, da miséria, tu não tens matéria nenhuma no corpo, não tens também
na cabeça, nos sentidos e no coração, matéria
para a moral.'' --- ``A política tem de se tornar a nossa
religião'' etc.\endnote{\versal{FEUERBACH}, L. ``\emph{Grundsätze der Philosophie. Nothwendigkeit einer Veränderung. 1842/43}''. \versal{IN}: \emph{Ludwig Feuerbach in seinem
  Briefwechsel und Nachlass 1850--1872.} Leipzig; Heidelberg: Winter
  Verlag, Bd. 2, 1874. Citado a partir de Starcke: Ludwig Feuerbach, 1ª
  edição, p. 280. {[}\versal{N.\,T.}{]}}
Mas Feuerbach não
sabe absolutamente como começar a agir com essas proposições, elas
permanecem puros modismos de fala, e o próprio Starcke tem de admitir
que a política era
para Feuerbach um
limite intransponível e que a ``doutrina da sociedade, a sociologia, era
para ele uma \emph{terra
incógnita}''.\endnote{Starcke: Ludwig Feuerbach, 1ª edição, p. 280. {[}\versal{N.\,T.}{]}}

Diante de
 Hegel,
parece igualmente superficial no tratamento da oposição entre bem e mal.
``Crê"-se que se diz algo muito grande'' --- vemos
em Hegel ---
``quando se diz: o homem é bom por natureza; mas esquecemos que dizemos
algo ainda maior com as palavras: o homem é mau por natureza.''\endnote{\emph{Grundlinien der Philosophie des Rechts},
  \emph{oder Naturrecht und Staatswissenschaft}. Hrsg. von Eduard Gans.
  2. Aufl. Berlin, 1840, § 18: ``Em relação à \emph{adjudicação}
  (\emph{Beurtheilung}) dos impulsos na dialética do fenômeno
  (\emph{Erscheinung}), por {[}aparecerem{]} como \emph{imanentes}, por
  isso, \emph{positivas}, as determinações imediatas da vontade são
  \emph{boas}; o homem é \emph{assim bom por natureza}. Porém, na medida
  em que são \emph{determinações da natureza}, isto é, opostas à
  liberdade e ao conceito de espírito em geral, e que são o
  \emph{elemento negativo}, precisam ser \emph{exterminadas}, \emph{o
  homem é assim mau por natureza}. O elemento decisivo a favor ou contra
  uma ou outra afirmação é, a partir de seu posicionamento
  (\emph{Standpunkt}), igualmente o arbítrio subjetivo''. {[}\versal{N.\,T.}{]}}
Em Hegel,
o mal é a forma em que a força motriz do desenvolvimento histórico se
apresenta. E de fato aqui reside o duplo sentido segundo o qual, por um
lado, cada novo progresso aparece necessariamente como um sacrilégio
contra um elemento sagrado, como rebelião contra situações antigas,
atrofiadas, mas sacralizadas pelo hábito, e, por outro lado, desde o
aparecimento das oposições de classes, são justamente as piores paixões
dos homens, cobiça e ânsia de domínio, que se 
tornaram alavancas do desenvolvimento histórico, das quais, por exemplo, %e das quais
a história do feudalismo e da burguesia são uma única e contínua prova.
Não ocorre, porém, a 
Feuerbach 
investigar o papel histórico do mal moral. Em geral, a história é para %Não ocorre a Feuerbach, porém
ele um campo desagradável, monstruoso. Nesse sentido, inclusive, o seu
dito: ``O homem que originariamente surgiu da natureza era apenas também
uma pura essência da natureza, não era homem. O homem é um produto do
homem, da cultura, da história'',\endnote{\versal{FEUERBACH}, L. ``Fragmente zur Characteritik meines philosophischen
  Curriculum vitae''. In. \emph{Sämtliche Werke.} Bd. 2, p. 411. Citado
  segundo Starcke: Ludwig Feuerbach. 1ª edição, p. 114. {[}\versal{N.\,T.}{]}} mesmo esse
dito permanece para ele completamente improdutivo.

O
que Feuerbach nos
indica sobre moral somente pode, de acordo com isso, ser algo
extremamente pobre. O impulso para a felicidade é inato ao homem e tem
de formar, portanto, a fundamentação de toda a moral. Mas o impulso para
a felicidade experimenta uma dupla correção.\endnote{Essa
  dupla correção ou limitação recíproca dos impulsos naturais para
  felicidade de um individuo frente ao outro é apresentada por Feuerbach
  em sua ``filosofia moral'': ``Zur Moralphilosophie (1868)'' In:
  \emph{Ludwig Feuerbach in seinem Briefwechsel und Nachlass}, Bd. 2. {[}\versal{N.\,T.}{]}}
Em primeiro lugar, pelas consequências naturais das nossas ações: à
bebedeira segue"-se a ressaca, aos excessos habituais a doença. Em
segundo lugar, pelas suas consequências sociais: se não respeitamos o
mesmo impulso dos outros para a felicidade, eles irão se defender e
perturbaram o nosso próprio impulso para a felicidade. Segue"-se daqui %perturbarão?
que nós, para satisfazermos o nosso impulso, temos de estar em condições
de avaliar de modo correto as consequências das nossas ações e temos,
por outro lado, de estar em condições fazer valer a igualdade de direito
dos outros em relação ao impulso correspondente. Autodelimitação
racional em relação a nós próprios e ao amor --- sempre novamente o amor!
--- no intercâmbio com os outros são, portanto, as regras fundamentais da %da qual
moral feuerbachiana, a partir das quais todas as outras derivam. E nem
as mais espirituosas exposições
de Feuerbach,
nem os mais vigorosos elogios de Starcke, podem esconder a fraqueza e a
banalidade desse par de proposições.

O impulso para a felicidade satisfaz"-se apenas muito excepcionalmente e
de modo algum em benefício de si e de outras pessoas, através da
ocupação de um homem consigo mesmo. Requer, porém, ocupação com o mundo
exterior, com os meios de satisfação, portanto, alimentação, um
indivíduo do outro sexo, livros, conversas, debates, atividade, objetos
para uso e elaboração. A moral
de Feuerbach ou
pressupõe que estes meios e objetos de satisfação sejam dados sem mais a
todo homem, ou ela lhe dá, porém, apenas boas doutrinas inaplicáveis.
Não vale, portanto, absolutamente nada para as pessoas às quais que esses
meios faltam. E o
próprio Feuerbach nos
explica isso com palavras duras: ``Em um palácio pensa"-se de modo
diferente do que em uma cabana.'' ``Onde, diante da fome, da miséria, tu
não tens matéria nenhuma no corpo, não tens também na cabeça, nos
sentidos e coração, matéria para a moral.''


As coisas ficarão melhores com a igualdade de direito em relação ao
impulso de felicidade do
outro? Feuerbach 
apresenta essa reivindicação como absolutamente válida para todas as
épocas e circunstâncias. Mas desde quando ela vale? Na Antiguidade,
entre escravos e senhores, na Idade Média, entre servos e barões,
tinha"-se em vista a igualdade de direito em relação ao impulso para a
felicidade? O impulso para a felicidade da classe oprimida não era, de
modo brutal e ``de direito'', sacrificado em prol do impulso de
felicidade da dominante? --- Sim, isso também era imoral, mas agora a
igualdade de direito é reconhecida. --- Reconhecida na fraseologia, desde
que é visto que a burguesia, na sua luta contra a feudalidade e no
desenvolvimento da produção capitalista, foi obrigada a abolir todos os
privilégios estamentais, isto é, pessoais, e a introduzir a igualdade
jurídica de direito da pessoa, primeiro, a do direito privado, depois
também, gradualmente, a do direito estatal. Mas o impulso para a
felicidade não vive senão, minimamente, de direitos ideais e, na maior
parte, de meios materiais; e a produção capitalista cuida para que caiba
à grande maioria das pessoas com direitos iguais apenas o necessário a
uma vida estreita, e portanto mal respeita, se é que em geral respeita, a
igualdade de direito do impulso da maioria para a felicidade mais do que
a escravidão ou a servidão o fizeram. E essa é melhor no que concerne %não fica claro a que o "essa" se refere. Se a "produção capitalista", talvez "E essa produção é"
aos meios espirituais da felicidade, aos meios de formação cultural? Não
é o próprio ``mestre"-escola de Sadowa''\endnote{Possivelmente uma referência à Batalha de Königgrätz (ou de Sadowa, cidade da atual República Tcheca) ocorrida
  em três de julho de 1866, enfrentamento decisivo da Guerra Austro"-Prussiana, com vitória da Prússia.
  A disputa se deu no contexto do processo de unificação da Alemanha. O
  ``mestre"-escola de Sadowa'' aponta possivelmente para vitória da
  Prússia e a sobreposição de seu sistema educacional sobre o austríaco.
  Possível alusão também à obra do clérigo católico e membro da Câmara
  dos Deputados da Baviera, crítico das imposições estatais e vigilância
  dos sistemas educacionais, Josef Lukas. O livro em justamente o mesmo
  titulo: \emph{Der Schulmeister von Sadowa.} Mainz, 1878. {[}\versal{N.\,T.}{]}} uma
personagem mítica?

Mais ainda. Segundo a teoria da moral %Mas ainda: segundo...
de Feuerbach,
a bolsa de valores é o templo supremo da eticidade --- 
pressupondo apenas que se especula sempre corretamente. Se o meu impulso
para a felicidade me conduz à bolsa de valores e lá eu pondero
corretamente as consequências das minhas ações de tal modo que elas só
me trazem vantagem e nenhum prejuízo, isto é, eu sempre ganho, a
prescrição
de Feuerbach está
cumprida. Também não interfiro no mesmo impulso de felicidade de outra
pessoa, afinal o outro, assim como eu, dirigiu"-se por livre vontade à
bolsa, seguiu seu impulso de felicidade ao fechar o negócio especulativo
tanto quanto eu fizera. E se ele perde seu dinheiro, sua ação
comprova"-se por meio disso ser mal\est\ calculada, como imoral,
e ao levar a cabo a pena que ele merece, posso até
ufanar orgulhosamente como um Rhadamanthus moderno. O amor domina também
na Bolsa, na medida em que ele não é mera fraseologia sentimental,
afinal, cada um encontra no outro a satisfação do seu impulso para a
felicidade, e é justamente isso que o amor deve cumprir e a isso que ele
se dedica na pratica. E se eu aí jogar possuindo a previsão correta das
consequências das minhas operações, portanto, se eu jogar com sucesso,
realizarei todas as mais rigorosas exigências da moral
de Feuerbach e
me tornarei, além disso, um homem rico. \textbar{} Dito de outro modo: a moral
de Feuerbach está
talhada pela atual sociedade capitalista, por mais que ele próprio não
queira isso ou possa suspeitar.\textbar{}

Mas o amor! --- Sim, o amor é em toda parte e sempre o Deus da fascinação
que,
em Feuerbach,
deve ajudar a superar todas as dificuldades da vida prática --- e isto
numa sociedade que está cindida em classes com interesses diametralmente
contrapostos. Desse modo, desapareceu da filosofia, portanto, o último %talvez cortar o "portanto": "desse modo" já introduz a explicativa
resto do seu caráter revolucionário, e permanece apenas a velha
lenga"-lenga: amai"-vos uns aos outros, derramai"-vos sobre os braços uns
dos outros, sem diferença de gênero e estamento --- o devaneio da
reconciliação universal!

Em poucas palavras. Passa"-se pela teoria moral %"Em poucas palavras: passa"-se"?
de Feuerbach
assim como se passa pela de todos seus predecessores. Tal teoria está
talhada em todos os tempos, em todos os povos, em todas as situações, e,
precisamente por isso, ela nunca, e em parte alguma, é aplicável,
permanecendo diante do mundo efetivo tão impotente quanto o imperativo
categórico
de Kant.
Na realidade, cada classe, inclusive cada tipo profissional, tem sua
própria moral, e rompe com esta onde o pode fazer impunemente, e o amor,
que tudo deve unir, vem à luz do dia em guerras, conflitos, processos,
barulhos domésticos, divórcios e na máxima exploração possível de uns
pelos outros.

Mas como era possível que o impulso violento, dado
por Feuerbach, tenha
chegado a ele próprio de modo tão improdutivo? Simplesmente por
Feuerbach não
conseguir encontrar o caminho que parte do reino das abstrações,
mortalmente odiadas por ele mesmo, em direção à realidade efetiva viva.
Ele se agarrou com toda força à natureza e ao homem; mas, natureza e
homem permanecem para ele meras palavras. Ele não sabe nos dizer algo
determinado nem sobre a natureza efetiva, nem sobre o homem efetivo. No
entanto, somente se chega do homem abstrato
de Feuerbach aos
homens vivos efetivos caso estes sejam considerados agindo no interior
da história. Contra isso,
Feuerbach se
opõe e, por isso, o ano de 1848, que ele não compreendeu, significou %tanto "isso" como "portanto" configuram certa repetição. Talvez "dessa maneira"?
para ele apenas a ruptura definitiva com o mundo efetivo, o recolhimento
para a solidão. Por outro lado, a culpa disso se deve principalmente às
relações na Alemanha, que o degeneraram miseravelmente.

Mas o passo
que Feuerbach não
deu, precisava, ainda assim, ser dado; o culto ao homem abstrato, que
formava o núcleo da nova religião
de Feuerbach,
tinha de ser substituído pela ciência dos homens efetivos e de seu
desenvolvimento histórico. Esse desenvolvimento posterior, a partir da
posição
de Feuerbach, e para além dele, foi inaugurado por Marx, em 1845, na \emph{Sagrada
Família.}

\quebra

\mbox{}
\vspace{2cm}

\addcontentsline{toc}{chapter}{Quarta parte \medskip}
\section{IV}

\noindent{}Strauss, Bauer, Stirner, Feuerbach,
eram esses os continuadores da filosofia
de Hegel,
enquanto não abandonarem o solo
filosófico. Strauss, %confuso. "Na medida em que não abandonaram..."?
depois da \emph{A vida de Jesus }\textbar{} e
da \emph{Dogmática}\endnote{Strauß, David Friedrich. \emph{Das Leben Jesu, kritisch bearbeitet}
  (1835). Erster Band. Tübingen: Verlag von C. F. Osiander, 1864
  / Strauß, David Friedrich. \emph{Die christliche Glaubensiehre in
  ihrer geschichtlichen Entwicklung und im Kampfe mit der modernen
  Wissenschaft}. Tübingen, Stuttgart, 1840--1841,2 Bände. {[}\versal{N.\,T.}{]}} \textbar{}, praticou
apenas ainda a beletrística filosófica e histórico"-eclesial \emph{à
la} Renan; Bauer só
realizou algo no âmbito da gênese do cristianismo, mas aqui também algo
significativo; Stirner permaneceu
uma curiosidade, mesmo depois de  Bakunin o ter combinado com Proudhon e
batizado essa combinação de ``anarquismo''; apenas
Feuerbach foi
significativo como filósofo. Mas não apenas a filosofia --- a ciência das
ciências, que supostamente pairava acima e vinculava todas as ciências
especiais --- permaneceu uma barreira intransponível para ele, um elemento
sagrado inviolável; ele permanece no meio do caminho, embaixo, foi
materialista, em cima, idealista; não liquidou criticamente
com Hegel,
simplesmente o deixou de lado como inutilizável, enquanto ele mesmo,
diante da riqueza enciclopédica do sistema
de Hegel,
não levou a cabo nada de positivo além de uma empolada religião do amor
e de uma pouco satisfatória, impotente, moral.

Da dissolução da escola hegeliana surgiu, porém, ainda outra orientação,
a única que efetivamente deu frutos e esta orientação vincula"-se
essencialmente ao nome de Marx.\endnote{Permitam"-me aqui um esclarecimento pessoal. Recentemente, mais de uma vez, aludiu"-se à minha participação nessa teoria e, portanto, eu não posso deixar de dizer aqui algumas poucas palavras que colocam um fim
  nesse ponto. Não posso negar que, antes e durante a minha colaboração
  de quarenta anos com Marx, tive certa participação autônoma, tanto na
  fundamentação como, nomeadamente, na elaboração da teoria. Mas a maior
  parte dos pensamentos fundamentais orientadores, particularmente no
  domínio econômico e histórico e, especialmente a precisa apreensão
  definitiva desse domínio, pertencem a Marx. Com aquilo que eu possa
  ter contribuído, Marx poderia --- excetuando, quando muito, algumas
  disciplinas especiais --- ter muito bem levado a cabo sem mim. O que
  Marx realizou, eu não teria levado a cabo. Marx estava mais acima, via
  mais longe, abarcava mais e mais rapidamente do que o resto de nós. \textbar{}
  Marx era um gênio, nós, no máximo, talentosos \textbar{}. Sem ele a teoria não
  seria hoje, nem de longe, aquilo que ela é. Ela tem, portanto, também
  com razão, seu nome. {[}\versal{N.\,E.}{]}}

A separação da filosofia hegeliana resultou aqui também de um regresso à
posição materialista. Isso significa que se decidiu apreender o mundo
efetivo\est\ --- natureza e história, tal como ele próprio se apresenta a quem %tal como se apresentam...
quer que se aproxime dele sem ideias fixas
idealisticamente preconcebidas; decidiu"-se sacrificar impiedosamente
toda a ideia fixa idealista que não pudesse ser posta em consonância com
os fatos apreendidos em seu próprio nexo, e não um nexo fantástico
qualquer. Para além disso, o materialismo não significa absolutamente
nada. Só que aqui, pela primeira vez, a visão de mundo materialista foi
realmente levada a sério, de tal modo que foi consequentemente conduzida
em todas as áreas relevantes do saber --- pelo menos em seus traços
fundamentais.

Hegel não
foi simplesmente deixado de lado; pelo contrário, vinculou"-se ao seu
lado revolucionário acima desenvolvido, ao método dialético. Porém, esse
método, na sua forma hegeliana, era inutilizável.
Para Hegel,
a dialética é o autodesenvolvimento do conceito. O conceito absoluto não
existe apenas desde a eternidade --- não se sabe onde? ---, ele é também a %o trecho entre travessões está confuso
autêntica e viva alma de todo mundo existente. Ele desenvolve"-se para si
mesmo por meio de todos os estágios preliminares, amplamente tratados na
\emph{Lógica} e que estão todos contidos nele; depois, ele se ``externa
alienadamente'', convertendo"-se em natureza, onde, sem consciência de si
próprio, disfarçado de necessidade natural, sofre um novo
desenvolvimento e, por fim, explicita"-se novamente, no homem, na
consciência"-de"-si; essa consciência"-de"-si elabora a si mesma
novamente na história a partir do estado bruto, até
finalmente o conceito absoluto novamente voltar completamente a si
próprio na filosofia
de Hegel.
Para Hegel,
o desenvolvimento dialético que se explicita na natureza e na história ---
isto é, a conexão causal do ato de progressão do elemento inferior para
o superior que se impõe através de todos os movimentos em zigue"-zague e
retrocessos momentâneos --- é, portanto, apenas a imitação 
do automovimento do conceito que se processa desde a
eternidade, não se sabe onde, mas, em todo caso, independentemente de
qualquer cérebro humano\est\ pensante. Tratava"-se de eliminar essa distorção
ideológica. Voltamos a apreender materialistamente os conceitos da nossa
cabeça como imagens derivadas de coisas efetivas, em 
vez de apreender as coisas efetivas como imagens derivadas
deste ou daquele estágio do conceito absoluto. 
Reduziu"-se, com isso, a dialética a ciência das leis universais do
movimento, tanto do mundo exterior como do pensar humano --- duas séries
de leis que, segundo o movimento da coisa em questão, são
idênticas, mas que, na expressão, são diversas, na medida em que a
cabeça humana as pode aplicar com consciência, enquanto que elas, na
natureza e, até agora, em grande parte da história humana,
impõe"-se de modo inconsciente, na forma de necessidade exterior, em meio
a uma série sem fim de contingências aparentes. Com isso, porém, a
própria dialética do conceito tornava"-se apenas reflexo consciente do
movimento dialético do mundo efetivo, e assim a dialética
de Hegel era
posta acima da cabeça, ou, antes: da 
cabeça, sobre a qual estava, foi posta novamente sobre os pés. E esta
dialética materialista, que era há anos o nosso melhor meio de trabalho
e a nossa arma mais afiada, foi, de modo notável, novamente descoberta,
não apenas por nós, mas ainda, independentemente de nós e do próprio
Hegel,
por um trabalhador
alemão, Josef Dietzgen.\endnote{Cfr. ``\emph{Das Wesen der Kopfarbeit,
  von einem Handarbeiter}'' {[}A essência do trabalho intelectual, por
  um trabalhador manual{]} Hamburg, Meißner, 1869. {[}\versal{N.\,E.}{]}}

Deste modo, porém, o lado revolucionário da filosofia
de Hegel foi
novamente retomado e, ao mesmo tempo, libertado de suas dissimulações
idealistas que,
em Hegel,
haviam impedido a sua efetivação consequente. O grande pensamento
fundamental, segundo o qual não se deve apreender o mundo como um
complexo de \emph{coisas }prontas, mas como um complexo
de \emph{processos}, no qual as coisas, aparentemente estáveis, não
passam de imagens derivadas do pensamento delas na nossa cabeça, os
conceitos, que passam por uma ininterrupta transformação no devir e perecer,
na qual, em toda a aparente contingencia, e apesar de todo o retrocesso
momentâneo, impõe"-se no fim um desenvolvimento progressivo --- este grande
pensamento fundamental, expressamente
desde Hegel,
transformou"-se na consciência habitual que já quase não encontra
contradição nessa universalidade. Mas, reconhecê"-lo na fraseologia e
executá"-lo na realidade efetiva, nos pormenores, em todo o domínio que
venha a ser investigado, são duas coisas diversas. Mas se na
investigação partimos sempre desse ponto de vista,
 a exigência de soluções definitivas e de
verdades eternas se encerra de uma vez por todas; sempre estamos
conscientes da necessária limitação de todo o conhecimento adquirido, do
seu condicionamento pelas circunstâncias em que foi adquirido; mas
também não nos deixamos mais impor pelas insuperáveis oposições da velha %talvez "sobrepujar" no lugar de "impor"?
metafísica, ainda sempre em voga, entre verdadeiro e falso, bom e mau,
idêntico e diverso, necessário e contingente; sabe"-se que essas
oposições só têm validade relativa, que aquilo que agora é considerado
como verdadeiro tem igualmente o seu lado falso, oculto, que aparecerá
mais tarde, assim como aquilo que agora é tomado como falso tem o seu
lado verdadeiro, devido ao fato de que, anteriormente, pode ter sido
tomado como verdadeiro; que o elemento afirmado como necessário é
composto de elementos evidentemente contingentes, e que o elemento
pretensamente contingente é a forma atrás da qual a necessidade se
esconde, e assim por diante.

O velho método de investigação e pensamento
que Hegel 
denomina ``metafísico'', que se ocupava preferencialmente com a
investigação das \emph{coisas }como elementos duradores, 
consistentes e dados, cujos restos ainda assombram
fortemente as nossas cabeças, teve, no seu tempo, uma grande
justificação histórica. As coisas tinham de ser investigadas primeiro,
antes que os processos pudessem ser investigados. Era necessário
primeiro saber o que uma coisa qualquer era, antes que fosse possível
perceber as transformações que se processavam nela. E assim era na
ciência da natureza. A velha metafísica, que tomava as coisas como
prontas, surgiu a partir de uma ciência da natureza que investigava as
coisas mortas e vivas como coisas prontas. Porém, quando essa
investigação se estendeu a tal ponto que tornou possível um progresso
decisivo, a transição para a investigação sistemática das transformações
que se processam com essas coisas na própria natureza, nesse momento,
também dobram no âmbito filosófico os sinos da morte da velha
metafísica. E, de fato, se a ciência da Natureza até no final do século
passado foi, predominantemente, uma ciência \emph{coletora}, uma ciência
das coisas prontas, no nosso século, ela é essencialmente um
ciência \emph{ordenadora}, uma ciência dos processos, da origem e do
desenvolvimento dessas coisas e da conexão que vincula esses processos
naturais em um grande todo. A fisiologia, que investiga os processos no
organismo vegetal e animal, a embriologia, que trata do desenvolvimento
do organismo singular do embrião até a maturidade, a geologia, que
persegue a formação gradual da superfície terrestre, todas elas são
filhas do nosso século.

Sobretudo há, porém, três grandes descobertas que fizeram o nosso
conhecimento da conexão dos processos naturais avançar passos
gigantescos: em primeiro lugar, a descoberta da célula como unidade em
multiplicação \textbar{} e diferenciação \textbar{}, a partir da qual todo corpo vegetal
e animal se desenvolve, de tal modo que não apenas o desenvolvimento e
o crescimento de todos os organismos superiores são reconhecidos como
algo que se processa segundo uma única lei universal, \textbar{} mas também na
capacidade de transformação da célula está mostrado o caminho pelo qual
os organismos podem mudar a sua espécie e, assim, percorrer um
desenvolvimento mais do que individual. Em segundo lugar, \textbar{} a
transformação da energia que nos comprovou todas as chamadas forças que %discriminou no lugar de comprovou?
atuam, antes de tudo, na natureza inorgânica, a força mecânica e o seu
complemento, a chamada energia potencial, calor, radiação (luz, ou calor
radiante), eletricidade, magnetismo, energia química --- como diversas
formas de aparição do movimento universal que\est\ em determinadas
proporções transitam 
de uma para outra, de tal modo que, para a quantidade de uma que
desaparece volta a aparecer uma determinada quantidade de outra,
reduzindo assim todo o movimento da natureza a esse incessante processo
de transformação de uma forma em outra. Por fim, a prova desenvolvida por Darwin,
pela primeira vez nesse contexto,
de que o elemento duradouro dos produtos orgânicos da 
natureza que hoje nos rodeia, incluindo os homens, é o produto de um
longo processo de desenvolvimento a partir de alguns embriões
originalmente unicelulares, e que esses surgidos, por sua vez, por meio
químico provieram do protoplasma ou albumina. %por sua vez provenientes, por meio químico, do protoplasma ou da albumina.

Graças a estas três grandes descobertas e aos restantes poderosos
progressos da ciência da natureza, chegamos agora ao ponto de poder
demonstrar a conexão entre os processos no interior da natureza, não
apenas nos domínios isolados, mas também dos domínios isolados entre si
e, assim, poder apresentar uma imagem nítida da conexão da natureza, em
uma forma aproximadamente sistemática, por meio dos fatos fornecidos
pela própria ciência empírica da natureza. Fornecer esta imagem do todo
era, anteriormente, a tarefa da chamada filosofia da natureza. Ela
somente era capaz disso na medida em que substituía as conexões
efetivas ainda desconhecidas por conexões ideais, fantásticas, que
completavam os fatos com imagens do pensamento, que preenchiam lacunas
efetivas na pura imaginação. Como não era possível ser diferente, ao
proceder assim, alcançou muitos pensamentos geniais, anteviu muitas
descobertas ulteriores, mas também trouxe à luz consideráveis absurdos.
Hoje, onde apenas é preciso apreender dialeticamente --- isto é, no sentido
da sua conexão própria --- os resultados da investigação da natureza para
chegar a um ``sistema da natureza'' suficiente para o nosso tempo, onde
o caráter dialético dessa conexão se impõe às cabeças metafisicamente
formadas\est\ dos naturalistas, mesmo contra a sua vontade, hoje, a filosofia
da natureza está definitivamente posta de lado. Qualquer tentativa de
ressuscitá"-la não seria apenas supérflua, \emph{seria um retrocesso.}

Porém, o que vale para a natureza, que também é reconhecido por meio
disso como um processo de desenvolvimento histórico, vale também para a
história da sociedade em todos os seus ramos e para a totalidade de
todas as ciências que se ocupam de coisas humanas (e divinas). Também
aqui a filosofia da história, do direito, da religião etc. consistia em
substituir a conexão efetiva a ser demonstrada nos acontecimentos
singulares por uma conexão feita na cabeça do 
filósofo, de tal modo que a história fosse apreendida como a efetivação gradual de ideias, tanto no todo
como em suas partes singulares --- e, naturalmente,
de fato, sempre apenas das ideias prediletas do próprio %é possível tirar o "de fato"?
filósofo. De acordo com isso, a história trabalhava aqui
inconscientemente, mas com necessidade de iniciar por uma finalidade
ideal, estabelecida de antemão, como por exemplo,
em Hegel,
pela efetivação da sua ideia absoluta, e a orientação inalterável por
essa ideia absoluta formava a conexão interna no interior dos
acontecimentos históricos singulares. No lugar da conexão efetiva, ainda
desconhecida, estabelecia"-se constitutivamente, assim,
uma nova providência misteriosa --- inconsciente ou que alcançava gradualmente consciência.
Aqui, justamente como no âmbito da natureza, o
que valia, portanto, era eliminar as conexões feitas artificialmente
pela adivinhação  das efetivas; uma tarefa que
definitivamente acaba por descobrir as leis universais do movimento que
se impõem como dominantes na história da sociedade humana.

Aqui, no entanto, a história do desenvolvimento da sociedade mostra"-se %Remover o "aqui" repetido?
em um ponto essencialmente diverso da história do desenvolvimento da
natureza. Na Natureza --- desde que deixemos de fora a consideração da
repercussão da ação do homem sobre a natureza --- há 
somente agenciamentos cegos, desprovidos de consciência, que geram
efeitos uns sobre os outros e em cuja interação recíproca a lei
universal torna"-se valida. De tudo o que acontece --- tanto das inúmeras
contingências aparentes, que se tronam visíveis na superfície, como dos
resultados que confirmam a regularidade no 
interior dessas contingências ---, nada acontece enquanto uma finalidade
consciente fruto da vontade. Em contrapartida, na história da sociedade,
os agentes estão nitidamente dotados de consciência, são homens que se
propõem a agir com reflexão ou paixão, em determinadas finalidades; nada
acontece sem propósito consciente, sem uma finalidade que seja fruto da
vontade. Mas essa diferença, por mais importante que seja para a
investigação histórica, especialmente de épocas e eventos, não altera em
nada o fato de que o curso da história é regido por leis internas
universais. Afinal, também aqui, apesar das finalidades frutos %finalidades serem frutos?
conscientes da vontade de todos os indivíduos singulares, aparentemente
rege sobre superfície, em geral, a contingência. Apenas raramente
acontece o elemento que é fruto da vontade; na maioria dos casos, as
múltiplas finalidades que são frutos da vontade entrecruzam"-se e se
contradizem, ou essas mesmas finalidades são, a princípio, irrealizáveis
ou os meios são insuficientes. Assim, as colisões das inúmeras vontades
singulares e ações singulares no âmbito histórico proporcionam um estado
que é totalmente análogo ao que domina na natureza desprovida de
consciência. As finalidades das ações são frutos da vontade, mas os
resultados que efetivamente decorrem das ações não são frutos da vontade,
ou na medida em que, antes de tudo, parecem no entanto corresponder à
finalidade que é fruto da vontade, têm no fim consequências totalmente
diversas das queridas. Os acontecimentos históricos singulares aparecem, %desejadas no lugar de queridas
no geral, como se fossem em todo caso dominados pela
contingência. Mas onde sobre superfície a contingência joga seu jogo,
ela é sempre dominada por leis internas ocultas e a questão é apenas
descobrir essas leis.

Os homens fazem a sua história, aconteça ela como acontecer, na medida
em que cada um persegue conscientemente as finalidades que eles mesmos %as finalidades que deseja?
querem, e a resultante destas várias vontades que atuam em direções
diversas e da sua influencia múltipla sobre o mundo exterior é
justamente a história. Depende, portanto, do que os muitos indivíduos
querem. A vontade é determinada por paixão ou reflexão. Mas as alavancas
que, por sua vez, determinam imediatamente a paixão ou reflexão, são de
tipos muito diversos. Em parte podem ser finalidades exteriores, em
parte \emph{fundamentos ideais do movimento}, 
ambição, ``entusiasmo pela verdade e pela justiça'', ódio
pessoal, ou também caprichos puramente individuais de toda a espécie.
Mas, por outro lado, vimos que as várias vontades individuais ativas na
história, na maioria dos casos, produzem resultados totalmente
diferentes dos queridos --- \emph{muitas vezes contrapostos} --- e que,
portanto, para o resultado do todo, seus fundamentos de movimento têm um
significado subordinado. Por outro lado, é possível questionar ainda
mais: quais forças impulsionadoras estão novamente por detrás destes
fundamentos do movimento, que causas históricas transformam, na cabeça
dos agentes, esses fundamentos de movimento?

O velho materialismo nunca se colocou essa questão. Sua concepção da
história, se é que ele tem uma, é, portanto, também essencialmente
pragmática, adjudica tudo segundo os motivos da ação, divide os homens
que agem historicamente em nobres e não nobres e então descobre, em
regra, que os nobres são os enganados e os não
nobres os vencedores; disso resulta para o velho materialismo que do
estudo da história nada de muito edificante se explicita e, para nós, no
âmbito da história, o velho materialismo se tornou infiel a si próprio,
pois toma as forças motrizes ideais aí atuantes como causas últimas, em
vez de investigar aquilo que está por detrás delas, quais são as forças
motrizes dessas forças motrizes. Não é nisso que se estabelece a
inconsequência, de reconhecer forças motrizes \emph{ideais}, mas no fato
de que a partir dessas não se investigue mais a fundo as causas de seu
movimento. A filosofia da história, em contrapartida, justamente como é
defendida
por Hegel,
reconhece que os fundamentos ostensivos, e também os efetivamente ativos
do movimento dos homens que agem historicamente, não são, de modo algum,
as causas últimas dos acontecimentos históricos {[};{]} reconhece que
por detrás desses fundamentos do movimento encontram"-se outras potências
móveis, que é preciso investigar; mas ela não procura 
essas potências na própria história, importa"-as, pelo contrário, de %na própria história. Pelo contrário, importa"-as de fora
fora, da ideologia filosófica para o interior da história. Em vez de
explicar a história da Grécia antiga a partir da sua conexão própria,
interna, Hegel afirma,
por exemplo, simplesmente que ela não mais é do que a elaboração das
``cofigurações da bela individualidade'', a realização da ``obra de
arte'' enquanto tal. Quando convém, ele diz algo de muito belo e
profundo sobre a Grécia antiga, mas isso não impede que nós hoje já não
nos contentemos com tal explicação, que não passa de uma mera expressão
idiomática.

Quando se trata, portanto, de investigar as potências impulsionadoras
que --- consciente ou inconscientemente e, de fato, frequentemente %as potências impulsionadoras --- conscientes ou inconscientes e, de fato, frequentemente inconscientes --- que estão
inconscientemente --- estão por detrás dos fundamentos dos movimentos dos
homens que agem historicamente, e que constituem propriamente as forças %talvez ainda extender o travessão, ficando "Quando se trata, portanto, de investigar as potências impulsionadoras --- conscientes ou inconscientes e, de fato, frequentemente inconscientes, que estão por detrás dos fundamentos dos movimentos dos homens que agem historicamente, e que constituem propriamente as forças motrizes últimas da história ---, não se pode tratar do (...)"
motrizes últimas da história, não se pode, assim, tratar dos fundamentos %cortar o "assim"
de movimento dos indivíduos, mesmo os homens que agem de modo eminente, %mesmo daqueles que agem de modo eminente, e põem em
enquanto aqueles que põem em movimento grandes massas, povos inteiros e,
em cada povo, por sua vez, classes inteiras; e isso também não
momentaneamente, em uma explosão temporária, fogo de palha que queima %melhor cortar ou o "momentaneamente" ou o "em uma explosão temporária"
rapidamente, mas em uma ação duradoura que se alastra em uma grande
transformação histórica. Fundamentar as causas motrizes que aqui se
refletem clara ou obscuramente, imediatamente ou na forma ideológica,
mesmo a sacralizada, na cabeça das massas que agem e de seus condutores
--- os chamados grandes homens --- como fundamentos conscientes de movimento
é o único caminho que nos pode colocar no rastro das leis que, tanto
em geral como em períodos e países singulares, dominam a história. Tudo
o que põe os homens em movimento tem de passar por sua cabeça; mas que
configuração toma nessa cabeça, depende muito das circunstâncias. Os
trabalhadores, sob nenhuma circunstância, reconciliaram"-se com o
maquinário fabril capitalista, mesmo que não mais o tenham simplesmente
quebrado em pedaços, como ainda em 1848 no Reno.\endnote{Possível alusão aos ocorridos na noite de 16 e 17 de Março de 1848 na
  fabrica de produção de ferro fundido da cidade de Solinger, onde uma
  rebelião dos trabalhadores destruiu quatro oficinas de fundição e uma
  maquina a vapor. {[}\versal{N.\,T.}{]}}

Porém, enquanto em todos os períodos anteriores era quase impossível a
investigação destas causas impulsionadoras da história --- devido às
complicadas e encobertas conexões com os seus efeitos --- o nosso período
atual simplificou tanto essas conexões que foi possível resolver o
enigma. Desde a efetivação da grande indústria, 
portanto pelo menos desde a paz europeia de 1815, não era mais segredo
para homem nenhum em Inglaterra que lá toda a luta política girava em
torno das pretensões à dominação de duas classes: a aristocracia
possuidora de terras (\emph{landed
aristocracy}) e a burguesia (\emph{middle
class}). Na França, a consciência do mesmo
fato foi obtida com o regresso dos Bourbon; os historiadores da época
da Restauração, de Thierry a Guizot, Mignet e Thiers,
falam disso, por toda a parte, como a chave para a compreensão da %por toda parte falam disso como a
história francesa desde a Idade Média. E, desde 1830, em ambos os
países, a classe dos trabalhadores, o proletariado, foi reconhecida como
a terceira força por essa dominação. As relações se 
simplificaram tanto que era preciso fechar os olhos propositalmente para
não ver na luta dessas três grandes classes e, no conflito de seus
interesses, a força impulsionadora da história moderna --- pelo menos, nos
dois países mais avançados.\est\

Como haviam, porém, surgido essas classes? Se à primeira vista ainda
se podia atribuir à grande propriedade fundiária, antes feudal, uma
origem --- pelo menos a princípio --- a partir de causas políticas,
de uma apropriação violenta, isso não dizia respeito à burguesia
e ao proletariado. A origem e desenvolvimento de duas grandes classes
eram aqui claras e palpáveis a partir de causas puramente econômicas. E %clara e palpavelmente?
era igualmente claro que, na luta entre possuidores de terras e
burguesia, não menos do que na luta entre burguesia e proletariado, o
que estava em disputa, em primeira linha, eram interesses econômicos, %em primeiro lugar?
para cuja efetivação o poder político devia servir de mero meio.
Burguesia e proletariado haviam surgido ambos em decorrência de uma
transformação das relações econômicas, ou, dito de modo mais exato, do modo
de produção. A passagem, primeiro das corporações de oficio artesanais
para a manufatura, e depois da manufatura para a grande indústria com o
emprego do vapor e das máquinas, havia desenvolvido estas duas classes.
Em certo estágio, as novas forças de produção postas em movimento pela
burguesia --- antes de tudo, a divisão do trabalho e a reunião de vários
trabalhadores seccionais em uma manufatura conjunta --- e as condições de
troca e necessidades de troca por ela desenvolvidas tornaram"-se
incompatíveis com a ordem da produção existente, historicamente
transmitida e consagrada pela lei, isto é, com os privilégios
corporativos e incontáveis outros privilégios pessoais e locais (que,
para os estamentos não privilegiados, eram igualmente muitos grilhões)
da constituição da sociedade feudal. As forças de produção,
representadas pela burguesia, rebelaram"-se contra a ordem de produção
representada pelos senhores de terras feudais e mestres"-artesões; o
resultado é conhecido: os grilhões feudais foram quebrados, na
Inglaterra de modo gradual, na França com um só golpe, na Alemanha ainda %de modo gradual na Inglaterra, com um só golpe na França; na Alemanha ainda não se acabou com eles.
não se acabou com eles. Mas, assim como a manufatura, em um estágio
determinado do desenvolvimento, entrou em conflito com a ordem feudal de %assim como a manufatura entrou em conflito com a ordem feudal de prdução em um estágio determinado de seu desenvolvimento,
produção, também agora a grande indústria entrou já em conflito com a
ordem burguesa de produção posta no lugar daquela. Mantida por esta
ordem, pelas estreitas barreiras do modo de produção capitalista, ela
produz, por um lado, uma proletarização sempre crescente de toda a
grande massa do povo, e, por outro lado, uma massa cada vez maior de
produtos que não podem ser vendidos. Sobreprodução e miséria das massas,
cada uma a causa da outra, é essa a contradição absurda na qual essa
ordem desemboca e que demanda necessariamente retirar os grilhões das
forças produtivas por meio da mudança do modo de produção.

Pelo menos, na história moderna está assim demonstrado que todas as %Na história moderna, pelo menos,
lutas políticas são lutas de classes, e que todas são lutas por
emancipação das classes, apesar da sua forma necessariamente política ---
afinal, toda luta de classes é uma luta política ---, e que giram, no fim, em
torno da emancipação \emph{econômica. }Pelo menos aqui, o Estado, a
ordem política, é o elemento subordinado; a sociedade civil"-burguesa, o
reino das inter"-relações econômicas, é o elemento decisivo. A visão
tradicional, também acatada por Hegel, via no Estado o elemento
determinante, na sociedade civil"-burguesa o elemento por ele
determinado. A aparência corresponde a isso. Assim como no homem
singular todas as forças impulsionadoras das suas ações têm de passar
pela cabeça dele, têm de se transformar em fundamentos do movimento da
sua vontade para levá"-lo a agir, também todas as necessidades da
sociedade civil"-burguesa --- qualquer que seja a classe que no momento
a domina --- têm de passar pela vontade do Estado para obter validade
universal na forma de leis. Esse é o lado formal da coisa, que se
compreende por si; mas a questão é qual o conteúdo que esta vontade
apenas formal --- tanto do indivíduo singular como do Estado --- tem, e de
onde vem esse conteúdo, por que é precisamente este e não outro que é
fruto da vontade. E se perguntarmos por isso, verificamos que, na
história moderna, a vontade do Estado, em geral, é determinada pelas
carências mutáveis da sociedade civil"-burguesa, pelo predomínio dessa ou
daquela classe, e, em última instância, pelo desenvolvimento das forças
produtivas e das relações de troca.

Mas, se já na nossa época moderna, com os seus gigantescos meios de
produção e intercâmbio, o Estado não é um domínio autônomo com
desenvolvimento autônomo --- pelo contrário, tanto sua existência como o
seu desenvolvimento precisam ser esclarecidos, em última instância, a
partir das condições econômicas de vida da sociedade ---, isto tem que ser
válido, ainda muito mais, para todos as épocas anteriores, em que a
produção da vida material dos homens ainda não era empreendida com esses
recursos abundantes, e onde, portanto, a necessidade dessa produção tinha
de exercer uma dominação ainda maior sobre os homens. Se ainda
hoje, na época da grande indústria e das estradas de ferro, o Estado é em geral
reflexo, em forma vinculativa, das carências 
econômicas da classe que domina a produção, então isso precisaria ser
assim, ainda muito mais, em uma época na qual uma geração de homens
tinha de consagrar uma parte muito maior do seu tempo total de vida à
satisfação das suas carências materiais, portanto, estava muito mais %"logo estando", para não repetir "portanto"
dependente delas do que nós hoje estamos. A investigação da história de
épocas anteriores, desde que seriamente comprometida com esse lado,
confirma isso na mais rica medida; porém, naturalmente, isso não poderá
ser tratado aqui.

Se o Estado e o direito do Estado são determinados pelas relações
econômicas, também o é, evidentemente, o direito privado, o que
essencialmente apenas sanciona, sob as circunstâncias dadas, as
inter"-relações econômicas normais existentes entre os indivíduos. A
forma na qual isso acontece pode, porém, ser muito diversa. É possível,
como aconteceu na Inglaterra, em consonância com todo o desenvolvimento
nacional, que formas do velho direito feudal, em grande parte, sejam
conservadas e lhes sejam dadas um conteúdo burguês, imputando
diretamente ao nome feudal um sentido burguês;\est\ mas também é possível,
como na Europa Ocidental continental, tomar por base o primeiro direito
mundial de uma sociedade produtora de mercadorias, o romano, com a sua
insuperavelmente precisa elaboração de todas as inter"-relações jurídicas
essenciais dos possuidores simples de mercadorias (comprador e vendedor,
devedor e credor, contrato, obrigação, etc.). Com isso, para utilidade e
proveito de uma sociedade ainda pequeno"-burguesa e semifeudal, ou se
pode simplesmente reduzi"-lo ao patamar dessa sociedade por meio da
\emph{práxis} jurídica (direito comum), ou então, com a ajuda de
juristas pretensamente esclarecidos, moralistas, pode"-se elaborá"-lo num
código à parte, correspondente a esse estado da sociedade, código esse
que, nessas circunstâncias, será também juridicamente perverso
(\emph{Landrecht} prussiano);\endnote{No contexto do processo de codificação europeia pós"-revolução francesa, que tem como marco o código civil napoleônico de 1804, o
  \emph{Código Geral da Prússia} (\emph{Allgemeine Preußische
  Landrecht)} de 1794, com seus 19.000 parágrafos, desenha na estrutura
  legal e social da Alemanha muitos aspectos dos compromissos de classes
  que caracteriza os efeitos concretos da Revolução de 1789 fora da
  França, já que nele é possível vislumbrar com muita clareza o
  desdobramento interno e constitutivo da inter"-relação entre Estado na
  forma burguesa e sociedade estamental pré"-burguesa (Sobre isso:
  \versal{KOSELELCK}, R. \emph{Preußen zwischen Reform und Revolution,
  Allgemeines Landrecht, Verwaltung und soziale Bewegung von 1791 bis
  1848}). A distorção que essa inter"-relação impõe é certamente uma
  aspecto da perversidade jurídica mencionada por Engels. {[}\versal{N.\,T.}{]}} com isso,
porém, é possível também, após uma grande revolução burguesa, tendo como %será o "porém" necessário?
base justamente esse direito romano, elaborar um código da sociedade
burguesa tão clássico quanto o \emph{Code
civil} francês. Se, portanto, as
determinações jurídicas burguesas apenas expressam as condições
econômicas de vida da sociedade em forma jurídica, isso pode ocorrer, a
depender das circunstâncias, de modo satisfatório ou
perverso.

No Estado, apresenta"-se para nós a primeira potência 
ideológica sobre o homem. A sociedade cria para si um órgão para a
salvaguarda dos seus interesses comuns diante de ataques internos e
externos. Esse órgão é o poder do Estado. Assim que  
surge, tal órgão se autonomiza diante da sociedade, e isso, justamente,
quanto mais ele se torna órgão de uma classe determinada, um órgão que
valida diretamente a dominação dessa classe. A luta da classe oprimida
contra a classe dominante torna"-se necessariamente uma luta política;
uma luta, antes de tudo, contra a dominação política desta classe; a
consciência da conexão dessa luta política com os suas bases
econômicas torna"-se mais indeterminada  
e pode se perder totalmente. Onde isso não é  
completamente o caso em relação aos implicados\est\ na luta, isso quase %frase está difícil de entender
sempre acontece com os historiadores. Entre as velhas fontes acerca das
lutas no interior da república romana, apenas Apiano\endnote{Apiano de Alexandria, autor da \emph{História romana} (\emph{Ῥωμαϊκά -- Romaica}) escrita em 24 volumes. {[}\versal{N.\,T.}{]}} nos diz clara e distintamente do que definitivamente se tratava: justamente da
propriedade fundiária.

O Estado, porém, uma vez que se torna um poder autônomo  
diante da sociedade, logo em seguida produz uma ideologia ulterior. Nos
políticos de profissão, nos teóricos do direito do Estado e nos juristas
do direito privado, perde"-se, sobretudo, justamente a conexão com os %sobretudo, a própria conexão
fatos econômicos. Porque em cada caso individual os fatos econômicos têm
de tomar a forma de motivos jurídicos para serem sancionados na forma de
lei, e porque, ao fazê"-lo, é preciso também evidentemente considerar
todo o sistema jurídico já em vigor; por isso, a forma jurídica deve
aqui ser tudo e o conteúdo econômico nada. Direito do Estado e direito
privado são tratados como domínios autônomos, que têm o seu
desenvolvimento histórico independente, que são capazes em si mesmos de
uma exposição sistemática e a necessitam através da consequente extinção
de todas as suas contradições internas.

Ideologias ainda mais superiores, isto é, ainda mais afastadas do
fundamento econômico, material, tomam a forma da filosofia e da
religião. Aqui, a conexão das representações com as suas condições
materiais de existência torna"-se sempre mais complexa, sempre mais
obscurecida por elos intermediários. Mas ela existe. Assim como toda a
época do Renascimento, desde os meados do século \versal{XV}, foi essencialmente
um produto das cidades --- portanto, da burguesia ---, também o foi a filosofia
desde então renascida; o seu conteúdo era essencialmente apenas a
expressão filosófica do pensamento correspondente ao desenvolvimento da
pequena e média burguesia em grande burguesia. Isso se explicita
claramente nos ingleses e franceses do século passado que, em muitos
casos, tanto eram filósofos\est\ como economistas políticos, e, na escola %e também na
hegeliana, como já demonstramos acima.

Passemos, entretanto, ainda que apenas brevemente, para a religião, já
que essa se encontra o mais afastada possível da vida material e parece
ser a mais alheia possível. A religião surgiu em uma época
originariamente bastante silvestre, a partir 
de originariamente silvestres, equivocadas, representações dos homens
sobre a sua própria natureza e a natureza exterior circundante. Toda a
ideologia, porém, desde que ela exista, desenvolve"-se em conexão com o
material da representação dado, dá a ele uma forma ulterior; caso
contrário, ela não seria ideologia, isto é, ocupação com pensamentos
como essencialidades autônomas, desenvolvendo"-se independentemente,
submetidas apenas às suas próprias leis. O fato de as condições
materiais de vida dos homens, em cuja cabeça esse processo de pensamento
avança, determinarem definitivamente o curso desse processo, permanece
necessariamente inconsciente para esses homens, afinal, caso contrário,
toda a ideologia chegaria ao fim. Essas representações religiosas %Essas representações religiosas originárias, comuns a todo o grupo [a todo o grupo ou a todo grupo?] de povos aparentados na maior parte dos casos, desenvolvem"-se portanto após a separação do grupo
originárias, portanto, na maior parte dos casos, comuns a todo o grupo
de povos aparentados, desenvolvem"-se, após a separação do grupo, de modo
particular em cada povo, dependendo das condições de vida particulares,
e esse processo, para uma série de grupos de povos --- expressamente para
os arianos (chamados indo"-europeus) ---, está demonstrado detalhadamente
pela mitologia comparada. Os Deuses assim elaborados por cada povo eram
Deuses nacionais, cujo reino não ia além do território nacional a ser
protegido por eles, para além de cujas fronteiras outros Deuses detinham
incontestavelmente a última palavra. Eles somente podiam sobreviver na
representação enquanto a nação existisse; caíam com a sua decadência. O
império mundial romano, cujas condições econômicas de surgimento não
podemos investigar aqui, trouxe à tona a decadência das antigas
nacionalidades. Os antigos Deuses nacionais entraram em declínio, mesmo
os Deuses romanos que apenas estavam talhados para o estreito círculo da
cidade de Roma; a necessidade de completar o império mundial com uma
religião mundial apareceu claramente nas tentativas de reconhecer
erguer altares, ao lado dos nativos de Roma, a todos e quaisquer Deuses
estrangeiros respeitáveis. Mas uma nova religião mundial não se faz
dessa maneira, por decretos imperiais. A nova religião mundial, o
cristianismo, já tinha surgido em silêncio, a partir de uma mistura de
teologia oriental generalizada, nomeadamente judaica, e de filosofia
grega vulgarizada, nomeadamente estoica. O quanto ela parecia
originária, temos ainda que pesquisar exaustivamente, pois a sua
cofiguração oficial que nos foi transmitida é apenas aquela em que se
tornou religião de Estado, e que para esse fim foi adaptada pelo Concílio
de Niceia.\endnote{Primeiro
  concilio ecumênico de bispos cristãos convocado pelo Imperador Romano
  Constantino \versal{I} em 325, na cidade de Nicéia (atualmente, província de
  Bursa na Turquia). Organizado na forma do senado romano, o concilio
  discutiu, entre outras questões, sobre a divindade da figura de cristo
  e sua relação com o Deus"-Pai. Do fundo da disputa político"-religiosa
  entre Alexandre \versal{I} e Ário surge a questão da doutrina da revelação. Um
  dos resultados é a tentativa de estabelecimento de uma unidade do
  credo cristão em oposição às visões que deveriam ser consideradas
  heréticas. Na passagem, Engels se refere justamente a esse aspecto
  quando fala da ``mistura de teologia oriental generalizada,
  nomeadamente judaica, e de filosofia grega vulgarizada, nomeadamente
  estoica'', assim como remete à questão da estrutura de organização
  institucional da Igreja, tendo como resultado principal a promulgação
  da primeira forma da lei canônica, uma lei que, naquela época,
  organizou tanto a vida espiritual como material. {[}\versal{N.\,T.}{]}} É suficiente o fato
de que apenas 250 anos depois tenha se tornado religião de Estado para
demonstrar que era a religião adequada às circunstâncias da época. Na
Idade Média, na exata medida em que o feudalismo se desenvolvia,
transformou"-se na religião que correspondia a ele, com hierarquia feudal %o cristianismo transformou"-se
correspondente. E quando a burguesia apareceu, desenvolveu"-se, em
oposição ao catolicismo feudal, a heresia protestante, primeiro, no Sul
da França, entre os Albigenses,\endnote{Os Albigenses
  eram membros de um movimento religioso herético que se estabeleceu
  desde o fim do século \versal{XII} no sul da França, em torno da cidade de
  Albi. Composto em grande parte por artesãos e comerciantes das
  cidades, além de alguns nobres, todos contrários, entre outras coisas,
  ao controle de terra pela Igreja e a hierarquia eclesiástica. {[}\versal{N.\,T.}{]}} na
época de maior florescimento das cidades dessa região. A Idade Média
tinha anexado à teologia todas as restantes formas da ideologia:
filosofia, política, prática do 
direito[.]\endnote{Engels já havia indicado isso em um artigo publicado na \emph{Nova
  Gazeta Renana} de 1850: \versal{ENGELS}, F. \emph{Der Deutsche Bauernkrieg}
  {[}A guerra camponesa alemã{]}. \versal{IN}: \versal{MEGA}, \versal{I}.10. {[}\versal{N.\,T.}{]}} Tinha"-as
tornado subdivisões da teologia. Obrigou, portanto, todo o movimento
social e político a assumir uma forma teológica; os ânimos das massas,
alimentadas como animais exclusivamente com religião, tiveram que
mostrar seus próprios interesses em disfarces religiosos para criar uma
grande tempestade. \textbar{} E assim como a burguesia criou desde o início um
apêndice de plebeus urbanos não reconhecidos por nenhum estamento,
trabalhadores que recebiam por uma jornada diária e prestadores de
serviços de todos os tipos, precursores do proletariado tardio, \textbar{} a
heresia se dividiu, desde o início, em um herege moderado"-burguês e um
revolucionário"-plebeu, também abominado pelos hereges burgueses.

O caráter inexterminável da heresia protestante correspondia à
invencibilidade da burguesia ascendente; quando a burguesia era forte o
suficiente, a sua luta começou a tomar dimensões nacionais, até então %a sua luta até então predominantemente local, com a nobreza feudal, começou a tomar dimensões nacionais
predominantemente local, com a nobreza feudal. A primeira grande ação
aconteceu na Alemanha --- a chamada Reforma. A burguesia não era
suficientemente forte, tampouco estava suficientemente desenvolvida,
para conseguir unificar sob a sua bandeira os estamentos rebeldes
restantes --- os plebeus das cidades, a baixa nobreza e os camponeses, no %tirar "no campo"
campo. Primeiro, a nobreza foi abatida; os camponeses levantaram"-se em
uma insurreição que formou o ponto culminante de todo este movimento
revolucionário; as cidades os abandonaram e, assim, a revolução sucumbiu
aos exércitos dos príncipes da terra, que embolsaram todos os ganhos. A
partir de então, a Alemanha desaparece por três séculos da série de
países que intervêm na história de forma autônoma. Mas, ao lado do
alemão Lutero, surgiu o
francês Calvino;
com a fina precisão francesa, trouxe para primeiro plano o caráter
burguês da Reforma, republicanizou e democratizou a Igreja. Enquanto a
Reforma luterana estagnava e levava a Alemanha à ruína, a calvinista
servia de bandeira aos republicanos em Genebra, na Holanda, na Escócia,
libertava a Holanda da Espanha e do Império alemão\endnote{Período em que a Holanda fez parte do Sacro Império Romano"-Germânico
  entre 1477 e 1555. {[}\versal{N.\,T.}{]}} e fornecia o
disfarce ideológico ao segundo ato da revolução burguesa que estava em
processo na Inglaterra. O calvinismo comprovava"-se aqui como o autêntico
disfarce religioso dos interesses da burguesia daquela época e, por
isso, não foi plenamente reconhecido quando a revolução de 1689 chegou a
um fim por um compromisso de uma parte da nobreza com os burgueses.\endnote{Referência à Revolução Gloriosa de 1689. No
  capítulo sobre a \emph{Acumulação primitiva} (\emph{originária}), ao
  mencionar a Revolução Gloriosa como um dos impulsos da dominação
  violenta constitutiva da expropriação do trabalhador da propriedade
  rural, Marx indica justamente o caráter histórico fundamental desse
  compromisso entre aristocracia rural e capitalistas, ambos
  ``extratores de mais"-valor'': ``A \emph{Glorious Revolution}, com
  Guilherme \versal{III} de Orange, levou ao poder extratores do mais"-valor
  fundiários e capitalistas. Estes inauguraram, em escala colossal, a
  nova Era de roubo de domínios do Estado, até então realizado em
  proporções apenas modestas. Essas terras foram presenteadas, vendidas
  a preços irrisórios ou, mediante usurpação direta, anexadas a
  propriedades privadas. Tudo isso ocorreu sem nenhuma observância da
  etiqueta legal''. (\versal{MARX}, K. \emph{Das Kapital}. Erster Band. 39 Aufl.
  2008, p. 751). {[}\versal{N.\,T.}{]}} A Igreja de Estado
inglesa foi restabelecida, não em sua configuração anterior, enquanto
catolicismo com o rei como papa, mas fortemente calvinizada. A velha
igreja de Estado tinha celebrado o alegre domingo católico e combatido o
maçante domingo calvinista; este foi introduzido pela nova Igreja de
Estado aburguesada, e ainda hoje ele embeleza a Inglaterra.

Na França, a minoria calvinista foi oprimida, catolicizada ou expulsa em
1685;\endnote{Fim do Édito de
  Nantes em 1685 que, desde 1598, garantia aos calvinistas franceses
  tolerância religiosa. {[}\versal{N.\,T.}{]}} mas, para que isso serviu? Já nessa época, o
livre"-pensador Pierre
Bayle estava no trabalho e, em 1694,
nascia Voltaire.
A medida violenta
de Luís \versal{XIV} apenas facilitou à burguesia francesa para que pudesse fazer a sua %facilitou à burguesia francesa fazer
revolução sob a forma não religiosa, exclusivamente política, a única
apropriada à burguesia desenvolvida. Em vez de protestantes, foram
livres"-pensadores que se sentaram nas assembleias nacionais. O
cristianismo havia entrado por meio disso em seu último estágio. Tinha
se tornado incapaz de servir a qualquer classe progressiva como disfarce
ideológico das suas aspirações; tornou"-se cada vez mais posse exclusiva
das classes dominantes e essas o aplicavam como mero meio de governo
pelo qual as classes inferiores eram mantidas dentro das barreiras. Com
isso, então, cada uma das diversas classes utiliza a própria religião
que lhe corresponde: a aristocracia rural possuidora de 
terras utiliza o jesuitismo católico ou a ortodoxia protestante; o
burguês liberal e radical, o racionalismo; e não faz nenhuma diferença
se os próprios senhores acreditam nas respectivas religiões ou não.

Vemos, portanto, que a religião, uma vez formada, contém sempre uma
matéria tradicional, assim como que, em todos os âmbitos ideológicos, a
tradição é uma grande força conservadora. Mas as transformações que se
processam nessa matéria resultam das relações das classes, portanto,
das relações econômicas dos homens que empreendem essas transformações.
E isso é o suficiente aqui.

No exposto, somente é possível oferecer um esboço geral da concepção de
história de Marx, no máximo mais algumas ilustrações. A prova deve\est\ ser
fornecida na própria história, e posso dizer que já foi suficientemente
fornecida em outros escritos. Essa concepção põe fim, porém, à filosofia
no domínio da história, assim como a concepção dialética da natureza
torna tão desnecessária quanto impossível toda a filosofia da Natureza. %filosofia em caixa baixa e Natureza em alta mesmo?
Não se trata mais de conceber conexões na cabeça, mas descobri"-las nos
fatos. A única coisa que resta para a filosofia expulsa da natureza e da
história é o domínio do pensamento puro, na medida em que resta algo: a
doutrina das leis do próprio processo do pensamento, a lógica e a
dialética.

\asterisc

Com a revolução de 1848, a Alemanha ``culta'' recusou a teoria e, do
alto, desceu para o chão da práxis. O pequeno ofício e a manufatura, que
se baseavam no trabalho manual, foram substituídos por uma grande
indústria efetiva; a Alemanha voltou a aparecer no mercado mundial; o
novo império pequeno"-alemão eliminou pelo
menos as mais gritantes inconveniências que a medíocre divisão em pequenos
Estados, os restos 
do feudalismo e a economia burocrática haviam deixado para o caminho
desse desenvolvimento. Porém, à medida que a especulação se
mudava do gabinete de estudo filosófico para instituir o seu templo na
bolsa de valores, à medida que se perdia também para a
Alemanha culta aquele grande sentido teórico que havia sido a glória da %frase confusa. À medida que a especulação se mudava e se perdia o sentido teórico, o quê?
Alemanha durante o tempo da sua mais profunda degradação política --- o
sentido de uma pesquisa puramente científica, independentemente se o
resultado alcançado fosse aproveitável na prática ou não, ou contrário
às regras. De fato, a ciência da natureza oficial alemã, justamente no
âmbito da investigação singular, manteve"-se à altura da época, mas a
revista americana \emph{Science} já observa, com razão, que os
progressos decisivos\est\ no âmbito das grandes conexões entre fatos
singulares, da sua generalização em leis, são agora feitos muito mais na
Inglaterra do que, como anteriormente, na Alemanha. E, no âmbito das
ciências históricas, incluindo a filosofia, desapareceu, junto à
filosofia clássica, com maior razão, o velho 
espírito teórico"-brutal:\endnote{Aqui é possível remeter a uma conhecida passagem de Marx, onde defende a necessidade de uma crítica teórica sem restrições (\textit{rücksichtlos}), aquela que não olha para trás (\textit{rück"-sicht}) impulsionada pela investigação da estrutura de poder do presente. O contexto é uma carta de Marx a Ruge de 1843, na qual reflete sobre a situação política na Alemanha do período e a função do crítico. A crítica deve ser ``brutal (\textit{rücksichtlos}) tanto no sentido de [\ldots{}] não pode temer os seus próprios resultados quando no sentido de que não pode temer os conflitos com os poderes estabelecidos''. \versal{MARX}, K. Deutsche Französische Jahrbücher 1.
  Doppellieferund, Februar, 1844. In: \versal{MARX}, K.; \versal{ENGELS}, F. \emph{Werke}. Band 1. Berlin/\versal{DDR}: Dietz Verlag, 1976, p. 344. Comparando a situação da filosofia alemã do período clássico com a nova função da filosofia pós"-1870, é que Engels pode afirmar, em um sentido próximo ao de Marx em 1843, o fim ``o velho espírito teórico"-brutal'' da crítica na Alemanha, completamente absorvido pela estrutura de poder do Estado cada vez mais burguês, e seu único caminho agora possível: a consciência crítica da classe trabalhadora.  {[}\versal{N.\,T.}{]}} ecletismo desprovido
de pensamento, preocupação angustiada com carreiras e rendimentos %tomam seu lugar do ecletismo desprovido [...] até o arrivismo [...]
descendo até ao arrivismo  mais ordinário, tomam seu 
lugar. Os representantes oficiais desta ciência tornaram"-se ideólogos
não encobertos da burguesia e do Estado existente --- mas em um tempo em
que ambos estão em oposição aberta à classe trabalhadora.

E é apenas na classe trabalhadora que continua a subsistir intacto o
sentido teórico alemão. Aqui ele não pode ser exterminado; aqui não têm
lugar as preocupações com a carreira, as pequenas atitudes sorrateiras
para tirar proveito, a benevolente proteção vinda de cima; pelo
contrário, quanto mais sem restrições e 
imparcialmente a ciência procede, tanto mais se encontra em consonância
com os interesses e as aspirações dos trabalhadores. A nova orientação,
que reconheceu na história do desenvolvimento do trabalho a chave para a
compreensão de toda história da sociedade, voltou"-se, antes de tudo,
preferencialmente à classe trabalhadora e encontrou aí a receptividade
que não procurou, tampouco esperava, na ciência oficial. O movimento
trabalhador alemão é o herdeiro da filosofia clássica alemã.

\quebra

} %
%\renewcommand{\ParallelAtEnd}{\noindent{}\vspace{1cm}\Large{Notas}}
\end{Parallel}
\pagebreak
\thispagestyle{empty}
\movetooddpage
\addcontentsline{toc}{chapter}{Notas}
\theendnotes

\pagebreak
\thispagestyle{empty}
\movetooddpage
\chapter[Posfácio, \emph{por Vinicius Matteucci de Andrade Lopes}]{Posfácio \subtitulo{Friedrich Engels e o ponto de saída\\ da filosofia clássica alemã}}

\begin{flushright}
\textsc{vinicius matteucci de andrade lopes}
\end{flushright}

\noindent{}Em sua clássica biografia teórica sobre o jovem Marx, Lukács sugere,
para além das qualidades investigativas, determinação e rigor, a
existência na obra de Marx de uma intima coincidência, que cabe a
``poucas personalidades da história mundial'', ``entre desenvolvimento
teórico individual e desenvolvimento social geral''.\footnote{``O
  processo de superação do hegelianismo, também a ultrapassagem por
  Feuerbach, a fundação da dialética materialista, coincide, no seu
  processo formativo, com o desenvolvimento da posição da democracia
  revolucionaria em direção ao socialismo consciente. Ambas as linhas
  formam uma unidade necessária, todo o processo, porém, transcorre ---
  de modo algum contingente --- em um período da história alemã, na qual
  --- após a ascensão do trono Frederico Guilherme \textsc{iv} na
  Prússia, após a conversão reacionário"-romântica da política interna
  prussiana --- estabelece"-se uma agitação universal política e
  ideológica: o período de preparação da revolução democrático"-burguesa
  de 1848''. (\textsc{lukács}, Georg. \emph{Der junge Marx. Seine
  philosophische Entwiklung von 1840--1844}. Stuttgart. Verlag Günther
  Neske Pfullingen, 1965, p.6)}

Como se sabe, um elemento fundamental para construção dessa ``intima
coincidência'', que se expressa em diversos textos de Marx,
principalmente os com veiculação jornalística, é que muitos eram
respostas diretas a momentos políticos, acontecimentos históricas e
debates teóricos contemporâneos. O alcance e profundidade da crítica é
traço a ser medido pela singularidade de cada texto. E aqui sem
mistificar a abstração \emph{marxismo} --- constituída principalmente a
partir do século \textsc{xix} como sintoma conjuntural, muito mais a
favor dos seus detratores do que dos seus defensores --- termo que
Engels reputa como correto em relação ao seu próprio modo investigativo,
já que a rubrica \emph{marxismo} serve justamente para diferenciar dos
materialismos, socialismos e positivismos do final do século
\textsc{xix}, afinal sem Marx, já morto há cerca de três anos, ``a
teoria não seria hoje, nem de longe, aquilo que ela é. Ela tem,
portanto, também com razão, seu nome''.\footnote{\emph{Ludwig Feuerbach
  e o fim da filosofia clássica alemã (parte \textsc{iv})}.}

A obra de Engels, por sua vez, também é exemplo dessa rara
\emph{confluência}. Um fator fundamental é a mesma interlocução direta
via textos jornalísticos com seu tempo histórico. Nesses textos, tanto
Marx como Engels enfrentaram abertamente o que na metade do século
\textsc{xix} europeu era inevitável: o impacto da expansão do mundo
burguês em \emph{todas} as suas dimensões. Mais de meio século depois da
burguesia francesa estabelecer a retórica política de legitimação de sua
estrutura de poder, a totalidade dessa expansão, já há muito ``visível''
do \emph{Standpunkt} europeu em uma de suas figuras centrais, o avanço
do poder do ``mercado mundial'', torna"-se inquestionável no século
\textsc{xix} com o estabelecimento da grande indústria, como o próprio
Engels escreve:

\begin{quote}
Desde a efetivação (\emph{Durchführung}) da grande indústria,
portanto, pelo menos, desde a paz europeia de 1815, não era mais segredo
para homem nenhum na Inglaterra que lá toda a luta política girava em
torno das pretensões de dominação de duas classes: a aristocracia
possuidora de terras (\emph{landed aristocracy}) e a burguesia
(\emph{middle class}). Na França, a consciência do mesmo fato foi obtida
com o regresso dos Bourbons; os historiadores da época da Restauração de
Thierry a Guizot, Mignet e Thiers, falam disso, por toda a parte, como a
chave para a compreensão da história francesa desde a Idade Média. E,
desde 1830, em ambos os países, a classe dos trabalhadores, o
proletariado, foi reconhecido como a terceira força por essa dominação.
\end{quote}

A consciência desse processo se explicita não apenas na reestruturação
dos modos e relações de produção, na transformação da relação
campo"-cidade em diversas regiões europeias, mas também nas mais
abstratas formas de consciência, condicionadas à revelia de suas mais
nobres intenções. Afinal, como também aponta Engels, ``a moral de
Feuerbach esta talhada pela atual sociedade capitalista, por mais que
ele próprio não queira isso ou possa suspeitar''.\footnote{\emph{Ludwig
  Feuerbach e o fim da filosofia clássica alemã (parte \textsc{iii})}.}

A presente obra é, portanto, um exemplo paradigmático desse
enfrentamento de conjuntura. Tem sua origem justamente em um convite que
Engels recebeu dos editores da revista social"-democrata, \emph{Die
Neue Zeit}, provavelmente de Kautsky,\footnote{\emph{Apparat},
  \textsc{mega} (\textsc{i}.30, 2011), p.738.} para realizar uma
resenha crítica da mencionada tese de doutorado, recém"-lançada em
1885, do filósofo dinamarquês Carl Nicolaj Starcke: uma exposição
filosófica do desenvolvimento teórico de Feuerbach. Essa informação
importa aqui, já que, conforme adiantado, a resenha empreendida por
Engels não corresponde a uma revisão da obra de Starcke, uma retomada
crítica de Feuerbach ou, a partir disso, uma crítica conceitual interna
dos pressupostos da filosofia de Feuerbach. Quem buscar qualquer uma
desses pontos no texto dificilmente acompanhará a sua potência crítica.
Como indicado no \emph{Apparat} da \textsc{mega} (\textsc{i}.30, 2011),
trabalhado por Renate Merkil"-Melis:

\begin{quote}
A monografia de Starcke sobre Feuerbach ofereceu a Engels a
possibilidade de confrontar de modo exemplar as novas tendências no
pensamento filosófico, recusa de Hegel, neokantismo ético, positivismo.
O livro {[}de Starcke{]} destacava"-se tanto como uma medição
intencional entre idealismo e realismo, como enquanto um direcionamento
para metafísica e ética.\footnote{\emph{Idem}.}
\end{quote}

Essa tentativa de mediação entre uma metafísica e uma ética que Starcke
vê na obra de Feuerbach é um traço central do embate das ideias na
Alemanha desde 1830, principalmente tendo em vista os efeitos práticos
da fraseologia da Revolução Francesa, ou como coloca Starcke no prefácio
da obra, dos efeitos da ``indeterminação'' da Revolução de 1789, por
meio da qual direitos abstratos como a liberdade universal\footnote{``Na
  Revolução de 1789 tudo ainda era indeterminado; falava"-se de direitos
  humanos, mas esses eram apenas frases feitas, na medida em que não
  conseguiam oferecer para grande maioria um elemento determinado, real,
  para que os homens tivessem direitos''. \textsc{starcke}, C. N.
  \emph{Ludwig Feuerbach.} Stuttgart: \emph{Verlag von Ferdinand Enke},
  1885, p.\textsc{vii}} passam a ser considerados como inatos ao homem,
deixando em abertos questões como: qual homem? Qual liberdade? Qual a
estrutura desse caráter inato? A ética e filosofia da religião de
Feuerbach são, para Starcke, tentativas de respostas \emph{não
sistêmicas} a essas questões.

\section*{A ilusão burguesa da autonomia da vontade}

Para Engels, leitor do real via obra de Starcke, ou mesmo para Marx e
Engels, leitores do real via Feuerbach em 1844/45 nos manuscritos da
\emph{Ideologia Alemã}, qualquer formulação de resposta metafísica,
moral ou ética a essas questões conduz inevitavelmente às confusões
fundamentais do mundo burguês: entre elas, apreender o Estado moderno
enquanto ``realidade efetiva da liberdade concreta'';\footnote{\textsc{hegel},
  G. W. F. \emph{Grundlinien der Philosophie des Rechts oder Naturrecht
  und Staatswissenschaft im Grundrisse}, Frankfurt am Main, Suhrkamp
  Verlag, 1970, p.415 (§ 270). ``Que o fim do Estado seja o interesse
  universal como tal e que, nisso, seja a conservação dos interesses
  particulares como substância destes últimos, isso é 1) sua
  realidade"-efetiva abstrata ou substancialidade; mas esta última é 2)
  sua necessidade, enquanto ela se divide nas distinções conceituais de
  sua atividade, que são, do mesmo modo, graças àquela substancialidade,
  determinações estáveis e reais, poderes; 3) porém, tal
  substancialidade é, precisamente, o espírito que, por haver passado
  pela forma de uma tradição cultural {[}Die \emph{Form der Bildung}{]},
  sabe"-se e quer a si mesmo. O Estado sabe, por isso, o que quer, e o
  sabe em sua universalidade, como algo pensado; ele age e atua, por
  isso, segundo fins sabidos, princípios conhecidos e segundo leis que
  não são somente em si, mas para a consciência; e, do mesmo modo, na
  medida em que suas ações se atêm às circunstâncias e relações
  existentes, age e atua segundo o conhecimento determinado que tem
  delas.''} ou ainda, a dimensão interna dessa concepção, apreender os
imperativos político"-morais (o \emph{Sollen, o dever"-ser}) como
resoluções formais da relação entre vontade e devir, isto é, como
\emph{locus} concreto da resolução da tensão entre vontade subjetiva e
história.

Na semântica de uma metafísica burguesa sobre a forma da autonomia da
vontade, o problema se coloca como uma tentativa de delimitação da
diferença entre um \emph{sollen subjetivo} e um \emph{objetivo}, entre o
elemento posto objetivamente pela vontade e a indeterminação subjetiva
da constituição dessa vontade. Essa delimitação passa pela justificação
dos \emph{imperativos sociais como elementos subjetivos sempre já
superados} pelo \emph{devir}, pela \emph{``forma da formação
cultural''}, nos termos de Hegel. O Estado aparece como a conservação
(\emph{Erhaltung}) da ``substância'' do interesse particular que se
expressa enquanto interesse universal justamente por ter atravessado
\emph{uma} \emph{forma de formação cultural}. A crítica do jovem Marx a
Hegel aponta justamente para neutralização dessa relação recíproca entre
universal e particular, questionando a ``omissão das determinações
concretas''\footnote{\textsc{marx}, K. \emph{Zur Kritik der Hegelschen
  Rechtsphilosophie. Kritik des Hegelschen Staatsrechts}. \textsc{mew},
  Band 1, p.217.} diante das determinações abstratas. Omissão esta,
imposta pelo mesmo movimento da concretude, que revela o caráter
arbitrário das abstrações, sempre a favor de \emph{uma forma} de
``formação cultural'' que \emph{aparece} como \emph{única forma
possível}. Aqui a conhecida proposição de Marx, para ficarmos ainda em
1843, por meio da qual essa distorção funciona: ``o momento filosófico
não é a lógica da determinação prática (\emph{Sache}) {[}como pretende
Hegel e as variadas formas de consciência no mundo burguês{]}, mas a
determinação pratica (\emph{Sache}) da lógica {[}como ocorre à revelia
da intenção dos filósofos{]}''.\footnote{Idem, p.216.}

Marx e Engels, enquanto historiadores do processo de formação do mundo
burguês, questionam justamente o \emph{locus} de aparecimento da relação
(\emph{Verhältnis}) constitutiva dos imperativos sociais. A crítica de
Marx à relação"-capital (\emph{Kapitalverhältnis}) implode justamente
qualquer pretensão de localizar essa relação em uma mediação, mais ou
menos direta, entre \emph{razão objetiva} e \emph{subjetividade moral.}
Em termos kantianos, criticados via Feuerbach por Engels (parte
\textsc{iii}), a ``impotência'' do imperativo categorial, a pretensão de
se livrar das condições empíricas, escapar às limitações arbitrarias e
imposições objetivas, e alcançar uma forma universal da subjetividade
não passa de uma potência sistêmica. Na \emph{Fundamentação da
metafísica dos costumes} (1785), Kant defende que a máxima subjetiva não
seria dada pelas condições objetivas, mas já expressaria um conteúdo
externo à lei posta --- pelo Estado ou tradição cultural --- localizada
em um espaço imaginado \emph{entre} o subjetivo e o objetivo. A
universalidade se provaria, nesse lugar, como forma ``\emph{natural}''
da vontade, como se fosse uma decantação subjetiva da realidade. Como se
o Estado, ou o próprio elemento político do mundo burguês, fosse uma
\emph{decantação natural} de uma \emph{vontade histórica} universal. O
\emph{caráter natural}\footnote{Citar o Kant: ``Devido ao fato da
  universalidade da lei, segundo a qual efeitos acontecem, constituir
  aquilo a que, na realidade, chama"-se \emph{natureza} no sentido mais
  amplo da palavra (quanto à forma), isto é, em relação à existência das
  coisas, enquanto determinada por leis universais, o imperativo
  universal do dever (\emph{Pflicht --- obrigação}) poderia também
  exprimir"-se assim: \emph{age como se a máxima da tua ação devesse se
  tornar, pela tua vontade, lei universal da natureza}.''
  (\textsc{kant}, I. \emph{Grundlegung zur Metaphysik der
  Sitten.}Frankfurt am Main, Suhrkamp Verlag, 2007, p.53)} se
expressaria como aquilo que é determinado pelo elemento universal, isto
é, localizado tanto \emph{fora da vontade subjetiva} como \emph{fora das
condições objetivas}. A \emph{vontade histórica} seria o outro lado da
mesma moeda, já que explicitaria, enquanto devir histórico, o caráter
\emph{interno} de uma suposta autorresolução da tensão entre vontade
subjetiva e condições objetivas.

É bastante ilustrativa, nesse sentido, a conhecida passagem de Marx na
\emph{Miséria da filosofia}, onde critica o fato dos economistas
burgueses dividirem as instituições em ``artificiais''
(\emph{künstliche} / \emph{de l'art}) e ``naturais'' (\emph{natürliche /
de la nature}). Assim como os teólogos defendem a religião dos outros
como criação dos homens e a sua como revelação de Deus, os economistas
burgueses defendem suas instituições, ``suas relações no presente'',
como revelações de leis naturais eternas que devem governar toda
sociedade, as instituições feudais, as relações de produção feudal,
seriam históricas, necessariamente transitórias\emph{:} ``Havia assim
uma história, mas agora não há mais.''\footnote{\textsc{marx}, K.
  \emph{Das Elend der Philosophie,} \textsc{mew} 6, p.139.} Ou, de modo
mais enfático: havia uma história assim, \emph{justamente por isso}, não
há mais.

Na realidade, como não cansaram de demonstrar Marx e Engels, o Estado
moderno, o elemento político burguês, ou a forma da vontade (\emph{a
Sitllichkeit}) burguesa, nada mais são do que uma decantação
\emph{historicamente imposta} (aqui a igualação entre interesse
universal e a forma da formação cultural apontada por Hegel) de uma
vontade igualmente histórica, mas que não se constitui como caráter
\emph{interno} da autorresolução da tensão entre vontade subjetiva e
condições objetivas. Essa autorresolução não é a revelação da verdade,
ainda que contenha elementos concretos, justamente porque ela não é o
resultado automático dessa autorresolução. E ela não é automática
porque não há o \emph{fora idealizado.} Nem como transcendental, nem
como transcendente. Esse seria um elemento básico da dimensão
materialista, necessariamente histórica. Toda a crítica da
economia"-política de Marx explicita as diversas dimensões desse
\emph{fora idealizada}. Um fora que é idealizado e conjurado pelo
movimento concreto, basta atentar, por exemplo, para a cisão entre
potencias intelectuais do processo de produção e o trabalho manual,
mediado pelo fora idealizado do trabalho abstrato, que se apresenta
historicamente quando o autômato da grande indústria se põe.\footnote{``Toda
  produção capitalista, por ser não apenas processo de trabalho, mas, ao
  mesmo tempo, processo de valorização do capital, tem em comum o fato
  de que não é o trabalhador que emprega as condições de trabalho, mas,
  ao contrário, são estas últimas que empregam o trabalhador; porém,
  apenas com a maquinaria essa inversão adquire uma realidade tangível.
  Transformado num autômato, o próprio meio de trabalho se confronta,
  durante o processo de trabalho, com o trabalhador como capital, como
  trabalho morto a dominar e sugar a força de trabalho viva. A cisão
  entre as potências intelectuais do processo de produção e o trabalho
  manual, assim como a transformação daquelas em potências do capital
  sobre o trabalho, consuma"-se, como já indicado anteriormente, na
  grande indústria, erguida sobre a base da maquinaria``(Das Kapital,
  \emph{p.495}). Para uma indicação clara e concisa do trabalho
  abstrato, entendido não como uma categoria lógica abastrata,mas como
  um movimento histórico que se expressa pela divisão do trabalho na
  grande industria, conferir: \textsc{versolato}, Rafael. \emph{O
  mistério do real: capital e trabalho assalariado}. 2016. Faculdade de
  Filosofia Letras e Ciências Humanas. Departamento de Filosofia.
  Universidade de São Paulo, São Paulo, 2016, p.220 e seguintes} A
exposição de Engels toma esse ponto como pressuposto evidente.

Essa \emph{fora idealizado,} essa ``dominação violenta estanha''
(\emph{fremde Gewalt}), indicado por Marx e Engels já em 1844/45,\footnote{``O poder (\emph{Macht}) social, isto é, a força de produção
  multiplicada que nasce da cooperação (\emph{Zusammenwirken ---
  atividades que visam um fim comum}) dos diversos indivíduos
  condicionada pela divisão do trabalho, aparece a esses indivíduos,
  porque a própria cooperação não é voluntária, mas espontânea
  (\emph{naturwüchsig}), não como seu próprio poder (\emph{Macht})
  unificado, mas como um poder"-violência (\emph{Gewalt}) externo
  (\emph{fremde}), situado fora deles, sobre o qual não sabem de onde
  vêm ou aonde vão, uma violência (\emph{Gewalt}), portanto, que não
  podem mais dominar {[}e{]} que, pelo contrário, percorre agora uma
  sequência particular de fases e etapas de desenvolvimento,
  independente do querer e da conduta do ser humano (\emph{Luafen des
  Menschen}) e que até mesmo dirige esse querer e conduta.''
  \textsc{marx} K.;\textsc{engels} F. Die deutsche Ideologie.
  \emph{Werke}. p.29.} porém, e isso é central, não é
\emph{contingente}. Pelo contrário, sua existência é o sinal da
evidência da sua contradição: ``As relações se simplificaram tanto que
era preciso fechar os olhos propositalmente para não ver na luta dessas
três grandes classes {[}a aristocracia possuidora de terras, burguesia e
trabalhador assalariado{]} e no conflito de seus interesses, a força
impulsionadora da história moderna --- pelo menos, nos dois países mais
avançados.''\footnote{\emph{Ludwig Feuerbach e o fim da filosofia
  clássica alemã (parte \textsc{iv})}.} O desenvolvimento histórico do
mundo burguês, assim como a complexa luta de classes que lhe é imanente,
revela que a dinâmica do movimento do real, que aparece a todos como o
poder estranho concreto de uma formação cultural universal, é um
artifício \emph{sistêmico e arbitrário} de converter os interesses
particulares da \emph{forma burguesa} na \emph{forma universal do
interesse}.

A clareza com que essa distorção se expressa com a grande indústria,
como aponta Engels, ``atravessa'' (\emph{Durchführung}) a divisão social
do trabalho no século \textsc{xix}, alterando o \emph{ponto de saída da
reflexividade histórica} no mundo burguês. Tanto enquanto acúmulos de
círculos de reprodução do mais"-valor (\emph{reprodução ampliada}),
gerações de trabalhadores já nascem sob a forma da relação"-capital,
como enquanto processo de reposição, dos modos mais variados, da
expropriação das formas de consciência. Uma alteração complexa,
inquestionável, que não se reduz a uma relação imediata e causal entre
divisão social do trabalho e formas de consciência e permite questionar
tanto a ideia \emph{uma} \emph{pragmática} \emph{da} \emph{história} ---
que somente poderia ser ``apreendida (\emph{abgefasst}), se produz
(\emph{macht}) prudência (\emph{klug}), isto é, se ensina ao mundo como
ele poderia assegurar sua vantagem melhor, pelo menos tão bem quanto o
mundo precedente''\footnote{\textsc{kant}, I. \emph{Grundlegung zur
  Metaphysik der Sitten.}Frankfurt am Main, Suhrkamp Verlag, 2007, p.47.}
--- como a ideia de uma história que entende o presente como resultado
fatalmente contingente de algo sempre já superado, isto é, como
naturalização do elemento transitório. Ambas as concepções correspondem
às duas faces da ``forma da \emph{Bildung}'' burguesa.

A necessidade de enfrentar racionalmente a contingência das contradições
do mundo burguês como \emph{totalidade em movimento} é um ponto
fundamental do texto de Engels. Uma exigência primária para enfrentar
essa totalidade é justamente conseguir visualizá"-la, ter
\emph{consciência do alcance de sua dinâmica}. A discussão sobre o modo
de investigação, sobre o método da crítica, não é, portanto, uma questão
que se coloca separada da luta política concreta. Na realidade, a
possibilidade dessa separação, que irá contaminar o marxismo do século
\textsc{xx} por meio das mais variadas formas de positivismos, começa a
se colocar como tendência a partir desse período de 1880, como o próprio
Engels indica, com a reabilitação de Kant e Hume. Mais à frente
retonaremos a isso.

\section*{Consciência histórica e totalidade}

Em relação ao modo investigativo, o texto é uma \emph{tentativa de
exposição sintética} do materialismo histórico e de como somente por
meio dele uma crítica do desenvolvimento histórico burguês é possível.
Lukács em seu ensaio, \emph{Consciência de classe}, toma a ``famosa''
exposição de Engels como ponto de partida para introduzir sua definição
de \emph{consciência de classe}, mais precisamente de \emph{consciência
classe imputada/atribuída} (\emph{zugerechnet}). Em um diálogo
subjacente com a clássica crítica de Lenin (\emph{Que fazer?}) à
socialdemocracia russa (e alemã) sobre a formação de uma classe
revolucionária --- se espontânea, isto é, resultado automático do
desdobramento das contradições do mundo burguês, ou se consciente,
trazida ``de fora'' da relação patrão"-empregado por estruturas
pratico"-teóricas capazes de fazer frente à marcha inexorável do avanço
do desenvolvimento capitalista e da ideologia burguesa ---, o ponto de
partida de Lukács é a crítica à falsa objetividade concreta da
racionalidade histórica burguesa. ``Seu erro consiste em querer
encontrar esse elemento concreto no indivíduo histórico empírico (não
importa se trata de uma pessoa, uma classe, ou povo) e na consciência
empiricamente dada (isto é, dada por uma psicologia ou por uma
psicologia das massas)''.\footnote{\textsc{lukacs}, G. \emph{Geschichte
  und Klassenbewusstsein.Studien über marxistische Dialektik}. Verlag de
  Munter, Amsterdam, 1967, p.61.} Na sequencia de sua argumentação,
Lukács apresenta a retomada --- em sentido contrário ao
anti"-hegelianismo da social democracia alemã, principalmente em Kaustky
--- da categoria da \emph{totalidade concreta, da sociedade
civil"-burguesa como totalidade}. É curioso que Kautsky fora um
entusiasta desse mesmo texto de Engels, que permitiu, porém, outro
caminho de leitura a Lukács.

Um aspecto da dificuldade de uma crítica do desenvolvimento da
totalidade histórica burguesa reside justamente na particularidade do
processo de reprodução da racionalidade político"-moral no interior da
racionalidade histórica, já que na concretude da dinâmica histórica
ambas são a mesma coisa. Como vimos brevemente elementos dessa confusão
concreta haviam sido apontados por Marx já em 1843 na crítica a Hegel.
No presente texto, Engels expõe a inevitabilidade da constituição dessa
totalidade do ponto de vista da relação entre vontade e história de modo
bastante claro:

\begin{quote}
Os homens fazem a sua história, aconteça ela como acontecer, na medida
em que cada um persegue conscientemente as finalidades que eles mesmos
querem, e a resultante destas várias vontades que atuam em direções
diversas e da sua influencia múltipla sobre o mundo exterior é
justamente a história. Depende, portanto, do que os muitos indivíduos
querem. A vontade é determinada por paixão ou reflexão. Mas as alavancas
que, por sua vez, determinam imediatamente a paixão ou reflexão, são de
tipos muito diversos. Em parte podem ser finalidades exteriores, em
parte, \emph{fundamentos ideais do movimento} (\emph{ideelle
Beweggründe}), ambição, ``entusiasmo pela verdade e pela justiça'', ódio
pessoal, ou também caprichos puramente individuais de toda a espécie.
Mas, por outro lado, vimos que as várias vontades individuais ativas na
história, na maioria dos casos, produzem resultados totalmente
diferentes dos queridos --- \emph{muitas vezes contrapostos} --- e que,
portanto, para o resultado do todo, seus fundamentos de movimento têm um
significado subordinado. Por outro lado, é possível questionar ainda
mais: quais forças impulsionadoras estão novamente por detrás destes
fundamentos do movimento, que causas históricas transformam, na cabeça
dos agentes, esses fundamentos de movimento?
\end{quote}

A concepção materialista precisa necessariamente enfrentar esse enigma
de compreender \emph{o todo} das inter"-relações que movimentam
\emph{consciente e inconscientemente} suas ações:

\begin{quote}
Quando se trata, portanto, de investigar as potências impulsionadoras
que --- consciente ou inconscientemente e, de fato, frequentemente,
inconscientemente --- estão por detrás dos fundamentos de movimento dos
homens que agem historicamente e que constituem propriamente as forças
motrizes últimas da história, não se pode, assim, tratar dos fundamentos
de movimento dos indivíduos, mesmo os homens que agem de modo eminente,
enquanto aqueles que põem em movimento grandes massas, povos inteiros e,
em cada povo, por sua vez, classes inteiras do povo; e isso também não
momentaneamente em uma explosão temporária e fogo de palha que queima
rapidamente, mas em uma ação duradoura que se alastra em uma grande
transformação histórica.
\end{quote}

Os fundamentos do movimento, sejam eles ideias ou não, direcionam o
olhar necessariamente para além dos processos automáticos de
subjetivação ou repetição objetiva dos limites concretos, seja da psique
ou da classe social. Mas é justamente por isso que eles são apreensíveis
e podem ser explicados por suas causas históricas concretas, porque seu
ponto de saída não é simplesmente o amplo espectro que compõe a
subjetividade, mas justamente aquilo que escapa ao arco da vontade: ``as
várias vontades individuais ativas na história, na maioria dos casos,
produzem resultados totalmente diferentes dos queridos --- \emph{muitas
vezes contrapostos} --- e que, portanto, para o resultado do
\emph{todo}, seus fundamentos de movimento têm um significado
subordinado.''

A colisão das vontades, a contradição da luta em movimento, escapa ao
velho materialismo de Feuerbach, tanto quanto escapa a qualquer
racionalidade que não considera o arco de constituição dos fins da
vontade de um tempo histórico, e daquilo que é encoberto no presente,
podendo ou não se realizar no futuro. Ainda assim, algo se põe e
constitui uma estrutura interna do movimento, que não está ao alcance da
consciência, tampouco é resultado de uma autorresolução entre vontade
subjetiva e condição histórica objetiva, como indicado acima. ``Assim,
as colisões das inúmeras vontades singulares e ações singulares no
âmbito histórico proporcionam um estado que é totalmente análogo ao que
domina na natureza \emph{desprovida de consciência}.''\footnote{\emph{Ludwig
  Feuerbach e o fim da filosofia clássica alemã (parte \textsc{iv})}.} A
analogia, antes de configurar um princípio heurístico de apreensão da
realidade histórica, explicita justamente que somente há um acesso
possível e crítico da realidade histórica quando considerada a questão
básica do caráter de \emph{movimento do elemento histórico}. O interesse
subjetivo da ação sempre se põe em movimento, pois ele mesmo é resultado
ideal de um movimento anterior que se projeta objetivamente no limite
dessa idealidade no presente. Esse limite é a realidade contraditória
que forma a consciência de sua existência no presente, a qual não pode
ser buscada meramente naquilo ``que é pensado, sentido e querido
\emph{factualmente} em determinadas condições históricas em situações
determinadas de classe, etc.''.\footnote{``A inter"-relação com a
  totalidade concreta e as determinações dialéticas dela decorrentes
  ultrapassam a simples descrição e resultam na categoria da
  possibilidade objetiva. Ao relacionar a consciência com o todo da
  sociedade {[}como uma \emph{inter"-relação}{]}, são reconhecidos todos
  os pensamentos, sensações, etc, que os homens \emph{teriam} em uma
  situação determinada de vida, se eles \emph{fossem capazes de
  apreender completamente (vollkommende --- consumando até o fim)} essa
  situação, os interesses dela resultantes, tanto em relação à ação
  imediata como em relação a --- conforme tais interesses --- estrutura
  constitutiva (\emph{Aufbau}) de toda a sociedade; os pensamentos, etc,
  isto, os que estão em conformidade com tal situação objetiva. Em
  nenhuma, o numero de tais situações de vida é ilimitado. Mesmo caso se
  busque aperfeiçoar ainda mais sua tipologia por meio de pesquisas
  singulares detalhas, resulta porém, em alguns tipos fundamentais
  completamente afastados um do outro. Tipos fundamentais cujo elemento
  essencial (\emph{Wesenart}) é determinado por meio da tipologia da
  posição dos homens no processo de produção. Aqui a reação
  racionalmente adequada que é \emph{imputada}
  {[}\emph{zugerechnet} --- correta, justificada, adjudicada
  --- algo é lançado à adequação por uma ação
  imposta/objetiva{]} no modo de uma situação típica
  determinada no processo de produção, é a consciência de classe.''
  (\textsc{lukacs}, G.\emph{Geschichte und Klassenbewusstsein}.
  \emph{Op.Cit.} p.62)} Apesar da autocrítica lukacsiana da
incapacidade prática da \emph{consciência da falsa consciência}, da
consciência de classe historicamente imputada,\footnote{\textsc{lukacs},
  G. \emph{Geschichte und Klassenbewusstsein}. \emph{Op.,Cit. Vorwort}
  (Prefácio) 1962: ``A transformação da consciência ``imputada'' em
  consciência revolucionária aparece aqui, considerada objetivamente,
  como um puro milagre''.} o ponto de partida do problema continua o
mesmo apontado tanto por Engels como por Lukács: o caráter absolutamente
inevitável da relação \emph{consciente"-inconsciente} com a totalidade
histórica.

\begin{quote}
Tudo o que põe os homens em movimento tem de passar por sua cabeça; mas
que configuração toma nessa cabeça, depende muito das circunstâncias. Os
trabalhadores, sob nenhuma circunstância, reconciliaram"-se com o
maquinário fabril capitalista, mesmo que não mais tenham simplesmente
quebrado em pedaços as máquinas, como ainda em 1848 no Reno.\footnote{\emph{Ludwig Feuerbach e o fim da filosofia clássica alemã (parte \textsc{iv})}.}
\end{quote}

Aqui para muitos, como para a crítica social alemã dos anos 1920 e 1930,
salta aos olhos da concretude o erro da última passagem. A
\emph{reconciliação} entre uma posição de classe e circunstâncias
contrárias teria se mostrado como um fenômeno corriqueiro. Na realidade,
a questão dessa reconciliação, não por acaso, uma dimensão subjacente às
diversas discussões sociaisdemocratas do entre guerras, é expressão de
mais um sintoma da expansão da relação"-capital, que reforça os
processos de expropriação das formas de consciência\footnote{Dois
  exemplos de exposição dos mecanismos de \emph{expropriação das formas
  de consciência} podem ser encontrados na crítica da anatomia da classe
  média alemã reacionária dos anos 1920 e 1930, fração dos trabalhadores
  que irá apoiar o Nacional Socialismo, presente na análise
  literário"-sociológica de Siegfried Kracauer de 1930, \emph{Os
  empregados,} e na obra de Ernst Bloch de 1935, \emph{A herança dessa
  época,} onde inclusive há uma brilhante definição da estrutura
  ideológica que forma essa figura da ``classe média reacionária'',
  definida por Bloch com um termo ambíguo em alemão, mas que, bem
  entendido, é quase autoexplicativo: \emph{dispersão"-distrativa
  (Zerstreuung)}.} na época do \emph{imperialismo}, momento em que na
passagem do século \textsc{xix} para o \textsc{xx} a totalidade
sistêmica do capital já posta há mais de meio século começa a se
transfigurar em outros elementos totalizantes: nacionalismo, pátria,
progresso, técnica. A ``falsa consciência'' de classe é poderosa
justamente porque ela é resultado do impulso reflexivo dessa
``reconciliação'', inatingível por meio da dimensão subjetiva da
consciência, mas que explicita o limite objetivo do movimento histórico
de constituição da divisão social do trabalho. Esse limite expressa a
impossibilidade da ``reconciliação''. A começar porque nunca houve uma
unidade anterior a ser reconciliada entre o trabalhador no mundo burguês
(entendido na dinâmica do devir da escravidão e servidão para forma
assalariada) e as circunstâncias, mesmo se pensarmos na ``cisão entre o
trabalhador e a \emph{propriedade} das condições de efetivação de seu
trabalho'' da acumulação originária (primitiva), já que a relação das
passagens históricas da cooperação até a grande indústria com os
impulsos violentos que geram a cisão (\emph{Trennung}), entre eles, por
exemplo, a Revolução Gloriosa mencionada por Engels nesse texto, indicam
que não é possível considerar essa não"-cisão a partir de uma
\emph{conciliação anterior}. A subjetividade da consciência não alcança
os fins que coloca, pois eles nunca são apenas seus, ainda que se creia
profundamente nisso. A objetividade do desenvolvimento histórico se põe
a partir e para além da subjetividade. Para além da imposição sistêmica
do ``ter que trabalhar'', poucos trabalhadores ``se reconciliariam'' com
a vida que levam no mundo do império do livre"-arbítrio, como é \emph{a
aparência} do mundo burguês, sem um reforço sistêmico de elaboração de
\emph{convicções}, sem um reforço sistêmico de todas as ``pedagogias''
para \emph{servidão voluntária}, muito mais eficientes do que La Boétie
poderia imaginar em 1563.

\section*{O destino da filosofia alemã com\break a consolidação do mundo burguês}

As contradições do mundo burguês a partir do século \textsc{xix}
\emph{escapam} para todos os lados. É \emph{impossível não sentir os
efeitos da contradição histórica da expansão da relação"-capital na
Alemanha de 1886}. O \emph{fim} da época da filosofia clássica alemã,
nesse sentido, não é simplesmente o \emph{ponto final} da filosofia
clássica. A filosofia clássica não se esgotou com a morte de Hegel em
1831. E a referência ao neokantismo na Alemanha e à retomada de Hume na
Inglaterra pelo próprio Engels indica justamente que não se trata de um
fim da filosofia clássica alemã em si mesma, mas de um \emph{ponto de
saída} da \emph{relação} da filosofia alemã com \emph{o devir do
desenvolvimento histórico burguês}. A metade do século \textsc{xix} é o
ponto de transição, de saída, para o momento em que o mundo burguês
\emph{fica sobre os próprios} \emph{pés}, completa esse devir e revela
as contradições do seu próprio desenvolvimento, encobertas pelo seu
próprio devir. Essa filosofia torna"-se clássica justamente porque se
converte em ponto de mediação intransponível com o que vem antes dela, o
que expressa seu caráter sintomático, já que não pode ser pensada como
tendo o mesmo papel que tinha anteriormente, distorcida como uma
\emph{filosofia em si mesma}. Claro, Engels sabia há muito que não há
algo como uma \emph{filosofia em si mesma} separada da realidade:
platonismo, kantismo, hegelianismo, nem mesmo marxismo. O limite do
movimento do efetivo, do real, é a consciência (filosófica, jurídica,
política, econômica, religiosa, moral, etc), tanto quanto a consciência
está limitada pelo efetivo, aqui colocando em outras palavras a
conhecida e mal compreendia proposição da \emph{Ideologia Alemã}: ``Não
é a consciência que determina (\emph{bestimmt --- indica o limite, o
termo}) a vida, mas a vida que determina (\emph{bestimmt --- indica o
limite, o termo}) a consciência''.

Em 1848 estávamos diante do \emph{ponto de saída da filosofia clássica
alemã}. O titulo, o \emph{fim (Ausgang}) \emph{da filosófica clássica
alemã}, que desde a tradução francesa de 1894 revisada pelo próprio
Engels --- \emph{Ludwig Feuerbach et la fin de la philosophie classique
allemande} --- tornou"-se cânone das traduções latinas, induz, portanto,
a uma limitação. Não se trata, porém, de um erro de tradução.
Normalmente \emph{Ausgang} é traduzido como ``saída'', ``ponto de
saída'', ao contrário do termo ``fim'', \emph{Ende,} mas no campo
semântico em que foi empregado, relacionado a uma época histórica, cabe
a tradução por ``fim'', desde que se tenha em vista essa dimensão de ser
\emph{um ponto de passagem ou de saída} para outra época histórica. Esse
\emph{ponto de saída} é o que define o enfrentamento de conjuntura
empreendido por Engels em 1886, já que, como apontado acima, e
certamente ficará claro ao leitor do livro, a filosofia de Feuerbach
apresentada por Starcke, sua crítica ou defesa, não é o objeto da obra.
Feuerbach é apresentado no máximo como catalisador de um \emph{ponto de
transição} que a própria realidade já evidenciava nos anos 1840 alemão.

Engels define a conjuntura 1848/1886 como resultado e explicitação do
processo de expansão da relação"-capital, apreendido pela via
\emph{prussiana,} que virá a ser tematizada por ele próprio em
1887/1888, em outro artigo a ser publicado na \emph{Die Neue Zeit}.\footnote{\textsc{engels}, F. \emph{Die Rolle der Gewalt in der
  Geschichte.} \textsc{mew}, 21. Dietz Verlag, Berlin. 5. Auflage, 1975.}
Três vias de unificação da Alemanha abriram"-se com o fracasso das
Revoluções de 1848, a primeira, a \emph{via alemã}, revolucionária que
suprimiria a autonomia político"-econômica dos pequenos estados
(\emph{Einzelstaaten}) por uma guerra de unificação conduzida pelo povo
contras Napoleão \textsc{iii} e as dinastias prussianas e austricas, a
segunda, \emph{a via pela dominação da Áustria} e a terceira, a via da
\emph{Realpolitik} de Bismarck, \emph{a via prussiana}. Como sabemos, a
última, não por acaso, impôs"-se, já que vinha sendo gestada desde 1830
com a \emph{Zollverein} (união aduaneira), quando a estrutura de
estabilização do modo de produção e relação burguesa começa a forjar a
unidade territorial e política de parte do atual território alemão, que
terá como ponto de chagada o \textsc{ii} Reich (\emph{Deutsches
Kaiserreich}).

Partindo da via prussiana, a metáfora do \emph{ponto de saída} atravessa
toda a argumentação de Engels e se estrutura em duas dimensões
indissociáveis. A primeira refere"-se ao fato de que ``na Alemanha do
século \textsc{xix}, a revolução filosófica preparou o colapso
político''. A crítica que preaprou o colapso político não foi, portanto,
empreendida por uma burguesia politicamente revolucionária e liberal,
apesar de ser possível questionar o verdadeiro papel social dos
``mandarins alemães''. De fato, na conjuntura 1830/1848 e depois em 1870
com o processo de unificação da Alemanha com Bismarck, o tempo de uma
burguesia revolucionária já havia passado. Uma imagem de unidade de
época se rompe. No plano filosófico, o sistema de Hegel é a última
tentativa de refletir uma unidade, principalmente na milenar relação
\emph{teologia/política/direito/filosofia.}

O caráter anti"-sistêmico, em oposição à sistematicidade da filosofia
hegeliana, que agradou Starcke na obra de Feuerbach, já havia chamado
atenção de Marx e Engels há 40 anos. Na realidade, um aspecto
conjuntural fundamental das Revoluções de 1848 na Alemanha. A
\emph{unidade política alemã,} sob a égide do avanço da \emph{unidade do
mundo burguês}, traz para o chão da concretude a abstrata unidade
teórica da antiga relação política/religião. A intenção filosófica dessa
unidade, que --- \emph{no plano das ideias} --- pode saltar
arbitrariamente de Aristóteles a Hegel, antes de cair na unidade do
\emph{absoluto}, implode justamente com a consideração interna da
dinâmica de transitoriedade da racionalidade dialética. Uma questão
elementar, apontada por Marx já na tese de doutorado (1837), apreendida
como crítica ao modo de intuição filosófico idealista, é que o propósito
do filosofar, ``do espírito teórico'', não é
\emph{explicar"-compreender} o mundo, mas \emph{intervir no mundo},
mesmo quando quer apenas \emph{interpretá"-lo}:

\begin{quote}
É uma lei psicológica que o espírito teórico, que vem a ser livre em
si (\emph{in sich}) mesmo, torne"-se uma energia prática, emergindo
enquanto \emph{vontade} do reino das sombras de \emph{Amenthes},
volte"-se contra a mundana, existente (\emph{vorhanden}) sem o
espírito, realidade efetiva (\textsc{i}.1/ 67 e 68).
\end{quote}

A filosofia como vontade (``como impulso de se efetivar'') se contorce
para fora (\emph{herauskehren}) do mundo fenomênico o trazendo consigo.
A tensão contraditória entre a relação de reflexividade da ideia e
autonomia pré"-existente (\emph{vorhande}) do mundo não pode ser
superada por si só, pois o movimento da ideia consome a si mesmo ao se
efetivar no mundo e altera: \emph{a si mesmo}, \emph{o mundo} e a
\emph{própria reflexividade}.\footnote{``Assim surge a consequência de
  que o vir a ser filosófico do mundo é ao mesmo tempo um vir a ser
  mundano da filosofia, que a sua efetivação é, ao mesmo tempo, sua
  perda (\emph{Mangel}), que, aquilo que ela combate no lado de fora é
  seu próprio defeito interno, que justamente na luta ela incorre em
  danos (\emph{Schade} --- falta de vínculos), que ela combate como
  danos e que somente os supera (\emph{aufheben}), na medida em que
  incorre nos mesmos''.(\textsc{mega} \textsc{i}.1/67 e ss).} A
pressuposição de uma autonomia pré"-existente, sempre à disposição da
força pura da ideia, sem perceber (isto é, sem considerar fundamental a
compreensão do movimento real das coisas) a \emph{influência recíproca}
concreta dessa relação \emph{contraditória} incompleta, em que tanto do
lado subjetivo da ideia, como do lado objetivo da realidade --- na busca
de uma relação proporcional (estável) capaz de expressar a medida das
coisas --- é um erro. Os excessos e elementos evasivos, danos
(\emph{Schaden}) no vocabulário da tese de 1837, são sempre inevitáveis.
Nem mesmo a sistematicidade hegeliana escaparia desse movimento, ou do
\emph{próprio movimento da história}, como Engels demonstra em alguns
aspectos. Por isso, a relação de Marx e Engels com a tradição do
idealismo alemão, passando justamente pelos jovens hegelianos e
Feuerbach, sempre teve em mente a contradição e a tensão entre ideia e
realidade, mas percebendo a possibilidade de compreender nessa tensão os
excessos e elementos evasivos, os danos que o embate entre ideia e
realidade impõe.

A filosofia hegeliana seria a expressão dos limites desse embate:

\begin{quote}
Com Hegel encerrasse a filosofia em geral. Por um lado, porque ele
reuniu em seu sistema, do modo mais grandioso, todo o desenvolvimento da
filosofia; por outro, porque, ainda que inconscientemente, nos mostra o
caminho para fora desse labirinto de sistemas em direção ao conhecimento
positivo e efetivo do mundo.\footnote{\emph{Ludwig Feuerbach e o fim
  da filosofia clássica alemã (parte \textsc{i})}.}
\end{quote}

A segunda dimensão do ponto de saída corresponde à contradição imanente
da relação \emph{reação"-revolução} da expansão do mundo burguês. Entre
1789 e 1851, a revolução burguesa se resolve na indeterminação entre
\emph{a reação} feudal à revolução burguesa em 1789 e \emph{a reação}
burguesa à \emph{possibilidade da} revolução proletária, principalmente
a partir da comuna de Paris em 1870. As duas reações se fundem e se
sobrepõe sistemicamente enquanto totalização da relação"-capital entre o
bonapartismo na França em 1851 e via prussiana com Bismarck, apesar da
evidente particularidade do desenvolvimento histórico de cada processo.

Para sintetizar algo complexo que vale uma obra, questão trabalhada em
muitos aspectos por Lukács, \emph{a via prussiana é a via de saída da
filosofia clássica alemã,} restando às intenções filosóficas teóricas,
de desvelar os mistérios do real, converterem"-se, a partir da metade do
século \textsc{xix}, \emph{na forma farsesca dos seus conteúdos
trágicos}, forçando aqui a brilhante e conhecida metáfora de Marx. A
razão disso, de modo sucinto, é a modificação do papel social do saber
filosófico, em razão da, nos termos de Engels, ``simplificação'' da
estrutura social europeia: ``tanto que era preciso fechar os olhos
propositalmente para não ver na luta dessas três grandes classes {[}a
aristocracia possuidora de terras, burguesia e trabalhador
assalariado{]} e no conflito de seus interesses, a força impulsionadora
da história moderna --- pelo menos, nos dois países mais avançados.''

O lugar social do filósofo começa a ser determinado pelos problemas a
que é chamado a oferecer respostas e de onde ele formula suas respostas.
A mudança do papel das universalidades a partir da metade do século
\textsc{xix} em relação ao desenvolvimento das forças produtivas prepara
a passagem da centralidade do ``filósofo"-teólogo'' para o
``cientista"-empregado''. Essa passagem é subjacente à analise de
Engels, elaborada ainda nas ruínas do papel singular do papel do
filósofo"-teólogo na Alemanha. Essa dinâmica estabelece necessariamente
um novo tipo de ruptura com a realidade, estabelecendo um papel diverso
dos filósofos"-teólogos do mundo pré"-burguês. Uma ruptura que ocorre
tanto no nível cognitivo mais elementar, enquanto relação com a
objetividade e seus circuitos cotidianos, como também em um nível
social"-político mais amplo como autorreflexividade do mundo burguês
enquanto \emph{comunidade universal.} O caráter sistêmico da
desigualdade social e da falsa liberdade concreta do mundo burguês,
nessa passagem de uma filosofia burguesa crítica dos fundamentos do
mundo feudal para uma burguesia reacionária em crise com o mundo que se
põe, consciente ou inconscientemente, desloca o \emph{locus} do problema
e explicita justamente --- aqui o clássico problema do \emph{fetiche da
objetividade} inaugurado por Lukács\footnote{Conferir, por exemplo:
  \textsc{lukács}, G. \emph{Existentialismus oder Marxismus?} Berlin:
  Aufbau"-Verlag, 1951, p.8 e ss.} como questão para o marxismo do
século \textsc{xx} --- um \emph{descolamento específico} em relação à
realidade efetiva, que não \emph{pode mais ser apreendida por
fundamentos universais, sem vincular a si mesma aos
deslocamentos"-dissimuladores} impostos pela universalidade e unidade
social do mundo burguês.

Engels tenta, em certo momento, delimitar a alteração de função da
\emph{intelligentsia} filosófica alemã, opondo as descobertas
científicas à antiga filosofia da natureza alemã, indicando, por
exemplo, a materialidade da ``coisa em si'' de Kant: ``as matérias
químicas produzidas em corpos vegetais e animais eram as tais ``coisas
em si'' até que a química orgânica as ter começado a apresentar uma após
outra; com isso, a ``coisa em si'' se tornou uma coisa para nós''. A
questão é que Marx já havia mostrado que o enigma da objetividade da
experiência tem um \emph{locus} não enigmático. A reprodutibilidade da
relação"-capial é o que produz a simplificação da luta das três classes
apontadas por Engels, produz a divisão do acesso aos meios de
subsistência, a vinculação entre modo de produção e relação de produção.
A mobilidade dessa reprodutibilidade é inegável, a ilusão de um controle
\emph{externo} a essa reprodutibilidade é base de todos os enigmas da
forma político"-moral burguesa.

Nesse sentido, os compromissos de classes, entre Napoelão \textsc{iii} e
a burguesia francesa, Bismarck e a burguesia alemã, são resultados
sistêmicos desses deslocamentos. O irracionalismo que Lukács apontou
para a filosofia na época do imperialismo tem esse caminho. O caminho da
miséria da filosofia se abre com a complexa dinâmica de expansão do
mundo burguês. A partir de 1848 nenhum tema das sintomaticamente
denominadas \emph{Geistwissenschaften} escapará dessa relação, que
sempre voltará com suas mascaras. É nesse contexto que as retomadas,
tanto do agnosticismo, como do neokantismo, podem ser compreendidas:

\begin{quote}
Se, entretanto, a reabilitação da concepção kantiana é tentada na
Alemanha pelos neokantianos e a reabilitação de Hume na Inglaterra (onde
nunca morreu) pelos agnósticos, cientificamente, diante da refutação
teórica e prática há muito alcançada, isso é um retrocesso e,
praticamente, apenas um modo envergonhado de aceitar o materialismo
pelas costas e de o negar perante o mundo.\footnote{\emph{Ludwig
  Feuerbach e o fim da filosofia clássica alemã (parte \textsc{ii})}.}
\end{quote}

\section*{Engels crítico do neokantismo}

A partir de 1880 o neokantismo configura"-se como a escola predominante
nas universidades de filosofia na Alemanha, tendo efeitos diretos nas
chamadas \emph{Geistwissenschaften}, particularmente na \emph{Ciência do
Direito} (o exemplo paradigmático será a \emph{Teoria pura do Direito}
de Hans Kelsen na Áustria em 1934). Um tema central inaugurado por
Engels nesse texto: \emph{a} \emph{primeira crítica materialista do
neokantismo}. A pré"-história teórica da retomada de Kant delimita
justamente o problema da relação entre subjetividade e razão, tendo em
vista justamanete o problema do \emph{a priori}: poderia este ser
compreendido como ``um instituição orgânica inata ao gênero
(\emph{angeborene Gattungsorganisation})'' humano?\footnote{\textsc{ollig},
  Hans"-Ludwig, \emph{Der Neukantismus,} Stuttgart, Springer"-Verlag,
  GmbH, 1979, p.1}

A retomada de Hume, o agnosticismo inglês, não por acaso, compartilha a
mesma dimensão filosófica. Em termos teóricos a \emph{aporia} da
experiência sensível racional, entre ceticismo e filosofia
transcendental, gira em torno, como se sabe, do desafio assumido por
Kant diante de Hume, ``o geógrafo da razão humana'', de construir um
fundamento racional para proposições como: ``toda mudança necessita de
uma causa''. A existência da causa é faticamente, empiricamente,
inegável, afinal a água não esquenta sem o fogo, e a maior prova disso é
que se colocarmos a mão na água quente iremos nos queimar. Para além do
mau ``hábito'' de queimar a mão, caberia comprovar que a água sempre
esquenta com o fogo, comprovar como posso saber que entre esses dois
entes --- água quente e fogo --- há uma ligação que não é fruto da mera
contingência empírica. Em outras palavras: como é possível antecipar
racionalmente um vínculo entre fenômenos sem precisar sempre recorrer à
má experiência? Uma questão que nesse nível de empiria, água quente e
fogo, parece banal, mas que se torna complexo quando passamos para
dimensão metafísica: como justificar racionalmente Deus como causa do
mundo? Ou ainda: como entender a relação entre ``má experiência'' e
produtividade? A resposta kantiana ao desafio cético passa pela
aceitação da inacessibilidade metafísica da dimensão fenomênica do
mundo, mas que, justamente por isso, devido à indeterminação sensível do
fenômeno, revelaria a inevitabilidade do papel atuante e determinante da
\emph{noumenon} no conhecer e pensar da razão transcendental.

A dificuldade aqui, na realidade, é situar um fundo histórico desse
debate, principalmente tendo em vista a ideia de uma ``instituição
orgânica do gênero'' humano enquanto ser racional e livre, que contém
inato, em si, a potência \emph{a priori}, ao mesmo tempo, espontânea e
reflexionada, de constituir o \emph{critério} para avaliar os
automatismos da realidade efetiva e o sentido das transformações sociais
que o século \textsc{xix} trazia. O estabelecimento de um critério
transcendental, a partir de um vínculo interno entre \emph{autonomia da
vontade} e realidade efetiva, é a questão central da \emph{Fundamentação
da metafísica dos costumes}. Como indicado no inicio dessa apresentação,
a obra de Marx e Engels questiona, como nunca antes, a possibilidade da
\emph{autonomia da vontade em termos transcendentais}, e expande o
problema para além de uma questão meramente de critica da política, da
religião ou da filosofia. O desenvolvimento e protagonismo das ciências
da natureza no século \textsc{xix}, pressuposto da ``querela do
materialismo'' (\emph{Materialismusstreit}) diante do transcendental,
como aponta Engels, não se deu apenas pela ``força do puro pensamento'',
mas também pelo ``sempre mais veloz impetuoso progresso da ciência da
natureza e da indústria''. Esse progresso impetuoso implica tanto uma
mudança ``interna'', de avanço e novas descobertas, desenvolvimento da
matemática, física, química, biologia, como uma transformação
``externa'', pelo desdobramento histórico da indústria e alteração
continua do modo de produção e divisão social do trabalho, que chega a
um ponto culminante em 1848.

O neokantismo, que direcionou as discussões das universidades alemãs por
quase um século no âmbito das ciências do espírito, é uma resposta, mais
ou menos direta, ao posicionamento"-constitutivo da totalidade
sistêmica do capital com a grande indústria. As ciências duras, da
natureza, não podem, portanto, mais ser compreendidas unicamente pelo
desenvolvimento interno de seu caráter cognitivo ou capacidade de
explicar os mistérios da natureza e da humanidade, não ocupam mais um
espaço ao lado das discussões metafísicas, submetidos ao controle
político"-social da Igreja, inseridas, principalmente no idealismo
alemão, como o próprio Engels aponta, nas diversas ``filosofias da
natureza''. Na realidade, a querela entre o papel ativo da subjetividade
diante dos automatismos da realidade, pano de fundo do neokantismo, é um
dos complexos efeitos ideológicos do estabelecimento de uma nova divisão
do trabalho ``no interior da sociedade''\footnote{\textsc{marx}, K. \textit{Das Kapital. Erster Band}. 39 Aufl. 2008, p.\,371} que tem como um dos aspectos fundamentais, o longo processo de reposição das ciências da natureza no processo histórico de passagem da
manufatura para grande indústria.

Partindo do sistema de cooperação artesanal capitalista, passando pela
autonomização parcial do trabalhador, divisão territorial do trabalho e
concentração da produção na mão do capitalista na manufatura, até chegar
à completa autonomização abstrata do trabalho com a maquinaria. O núcleo
do argumento de Marx é a \emph{interação recíproca} entre divisão do
trabalho na manufatura capitalista, isto é, a divisão do processo de
produção ``no interior da manufatura'' e a divisão do trabalho no
``interior da sociedade civil"-burguesa''. ``Na medida em que a produção
e circulação de mercadorias é o pressuposto geral do modo de produção
capitalista, a divisão do trabalho mediada pela manufatura exige que a
divisão do trabalho tenha amadurecido até certo grau de desenvolvimento
no interior da sociedade. Às avessas, a divisão do trabalho mediada pela
manufatura desenvolve e multiplica por efeito retroativo
(\emph{rückwirkend}) aquela divisão social do trabalho''.\footnote{\textsc{marx}, K., idem.} Os
meios de subsistência somente existem enquanto meios de produção, ou em
função dos meios de produção, o que se explicita, por exemplo, com a
nova divisão entre campo e cidade que se intensifica sem precedentes a
partir do século \textsc{xix} europeu, constituindo a cidade como
\emph{telos} automático ideal de certo desenvolvimento histórico do
campo. Esse longo processo de transformação, como se sabe, acontece de
modo diverso em cada país e passa sempre pelo desdobramento das formas
de dominação precedentes.

Além de naturalizar, disciplinar e condicionar o trabalhador, como
aquele a quem não resta nada além de vender suar força de trabalho, o
devir da fábrica manufatureira em grande indústria, a divisão
``interna'' do trabalho, constitui a autoridade incondicional do
capitalista, na mesma medida em que permite a este, no âmbito da divisão
social do trabalho ``no interior da sociedade'', ``por seu efeito
retroativo'', aparecer como um produtor e vendedor de mercadoria
subemitido à mesma concorrência, imposição e pressão dos interesses que
parecem \emph{externos} à divisão do trabalho fabril, como se fossem
resultados unicamente do arbítrio e contingência, em última instância,
de uma ``necessidade'' histórica decorrente da abstração da
\emph{vontade} \emph{individual}.

A autoridade incondicional que o capitalista detém, iniciada com a
cooperação quando o capitalista representa a unidade e vontade do corpo
de trabalho, intensifica"-se com a hierarquização imposta pelo trabalho
parcial na manufatura e se completa com a grande indústria, pela
instituição da maquinaria, com a cisão entre ciência e força de
trabalho, antes unidas no camponês autônomo ou trabalhador manual, tudo
e todos agora a serviço da produção de mais"-valor.\footnote{\textsc{marx}, K., op.\,cit., p.\,377} A dimensão subjetiva do trabalhador é completamente suplantada
pela existência material da maquinaria, assim como a identificação
imediata do capitalista como unidade e vontade do corpo de trabalho. A
ciência da natureza torna"-se a ``consciência'' da produção capitalista
que substitui a rotina baseada na experiência: ``Enquanto maquinaria o
meio de trabalho passa a receber um modo de existência, que condiciona a
substituição da força de trabalho humana pelas forças naturais e as
rotinas mediadas pela experiência pela aplicação consciente da ciência
da natureza''.\footnote{\textsc{marx}, K., op.\,cit., p.\,407.} Essa substituição será fundamental, pois os efeitos
indiretos, antes de qualquer salto ``filosófico'' apressado entre
\emph{noumenon e phainomenon}, implicam uma reorganização e
redistribuição do controle dos centros de produção de saber nas
universidades, delimitando a centralidade das ciências da natureza. A
oposição entre materialismo e idealismo poderá assim alcançar um núcleo
social de reflexividade. As ciências da natureza terão um papel
fundamental como um dos pontos constitutivos da ``consciência'' da
sociedade civil"-burguesa, fundamental para entender a subsunção
intelectual efetiva do trabalho ao capital, às potências de controle das
forças produtivas. Uma ``consciência'' impessoal que necessariamente irá
se formar \emph{a partir e para além} de qualquer dicotomia entre
sujeito e objeto, autonomia da vontade e autonomia da natureza,
autonomia da vontade subjetiva"-individual e autonomia da vontade
objetiva"-social.

Na sua dimensão concreta, o neokantismo pode ser, portanto, compreendido
como uma das expressões do processo de legitimação da dominação da
relação"-capital por meio do movimento de suas ``formas de
consciência''. É justamente aqui que o problema de uma ética
transcendental ganha sentido, na relação entre \emph{autonomia da
vontade} e \emph{legitimação do monopólio da dominação
racional"-violenta}, não apenas pelo Estado e pela forma da política, do
direito, ou pelas racionalidades de gerenciamento social, mas como
tentativa de tradução filosófica do enfrentamento do automatismo por
meio do qual a realidade efetiva se impõe enquanto potência estranha
(\emph{fremde Macht}), fazendo com que o mundo apareça, ao mesmo tempo,
como \emph{possível} representação universal do gênero racional humano e
como \emph{necessária} imposição objetiva de um conteúdo empírico que
escapa a qualquer representação subjetiva possível. A ilusão de
adequação a essa contradição é o lugar de conforto do
estranhamento"-de"-si da burguesia, que assume tal ilusão como seu
próprio poder, mas de aniquilação do proletariado a quem está dada, no
chão da história, a impossibilidade de adequação a ela.

\section*{Todo o efetivo é irracional,\break todo o irracional é efetivo}

A irracionalidade torna"-se estrutural na medida em que a materialidade
da história é recusada por uma subjetividade forjada que sempre já não
alcançou a concretude do movimento histórico. A expansão da
relação"-capital determina o ritmo da miséria da razão e seu novo
\emph{Standpunkt}, um movimento mais do que transparente já em 1886 na
Alemanha:

\begin{quote}
E, no âmbito das ciências históricas, incluindo a filosofia,
desapareceu, junto à filosofia clássica, com maior razão, o velho
espírito teórico"-brutal (\emph{theoretisch"-rücksichtslose}):
ecletismo desprovido de pensamento, preocupação angustiada com carreiras
e rendimentos descendo até ao arrivismo (\emph{Strebertum}) mais
ordinário, tomam seu lugar. Os representantes oficiais desta ciência
tornaram"-se ideólogos não encobertos da burguesia e do Estado
existente --- mas em um tempo em que ambos estão em oposição aberta à
classe trabalhadora.\footnote{\emph{Ludwig Feuerbach e o fim da
  filosofia clássica alemã (parte \textsc{iv})}.}
\end{quote}

A crítica brutal da realidade efetiva é ainda a tarefa. Como Marx
menciona na conhecida carta a Ruge onde reflete a situação política na
Alemanha em 1843 e a função do crítico, este não pode retirar suas
``armas'' da história da filosofia, do direito ou da economia, mas da
``\emph{crítica brutal de toda realidade dada}''\footnote{\textsc{marx},
  K. \emph{Deutsche Französische Jahrbücher 1. Doppellieferund},
  Februar, 1844. In: \textsc{marx}, K.; \textsc{engels}, F. Werke. Band
  1. Berlin/\textsc{ddr}: Dietz Verlag, 1976, p.344.} (\emph{die
rücksichtlose Kritik alles Bestehenden}), ou seja, da crítica que tem em
vista o todo de uma tradição que se estabilizou (\emph{Bestehende}) e
cujos elementos, justamente por isso, \emph{aparecem como necessários}
no presente. A ciência crítica da história, que opera sem restrições,
que faz a \emph{crítica brutal da realidade}, é aquela que trabalha
\emph{no interior do movimento de realização} da ``determinação ideal e
seus pressupostos efetivos'', \emph{a partir e para além da necessidade
do movimento}.

Apenas assim a crítica consegue ser ``brutal'' ou ``sem restrições''
(\emph{rücksichtlos}), como define Marx, ``tanto no sentido de não poder
temer os seus próprios resultados quando no sentido de não pode temer os
conflitos com os poderes estabelecidos''.\footnote{\textsc{marx}, K., idem}
A filosofia alemã, que tem como auge a dialética hegeliana, antes de ser
absorvida pela consolidação do mundo burguês na Alemanha pós"-1848,
vendo ``\emph{de fora''} o avanço da relação"-capital na França e
Inglaterra, estava ainda em condições de questionar a \emph{necessidade
do movimento para si,} até o momento que esse mesmo movimento se torna
sua \emph{própria necessidade} e é incorporado pelas suas próprias
restrições.



\end{document}
