{\let\clearpage\relax\chapter[Nota prévia]{Nota prévia\protect\endnote{\versal{ENGELS}, F. \emph{Werke.
  Artikel. Entwürfe. Oktober 1886 bis Februar} 1891. (\versal{MEGA}, 31, \versal{I}).
  Bearbeitet von Renate Merkel"-Melis. Akademie Verlag GmbH, Berlin,
  2002. A \emph{nota prévia} não aparece na primeira versão publicada na
  revista \emph{Die Neue Zeit} em 1886 em dois cadernos, de abril e
  maio, antes da versão publicada em livro no ano de 1888: \emph{Ludwig
  Feuerbach und der Ausgang der klassischen deutschen Philosophie.} Mit
  Anhang: \emph{Karl Marx über Feuerbach vom Jahre 1845}. Stuttgart:
  Verlag von J.H.W. Dietz, 1888. {[}\versal{N.\,T.}{]}}}}

%\endnotetext

\noindent{}No prefácio da \emph{Crítica da economia política}, Berlim, 1859,
Karl Marx conta como nós em 1845, em Bruxelas, começamos ``a realizar em
conjunto o contraste de nossa visão'' --- a concepção materialista da
história elaborada por Marx --- ``em oposição à visão ideológica da 
filosofia alemã, visando, de fato, acertar contas com nossa consciência
filosófica da época. O propósito foi realizado na \emph{forma} \emph{da
filosofia pós"-hegeliana}. O manuscrito {[}\emph{A ideologia alemã}{]},
dois grossos volumes dividido em oitavos, há muito já estava na editora
em Westfalen quando recebemos a mensagem de que circunstâncias
alteradas não permitiriam a publicação. Deixamos o manuscrito para a
crítica roedora dos ratos, tanto mais de boa vontade na medida em que já
havíamos alcançado nosso objetivo principal: autocompreensão''.

Desde então se passaram mais de 40 anos, e Marx morreu sem que nos
tivesse sido oferecida a oportunidade de retornar ao objeto. Sobre nossa
relação com Hegel nos pronunciamos em partes separadas, porém nunca em
um contexto amplo. A Feuerbach, que, em mais de um aspecto, estabelece
um meio de ligação entre a filosofia de Hegel e nossa concepção, nunca
retornamos.

Entretanto, a visão de mundo de Marx encontrou representantes muito além
da Alemanha e Europa e em todas as línguas cultas do mundo. Por outro
lado, a filosofia clássica alemã vivencia uma espécie de renascimento no
exterior, particularmente na Inglaterra e Escandinávia,\est\ e mesmo na
Alemanha parece estar farta de receber as sopas ecléticas como esmolas
servidas nas universidades de lá sob o nome de filosofia.

Sob essas circunstâncias, uma breve e coerente exposição de nossa
relação com a filosofia hegeliana, nosso ponto de partida e nossa
ruptura com ela, parecia cada vez mais necessária. E também me pareceu
necessário um reconhecimento completo da influência que, acima de todos
os outros filósofos pós"-hegelianos, Feuerbach teve sobre nós durante
nosso período de \emph{Strum und Drang} [tempestade e ímpeto], como uma dívida de honra não
quitada. Por isso, aproveitei de bom grado a oportunidade quando os
editores da \emph{Neue Zeit} solicitaram"-me uma revisão crítica do
livro de Starcke sobre Feuerbach. Meu trabalho foi publicado no quarto e
quinto cadernos dessa revista em 1886, e aparece aqui em uma edição
especial revisada.

Antes de enviar essas linhas para a imprensa, olhei novamente para o antigo
manuscrito de 1845/46. A seção sobre Feuerbach não está
concluída. A parte terminada consiste em uma exposição da concepção
materialista da história que apenas demonstra o quanto, naquela época,
os nossos conhecimentos da história econômica eram ainda incompletos.
Falta a crítica à própria doutrina de Feuerbach; não era, portanto,
aproveitável para o objetivo atual. Por outro lado, em um antigo caderno
de Marx, encontrei as onze teses sobre Feuerbach reproduzidas em
apêndice.\endnote{A edição alemã de 1888 contém anexas as
  onze teses de Feuerbach, publicadas com o título ``Karl Marx sobre
  Feuerbach do ano de 1845'' (\emph{Karl Marx über Feuerbach vom Jahre
  1845}). {[}\versal{N.\,T.}{]}} São notas para elaboração posterior, escritas rapidamente,
absolutamente não destinadas à impressão, mas inestimáveis ​​como o
primeiro documento no qual é estabelecido o genial embrião da nova
visão de mundo.

\bigskip

\hfill{}Londres, 21 de fevereiro de 1888

\quebra

\begin{flushright}
\emph{Die Neue Zeit. Ano 4.}\\
\emph{Caderno 4, abril de 1886.}
\end{flushright}

\vspace{2cm}


\addcontentsline{toc}{chapter}{Primeira parte}
\section{I}

\noindent{}O presente escrito\endnote{Ludwig Feuerbach, de C.
  N. Starcke, Dr. Phil. Stuttgart, Ferd. Encke, 1885. {[}\versal{N.\,E.}{]}}
nos conduz novamente a um período que, de acordo com a época, já ficou
uma boa geração para trás, mas que já se tornou tão estranho 
para a atual geração na Alemanha como se tivesse
completado um século inteiro de idade. E foi, no entanto, o período de
preparação da Alemanha para a revolução de 1848; e tudo o que desde
então aconteceu conosco é apenas uma avanço de 1848, apenas execução do
testamento da revolução.

Assim como na França do século \versal{XVIII}, também na Alemanha do século \versal{XIX}
a revolução filosófica preparou o colapso político. Mas quão diversas
ambas parecem! Os franceses em luta aberta contra todo o saber
oficial, contra a Igreja, frequentemente também contra o
Estado; os seus escritos impressos além das fronteiras, na Holanda ou na
Inglaterra, e eles próprios, com frequência, prontos a marchar para
a Bastilha. Os alemães em contrapartida: professores universitários,
doutrinadores da juventude estabelecidos pelo Estado; seus escritos,
manuais doutrinários reconhecidos, e o sistema que encerra todo o desenvolvimento, o sistema hegeliano, que foi, inclusive, elevado ao nível de uma filosofia de Estado da realeza prussiana! E quem diria
que a revolução se esconderia por detrás desses professores, por detrás
das suas palavras pedantemente obscuras, em seus prolixos e maçantes
períodos? Afinal, não eram justamente os liberais as pessoas
consideradas, naquela época, os representantes da revolução, os
adversários mais duros dessa filosofia que confunde as cabeças? O que,
porém, nem os governos nem os liberais viram já foi visto, em 1833, pelo
menos por \emph{um} homem, e ele se chamava Heinrich Heine.\endnote{Referência ao livro de Heine, \emph{Zur
  Geschichte der Religion und Philosophie in Deutschland,} 1833 {[}Sobre
  a história da religião e filosofia na Alemanha{]}, em que o autor trabalha a
  relação entre reforma protestante, revolução francesa e o papel da
  ``revolução'' filosófica do idealismo alemão. Heine é o epicentro da
  expressão crítica que ficou conhecida como miséria alemã,
  \emph{Sonderweg}, ou ainda, via prussiana. A problemática da
  expansão"-estabilização do mundo burguês, de totalização histórica e
  sistêmica da relação"-capital, é o processo que define o século \versal{XIX}
  europeu. O tema é desenvolvido, principalmente, por Lenin, Trótski e
  Lukács. A reverberação política e social do conhecido descompasso em
  relação ao desenvolvimento histórico da Inglaterra (revolução
  econômica --- século \versal{XVII} e \versal{XVIII}) e da França (revolução política --- século \versal{XVIII}) é o que define a miséria alemã do século \versal{XIX}. O pano de fundo histórico é complexo e atravessa muitos séculos de conflitos. A
  disputa pela região da Alsácia"-Lorena, parte integrante do Sacro
  Império Romano"-Germânico em 1648, anexada à França de Luis \versal{XIV} depois
  da Paz de Vestfália, é um dos mais antigos embates envolvendo o povo
  francês e o alemão. Heinrich Heine também é aludido indiretamente por
  Marx em um texto de 1842 (\emph{Manifesto filosófico da Escola
  Histórica do Direito}) ao citar ironicamente a ``má'' influência do
  jurista Gustav von Hugo --- ao defender a irracionalidade da
  exclusividade do impulso sexual do homem no matrimônio --- para os
  \emph{jovens alemães}, movimento literário 
  alemão da primeira metade do século \versal{XIX} com inspiração em Heine e que
  dentre suas principais revindicações exigia o amor livre. {[}\versal{N.\,T.}{]}}

Tomemos um exemplo. Nenhuma proposição filosófica carregou consigo
igualmente a gratidão de governos limitadores e a cólera de liberais
como a famosa proposição
de Hegel:
``Tudo o que é efetivo é racional, e tudo o que é racional é
efetivo.''\endnote{Passagem presente em: \emph{Vorrede,}
   \emph{Grundlinien der Philosophie des Rechts}, oder Naturrecht und
  Staatswissenschaft. Hrsg. von Eduard Gans. 2. Aufl. Berlin, 1840 e
  retomada em  \emph{Enzyklopädie der philosophischen Wissenschaften \versal{I}}
  (Dritte Ausgabe, 1830), § 6. {[}\versal{N.\,T.}{]}}

Isso era, porém, a evidente santificação de todo o elemento
existente, a consagração filosófica do despotismo, do Estado policial,
da justiça de gabinete, da censura. E tal
como Frederico
Guilherme \versal{III} assim entendeu, assim entenderam os seus súditos. Mas,
em Hegel,
de modo nenhum tudo aquilo que existe é também automaticamente efetivo. Para ele, o atributo da realidade 
efetiva cabe apenas àquilo que, ao mesmo tempo, é necessário: ``a
realidade efetiva mostra"-se em seu desdobramento como a
necessidade''.\endnote{\emph{Enzyklopädie
  der philosophischen Wissenschaften~\versal{I}} (1830), § 143,
  \emph{Zusatz}: ``Se isso é possível ou impossível, depende do
  conteúdo, isto é, da totalidade dos momentos da realidade efetiva,
  \emph{que se mostra em seu desdobramento como a necessidade}''.
  (grifei) {[}\versal{N.\,T.}{]}} Um regulamento governamental arbitrário --- o próprio Hegel
remete ao exemplo ``de certa instituição
fiscal''\endnote{\emph{Enzyklopädie der philosophischen Wissenschaften \versal{I},} § 142, \emph{Zusatz}. Ao tratar da \emph{realidade efetiva} como unidade
  entre essência e existência, compreendendo o elemento efetivo como o
  ser"-posto que delimita um modo de relação que, apesar de racional ---
  adequada à ideia --- pode não ser efetivo, Hegel critica justamente a
  suposta cisão entre ideia e realidade efetiva. Ao tomar a ideia como
  representação subjetiva e a realidade como exterioridade sensível, a
  dinâmica de necessidade que vincula uma à outra se perderia em
  estruturas lógico"-formais abstratas: ou como uma dinâmica de
  necessidade manifesta de modo meramente ideal, ou como uma dinâmica de
  necessidade efetiva, expressa por automatismos irracionais. O exemplo
  que ilustra essa situação mencionada por Engels é justamente o ``plano, ou assim chamada ideia, de certa instituição fiscal'', 
  que pode ser ``em si boa ou adequada aos fins, mas 
  que não se encontra, do mesmo modo, na assim chamada realidade efetiva
  e não pode ser executada sob as relações dadas.'' (§ 142,
  \emph{Zusatz}) {[}\versal{N.\,T.}{]}} --- de modo algum tem, para ele, validade automática
como algo efetivo. O que é necessário, porém, comprova"-se, em última
instância, também como racional e, aplicada ao Estado prussiano daquela
época, a proposição
de Hegel quer
dizer apenas: este Estado é racional, correspondente à razão, na mesma
medida em que é necessário; e se ele, porém, apresenta"-se para nós como
perverso, mas apesar da sua perversidade continua a 
existir, a perversidade do governo encontra a sua justificação e a sua
explicação na correspondente perversidade dos súditos. Os prussianos
daquela época tinham o governo que mereciam.

Afinal,
segundo Hegel,
a realidade efetiva não é de modo algum um atributo que condiz com um
estado de coisas social ou político dado em todas as circunstâncias e em
todos os tempos. Pelo contrário. A República Romana era efetiva, mas o
Império Romano \textbar{} que a suplantou \textbar{} também. A Monarquia Francesa, \textbar{} em
1789 \textbar{}, tinha"-se tornado tão inefetiva, isto é, tão desprovida de toda a
necessidade, tão irracional, que tinha de ser aniquilada pela grande
revolução sobre a qual
Hegel sempre
fala com o maior entusiasmo. Aqui, portanto, a monarquia era o elemento
inefetivo, e a revolução o efetivo. E, no curso do desenvolvimento, todo o
elemento anteriormente efetivo se torna inefetivo, perde a sua
necessidade, o seu direito de existência, seu caráter racional; toma o
lugar do efetivo que padece uma nova e viável realidade efetiva:
pacificamente, se a antiga é suficientemente sensata para marchar para
a morte sem resistência; pela força da violência, caso se oponha a essa
necessidade. E, assim, a proposição
de Hegel, por meio da própria dialética
hegeliana, inverte"-se no seu contrário: tudo o que no âmbito da história humana é
efetivo torna"-se, com o tempo, irracional, é, portanto, já segundo sua
determinação, irracional, está desde o princípio afetado com a
irracionalidade; e tudo o que na cabeça dos homens é racional está
determinado a se tornar efetivo, caso esse também ainda possa
contradizer a aparente realidade efetiva existente. A proposição da
racionalidade de todo o elemento efetivo real dissolve"-se, segundo todas
as regras do método de pensar
de Hegel,
nesta outra: tudo o que existe é digno de perecer.

Mas a verdadeira significação e o caráter revolucionário da filosofia
de Hegel (na
qual temos que nos limitar aqui como o desfecho de todo o movimento
desde Kant)
consistia justamente no fato que ele, de uma vez por todas, acabou com o
caráter definitivo de todos os resultados do pensar e do agir humanos. A
verdade, que valia conhecer na filosofia, não era mais
para Hegel uma
coleção de proposições dogmáticas prontas que, uma vez encontradas,
apenas se buscava decorar; a verdade consistia agora no processo do
próprio conhecer, no longo desenvolvimento histórico do saber,
que se eleva de estágios inferiores do 
conhecimento para estágios sempre superiores sem jamais, porém,
alcançar, por meio do processo de localização de uma, assim chamada, verdade absoluta, o ponto em que não pode mais avançar, em que não lhe resta
mais nada além de ficar de braços cruzados e admirar imóvel a verdade
absoluta obtida. E isso tanto no âmbito do conhecimento filosófico como no
de qualquer outro conhecimento e ação prática. Tanto quanto o
conhecimento, também a história não pode encontrar um desfecho pleno
em um estado ideal perfeito de 
humanidade; uma sociedade perfeita, um ``Estado'' perfeito, são coisas
que só podem existir na fantasia; pelo contrário, todos os estados
históricos que se seguem uns aos outros são apenas estados transitórios
no curso de desenvolvimento sem fim da sociedade humana, 
do inferior para o superior. Portanto cada estágio é necessário, está
justificado para a época e condições às quais deve a sua origem; mas cada estágio se %MANTIVE
torna caduco e injustificado diante das novas condições superiores que
gradualmente se desenvolvem no seu próprio âmago; precisam dar lugar a
um estágio superior que ingressa, por sua vez, novamente na série de
declínio e decadência. Assim como a burguesia, através da grande
indústria, da concorrência e do mercado mundial, dissolve na prática
todas as instituições estáveis e veneráveis pela longevidade, essa
filosofia dialética também dissolve todas as representações da verdade
absoluta definitiva e os correspondentes estados absolutos da
humanidade. Diante dela não existe nada de definitivo, de absoluto, de
sagrado; ela mostra a transitoriedade de tudo e em tudo, e nada subsiste
diante dela a não ser o ininterrupto processo do devir e perecer, da
ascensão sem fim do inferior ao superior, da qual ela própria é mero
reflexo no cérebro pensante. Ela também tem, certamente, um lado
conservador: reconhece a justificação de determinados estados do
conhecimento e da sociedade para a sua época e circunstâncias; mas
apenas até aqui. O conservadorismo desse modo de intuição
é relativo, seu caráter revolucionário é 
absoluto --- é o único elemento absoluto que ela pode admitir como válido.\est\

Não precisamos entrar aqui na questão se este modo de intuição está de
acordo com o atual estado da ciência da natureza, a qual prevê para a
existência da Terra um fim possível~--- para o seu caráter
habitável, porém, um fim bastante certo ---, que, portanto, atribui também
à história humana não só um ramo ascendente como também um descendente.
Estamos, de qualquer forma, ainda bastante distantes do ponto de mudança
a partir do qual a história da sociedade entra em declínio, e não podemos
exigir da filosofia
de Hegel que
se ocupe de um objeto que, no tempo dela, a ciência da natureza ainda
não tinha posto na ordem do dia.

Mas o que, de fato, podemos dizer aqui é: o desenvolvimento acima
referido não se encontra com essa precisão
em Hegel.
Tal desenvolvimento é uma consequência necessária do seu método,
consequência esta, porém, que ele próprio nunca tencionou com tal
expressividade. E isso, sem dúvida, pela simples razão de que estava
obrigado a empreender um sistema, e um sistema filosófico, segundo as
exigências tradicionais, tem de finalizar com algum tipo de verdade
absoluta. Portanto, por mais que Hegel também acentue, nomeadamente
na \emph{Lógica}, que esta verdade eterna
nada mais é do que o próprio processo lógico correspondente ao
histórico, ele próprio se vê compelido a dar um fim a esse processo,
porque necessita de algum modo, precisamente, chegar ao fim com o seu
sistema. Na \emph{Lógica} ele pode voltar a tornar esse fim um início,
na medida em que aí o ponto final, a ideia absoluta --- que só é absoluta por não saber dizer absolutamente nada acerca dela ---
``exterioriza"-se'', isto é, transforma"-se na 
natureza e, mais tarde, regressa a si própria no espírito, ou seja, no
pensar e na história. Mas, na conclusão de toda filosofia, tal regresso
ao início somente é possível por \emph{um} caminho. Ou seja, ao
estabelecer o fim da história de tal modo que a humanidade chega
justamente ao conhecimento dessa ideia absoluta \textbar{} e esclarecer que esse
conhecimento da ideia absoluta foi alcançado na filosofia hegeliana. \textbar{}
Com isso, porém, todo o conteúdo dogmático do sistema
de Hegel é
qualificado como verdade absoluta, em contradição com o seu método
dialético que dissolve todo elemento dogmático; o lado revolucionário é,
assim, abafado pelo lado conservador que o encobre completamente. E o
que vale para o conhecimento filosófico vale também para a
\emph{práxis} histórica. A humanidade que, na pessoa
de Hegel,
conduziu até a elaboração da ideia absoluta precisa, também na prática,
ter chegado ao ponto de poder levar a cabo essa ideia absoluta na
realidade efetiva. As reivindicações políticas práticas da ideia
absoluta, em relação aos contemporâneos, não podem, portanto, ser
demasiadamente ambiciosas. E, assim, encontramos na conclusão
da \emph{Filosofia do direito} que a ideia absoluta deve efetivar"-se
naquela monarquia de estamentos 
que Frederico Guilherme \versal{III}, tão obstinadamente em vão, prometeu aos seus súditos,
portanto, em uma dominação indireta das classes de possuidores,
limitada, adaptada e mediada pelas relações da pequena burguesia alemã
daquela época; com isso nos é demonstrada, pela via especulativa, ainda
a necessidade da nobreza.

Portanto, as necessidades internas do sistema somente são suficientes
para, por intermédio de um método de pensar revolucionário de ponta a
ponta, esclarecer a produção de uma conclusão política
bastante dócil. A forma específica dessa conclusão 
resulta, porém, do fato de que Hegel era alemão e de que, tal como do seu
contemporâneo Goethe, pendia"-lhe uma pedaço de trança de filisteu. Goethe, assim como Hegel, eram, cada um no seu âmbito, um Zeus do Olimpo, mas ambos nunca se
libertaram completamente do filisteu alemão.

Isso tudo não impediu, contudo, o sistema
de Hegel de abarcar um âmbito incomparavelmente maior do que qualquer sistema
anterior e de desenvolver nele uma riqueza de pensamento que
ainda hoje causa espanto. \emph{Fenomenologia do espírito} (que %MANTIVE
poderíamos denominar como um paralelo\est\ entre a embriologia e a
paleontologia do espírito, um desenvolvimento da consciência individual
por meio dos seus diversos estágios, apreendido como reprodução %MANTIVE
encurtada dos estágios pelos quais a consciência dos homens passa
historicamente), \emph{Lógica}, \emph{Filosofia da natureza},
\emph{Filosofia do espírito}, e esta última, novamente, elaborada em suas
subdivisões históricas isoladas: Filosofia da História, do Direito, da
Religião, História da Filosofia, Estética, etc.
Hegel trabalha
para encontrar e demonstrar, em todos esses diversos domínios
históricos, o fio do desenvolvimento que os perpassa; e nesse processo
ele não foi apenas um gênio criador, mas também um homem de erudição
enciclopédica, fazendo assim época em todos os domínios. E evidente que,
em virtude das necessidades do ``sistema'', com muita frequência ele
teve de refugiar"-se em construções forçadas, acerca das quais
os seus inimigos, apegados a questões menores, até hoje fazem uma
gritaria tão descomunal. Mas estas construções são apenas a armação e o
andaime da sua obra; se não nos retemos aí inutilmente, se penetramos
mais profundamente no poderoso edifício, inúmeros tesouros que ainda
hoje conservam o seu pleno valor serão encontrados. Para todos os
filósofos é precisamente o ``sistema'' o elemento perecível, e isto
justamente por decorrer de uma necessidade 
imperecível do espírito humano: a necessidade de superação
de todas as contradições. Mas, se todas as 
contradições são eliminadas de uma vez por todas, atracamos na assim
chamada verdade absoluta: a história mundial está no fim e, no entanto,
deve continuar, embora não lhe reste mais nada para fazer --- portanto,
uma nova contradição, insolúvel. Assim que compreendermos --- e
definitivamente ninguém nos ajudou mais nessa intelecção do que o
próprio Hegel ---
que a tarefa da filosofia, assim estabelecida, não significa outra coisa
além do fato de que um filósofo singular deve realizar aquilo que só a
humanidade inteira no seu desenvolvimento progressivo\est\ pode realizar ---
assim que compreendermos isso, estará também no fim toda a filosofia no
sentido em que a palavra é conhecida até hoje. Abandona"-se a ``verdade
absoluta'', inalcançável por esta via e por cada um individualmente e,
em troca, perseguimos as verdades relativas alcançáveis pela via das
ciências positivas e da conexão dos seus resultados por meio do pensar
dialético.
Com Hegel
encerra"-se a filosofia em geral. Por um lado, porque ele reuniu em seu
sistema, do modo mais grandioso, todo o desenvolvimento da filosofia;
por outro, porque, ainda que inconscientemente, mostra"-nos o caminho
para fora desse labirinto de sistemas em direção ao conhecimento
positivo e efetivo do mundo.

Concebemos qual efeito monstruoso este sistema hegeliano teve de
produzir na atmosfera filosoficamente tingida da Alemanha. Foi uma
marcha triunfal que durou décadas e que de modo nenhum parou com a morte
de Hegel.
Pelo contrário, precisamente de 1830 a 1840 o ``hegelianismo
charlatão'' dominou do modo mais exclusivo possível e 
havia contagiado, mais ou menos, até mesmo seus adversários;
precisamente nesse tempo as concepções
de Hegel penetraram
com a maior abundância, consciente ou inconscientemente, nas mais
variadas ciências e azedaram completamente também a literatura popular e
a imprensa diária, nas quais a ``consciência culta'' habitual adquire a
matéria de seu pensamento. Mas esse triunfo em todas as linhas era
apenas o prelúdio de uma luta interna.

A doutrina de Hegel como
um todo, vimos, deixava um amplo espaço para alocar as mais diversas
intuições tendenciosas da prática; e, na
prática, na Alemanha teórica daquele tempo, tais intuições eram, antes
de tudo, a religião e a política. Quem colocasse ênfase
no \emph{sistema} de Hegel
podia ser bastante conservador em ambos os domínios; quem visse o
principal no \emph{método} dialético podia, tanto religiosa como
politicamente, pertencer à oposição mais extrema. O
próprio Hegel,
apesar dos ataques de ira revolucionários bastante frequentes em suas
obras, parecia, no conjunto, inclinar"-se mais para o lado conservador;
se o seu sistema não lhe tivesse custado muito mais ``trabalho amargo do
pensamento'' do que o seu método. Perto do fim dos anos trinta a tensão
na escola se evidenciou cada vez mais. A ala da esquerda, os chamados
jovens hegelianos, na luta com ortodoxos pietistas e feudais
reacionários, desistiu, pedaço por pedaço, daquela reserva
filosoficamente distinta diante das questões ardentes do dia a dia que,
até então, haviam assegurado à sua doutrina tolerância estatal e
inclusive proteção; e quando, em 1840, a hipocrisia 
ortodoxa e a reação feudal"-absolutista subiram ao trono com Frederico
Guilherme \versal{IV}, uma aberta tomada de partido tornou"-se inevitável. A luta
seria travada ainda com armas filosóficas, mas não mais por fins
abstratamente filosóficos; tratava"-se do aniquilamento
da religião tradicional e do Estado existentes. E se 
nos \emph{Anais alemães}
os fins últimos práticos ainda se mostravam 
preponderantemente sob disfarce filosófico, a escola jovem"-hegeliana
revelou"-se na \emph{Gazeta renana }de 1842 diretamente como a filosofia
da burguesia com aspirações radicais e se valeu do pretexto filosófico
apenas para ainda enganar a censura.

A política era, nessa altura, um âmbito muito espinhoso e, por isso, a
luta principal voltou"-se contra a religião; esta era, certamente desde
1840, indiretamente também uma luta política.
A \emph{Vida de Jesus} de Strauss,
em 1835, tinha dado o primeiro impulso. À teoria da formação evangélica
dos mitos aí desenvolvida,
Bruno Bauer opôs"-se mais tarde ao demonstrar que toda uma série de narrativas
evangélicas haviam sido fabricadas pelos próprios autores. A disputa
entre ambos foi conduzida sob o disfarce filosófico de uma luta da
``consciência"-de"-si'' contra a ``substância''; a questão se as histórias
dos milagres evangélicos surgiram no seio do elemento comunitário
por meio da formação mitológica\est\ inconscientemente 
tradicional, ou se seriam fabricadas pelos próprios evangelistas, foi
exagerada na questão se na história mundial a ``substância'' ou a
``consciência"-de"-si'' seria a potência decisivamente 
ativa; e, por fim, veio Stirner,
o profeta do anarquismo atual --- Bakunin tomou muito dele ---, e ultrapassou
o ponto culminante da soberana ``consciência"-de"-si'' com o seu soberano
``Único''.\endnote{Max Stirner.
  \emph{Der Einzige und sein Eigenthum }(\emph{O único e sua
  propriedade}), Leipzig 1845. {[}\versal{N.\,T.}{]}}

Não aprofundaremos mais esse lado do processo de decomposição da escola
hegeliana. Para nós o mais importante é a massa dos jovens hegelianos
mais decisivos que, pelas necessidades práticas da sua luta contra a
religião positiva, retrocedeu em direção ao materialismo anglo"-francês.
E aí entrou em conflito com o sistema de sua escola. Enquanto o
materialismo apreendia a natureza como o único elemento efetivo, esta
representava, no sistema
de Hegel, apenas a ``exteriorização alienante'' da ideia 
absoluta, por assim dizer, uma degradação da ideia; em todas as
circunstâncias, o pensar e o produto de seu pensamento, a ideia, são %MANTIVE
aqui o elemento originário, a natureza, o elemento derivado que, em
geral, apenas existe por meio da condescendência da ideia. E era em torno
dessa contradição que se vagueava, tão bem ou mal quanto se queria
prosseguir.

Aí surgiu a \emph{Essência do cristianismo} de Feuerbach.
Com um só golpe pulverizou a contradição ao colocar, sem rodeios,
o materialismo novamente no trono. A natureza existe independentemente
de qualquer filosofia; ela é o fundamento sobre o qual nós, seres
humanos, crescemos, nós mesmos produtos da natureza; fora da natureza e
dos homens não existe nada, e as essências superiores criadas por nossa
fantasia religiosa são apenas o reflexo fantástico da nossa própria
essência. O encanto estava quebrado; o ``sistema'' foi pelos ares e
jogado para o lado, a contradição, enquanto elemento existente apenas na
imaginação, foi dissolvida. É preciso ter vivido o
efeito libertador desse livro para ter uma noção disso. O entusiasmo\est\ foi
geral: éramos, momentaneamente, todos feuerbachianos. O quanto Marx
saudou com entusiasmo a nova concepção \textbar{} e o quanto ele --- apesar de
todas as reservas críticas --- foi por ela influenciado \textbar{} pode se ler
na \emph{Sagrada família}.

Mesmo os erros do livro contribuíram para o seu efeito momentâneo. O
estilo beletrista, por vezes também pomposo, assegurou"-lhe um público
grande e foi, ainda assim, um alívio após longos anos de hegelianismo
charlatão abstrato e abstruso. O mesmo vale para a efusiva divinização
do amor que, perante a soberania do ``pensar puro'' que se tornou
insuportável, encontrou uma desculpa, até mesmo uma justificação. Mas o
que não podemos esquecer: justamente a ambas essas fraquezas
de Feuerbach
atrelou"-se o ``socialismo verdadeiro'', que desde 1844 se espalhava pela
Alemanha ``culta'' como uma praga, colocando, no lugar do conhecimento
científico, a fraseologia beletrista, no lugar da emancipação do
proletariado pela transformação econômica da 
produção, a libertação da humanidade por meio do ``amor'', em suma,
perdeu"-se no fastidioso estilo beletrista e no caráter asfixiantemente
amoroso do tipo do senhor Karl Grün.\endnote{Marx e Engels criticam o ``socialismo
  verdadeiro'' de Karl Grün em um dos manuscritos da \emph{Ideologia
  Alemã: ``V. Karl Grün: Die soziale Bewegung in Frankreich und Belgien``
  (Darmstadt 1845) oder Geschichtschreibung des wahren Sozialismus}''
  (Karl Grün: o movimento social na França e Belgica ou a
  historiografica do verdadeiro socialismo). {[}\versal{N.\,T.}{]}}

O que, além disso, não se deve esquecer: a escola
hegeliana estava dissolvida, mas a filosofia hegeliana não havia sido criticamente
superada. Strauss e Bauer extraíram
algo, cada um do seu lado, e viraram a filosofia hegeliana polemicamente
um contra o
outro. Feuerbach quebrou
o sistema e simplesmente o jogou para o lado. Mas não se extermina uma
filosofia simplesmente a qualificando de falsa. E uma obra tão poderosa
como a filosofia
de Hegel,
que teve uma influência tão grande sobre o desenvolvimento espiritual da
nação, não se permitiu ser deixada de lado pelo fato de ser ignorada sem
rodeios. Ela tinha de ser ``superada'' no seu sentido 
próprio, isto é, no sentido em que a sua forma fosse criticamente
aniquilada, porém o novo conteúdo obtido fosse salvo.
Veremos abaixo como isso se deu.

Nesse momento, porém, a revolução de 1848 colocou de lado, sem % MANTIVE Revolução
cerimônias, toda filosofia, assim como Feuerbach fizera com seu Hegel. E assim também o próprio Feuerbach foi impelido para o pano de fundo.

\quebra

\mbox{}
\vspace{2cm}

\addcontentsline{toc}{chapter}{Segunda parte}
\section{II}

\noindent{}A grande questão fundamental de toda a filosofia, especialmente da % MANTIVE grande e fundamental
moderna, é a relação entre pensar e ser. Desde tempos muito
remotos, em que os homens, ainda em total ignorância sobre a sua própria
constituição corporal e incitados por aparições em sonhos,\endnote{Ainda hoje, entre
selvagens e bárbaros inferiores, é a representação
  universal que as figuras humanas que aparecem em sonhos seriam almas
  que abandonam temporariamente os corpos; o homem efetivo é, portanto,
  considerado também responsável pelas ações que a sua aparição em sonho
  comete diante daquele que sonha. Em \emph{Thurn} {[}Everard Ferdinand
  im Thurn: \emph{Among the Indians of Guiana being sketches chiefly
  antropologic from the interior of British Guiana}{]}, por exemplo,
  isso se encontrou, em 1884, entre os índios na Guiana. {[}\versal{N.\,E.}{]}}
chegaram à representação de que o seu pensar e sentir não seriam uma
atividade do seu corpo, mas de uma alma particular que habita esse
corpo e o abandona com a morte. Desde esses tempos, os seres humanos
tinham de criar pensamentos sobre a relação dessa alma com o mundo
exterior. Se na morte a alma se separava do corpo e continuava a viver,
não havia nenhum motivo para lhe imputar ainda uma morte particular;
assim surgiu a representação de sua imortalidade que, naquele estágio de
desenvolvimento, de modo algum aparece como um consolo, mas como um
destino contra o qual nada se pode e, de modo bastante frequente, como
entre os gregos, como uma positiva infelicidade. Não foi a necessidade
religiosa de consolação, mas a aporia proveniente 
da limitação, igualmente universal, do que começar a fazer com a suposta
alma depois da morte do corpo que levou, de modo geral, à fastidiosa
imaginação da imortalidade pessoal. Por um caminho semelhante surgiram,
através da personificação dos poderes da natureza, os primeiros Deuses
que, na ulterior elaboração das religiões, supõem cada vez mais uma
configuração extramundana, até finalmente surgir --- por meio de um processo
de abstração que se orienta naturalmente pelo curso do desenvolvimento
espiritual, diria quase que se trata de um processo de destilação ---,
na cabeça dos seres humanos, a partir dos muitos Deuses mais ou menos
limitados e que se limitam reciprocamente, a representação de um único e
exclusivo Deus das religiões monoteístas. %MANTIVE

A questão da relação entre pensar e ser, espírito e natureza, a questão
suprema da filosofia como um todo tem a sua raiz, portanto, não menos
do que todas as religiões, nas representações estreitas e ignorantes do
estado de selvageria. Mas ela somente podia ser posta em sua plena
clareza, somente podia alcançar toda a sua significação, quando a
humanidade europeia despertasse da longa hibernação da Idade Média
cristã. A questão da posição do pensar em relação ao ser que, de
qualquer forma, também desempenhou o seu grande papel na escolástica da
Idade Média. A questão é: qual é o elemento originário, o espírito ou a
natureza? Esta questão aguçou"-se diante da Igreja da seguinte forma:
criou Deus o mundo ou o mundo está aí desde a eternidade?

Na medida em que essa questão era respondida de um modo ou de outro, os
filósofos se dividiram em dois grandes campos. Aqueles que afirmavam a
originariedade do espírito diante da natureza, que em última instância admitiam, portanto, uma criação do mundo, de qualquer tipo que fosse ---
e essa criação frequentemente é entre os filósofos, por exemplo
em Hegel,
ainda muito mais complicada e impossível do que no cristianismo ---,
formavam o campo do idealismo. Os outros, que viam a natureza como o
elemento originário, pertenciam às diversas escolas do materialismo.

As duas expressões significam algo diverso: idealismo e materialismo não
são utilizados aqui em sentido original, e tampouco em outros sentidos.
Veremos abaixo qual confusão surge quando se acrescenta algo mais nelas.

Mas a questão da relação entre pensar e ser tem ainda outro lado: como
se relacionam os nossos pensamentos sobre o mundo que nos circunda com
esse mesmo mundo? O nosso pensar está em condições de conhecer o mundo
efetivo, de produzir nas nossas representações e concepções do mundo
efetivo uma imagem especular correta da realidade 
efetiva? Na linguagem filosófica, tal questão corresponde à da
identidade entre pensar e ser, e é respondida afirmativamente pela
grande maioria dos filósofos.
Em Hegel,
por exemplo, a sua resposta afirmativa compreende"-se por si mesma. Afinal, aquilo que nós conhecemos no mundo efetivo é precisamente o seu
conteúdo que está conforme o pensamento, aquilo que torna o mundo uma
efetivação gradual da ideia absoluta. Tal ideia absoluta existiu em
alguns lugares desde a eternidade, independente e anteriormente ao
mundo; mas de tal modo que parece evidente ao pensamento a capacidade de reconhecer um conteúdo que, desde o início, já é um conteúdo do pensamento. É
igualmente evidente que o elemento a ser comprovado aqui já está contido
no pressuposto. Mas isso de forma alguma impede Hegel, a partir de sua
comprovação da identidade entre pensamento e ser, de concluir que sua
filosofia, por ser adequada ao seu pensamento, é também agora a única
adequada; e que a identidade entre pensar e ser pode se comprovar pelo %MANTIVE
fato de que a humanidade traduz imediatamente sua filosofia em
\emph{práxis} a partir da teoria e transforma o mundo inteiro segundo as
proposições fundamentais hegelianas. Essa é uma ilusão que ele
compartilha com praticamente todos os filósofos.

Ao lado desses ainda há, porém, uma série de outros filósofos que
contestam a possibilidade de um conhecimento do mundo ou ainda de um
conhecimento exaustivo. Há entre eles os
modernos, Hume e Kant,
que desempenharam um papel muito significativo no desenvolvimento
filosófico. O elemento decisivo para a refutação dessa perspectiva já foi
dito por Hegel, tanto quanto isso o era possível a partir da posição
idealista; o elemento materialista
que Feuerbach acrescenta
é mais intelectualmente estimulante do que profundo. A refutação mais
convincente desse elemento materialista, como de todas as outras ideias fixas
da filosofia, é a \emph{práxis}, a saber, o 
experimento e a indústria. Se nós podemos demonstrar a correção da nossa
concepção de um processo natural, na medida em que nós mesmos ao
empreendê"-lo o engendramos a partir das suas condições, podemos, acima
de tudo, torná"-lo utilizável para nossos objetivos, estabelece"-se o fim
da ``coisa em si'' inapreensível
de Kant.
As matérias químicas produzidas em corpos vegetais e animais eram as
tais ``coisas em si'', até a química orgânica começou a
apresentá-las uma após a outra; com isso, a ``coisa em si'' se tornou uma
coisa para nós, como, por exemplo, a matéria corante {[}da planta{]}
ruiva dos tintureiros, a alizarina, que já não permitimos que cresça em
campos nas raízes de ruiva dos tintureiros, mas a produzimos de modo
muito mais barato e simplesmente a partir do alcatrão do carvão mineral.
O sistema solar copernicano foi durante trezentos anos uma hipótese, em
que se podia apostar cem, mil, dez mil contra um, mas, ainda assim, uma
hipótese; quando Le Verrier, porém, a partir dos dados fornecidos por
este sistema, calculou não só a necessidade da existência de um planeta
desconhecido, como também o lugar desse planeta no céu, e quando então
Galle efetivamente encontrou esse planeta,
o sistema copernicano foi, naquele momento, comprovado.\endnote{Referência ao astrônomo Johann Gottfried Galle, o primeiro a
  visualizar Netuno em 23 de setembro de 1846, a partir, como indicado
  por Engels, dos cálculos do astrônomo e matemático francês Urbain Le
  Verrier do mesmo ano. {[}\versal{N.\,T.}{]}} Se, entretanto, a reabilitação da concepção
kantiana é tentada na Alemanha pelos neokantianos\endnote{O
  neokantismo tem início nos anos 1850 e 1860. Sua influência se estende
  até a segunda guerra mundial. Enquanto orientação filosófico"-acadêmica,
  a intenção era um retorno a Kant por meio de uma mediação crítica com
  o idealismo hegeliano e diversos materialismos do século \versal{XIX}. A partir
  de 1880 configura"-se como a escola predominante nas universidades de
  filosofia na Alemanha, tendo efeitos diretos em todas as chamadas
  \emph{Geistwissenschaften}. A sua pré"-história teórica delimita
  justamente o problema da relação entre subjetividade e razão, tendo em
  vista o problema do \emph{a priori} kantiano: poderia este ser
  compreendido como ``uma instituição orgânica inata ao gênero
  humano'' (Ollig, Hans"-Ludwig, 
  \emph{Der Neukantismus}, p.\,1)? Esse entendimento simplista de um
  ``neokantismo fisiológico'' foi reelaborado e criticado pelas escolas
  neokantianas seguintes. O centro da abordagem passou a ser a filosofia
  prática e metafísica no interior de um sistema transcendental que
  compreende a unidade das três grandes críticas, principalmente a
  partir de 1890. O texto de Engels dialoga justamente com o período de
  prelúdio dessa problemática sistêmica. Os anos 1870 e 1880
  correspondem à fase dos comentadores da obra kantiana, que serviu de
  impulso para fase de discussão sobre a legitimidade de um sistema
  transcendental (\emph{Der Neukantismus}, p. 2), tendo como principais
  representes, com suas diferenças de abordagem, a Escola de Marburg,
  Herman Cohen, Paul Nartop e Ernst Cassier, e a Escola do sudoeste
  alemão, como Wilhelm Windelband, Heinrich Rickert e Emil Lask. É
  também determinante a influência do neokantismo no revisionismo da
  social"-democracia alemã, de Kaustky e Bernstein, chegando até o
  marxismo austríaco e ao jovem Lukács, aspecto que reflete a dimensão
  da \emph{distorção sistêmica} \emph{da autonomia da vontade} diante da
  questão da organização revolucionária, isto é, de constituição e
  execução de uma \emph{práxis revolucionária}, de desenvolvimento
  concreto de uma \emph{vontade revolucionária} em interação recíproca
  com as formas políticas burguesas. {[}\versal{N.\,T.}{]}} e a reabilitação de Hume na
Inglaterra (onde nunca morreu) pelos agnósticos,\endnote{A
  alusão aos agnósticos remete certamente à obra do biólogo inglês Thomas Henry Huxley,
  amigo de Darwin e defensor de sua teoria da evolução, que teria cunhado, em 1869, sob influência da própria teoria da
  evolução e do ceticismo de Hume, o sentido moderno do termo
  ``agnóstico''. O pano de fundo teórico entre o ceticismo de Hume e
  \emph{a aporia} da ``coisa em si'' kantiana é reflexo de um longo
  processo de alteração do lugar de legitimação teórica e social das
  ciências da natureza no século \versal{XIX}. {[}\versal{N.\,T.}{]}} isso é, cientificamente, um retrocesso diante da
refutação teórica e prática há muito alcançada e,
na prática, apenas um modo envergonhado de aceitar o materialismo
pelas costas e de o negar perante o mundo.

Os filósofos, porém, nesse longo período
de Descartes a Hegel e de Hobbes a Feuerbach,
de modo algum foram impelidos a avançar, como acreditavam, apenas pela
força do puro pensamento. Pelo contrário. O que, na verdade, os impeliu
a avançar foi, nomeadamente, o poderoso e sempre mais veloz
progresso impetuoso da ciência da natureza \textbar{} e da indústria \textbar{}. Nos materialistas
isso já se mostrava na superfície, mas também os sistemas idealistas se
completaram cada vez mais com um conteúdo materialista e procuraram
conciliar a oposição entre espírito e matéria panteisticamente; de tal
modo que, no final, o sistema de Hegel representa apenas um materialismo, segundo método e conteúdo, idealisticamente posto de cabeça para baixo. 

Por isso é concebível que Starcke, em sua caracterização
de Feuerbach,
investigue, antes de tudo, a posição dele no que concerne à questão
fundamental da relação entre pensar e ser. Após uma curta
introdução na qual, em linguagem desnecessária e filosoficamente
prolixa, é exposta a concepção dos filósofos precedentes, isto é,
desde Kant, %MANTIVE
e na qual
Hegel,
por uma retenção demasiadamente formalista a passagens isoladas de suas
obras, é minimizado, segue"-se uma exposição pormenorizada do curso do
desenvolvimento da própria ``metafísica''
de Feuerbach,
tal como resulta da sequência dos escritos deste filósofo.
Essa exposição é trabalhada de modo fluido e claro, apenas
sobrecarregado, como todo o livro, por um lastro, não de
todo inevitável, de modos de expressão filosóficos, que é tanto mais
perturbador quanto menos o autor se atém ao modo de expressão de uma só
e mesma escola --- ou então do
próprio Feuerbach ---
e quanto mais ele mistura expressões no interior das mais diversas
orientações, justamente das que agora se espalham e denominam a si
mesmas de filosóficas.

O curso do desenvolvimento
de Feuerbach é
o de um hegeliano --- nunca, porém, totalmente ortodoxo --- em direção ao
materialismo, um desenvolvimento que, em um determinado estágio,
condiciona uma ruptura total com o sistema idealista de seu predecessor.
No fim, com uma força irresistível, impõe"-se"-lhe a intelecção de que a
existência pré"-mundana da ``ideia absoluta''
de Hegel,
a ``pré"-existência das categorias lógicas'', antes, portanto, de haver
mundo, nada mais é do que um resto fantástico da crença em um criador
extramundano; já que o mundo material, sensivelmente perceptível, ao
qual\est\ nós mesmos pertencemos, é o único elemento efetivo e que a nossa
consciência e o nosso pensamento, por mais que pareçam suprassensíveis, são o
produto de um órgão material, corpóreo, o cérebro. A matéria não é um
produto do espírito, mas o próprio espírito é apenas 
o produto supremo da matéria. Isso é, naturalmente, 
puro materialismo. Ao chegar aqui, Feuerbach empaca. Ele não pode
superar o preconceito filosófico habitual, não contra a
coisa em questão, mas contra o nome materialismo. Ele diz: ``O
materialismo é para mim a base do edifício do ser e saber humanos; mas,
para mim, ele não é nada do que é para o fisiólogo, para o naturalista em
sentido estrito, por exemplo Moleschott,
para o qual, devido a sua posição e profissão, o materialismo é necessariamente o próprio
edifício. No que precede concordo inteiramente com os materialistas, mas
não no que procede.''\endnote{\emph{Ludwig Feuerbach in
  seinem Briefwechsel und Nachlass 1850--1872.} Leipzig, Heidelberg:
  Winter Verlag, Bd. 2, 1874. Citado a partir de Starcke: Ludwig
  Feuerbach, p. 166. {[}\versal{N.\,T.}{]}}

Aqui,
Feuerbach coloca
no mesmo saco o materialismo, que é uma visão geral do mundo que repousa
sobre uma determinada concepção da relação entre matéria e espírito,
juntamente com a forma particular pela qual essa visão do mundo se
explicitou em um estágio histórico determinado \textbar{} no século \versal{XVIII}~\textbar{}. %COLOCAR textbar no alemão
Mais ainda, coloca"-o junto com a configuração vulgar, superficial, na qual
o materialismo do século \versal{XVIII} ainda continua a existir na cabeça
de naturalistas e médicos e que, nos anos cinquenta, foi pregado por
todos os cantos
por Büchner, Vogt e Moleschott. Porém,
assim como o idealismo passou por uma série de estágios de
desenvolvimento, o materialismo também passou. Juntamente com toda a descoberta
que faz época mesmo no domínio da ciência da natureza, ele tem que mudar
a sua forma; e na medida em que também a história está submetida ao
tratamento materialista, abre"-se aqui um novo traço de
desenvolvimento.

O materialismo do século passado era, sobretudo, mecânico, já que, de
todas as ciências da natureza daquele tempo, apenas a mecânica, e
justamente apenas a mecânica dos corpos\est\ sólidos --- celestes e terrestres ---,
em suma, a mecânica dos corpos pesados, tinha chegado a certa conclusão.
A química somente existia em sua configuração infantil, flogística.
A biologia ainda
usava fraldas; apenas de modo grosseiro o organismo vegetal e animal
era investigado e explicado por causas puramente mecânicas; assim
como
para Descartes o
animal era uma máquina, o homem o era para os materialistas do século
\versal{XVIII}. Essa aplicação exclusiva do padrão da mecânica a processos que
são de natureza química e orgânica e para os quais as leis mecânicas
certamente também se aplicam, apesar de impelidas para o segundo plano
por leis superiores, forma a primeira limitação específica do
materialismo francês clássico, ainda que inevitável para seu tempo.

A segunda limitação específica desse materialismo consistiu na sua
incapacidade de apreender o mundo como um processo, como uma matéria
concebida em uma formação historicamente contínua. Isso correspondia ao
estado da ciência da natureza daquela época e ao modo
metafísico --- isto é, antidialético --- do filosofar a ela vinculado. A natureza era
concebida, isso era consciente, como um movimento eterno. Mas esse
movimento, segundo a representação daquela época, girava em um círculo
eterno e, portanto, nunca saía do lugar; produzia novamente os
mesmos resultados. Essa representação era inevitável naquela época. A
teoria
de Kant sobre
a gênese do sistema solar mal havia se estabelecido e ainda não passava
de mera curiosidade. A história do desenvolvimento da Terra, a geologia,
era ainda totalmente desconhecida, e a representação de que os atuais seres
vivos naturais são o resultado de uma longa série de
desenvolvimentos do simples ao complexo não podia, naquela época, ser,
em geral, cientificamente estabelecida. A concepção a"-histórica
da natureza era, portanto, inevitável. 
 \textbar{}\,Tão pouco é possível censurar os filósofos do século \versal{XVIII} por\est\ isso,
como também é pouco possível censurar Hegel. Para ele, a natureza, como mera
``exteriorização alienante'' da ideia, não é capaz 
de nenhum desenvolvimento no tempo, mas apenas de uma extensão de sua
multiplicidade no espaço, de tal modo que expõe simultânea e
sucessivamente todos os estágios de desenvolvimento nela compreendidos e
está condenada à repetição eterna do mesmo processo. E nesse
absurdo de um desenvolvimento no espaço, porém fora do tempo --- a
condição fundamental de todo o desenvolvimento ---,
Hegel coloca
um grande peso na natureza, justamente ao mesmo tempo em que a geologia,
a embriologia, a fisiologia vegetal e animal e a química orgânica foram
desenvolvidas e em que, por toda a parte, na fundamentação dessas novas
ciências, emergiam pressentimentos geniais da posterior teoria da
evolução (por exemplo Goethe e Lamarck).
Mas o sistema assim exigia, e o método precisava, por amor ao sistema,
ser infiel consigo mesmo.\,\textbar{} A mesma concepção a"-histórica
vigorava no âmbito da história. Aqui, a luta contra os restos da Idade
Média tornava a visão parcial. A Idade Média era considerada como
simples interrupção da história por uma barbárie 
universal de mil anos; os grandes progressos da Idade Média não eram vistos --- a expansão
do âmbito cultural europeu, as grandes nações que sobreviveram até hoje,
que ali se formaram uma ao lado da outra, por fim, os enormes progressos
técnicos dos séculos \versal{XIV} e \versal{XV}. Uma
intelecção
racional do grande nexo histórico tornou"-se assim
impossível, e a história servia, no máximo, como uma coleção de exemplos
e ilustrações para uso dos filósofos.

Os vendedores ambulantes vulgarizadores que nos anos cinquenta na %Os vendedores ambulantes que, nos anos cinquenta na Alemanha, vulgarizaram o materialismo?
Alemanha se fizeram no materialismo, de modo algum ultrapassaram a
barreira de seus mestres. Todos os progressos da ciência da natureza
feitos desde então lhes serviam apenas como novos fundamentos de
comprovação contra a existência do criador do mundo; e, de fato, estava
totalmente fora de questão continuar a desenvolver a teoria. Se o
idealismo tinha esgotado o seu latim e tinha se deparado com a morte por %o idealismo esgotara o seu latim e se deparara...?
meio da revolução de 1848, vivenciou assim a satisfação de ver que o
materialismo momentaneamente ainda tinha caído mais %momentaneamente caíra ainda mais baixo?
baixo. Feuerbach tinha
decididamente razão quando declinava a responsabilidade por esse
materialismo; apenas não podia confundir a doutrina dos pregadores %não podia ter confudido?
ambulantes com o materialismo em geral.

Entretanto, é preciso aqui observar duas coisas. Em primeiro lugar, ao
longo da vida
de Feuerbach,
a ciência da natureza era ainda compreendida naquele intenso processo de
fermentação que só nos últimos quinze anos recebeu um desfecho relativo, %relativo e esclarecedor como sinônimos?
esclarecedor; um novo material de conhecimento foi fornecido em uma
medida até aqui sem precedentes, mas o estabelecimento
da conexão e, assim, da ordem nesse caos de 
descobertas precipitadas só se tornou possível muito recentemente. De
fato,
 Feuerbach ainda
vivenciou as três descobertas decisivas: a da célula, a da
transformação da energia e da, denominada %como denominada por Darwin,?
por Darwin,
teoria da evolução. Mas como o solitário
filósofo poderia, no campo, perseguir suficientemente a ciência para %poderia perseguir suficientemente no campo ...?
apreciar plenamente descobertas que os próprios naturalistas daquele
tempo em parte ainda contestavam, em parte não compreendiam %naturalistas daquele tempo ou contestavam ainda, ou não compreendiam? %compreendiam bem, em vez de compreendiam suficientemente?
suficientemente? A culpa reside aqui unicamente nas miseráveis condições
alemãs, em virtude das quais as cátedras da filosofia eram monopolizadas %foram monopolizadas?
por espirituosos e ecléticos esmagadores de pulgas
enquanto Feuerbach,
que os superava do alto de uma torre, tinha que se tornar um camponês e
se atrofiar em um pequeno \emph{Dorf}. Não é, portanto, culpa %NT para Dorf?
de Feuerbach
se a agora tornada possível concepção histórica da natureza, que põe de %se a concepção histórica da natureza agora possível, que põe de lado,...?
lado \textbar{} afasta \textbar{} todas as parcialidades do materialismo francês,
permanecesse inacessível para ele.

Em segundo lugar,
porém, Feuerbach tem
toda a razão, já que o materialismo meramente científico"-natural é ``o
fundamento do edifício do saber humano, mas não o próprio edifício''.
Afinal, não vivemos apenas na natureza, mas também na sociedade humana,
e essa tem também a sua história de desenvolvimento e a sua ciência
tanto quanto a natureza. Tratava"-se, portanto, de estabelecer uma
harmonia entre a ciência da sociedade, isto é, o complexo interno
das assim chamadas ciências históricas e filosóficas, 
e a fundamentação materialista, e de reconstruí"-las sobre esse 
fundamento. Isto, porém, não foi concedido
a Feuerbach.
Ele permaneceu aqui, apesar da ``fundamentação'', preso aos laços
idealistas tradicionais e reconheceu isso nominalmente:
``Retrospectivamente concordo com os materialistas, mas não
progressivamente.'' Mas quem aqui, no domínio social, não avançou
``progressivamente'', não ultrapassou a sua posição de 1840 ou de 1844,
foi o
próprio Feuerbach e,
de fato, uma vez mais principalmente na sequência do seu isolamento,
que o impeliu a produzir pensamentos a partir da sua 
cabeça solitária --- ele que, mais do que todos os outros filósofos, era
predisposto ao intercâmbio sociável --- em vez de os produzir no encontro
amigável e hostil com outros homens de seu calibre. O quanto, neste
âmbito, ele permaneceu idealista veremos mais à frente em detalhes.

Aqui é preciso apenas observar que Starcke procura o idealismo
de Feuerbach no
lugar errado.
``Feuerbach é
idealista, ele acredita no progresso da humanidade'' (p. 19). --- ``A
fundamentação, a infraestrutura do todo permanece, 
não obstante, o idealismo. O realismo não é para nós senão uma proteção
contra caminhos equivocados enquanto seguimos as nossas correntes
ideais. Não são compaixão, amor e entusiasmo pela verdade e pela %Não são forças ideais a compaixão, o amor e o entusiasmo pela verdade e pela justiça?
justiça, forças ideais?'' (p. \versal{VIII}).

Em primeiro lugar, idealismo aqui nada mais é do que persecução de
finalidades ideais. Tais finalidades, porém, remetem necessariamente, no
máximo, ao idealismo
de Kant e
ao seu ``imperativo categórico''; porém, %porém, de modo algum Kant denominou sua filosofia de ``idealismo transcendental'' por tratar de ideais éticos?
mesmo Kant 
denominou sua filosofia de ``idealismo transcendental'', de modo algum
por tratar de ideais éticos, mas por razões totalmente diferentes,
como Starcke recordará. A superstição segundo a qual o idealismo
filosófico giraria em torno da crença em ideais éticos, isto é, sociais,
surgiu do lado de fora da filosofia, entre filisteus alemães que
aprenderam a decorar nos poemas
de Schiller alguns %alguns pedaços da cultura filosófica que lhes...?
pedaços da formação cultural filosófica que lhes era necessária. Ninguém
criticou mais agudamente o impotente ``imperativo categórico''
de Kant ---
impotente porque ele exige o elemento impossível, portanto nunca %substituir "portanto" por "logo"?
chega a ser algo efetivo ---, ninguém zombou de modo mais agudo do
entusiasmo filisteu por ideais irrealizáveis, transmitidos
por Schiller,
do que justamento o idealista perfeito, Hegel (ver, por exemplo,
a \emph{Fenomenologia}).\endnote{A crítica ao imperativo
  categórico kantiano na \emph{Fenomenologia do Espírito} é
  desenvolvida, principalmente, na formulação da \emph{Verstellung}
  (deslocamento dissimulador). Uma passagem fundamental da recepção
  crítica da \emph{Razão prática} de Kant, ponto em que Hegel expõe a
  fragilidade conceitual que julga ver no dualismo kantiano: a resolução
  da antinomia entre natureza e moralidade, entre a estrutura
  normativa"-causal da natureza e a estrutura normativa da moralidade,
  via postulado de um ``\emph{sumo bem originário''} (\emph{KpV}, A 226
  ). O centro da crítica de Hegel é justamente ao apriorismo da solução
  kantiana, à fragilidade dessa indeterminação (aqui a distorção e
  dissimulação) do lugar (do \emph{Stellung}) da consciência moral, que,
  justificada como \emph{a priori}, ignora o caráter ativo da
  consciência e se torna uma mera projeção ideal que pode assumir
  qualquer conteúdo. É provavelmente a esse aspecto da crítica que
  Engels se refere. {[}\versal{N.\,T.}{]}}

Em segundo lugar, não é possível evitar que tudo o que move o homem
tenha que passar por sua cabeça --- até mesmo comer e beber, que começam
como resultado da fome e da sede sentidas pela cabeça e também terminam como %terminam com a satisfação por meio da cabeça?
resultado da satisfação por meio da cabeça. Os efeitos do mundo exterior
sobre o homem se expressam na sua cabeça, refletem"-se no interior dela
como sentimentos, pensamentos, impulsos, determinações da vontade, em
suma, como ``correntes ideais'', e se tornam, nessa cofiguração,
``poderes ideais''. Aqui, se a circunstância segundo a qual esse homem, %suprimir Aqui?
em geral, ``segue correntes ideais'' e concede aos ``poderes ideais''
uma influência sobre si --- se isto faz dele um idealista, então todo o
homem, de algum modo desenvolvido, é um idealista nato, e como é
possível, em geral, ainda existirem aí materialistas? %suprimir aí e em geral?

Em terceiro lugar, a convicção de que a humanidade, pelo menos
momentaneamente, move"-se, de maneira geral, em direção progressiva não %momentaneamente e de maneira geral? escolher só um uso para não ficar redundante?
tem absolutamente nada a ver com a oposição entre materialismo e
idealismo. Os materialistas franceses tinham esta convicção em grau
quase fanático, não menos do que os
deístas Voltaire e Rousseau,
e fizeram, de modo suficientemente frequente, os maiores sacrifícios %e fizeram-lhe frequentemente os maiores sacrifícios pessoais (substituir "de modo suficientemente frequente" por "frequentemente" e substituir "a ela" por "lhe)?
pessoais a ela. Se alguma vez alguém consagrou a vida toda ao
``entusiasmo pela verdade e pela justiça'' --- tomando a fraseologia no
bom sentido --- foi, por
exemplo, Diderot. Se Starcke esclarece %esclarece mesmo? não seria melhor "define" como idealismo, ou algo nessa linha?
assim tudo isso como idealismo, demonstra"-se apenas que a palavra %subtrair "assim"?
materialismo, e toda a oposição em ambas as direções, perdeu aqui para ele %subtrair aqui
todo o sentido.

O fato é que Starcke faz aqui uma concessão imperdoável, ainda que %subtrair aqui
talvez inconscientemente, ao preconceito filisteu contra o nome
materialismo resultante de longos anos de abuso sacerdotal. O filisteu
entende por materialismo voracidade, bebedeira, cobiça do olhar, prazer
carnal e ambição, avidez monetária, avareza, ânsia de posse, maquinação
do lucro e agiotagem, em suma, todos os vícios engordurados aos quais
ele próprio se entrega em segredo; e por idealismo a crença na virtude,
no amor humano universal e, em geral, em um ``mundo melhor'', com o que
se exibe diante dos outros mas nos quais ele próprio acredita, no
máximo, enquanto se preocupa em atravessar a ressaca moral 
ou a bancarrota que necessariamente se seguem aos
seus habituais excessos ``materialistas'' e, assim, canta a sua cantiga
predileta: que é o homem --- meio animal, meio anjo. %que o homem é meio animal...?

De resto, Starcke esforça"-se muito para
defender Feuerbach dos
ataques e teoremas dos professores de segundo escalão
que hoje se propagam na Alemanha sob o nome de filósofos. Para as pessoas
que se interessam por essa placenta da filosofia clássica alemã isso é
certamente importante; para o próprio Starcke isso pode até
parecer necessário. Nós pouparemos os leitores disso.

\quebra

\begin{flushright}
\emph{Die Neue Zeit. Ano 4.}\\
\emph{Caderno 5, maio de 1886.}
\end{flushright}

\vspace{2cm}

\addcontentsline{toc}{chapter}{Terceira parte}
\section{III}

\noindent{}O idealismo efetivo
de Feuerbach se
explicita assim que chegamos à sua filosofia da religião e da ética. Ele
não quer de modo algum abolir a religião, ele quer realizá"-la. A própria
filosofia deve se transformar em religião. ``Os 
períodos da humanidade se diferenciam apenas pelas transformações
religiosas. Um movimento histórico somente aceita o fundamento quando
aceita o coração do homem. O coração não é uma forma da religião, como
se ela também devesse estar no coração; o coração é a essência da
religião'' (citado por Starcke, p.\,168). A religião é,
segundo Feuerbach,
a relação de sentimentos, a relação de corações entre os homens, que até
então procurava a sua verdade em uma imagem especular fantástica da
realidade efetiva --- na mediação por meio de um ou de muitos Deuses, %na mediação por meio? escolher apenas um uso, a mediação já implica um meio, e vice-versa...
imagens especulares fantásticas de qualidades humanas ---, mas agora a
encontra diretamente e sem mediação no amor entre o Eu e o Tu.
Para Feuerbach,
o amor sexual torna"-se, assim, uma das mais elevadas, se não a mais
elevada, forma de exercício de sua nova religião.

Afinal, as relações de sentimentos entre os homens, isto é, entre os
dois sexos, têm existido desde que há seres humanos. Particularmente o
amor sexual conheceu um desenvolvimento nos últimos oitocentos anos e
conquistou uma posição que, durante este tempo, fizeram"-no eixo
obrigatório de toda poesia. As religiões positivas existentes se
limitaram\est\ a consagrar de modo elevado a regulação estatal do amor
sexual, isto é, a legislação do matrimônio, e amanhã podem desaparecer
conjuntamente sem que na prática do amor e da amizade o mínimo tenha
sido alterado. Assim como a religião cristã na França também desapareceu %substituir esse Assim, que na primeira leitura dá a ideia de que se vai fazer uma comparação?
factualmente de 1793 a 1798, de tal modo que nem o próprio Napoleão pôde
reintroduzi"-la sem relutância e dificuldade, sem que, porém, tenha
surgido nesse intervalo a necessidade de uma substituição no
sentido de Feuerbach.

Para Feuerbach, o idealismo consiste aqui no fato de simplesmente %consiste na permissão de simplesmente considerar as relações dos homens...
permitir considerar as relações dos homens com base na inclinação mútua
entre si, amor sexual, amizade, compaixão, sacrifício etc., relações
estas que não são recordadas a partir de si mesmas sem que se recorde de
uma religião particular, mesmo as que para ele pertencem ao passado,
afirma, pelo contrário, que elas só alcançam sua validade plena assim %quem, ou o que afirma? ficou confuso... talvez: "mesmo as que para ele pertencem ao passado, pelo contrário, só alcançam sua validade plena QUANDO [substituir "assim que" por "quando"] recebem uma..."
que recebem uma consagração superior sob o nome religião. A questão
principal para ele não é que essas inter"-relações puramente humanas
existam, mas que elas sejam apreendidas como a nova, verdadeira
religião. Elas só devem ter validade plena se receberam o selo
religioso. Religião vem
de \emph{religare} e, originariamente,
significa ligação. Toda a ligação entre dois homens é, portanto, uma
religião. Tais artifícios etimológicos formam o último meio de
transmissão da filosofia idealista. O que deve valer não é o que a
palavra significa segundo o desenvolvimento histórico do seu uso
efetivo, mas o que deveria significar segundo sua descendência. E assim
o amor sexual e o vínculo sexual são elevados ao céu de uma
``religião'' para que a palavra religião, cara à lembrança idealista,
não desapareça da linguagem. É justamente assim que, nos anos quarenta,
falavam os reformistas de Paris da orientação de Louis Blanc, os quais,
igualmente, só podiam imaginar um homem sem religião como um monstro e
nos diziam: \emph{Donc, l'athéisme c'est votre
religion}!\endnote{``Portanto,
  o ateísmo é vossa religião!''. {[}\versal{N.\,T.}{]}}\est\
Feuerbach querer
estabelecer a verdadeira religião tendo como fundamento uma visão da
natureza essencialmente materialista significa o mesmo que apreender a
química moderna como a verdadeira alquimia. Se a religião pode existir
sem o seu Deus, então a alquimia também pode sem a sua pedra filosofal.
Existe, aliás, uma conexão muito estreita entre alquimia e religião. A
pedra filosofal tem muitas propriedades semelhantes às divinas, e os
alquimistas greco"-egípcios dos dois primeiros séculos da nossa era
influenciaram a formação da doutrina cristã, como comprovam os dados
fornecidos por Kopp e Berthelot.\endnote{A relação
  entre alquimia e religião aparece em Hermann Kopp, \emph{Geschichte
  der Chemie} {[}História da química{]}, 1843, e em Marcelin Berthelot,
  \emph{Les origines de l'alchimie}, 1885. {[}\versal{N.\,T.}{]}}

É absolutamente falsa a afirmação
de Feuerbach
de que os ``períodos da humanidade se diferenciam apenas por
transformações religiosas''. \textbar{}\,Grandes pontos de mudança histórica
foram \emph{acompanhados} por transformações religiosas na medida em
que sejam consideradas apenas as três religiões mundiais que até agora %na medida em que se considerem apenas...
existiram: budismo, cristianismo, islamismo.\,\textbar{} As velhas religiões
tribais e nacionais, que surgiram de modo espontâneo"-natural, \textbar{} não
faziam propagandas e perderam \textbar{} todo o poder de resistência logo que a %"perderam" dentro da barra mesmo? assim dá a impressão que a primeira versão do texto estava incompleta: "que surgiram de modo espontâneo-natural, todo o poder..."
autonomia das tribos e povos foi rompida; entre os germanos bastou,
inclusive, o simples contato com o império mundial romano em decadência e
com a religião mundial cristã por ele recentemente adotada e conveniente
a seu estado econômico, político e ideal. Somente nessas religiões
mundiais, que surgem de modo mais ou menos artificial, particularmente
no cristianismo e islamismo, encontramos movimentos históricos mais
gerais que adotam um caráter religioso e, \textbar{} mesmo no âmbito do
cristianismo, \textbar{} o caráter religioso limita"-se, nas revoluções com
significado efetivamente universal, aos primeiros estágios da luta de
emancipação da burguesia do século \versal{XIII} ao século \versal{XVII}, e não se
explica, como pensa
Feuerbach,
pelo coração do homem\est\ e por sua carência religiosa, mas por toda a
pré"-história medieval que não conhecia outra forma de ideologia além da
religião e da teologia. Quando, porém, a burguesia se
fortaleceu no século \versal{XVIII} o suficiente para ter a sua própria ideologia, adequada à
sua posição de classe, fez, então, a sua grande e definitiva revolução,
a francesa, sob o apelo exclusivo às ideias jurídicas e políticas, e só se
preocupou com a religião na medida em que ela se colocava no caminho;
mas não lhe ocorreu estabelecer uma nova religião no lugar da antiga; \textbar{}
sabemos como Robespierre\endnote{Referência ao \emph{Culte de l'Être suprême} (Culto do Ser supremo) desenvolvido por Robespierre, que, inspirado pelo deísmo de Voltaire e
  pelo teísmo cristão de Rousseau, visava opor"-se ao ateísmo radical do
  \emph{Culto da Razão} de Joseph Fouché. {[}\versal{N.\,T.}{]}} fracassou nisso. \textbar{}

Hoje em dia, a possibilidade de um sentimento puramente humano no
intercâmbio com outros homens já se atrofiou para nós o suficiente
devido à sociedade, na qual temos que nos movimentar, fundada na
oposição de classes e na dominação de classe: não há razão para deixar
ela nos atrofiar ainda mais elevando aos céus esses sentimentos em uma
religião. E, do mesmo modo, a compreensão das grandes lutas de classes
históricas torna"-se para nós já suficientemente obscurecida pela
historiografia corrente, especialmente na Alemanha, sem que nós, pela
transformação dessa história de lutas em um mero apêndice da história da
Igreja, também tenhamos necessidade de torná"-la completamente impossível %também necessitemos torná-la...?
para nós. Já aqui fica claro o quanto hoje estamos distantes %subtrair "aqui"
de Feuerbach.
As suas ``mais belas passagens'', de celebração dessa nova religião do
amor, são absolutamente inteligíveis hoje.

A única religião
que Feuerbach investiga
seriamente é o cristianismo, a religião mundial do Ocidente, fundado no
monoteísmo. Ele demonstra que o Deus cristão é apenas o reflexo
fantástico, a imagem especular do homem. Afinal, esse mesmo Deus é, %Entretanto, esse mesmo Deus é o produto...?
porém, o produto de um longo processo de abstração, a quintessência
concentrada dos muitos Deuses anteriores, de tribos e nações. E,
correspondente, o homem cuja imagem\est\ é esse Deus também não é um homem %correspondentemente, em correspondência?
efetivo, mas igualmente a quintessência dos muitos homens efetivos, o %precisa usar esse "igualmente" aqui, já que já se afirmou a correspondência entre a religião e seu homem?
homem abstrato, portanto, ele próprio novamente uma imagem do
pensamento. O
mesmo Feuerbach que
a cada página prega a sensibilidade, o mergulho no elemento concreto, na
realidade efetiva, torna"-se, por todos os lados, abstrato, na medida em
que fala de um intercâmbio entre homens mais amplo do que o mero
intercâmbio sexual.

Esse intercâmbio só lhe oferece um lado: a moral. E aqui nos
surpreendemos novamente com a espantosa pobreza
de Feuerbach comparado
com Hegel.
A ética ou doutrina da eticidade de Hegel é a filosofia do direito,
abarcando: 1. o direito abstrato, 2. a moralidade, 3. a eticidade, sob
a qual, por sua vez, estão reunidos: a família, a sociedade
civil"-burguesa e o Estado. A forma é tão idealista quanto o conteúdo é
aqui realista. Todo o domínio do direito, da economia, da política, é
apreendido conjuntamente com a moral.
Em Feuerbach,
ocorre justamente o contrário. Ele é realista segundo a forma, ele parte %segundo a forma, e parte
do homem; mas não se fala absolutamente nada do mundo onde esse homem
vive e, assim, esse homem permanece sempre o mesmo homem abstrato que na
filosofia da religião detinha a palavra. Esse homem não nasceu %Esse homem não nasceu do corpo da mãe, mas sim revelou"-se (...), e, consequentemente, (...)
justamente do corpo da mãe, revelou"-se do Deus das religiões
monoteístas, consequentemente, também não vive em um mundo efetivo que
surgiu historicamente e foi determinado historicamente; de fato, ele
entra em intercâmbio com outros homens, mas cada um dos outros são tão
abstratos quanto ele. Na filosofia da religião, temos ainda homem e
mulher, mas na ética essa última diferença também desaparece.
Em Feuerbach
aparecem, de fato, longos intervalos, proposições como: ``Em um palácio
pensa"-se de modo diferente do que em uma cabana.'' --- ``Onde, diante da
fome, da miséria, tu não tens matéria nenhuma no corpo, não tens também
na cabeça, nos sentidos e no coração, matéria
para a moral.'' --- ``A política tem de se tornar a nossa
religião'' etc.\endnote{\versal{FEUERBACH}, L. ``\emph{Grundsätze der Philosophie. Nothwendigkeit einer Veränderung. 1842/43}''. \versal{IN}: \emph{Ludwig Feuerbach in seinem
  Briefwechsel und Nachlass 1850--1872.} Leipzig; Heidelberg: Winter
  Verlag, Bd. 2, 1874. Citado a partir de Starcke: Ludwig Feuerbach, 1ª
  edição, p. 280. {[}\versal{N.\,T.}{]}}
Mas Feuerbach não
sabe absolutamente como começar a agir com essas proposições, elas
permanecem puros modismos de fala, e o próprio Starcke tem de admitir
que a política era
para Feuerbach um
limite intransponível e que a ``doutrina da sociedade, a sociologia, era
para ele uma \emph{terra
incógnita}''.\endnote{Starcke: Ludwig Feuerbach, 1ª edição, p. 280. {[}\versal{N.\,T.}{]}}

Diante de
 Hegel,
parece igualmente superficial no tratamento da oposição entre bem e mal.
``Crê"-se que se diz algo muito grande'' --- vemos
em Hegel ---
``quando se diz: o homem é bom por natureza; mas esquecemos que dizemos
algo ainda maior com as palavras: o homem é mau por natureza.''\endnote{\emph{Grundlinien der Philosophie des Rechts},
  \emph{oder Naturrecht und Staatswissenschaft}. Hrsg. von Eduard Gans.
  2. Aufl. Berlin, 1840, § 18: ``Em relação à \emph{adjudicação}
  (\emph{Beurtheilung}) dos impulsos na dialética do fenômeno
  (\emph{Erscheinung}), por {[}aparecerem{]} como \emph{imanentes}, por
  isso, \emph{positivas}, as determinações imediatas da vontade são
  \emph{boas}; o homem é \emph{assim bom por natureza}. Porém, na medida
  em que são \emph{determinações da natureza}, isto é, opostas à
  liberdade e ao conceito de espírito em geral, e que são o
  \emph{elemento negativo}, precisam ser \emph{exterminadas}, \emph{o
  homem é assim mau por natureza}. O elemento decisivo a favor ou contra
  uma ou outra afirmação é, a partir de seu posicionamento
  (\emph{Standpunkt}), igualmente o arbítrio subjetivo''. {[}\versal{N.\,T.}{]}}
Em Hegel,
o mal é a forma em que a força motriz do desenvolvimento histórico se
apresenta. E de fato aqui reside o duplo sentido segundo o qual, por um
lado, cada novo progresso aparece necessariamente como um sacrilégio
contra um elemento sagrado, como rebelião contra situações antigas,
atrofiadas, mas sacralizadas pelo hábito, e, por outro lado, desde o
aparecimento das oposições de classes, são justamente as piores paixões
dos homens, cobiça e ânsia de domínio, que se 
tornaram alavancas do desenvolvimento histórico, das quais, por exemplo, %e das quais
a história do feudalismo e da burguesia são uma única e contínua prova.
Não ocorre, porém, a 
Feuerbach 
investigar o papel histórico do mal moral. Em geral, a história é para %Não ocorre a Feuerbach, porém
ele um campo desagradável, monstruoso. Nesse sentido, inclusive, o seu
dito: ``O homem que originariamente surgiu da natureza era apenas também
uma pura essência da natureza, não era homem. O homem é um produto do
homem, da cultura, da história'',\endnote{\versal{FEUERBACH}, L. ``Fragmente zur Characteritik meines philosophischen
  Curriculum vitae''. In. \emph{Sämtliche Werke.} Bd. 2, p. 411. Citado
  segundo Starcke: Ludwig Feuerbach. 1ª edição, p. 114. {[}\versal{N.\,T.}{]}} mesmo esse
dito permanece para ele completamente improdutivo.

O
que Feuerbach nos
indica sobre moral somente pode, de acordo com isso, ser algo
extremamente pobre. O impulso para a felicidade é inato ao homem e tem
de formar, portanto, a fundamentação de toda a moral. Mas o impulso para
a felicidade experimenta uma dupla correção.\endnote{Essa
  dupla correção ou limitação recíproca dos impulsos naturais para
  felicidade de um individuo frente ao outro é apresentada por Feuerbach
  em sua ``filosofia moral'': ``Zur Moralphilosophie (1868)'' In:
  \emph{Ludwig Feuerbach in seinem Briefwechsel und Nachlass}, Bd. 2. {[}\versal{N.\,T.}{]}}
Em primeiro lugar, pelas consequências naturais das nossas ações: à
bebedeira segue"-se a ressaca, aos excessos habituais a doença. Em
segundo lugar, pelas suas consequências sociais: se não respeitamos o
mesmo impulso dos outros para a felicidade, eles irão se defender e
perturbaram o nosso próprio impulso para a felicidade. Segue"-se daqui %perturbarão?
que nós, para satisfazermos o nosso impulso, temos de estar em condições
de avaliar de modo correto as consequências das nossas ações e temos,
por outro lado, de estar em condições de fazer valer a igualdade de direito
dos outros em relação ao impulso correspondente. Autodelimitação
racional em relação a nós próprios e ao amor --- sempre novamente o amor!
--- no intercâmbio com os outros são, portanto, as regras fundamentais da %da qual
moral feuerbachiana, a partir das quais todas as outras derivam. E nem
as mais espirituosas exposições
de Feuerbach,
nem os mais vigorosos elogios de Starcke, podem esconder a fraqueza e a
banalidade desse par de proposições.

O impulso para a felicidade satisfaz"-se apenas muito excepcionalmente e
de modo algum em benefício de si e de outras pessoas, através da
ocupação de um homem consigo mesmo. Requer, porém, ocupação com o mundo
exterior, com os meios de satisfação: alimentação, um
indivíduo do outro sexo, livros, conversas, debates, atividade, objetos
para uso e elaboração. A moral
de Feuerbach ou
pressupõe que estes meios e objetos de satisfação sejam dados sem mais a
todo homem, ou ela lhe dá, porém, apenas boas doutrinas inaplicáveis.
Não vale, portanto, absolutamente nada para as pessoas às quais esses
meios faltam. E o
próprio Feuerbach nos
explica isso com palavras duras: ``Em um palácio pensa"-se de modo
diferente do que em uma cabana.'' ``Onde, diante da fome, da miséria, tu
não tens matéria nenhuma no corpo, não tens também na cabeça, nos
sentidos e coração, matéria para a moral.''


As coisas ficarão melhores com a igualdade de direito em relação ao
impulso de felicidade do
outro? Feuerbach 
apresenta essa reivindicação como absolutamente válida para todas as
épocas e circunstâncias. Mas desde quando ela vale? Na Antiguidade,
entre escravos e senhores, na Idade Média, entre servos e barões,
tinha"-se em vista a igualdade de direito em relação ao impulso para a
felicidade? O impulso para a felicidade da classe oprimida não era, de
modo brutal e ``de direito'', sacrificado em prol do impulso de
felicidade da dominante? --- Sim, isso também era imoral, mas agora a
igualdade de direito é reconhecida. --- Reconhecida na fraseologia, desde
que é visto que a burguesia, na sua luta contra a feudalidade e no
desenvolvimento da produção capitalista, foi obrigada a abolir todos os
privilégios estamentais, isto é, pessoais, e a introduzir a igualdade
jurídica de direito da pessoa, primeiro, a do direito privado, depois
também, gradualmente, a do direito estatal. Mas o impulso para a
felicidade não vive senão, minimamente, de direitos ideais e, na maior
parte, de meios materiais; e a produção capitalista cuida para que caiba
à grande maioria das pessoas com direitos iguais apenas o necessário a
uma vida estreita, e portanto mal respeita, se é que em geral respeita, a
igualdade de direito do impulso da maioria para a felicidade mais do que
a escravidão ou a servidão o fizeram. E essa é melhor no que concerne %não fica claro a que o "essa" se refere. Se a "produção capitalista", talvez "E essa produção é"
aos meios espirituais da felicidade, aos meios de formação cultural? Não
é o próprio ``mestre"-escola de Sadowa''\endnote{Possivelmente uma referência à Batalha de Königgrätz (ou de Sadowa, cidade da atual República Tcheca) ocorrida
  em três de julho de 1866, enfrentamento decisivo da guerra austro"-prussiana, com vitória da Prússia.
  A disputa se deu no contexto do processo de unificação da Alemanha. O
  ``mestre"-escola de Sadowa'' aponta possivelmente para vitória da
  Prússia e a sobreposição de seu sistema educacional sobre o austríaco.
  Possível alusão também à obra do clérigo católico e membro da Câmara
  dos Deputados da Baviera, crítico das imposições estatais e vigilância
  dos sistemas educacionais, Josef Lukas. O livro tem justamente o mesmo
  titulo: \emph{Der Schulmeister von Sadowa.} Mainz, 1878. {[}\versal{N.\,T.}{]}} uma
personagem mítica?

Mais ainda. Segundo a teoria da moral %Mas ainda: segundo...
de Feuerbach,
a bolsa de valores é o templo supremo da eticidade --- 
pressupondo apenas que se especula sempre corretamente. Se o meu impulso
para a felicidade me conduz à bolsa de valores e lá eu pondero
corretamente as consequências das minhas ações de tal modo que elas só
me trazem vantagem e nenhum prejuízo, isto é, eu sempre ganho, a
prescrição
de Feuerbach está
cumprida. Também não interfiro no mesmo impulso de felicidade de outra
pessoa, afinal o outro, assim como eu, dirigiu"-se por livre vontade à
bolsa, seguiu seu impulso de felicidade ao fechar o negócio especulativo
tanto quanto eu fizera. E se ele perde seu dinheiro, sua ação
comprova"-se por meio disso ser mal\est\ calculada, como imoral,
e ao levar a cabo a pena que ele merece, posso até
ufanar orgulhosamente como um Rhadamanthus moderno. O amor domina também
na Bolsa, na medida em que ele não é mera fraseologia sentimental,
afinal, cada um encontra no outro a satisfação do seu impulso para a
felicidade, e é justamente isso que o amor deve cumprir e a isso que ele
se dedica na prática. E se eu aí jogar possuindo a previsão correta das
consequências das minhas operações, portanto, se eu jogar com sucesso,
realizarei todas as mais rigorosas exigências da moral
de Feuerbach e
me tornarei, além disso, um homem rico. \textbar{} Dito de outro modo: a moral
de Feuerbach está
talhada pela atual sociedade capitalista, por mais que ele próprio não
queira isso ou possa suspeitar.\textbar{}

Mas o amor! --- Sim, o amor é em toda parte e sempre o Deus da fascinação
que,
em Feuerbach,
deve ajudar a superar todas as dificuldades da vida prática --- e isto
numa sociedade que está cindida em classes com interesses diametralmente
contrapostos. Desse modo, desapareceu da filosofia, portanto, o último %talvez cortar o "portanto": "desse modo" já introduz a explicativa
resto do seu caráter revolucionário, e permanece apenas a velha
lenga"-lenga: amai"-vos uns aos outros, derramai"-vos sobre os braços uns
dos outros, sem diferença de gênero e estamento --- o devaneio da
reconciliação universal!

Em poucas palavras. Passa"-se pela teoria moral %"Em poucas palavras: passa"-se"?
de Feuerbach
assim como se passa pela de todos seus predecessores. Tal teoria está
talhada em todos os tempos, em todos os povos, em todas as situações, e,
precisamente por isso, ela nunca, e em parte alguma, é aplicável,
permanecendo diante do mundo efetivo tão impotente quanto o imperativo
categórico
de Kant.
Na realidade, cada classe, inclusive cada tipo profissional, tem sua
própria moral, e rompe com esta onde o pode fazer impunemente, e o amor,
que tudo deve unir, vem à luz do dia em guerras, conflitos, processos,
barulhos domésticos, divórcios e na máxima exploração possível de uns
pelos outros.

Mas como era possível que o impulso violento, dado
por Feuerbach, tenha
chegado a ele próprio de modo tão improdutivo? Simplesmente por
Feuerbach não
conseguir encontrar o caminho que parte do reino das abstrações,
mortalmente odiadas por ele mesmo, em direção à realidade efetiva viva.
Ele se agarrou com toda força à natureza e ao homem; mas, natureza e
homem permanecem para ele meras palavras. Ele não sabe nos dizer algo
determinado nem sobre a natureza efetiva, nem sobre o homem efetivo. No
entanto, somente se chega do homem abstrato
de Feuerbach aos
homens vivos efetivos caso estes sejam considerados agindo no interior
da história. Contra isso,
Feuerbach se
opõe e, por isso, o ano de 1848, que ele não compreendeu, significou %tanto "isso" como "portanto" configuram certa repetição. Talvez "dessa maneira"?
para ele apenas a ruptura definitiva com o mundo efetivo, o recolhimento
para a solidão. Por outro lado, a culpa disso se deve principalmente às
relações na Alemanha, que o degeneraram miseravelmente.

Mas o passo
que Feuerbach não
deu, precisava, ainda assim, ser dado; o culto ao homem abstrato, que
formava o núcleo da nova religião
de Feuerbach,
tinha de ser substituído pela ciência dos homens efetivos e de seu
desenvolvimento histórico. Esse desenvolvimento posterior, a partir da
posição
de Feuerbach, e para além dele, foi inaugurado por Marx, em 1845, em \emph{A sagrada
família.}

\quebra

\mbox{}
\vspace{2cm}

\addcontentsline{toc}{chapter}{Quarta parte \medskip}
\section{IV}

\noindent{}Strauss, Bauer, Stirner, Feuerbach,
eram esses os continuadores da filosofia
de Hegel,
na medida em que não abandonaram o solo
filosófico. Strauss,
depois da \emph{A vida de Jesus }\textbar{} e
da \emph{Dogmática},\endnote{Strauß, David Friedrich. \emph{Das Leben Jesu, kritisch bearbeitet}
  (1835). Erster Band. Tübingen: Verlag von C. F. Osiander, 1864
  / Strauß, David Friedrich. \emph{Die christliche Glaubensiehre in
  ihrer geschichtlichen Entwicklung und im Kampfe mit der modernen
  Wissenschaft}. Tübingen, Stuttgart, 1840--1841, 2 Bände. {[}\versal{N.\,T.}{]}} \textbar{} praticou
apenas ainda a beletrística filosófica e histórico"-eclesial \emph{à
la} Renan; Bauer só
realizou algo no âmbito da gênese do cristianismo, mas aqui também algo
significativo; Stirner permaneceu
uma curiosidade, mesmo depois de  Bakunin o ter combinado com Proudhon e
batizado essa combinação de ``anarquismo''; apenas
Feuerbach foi
significativo como filósofo. Mas não apenas a filosofia --- a ciência das
ciências, que supostamente pairava acima e vinculava todas as ciências
especiais --- permaneceu uma barreira intransponível para ele, um elemento
sagrado inviolável; ele permanece no meio do caminho, embaixo foi
materialista, em cima idealista; não liquidou criticamente
com Hegel,
simplesmente o deixou de lado como inutilizável, enquanto ele mesmo,
diante da riqueza enciclopédica do sistema
de Hegel,
não levou a cabo nada de positivo além de uma empolada religião do amor
e de uma pouco satisfatória, impotente, moral.

Da dissolução da escola hegeliana surgiu, porém, ainda outra orientação,
a única que efetivamente deu frutos e esta orientação vincula"-se
essencialmente ao nome de Marx.\endnote{Permitam"-me aqui um esclarecimento pessoal. Recentemente, mais de uma vez, aludiu"-se à minha participação nessa teoria e, portanto, eu não posso deixar de dizer aqui algumas poucas palavras que colocam um fim
  nesse ponto. Não posso negar que, antes e durante a minha colaboração
  de quarenta anos com Marx, tive certa participação autônoma, tanto na
  fundamentação como, nomeadamente, na elaboração da teoria. Mas a maior
  parte dos pensamentos fundamentais orientadores, particularmente no
  domínio econômico e histórico, e especialmente a precisa apreensão
  definitiva desse domínio, pertencem a Marx. Com aquilo que eu possa
  ter contribuído, Marx poderia --- excetuando, quando muito, algumas
  disciplinas especiais --- ter muito bem levado a cabo sem mim. O que
  Marx realizou, eu não teria levado a cabo. Marx estava mais acima, via
  mais longe, abarcava mais e mais rapidamente do que o resto de nós. \textbar{}
  Marx era um gênio, nós, no máximo, talentosos \textbar{}. Sem ele a teoria não
  seria hoje, nem de longe, aquilo que ela é. Ela tem, portanto, também
  com razão, seu nome. {[}\versal{N.\,E.}{]}}

A separação da filosofia hegeliana resultou aqui também de um regresso à
posição materialista. Isso significa que se decidiu apreender o mundo
efetivo\est\ --- natureza e história --- tal como ele próprio se apresenta a quem
quer que se aproxime dele sem ideias fixas
idealisticamente preconcebidas; decidiu"-se sacrificar impiedosamente
toda a ideia fixa idealista que não pudesse ser posta em consonância com
os fatos apreendidos em seu próprio nexo, e não um nexo fantástico
qualquer. Para além disso, o materialismo não significa absolutamente
nada. Só que aqui, pela primeira vez, a visão de mundo materialista foi
realmente levada a sério, de tal modo que foi consequentemente conduzida
em todas as áreas relevantes do saber --- pelo menos em seus traços
fundamentais.

Hegel não
foi simplesmente deixado de lado; pelo contrário, vinculou"-se ao seu
lado revolucionário acima desenvolvido, ao método dialético. Porém, esse
método, na sua forma hegeliana, era inutilizável.
Para Hegel,
a dialética é o autodesenvolvimento do conceito. O conceito absoluto não
existe apenas desde a eternidade --- não se sabe onde tal eternidade se localiza? ---, ele é também a
autêntica e viva alma de todo mundo existente. Ele desenvolve"-se para si
mesmo por meio de todos os estágios preliminares, amplamente tratados na
\emph{Lógica} e que estão todos contidos nele; depois, ele se ``externa
alienadamente'', convertendo"-se em natureza, onde, sem consciência de si
próprio, disfarçado de necessidade natural, sofre um novo
desenvolvimento e, por fim, explicita"-se novamente, no homem, na
consciência"-de"-si; essa consciência"-de"-si elabora a si mesma
novamente na história a partir do estado bruto, até
finalmente o conceito absoluto novamente voltar completamente a si
próprio na filosofia
de Hegel.
Para Hegel,
o desenvolvimento dialético que se explicita na natureza e na história ---
isto é, a conexão causal do ato de progressão do elemento inferior para
o superior que se impõe através de todos os movimentos em zigue"-zague e
retrocessos momentâneos --- é, portanto, apenas a imitação 
do automovimento do conceito que se processa desde a
eternidade, não se sabe onde, mas, em todo caso, independentemente de
qualquer cérebro humano\est\ pensante. Tratava"-se de eliminar essa distorção
ideológica. Voltamos a apreender materialistamente os conceitos da nossa
cabeça como imagens derivadas de coisas efetivas, em 
vez de apreender as coisas efetivas como imagens derivadas
deste ou daquele estágio do conceito absoluto. 
Reduziu"-se, com isso, a dialética à ciência das leis universais do
movimento, tanto do mundo exterior como do pensar humano --- duas séries
de leis que, segundo o movimento da coisa em questão, são
idênticas, mas que, na expressão, são diversas, na medida em que a
cabeça humana as pode aplicar com consciência, enquanto que elas, na
natureza e, até agora, em grande parte da história humana,
impõem"-se de modo inconsciente, na forma de necessidade exterior, em meio
a uma série sem fim de contingências aparentes. Com isso, porém, a
própria dialética do conceito tornava"-se apenas reflexo consciente do
movimento dialético do mundo efetivo, e assim a dialética
de Hegel era
posta acima da cabeça, ou, antes: da 
cabeça, sobre a qual estava, foi posta novamente sobre os pés. E esta
dialética materialista, que era há anos o nosso melhor meio de trabalho
e a nossa arma mais afiada, foi, de modo notável, novamente descoberta,
não apenas por nós, mas ainda, independentemente de nós e do próprio
Hegel,
por um trabalhador
alemão, Josef Dietzgen.\endnote{Cfr. ``\emph{Das Wesen der Kopfarbeit,
  von einem Handarbeiter}'' {[}A essência do trabalho intelectual, por
  um trabalhador manual{]} Hamburg, Meißner, 1869. {[}\versal{N.\,E.}{]}}

Deste modo, porém, o lado revolucionário da filosofia
de Hegel foi
novamente retomado e, ao mesmo tempo, libertado de suas dissimulações
idealistas que,
em Hegel,
haviam impedido a sua efetivação consequente. O grande pensamento
fundamental, segundo o qual não se deve apreender o mundo como um
complexo de \emph{coisas }prontas, mas como um complexo
de \emph{processos}, no qual as coisas, aparentemente estáveis, não
passam de imagens derivadas do pensamento delas na nossa cabeça, os
conceitos, que passam por uma ininterrupta transformação no devir e perecer,
na qual, em toda a aparente contingência, e apesar de todo o retrocesso
momentâneo, impõe"-se no fim um desenvolvimento progressivo --- este grande
pensamento fundamental, expressamente
desde Hegel,
transformou"-se na consciência habitual que já quase não encontra
contradição nessa universalidade. Mas, reconhecê"-lo na fraseologia e
executá"-lo na realidade efetiva, nos pormenores, em todo o domínio que
venha a ser investigado, são duas coisas diversas. Mas se na
investigação partimos sempre desse ponto de vista,
 a exigência de soluções definitivas e de
verdades eternas se encerra de uma vez por todas; sempre estamos
conscientes da necessária limitação de todo o conhecimento adquirido, do
seu condicionamento pelas circunstâncias em que foi adquirido; mas
também não nos deixemos mais impressionar pelas insuperáveis oposições da velha
metafísica, ainda sempre em voga, entre verdadeiro e falso, bom e mau,
idêntico e diverso, necessário e contingente; sabe"-se que essas
oposições só têm validade relativa, que aquilo que agora é considerado
como verdadeiro tem igualmente o seu lado falso, oculto, que aparecerá
mais tarde, assim como aquilo que agora é tomado como falso tem o seu
lado verdadeiro, devido ao fato de que, anteriormente, pode ter sido
tomado como verdadeiro; que o elemento afirmado como necessário é
composto de elementos evidentemente contingentes, e que o elemento
pretensamente contingente é a forma atrás da qual a necessidade se
esconde, e assim por diante.

O velho método de investigação e pensamento
que Hegel 
denomina ``metafísico'', que se ocupava preferencialmente com a
investigação das \emph{coisas }como elementos duradores, 
consistentes e dados, cujos restos ainda assombram
fortemente as nossas cabeças, teve, no seu tempo, uma grande
justificação histórica. As coisas tinham de ser investigadas primeiro,
antes que os processos pudessem ser investigados. Era necessário
primeiro saber o que uma coisa qualquer era, antes que fosse possível
perceber as transformações que se processavam nela. E assim era na
ciência da natureza. A velha metafísica, que tomava as coisas como
prontas, surgiu a partir de uma ciência da natureza que investigava as
coisas mortas e vivas como coisas prontas. Porém, quando essa
investigação se estendeu a tal ponto que tornou possível um progresso
decisivo, a transição para a investigação sistemática das transformações
que se processam com essas coisas na própria natureza, nesse momento,
também dobram no âmbito filosófico os sinos da morte da velha
metafísica. E, de fato, se a ciência da natureza até no final do século
passado foi, predominantemente, uma ciência \emph{coletora}, uma ciência
das coisas prontas, no nosso século, ela é essencialmente um
ciência \emph{ordenadora}, uma ciência dos processos, da origem e do
desenvolvimento dessas coisas e da conexão que vincula esses processos
naturais em um grande todo. A fisiologia, que investiga os processos no
organismo vegetal e animal, a embriologia, que trata do desenvolvimento
do organismo singular do embrião até a maturidade, a geologia, que
persegue a formação gradual da superfície terrestre, todas elas são
filhas do nosso século.

Sobretudo há, porém, três grandes descobertas que fizeram o nosso
conhecimento da conexão dos processos naturais avançar passos
gigantescos: em primeiro lugar, a descoberta da célula como unidade em
multiplicação \textbar{} e diferenciação \textbar{}, a partir da qual todo corpo vegetal
e animal se desenvolve, de tal modo que não apenas o desenvolvimento e
o crescimento de todos os organismos superiores são reconhecidos como
algo que se processa segundo uma única lei universal, \textbar{} mas também na
capacidade de transformação da célula está mostrado o caminho pelo qual
os organismos podem mudar a sua espécie e, assim, percorrer um
desenvolvimento mais do que individual. Em segundo lugar, \textbar{} a
transformação da energia que nos comprovou todas as chamadas forças que %MANTIVE
atuam, antes de tudo, na natureza inorgânica, a força mecânica e o seu
complemento, a chamada energia potencial, calor, radiação (luz, ou calor
radiante), eletricidade, magnetismo, energia química --- como diversas
formas de aparição do movimento universal que\est\ em determinadas
proporções transitam 
de uma para outra, de tal modo que, para a quantidade de uma que
desaparece volta a aparecer uma determinada quantidade de outra,
reduzindo assim todo o movimento da natureza a esse incessante processo
de transformação de uma forma em outra. Por fim, a prova desenvolvida por Darwin,
pela primeira vez nesse contexto,
de que o elemento duradouro dos produtos orgânicos da 
natureza que hoje nos rodeia, incluindo os homens, é o produto de um
longo processo de desenvolvimento a partir de alguns embriões
originalmente unicelulares, por sua vez provenientes, por meio químico, do protoplasma ou da albumina.

Graças a estas três grandes descobertas e aos restantes poderosos
progressos da ciência da natureza, chegamos agora ao ponto de poder
demonstrar a conexão entre os processos no interior da natureza, não
apenas nos domínios isolados, mas também dos domínios isolados entre si
e, assim, poder apresentar uma imagem nítida da conexão da natureza, em
uma forma aproximadamente sistemática, por meio dos fatos fornecidos
pela própria ciência empírica da natureza. Fornecer esta imagem do todo
era, anteriormente, a tarefa da chamada filosofia da natureza. Ela
somente era capaz disso na medida em que substituía as conexões
efetivas ainda desconhecidas por conexões ideais, fantásticas, que
completavam os fatos com imagens do pensamento, que preenchiam lacunas
efetivas na pura imaginação. Como não era possível ser diferente, ao
proceder assim, alcançou muitos pensamentos geniais, anteviu muitas
descobertas ulteriores, mas também trouxe à luz consideráveis absurdos.
Hoje, onde apenas é preciso apreender dialeticamente --- isto é, no sentido
da sua conexão própria --- os resultados da investigação da natureza para
chegar a um ``sistema da natureza'' suficiente para o nosso tempo, onde
o caráter dialético dessa conexão se impõe às cabeças metafisicamente
formadas\est\ dos naturalistas, mesmo contra a sua vontade, hoje, a filosofia
da natureza está definitivamente posta de lado. Qualquer tentativa de
ressuscitá"-la não seria apenas supérflua, \emph{seria um retrocesso.}

Porém, o que vale para a natureza, que também é reconhecido por meio
disso como um processo de desenvolvimento histórico, vale também para a
história da sociedade em todos os seus ramos e para a totalidade de
todas as ciências que se ocupam de coisas humanas (e divinas). Também
aqui a filosofia da história, do direito, da religião etc. consistia em
substituir a conexão efetiva a ser demonstrada nos acontecimentos
singulares por uma conexão feita na cabeça do 
filósofo, de tal modo que a história fosse apreendida como a efetivação gradual de ideias, tanto no todo
como em suas partes singulares --- e, naturalmente,
sempre apenas das ideias prediletas do próprio
filósofo. De acordo com isso, a história trabalhava aqui
inconscientemente, mas com necessidade de iniciar por uma finalidade
ideal, estabelecida de antemão, como por exemplo,
em Hegel,
pela efetivação da sua ideia absoluta, e a orientação inalterável por
essa ideia absoluta formava a conexão interna no interior dos
acontecimentos históricos singulares. No lugar da conexão efetiva, ainda
desconhecida, estabelecia"-se constitutivamente, assim,
uma nova providência misteriosa --- inconsciente ou que alcançava gradualmente consciência.
Aqui, justamente como no âmbito da natureza, o
que valia, portanto, era eliminar as conexões feitas artificialmente
pela adivinhação  das efetivas; uma tarefa que
definitivamente acaba por descobrir as leis universais do movimento que
se impõem como dominantes na história da sociedade humana.

Agora, porém, a história do desenvolvimento da sociedade mostra"-se
em um ponto essencialmente diverso da história do desenvolvimento da
natureza. Na natureza --- desde que deixemos de fora a consideração da
repercussão da ação do homem sobre a natureza --- há 
somente agenciamentos cegos, desprovidos de consciência, que geram
efeitos uns sobre os outros e em cuja interação recíproca a lei
universal torna"-se válida. De tudo o que acontece --- tanto das inúmeras
contingências aparentes, que se tronam visíveis na superfície, como dos
resultados que confirmam a regularidade no 
interior dessas contingências \mbox{---,} nada acontece enquanto uma finalidade
consciente fruto da vontade. Em contrapartida, na história da sociedade,
os agentes estão nitidamente dotados de consciência, são homens que se
propõem a agir com reflexão ou paixão, em determinadas finalidades; nada
acontece sem propósito consciente, sem uma finalidade que seja fruto da
vontade. Mas essa diferença, por mais importante que seja para a
investigação histórica, especialmente de épocas e eventos, não altera em
nada o fato de que o curso da história é regido por leis internas
universais. Afinal, também aqui, apesar das finalidades que são frutos
conscientes da vontade de todos os indivíduos singulares, aparentemente
rege sobre superfície, em geral, a contingência. Apenas raramente
acontece o elemento que é fruto da vontade; na maioria dos casos, as
múltiplas finalidades que são frutos da vontade entrecruzam"-se e se
contradizem, ou essas mesmas finalidades são, a princípio, irrealizáveis
ou os meios são insuficientes. Assim, as colisões das inúmeras vontades
singulares e ações singulares no âmbito histórico proporcionam um estado
que é totalmente análogo ao que domina na natureza desprovida de
consciência. As finalidades das ações são frutos da vontade, mas os
resultados que efetivamente decorrem das ações não são frutos da vontade,
ou na medida em que, antes de tudo, parecem no entanto corresponder à
finalidade que é fruto da vontade, têm no fim consequências totalmente
diversas das pretendidas. Os acontecimentos históricos singulares aparecem,
no geral, como se fossem em todo caso dominados pela
contingência. Mas onde a contingência joga seu jogo sobre a superfície,
ela é sempre dominada por leis internas ocultas e a questão é apenas
descobrir essas leis.

Os homens fazem a sua história, aconteça ela como acontecer, na medida
em que cada um persegue conscientemente as finalidades que eles mesmos %MANTIVE
querem, e a resultante destas várias vontades que atuam em direções
diversas e da sua influência múltipla sobre o mundo exterior é
justamente a história. Depende, portanto, do que os muitos indivíduos
querem. A vontade é determinada por paixão ou reflexão. Mas as alavancas
que, por sua vez, determinam imediatamente a paixão ou reflexão, são de
tipos muito diversos. Em parte podem ser finalidades exteriores, em
parte \emph{fundamentos ideais do movimento}, 
ambição, ``entusiasmo pela verdade e pela justiça'', ódio
pessoal, ou também caprichos puramente individuais de toda a espécie.
Mas, por outro lado, vimos que as várias vontades individuais ativas na
história, na maioria dos casos, produzem resultados totalmente
diferentes dos pretendidos --- \emph{muitas vezes contrapostos} --- e que,
portanto, para o resultado do todo, seus fundamentos de movimento têm um
significado subordinado. Por outro lado, é possível questionar ainda
mais: quais forças impulsionadoras estão novamente por detrás destes
fundamentos do movimento, que causas históricas transformam, na cabeça
dos agentes, esses fundamentos de movimento?

O velho materialismo nunca se colocou essa questão. Sua concepção da
história, se é que ele tem uma, é, portanto, também essencialmente
pragmática, adjudica tudo segundo os motivos da ação, divide os homens
que agem historicamente em nobres e não nobres e então descobre, em
regra, que os nobres são os enganados e os não
nobres os vencedores; disso resulta para o velho materialismo que do
estudo da história nada de muito edificante se explicita e, para nós, no
âmbito da história, o velho materialismo se tornou infiel a si próprio,
pois toma as forças motrizes ideais aí atuantes como causas últimas, em
vez de investigar aquilo que está por detrás delas, quais são as forças
motrizes dessas forças motrizes. Não é nisso que se estabelece a
inconsequência, de reconhecer forças motrizes \emph{ideais}, mas no fato
de que a partir dessas não se investigue mais a fundo as causas de seu
movimento. A filosofia da história, em contrapartida, justamente como é
defendida
por Hegel,
reconhece que os fundamentos ostensivos, e também os efetivamente ativos
do movimento dos homens que agem historicamente, não são, de modo algum,
as causas últimas dos acontecimentos históricos; reconhece que
por detrás desses fundamentos do movimento encontram"-se outras potências
móveis, que é preciso investigar; mas ela não procura 
essas potências na própria história. Pelo contrário, importa"-as de
fora, da ideologia filosófica para o interior da história. Em vez de
explicar a história da Grécia Antiga a partir da sua conexão própria,
interna, Hegel afirma,
por exemplo, simplesmente que ela não mais é do que a elaboração das
``cofigurações da bela individualidade'', a realização da ``obra de
arte'' enquanto tal. Quando convém, ele diz algo de muito belo e
profundo sobre a Grécia Antiga, mas isso não impede que nós hoje já não
nos contentemos com tal explicação, que não passa de uma mera expressão
idiomática.

Quando se trata, portanto, de investigar as potências impulsionadoras --- conscientes ou inconscientes e, de fato, frequentemente inconscientes --- que estão por detrás dos fundamentos dos movimentos dos
homens que agem historicamente, potências estas que constituem propriamente as forças 
motrizes últimas da história, não se pode levar em conta apenas os fundamentos
de movimento dos indivíduos, mesmo considerando aqueles que agem de modo eminente e põem em movimento grandes massas, povos inteiros e,
em cada povo, por sua vez, classes inteiras. Tampouco se pode considerar apenas as ações que se dão
por uma explosão momentânea passageira, fogo de palha que queima
rapidamente, mas a ação duradoura que se alastra em uma grande
transformação histórica. O único caminho que nos pode colocar no rastro das leis que dominam a história, tanto em geral como em períodos e regiões singulares, é averiguar, como fundamentos conscientes de movimento, as causas motrizes que aqui se
refletem clara ou obscuramente, imediatamente ou na forma ideológica,
por vezes sacralizada na cabeça das massas e de seus condutores, os chamados grandes homens. Tudo
o que põe os homens em movimento tem de passar por sua cabeça; mas que
configuração toma nessa cabeça, depende muito das circunstâncias. Os
trabalhadores, sob nenhuma circunstância, reconciliaram"-se com o
maquinário fabril capitalista, mesmo que não mais o tenham simplesmente
quebrado em pedaços, como ainda em 1848 no Reno.\endnote{Possível alusão aos ocorridos na noite de 16 e 17 de Março de 1848 na
  fábrica de produção de ferro fundido da cidade de Solinger, onde uma
  rebelião dos trabalhadores destruiu quatro oficinas de fundição e uma
  máquina a vapor. {[}\versal{N.\,T.}{]}}

Porém, enquanto em todos os períodos anteriores era quase impossível a
investigação destas causas impulsionadoras da história --- devido às
complicadas e encobertas conexões com os seus efeitos --- o nosso período
atual simplificou tanto essas conexões que foi possível resolver o
enigma. Desde a efetivação da grande indústria, 
portanto pelo menos desde a paz europeia de 1815, não era mais segredo
para homem nenhum em Inglaterra que lá toda a luta política girava em
torno das pretensões à dominação de duas classes: a aristocracia
possuidora de terras (\emph{landed
aristocracy}) e a burguesia (\emph{middle
class}). Na França, a consciência do mesmo
fato foi obtida com o regresso dos Bourbon; os historiadores da época
da Restauração, de Thierry a Guizot, Mignet e Thiers,
por toda a parte falam disso como a chave para a compreensão da
história francesa desde a Idade Média. E, desde 1830, em ambos os
países, a classe dos trabalhadores, o proletariado, foi reconhecida como
a terceira força por essa dominação. As relações se 
simplificaram tanto que era preciso fechar os olhos propositalmente para
não ver na luta dessas três grandes classes e, no conflito de seus
interesses, a força impulsionadora da história moderna --- pelo menos, nos
dois países mais avançados.\est\

Como haviam, porém, surgido essas classes? Se à primeira vista ainda
se podia atribuir à grande propriedade fundiária, antes feudal, uma
origem --- pelo menos a princípio --- a partir de causas políticas,
de uma apropriação violenta, isso não dizia respeito à burguesia
e ao proletariado. A origem e desenvolvimento de duas grandes classes
eram aqui claras e palpáveis a partir de causas puramente econômicas. E %MANTIVE
era igualmente claro que, na luta entre possuidores de terras e
burguesia, não menos do que na luta entre burguesia e proletariado, o
que estava em disputa, em primeiro lugar, eram interesses econômicos,
para cuja efetivação o poder político devia servir de mero meio.
Burguesia e proletariado haviam surgido ambos em decorrência de uma
transformação das relações econômicas, ou, dito de modo mais exato, do modo
de produção. A passagem, primeiro das corporações de ofício artesanais
para a manufatura, e depois da manufatura para a grande indústria com o
emprego do vapor e das máquinas, havia desenvolvido estas duas classes.
Em certo estágio, as novas forças de produção postas em movimento pela
burguesia --- antes de tudo, a divisão do trabalho e a reunião de vários
trabalhadores seccionais em uma manufatura conjunta --- e as condições de
troca e necessidades de troca por ela desenvolvidas tornaram"-se
incompatíveis com a ordem da produção existente, historicamente
transmitida e consagrada pela lei, isto é, com os privilégios
corporativos e incontáveis outros privilégios pessoais e locais (que,
para os estamentos não privilegiados, eram igualmente muitos grilhões)
da constituição da sociedade feudal. As forças de produção,
representadas pela burguesia, rebelaram"-se contra a ordem de produção
representada pelos senhores de terras feudais e mestres"-artesões; o
resultado é conhecido: os grilhões feudais foram quebrados, de modo gradual na Inglaterra, com um só golpe na França; na Alemanha ainda não se acabou com eles. Mas, assim como a manufatura entrou em conflito com a ordem feudal de produção em um estágio determinado de seu desenvolvimento, também agora a grande indústria entrou já em conflito com a
ordem burguesa de produção posta no lugar daquela. Mantida por esta
ordem, pelas estreitas barreiras do modo de produção capitalista, ela
produz, por um lado, uma proletarização sempre crescente de toda a
grande massa do povo, e, por outro lado, uma massa cada vez maior de
produtos que não podem ser vendidos. Sobreprodução e miséria das massas,
cada uma a causa da outra, é essa a contradição absurda na qual essa
ordem desemboca e que demanda necessariamente retirar os grilhões das
forças produtivas por meio da mudança do modo de produção.

Na história moderna, pelo menos, está assim demonstrado que todas as
lutas políticas são lutas de classes, e que todas são lutas por
emancipação das classes, apesar da sua forma necessariamente política ---
afinal, toda luta de classes é uma luta política ---, e que giram, no fim, em
torno da emancipação \emph{econômica. }Pelo menos aqui, o Estado, a
ordem política, é o elemento subordinado; a sociedade civil"-burguesa, o
reino das inter"-relações econômicas, é o elemento decisivo. A visão
tradicional, também acatada por Hegel, via no Estado o elemento
determinante, na sociedade civil"-burguesa o elemento por ele
determinado. A aparência corresponde a isso. Assim como no homem
singular todas as forças impulsionadoras das suas ações têm de passar
pela cabeça dele, têm de se transformar em fundamentos do movimento da
sua vontade para levá"-lo a agir, também todas as necessidades da
sociedade civil"-burguesa --- qualquer que seja a classe que no momento
a domina --- têm de passar pela vontade do Estado para obter validade
universal na forma de leis. Esse é o lado formal da coisa, que se
compreende por si; mas a questão é qual o conteúdo que esta vontade
apenas formal --- tanto do indivíduo singular como do Estado --- tem, e de
onde vem esse conteúdo, por que é precisamente este e não outro que é
fruto da vontade. E se perguntarmos por isso, verificamos que, na
história moderna, a vontade do Estado, em geral, é determinada pelas
carências mutáveis da sociedade civil"-burguesa, pelo predomínio dessa ou
daquela classe, e, em última instância, pelo desenvolvimento das forças
produtivas e das relações de troca.

Mas, se já na nossa época moderna, com os seus gigantescos meios de
produção e intercâmbio, o Estado não é um domínio autônomo com
desenvolvimento autônomo --- pelo contrário, tanto sua existência como o
seu desenvolvimento precisam ser esclarecidos, em última instância, a
partir das condições econômicas de vida da sociedade ---, isto tem que ser
válido, ainda muito mais, para todos as épocas anteriores, em que a
produção da vida material dos homens ainda não era empreendida com esses
recursos abundantes, e onde, portanto, a necessidade dessa produção tinha
de exercer uma dominação ainda maior sobre os homens. Se, ainda
hoje, na época da grande indústria e das estradas de ferro, o Estado é em geral
reflexo, em forma vinculativa, das carências 
econômicas da classe que domina a produção, então isso precisaria ser
assim, ainda muito mais, em uma época na qual uma geração de homens
tinha de consagrar uma parte muito maior do seu tempo total de vida à
satisfação das suas carências materiais, logo estando muito mais
dependente delas do que nós hoje estamos. A investigação da história de
épocas anteriores, desde que seriamente comprometida com esse lado,
confirma isso na mais rica medida; porém, naturalmente, isso não poderá
ser tratado aqui.

Se o Estado e o direito do Estado são determinados pelas relações
econômicas, também o é, evidentemente, o direito privado, o que
essencialmente apenas sanciona, sob as circunstâncias dadas, as
inter"-relações econômicas normais existentes entre os indivíduos. A
forma na qual isso acontece pode, porém, ser muito diversa. É possível,
como aconteceu na Inglaterra, em consonância com todo o desenvolvimento
nacional, que formas do velho direito feudal, em grande parte, sejam
conservadas e lhes sejam dadas um conteúdo burguês, imputando
diretamente ao nome feudal um sentido burguês;\est\ mas também é possível,
como na Europa Ocidental continental, tomar por base o primeiro direito
mundial de uma sociedade produtora de mercadorias, o romano, com a sua
insuperavelmente precisa elaboração de todas as inter"-relações jurídicas
essenciais dos possuidores simples de mercadorias (comprador e vendedor,
devedor e credor, contrato, obrigação, etc.). Com isso, para utilidade e
proveito de uma sociedade ainda pequeno"-burguesa e semifeudal, ou se
pode simplesmente reduzi"-lo ao patamar dessa sociedade por meio da
\emph{práxis} jurídica (direito comum), ou então, com a ajuda de
juristas pretensamente esclarecidos, moralistas, pode"-se elaborá"-lo num
código à parte, correspondente a esse estado da sociedade, código esse
que, nessas circunstâncias, será também juridicamente perverso
(\emph{Landrecht} prussiano);\endnote{No contexto do processo de codificação europeia pós"-revolução francesa, que tem como marco o código civil napoleônico de 1804, o
  \emph{Código Geral da Prússia} (\emph{Allgemeine Preußische
  Landrecht)} de 1794, com seus 19.000 parágrafos, desenha na estrutura
  legal e social da Alemanha muitos aspectos dos compromissos de classes
  que caracterizam os efeitos concretos da revolução de 1789 fora da
  França, já que nele é possível vislumbrar com muita clareza o
  desdobramento interno e constitutivo da inter"-relação entre Estado na
  forma burguesa e sociedade estamental pré"-burguesa (Sobre isso:
  \versal{KOSELELCK}, R. \emph{Preußen zwischen Reform und Revolution,
  Allgemeines Landrecht, Verwaltung und soziale Bewegung von 1791 bis
  1848}). A distorção que essa inter"-relação impõe é certamente um
  aspecto da perversidade jurídica mencionada por Engels. {[}\versal{N.\,T.}{]}} com isso é possível também, após uma grande revolução burguesa, tendo como
base justamente esse direito romano, elaborar um código da sociedade
burguesa tão clássico quanto o \emph{Code
civil} francês. Se, portanto, as
determinações jurídicas burguesas apenas expressam as condições
econômicas de vida da sociedade em forma jurídica, isso pode ocorrer, a
depender das circunstâncias, de modo satisfatório ou
perverso.

No Estado, apresenta"-se para nós a primeira potência 
ideológica sobre o homem. A sociedade cria para si um órgão para a
salvaguarda dos seus interesses comuns diante de ataques internos e
externos. Esse órgão é o poder do Estado. Assim que  
surge, tal órgão se autonomiza diante da sociedade, e isso, justamente,
quanto mais ele se torna órgão de uma classe determinada, um órgão que
valida diretamente a dominação dessa classe. A luta da classe oprimida
contra a classe dominante torna"-se necessariamente uma luta política;
uma luta, antes de tudo, contra a dominação política desta classe; a
consciência da conexão dessa luta política com os suas bases
econômicas torna"-se mais indeterminada  
e pode se perder totalmente. Onde os envolvidos não estão em situação de estabelecer essa conexão, isso quase sempre acontece por meio dos historiadores. Entre as velhas fontes acerca das
lutas no interior da república romana, apenas Apiano\endnote{Apiano de Alexandria, autor da \emph{História romana} (\emph{Ῥωμαϊκά -- Romaica}) escrita em 24 volumes. {[}\versal{N.\,T.}{]}} nos diz clara e distintamente do que definitivamente se tratava: justamente da
propriedade fundiária.

O Estado, porém, uma vez que se torna um poder autônomo  
diante da sociedade, logo em seguida produz uma ideologia ulterior. Nos
políticos de profissão, nos teóricos do direito do Estado e nos juristas
do direito privado, perde"-se, sobretudo, a própria conexão com os
fatos econômicos. Porque em cada caso individual os fatos econômicos têm
de tomar a forma de motivos jurídicos para serem sancionados na forma de
lei, e porque, ao fazê"-lo, é preciso também evidentemente considerar
todo o sistema jurídico já em vigor; por isso, a forma jurídica deve
aqui ser tudo e o conteúdo econômico nada. Direito do Estado e direito
privado são tratados como domínios autônomos, que têm o seu
desenvolvimento histórico independente, que são capazes em si mesmos de
uma exposição sistemática e a necessitam através da consequente extinção
de todas as suas contradições internas.

Ideologias ainda mais superiores, isto é, ainda mais afastadas do
fundamento econômico, material, tomam a forma da filosofia e da
religião. Aqui, a conexão das representações com as suas condições
materiais de existência torna"-se sempre mais complexa, sempre mais
obscurecida por elos intermediários. Mas ela existe. Assim como toda a
época do Renascimento, desde os meados do século \versal{XV}, foi essencialmente
um produto das cidades --- portanto, da burguesia ---, também o foi a filosofia
desde então renascida; o seu conteúdo era essencialmente apenas a
expressão filosófica do pensamento correspondente ao desenvolvimento da
pequena e média burguesia em grande burguesia. Isso se explicita
claramente nos ingleses e franceses do século passado que, em muitos
casos, tanto eram filósofos\est\ como economistas políticos, bem como na escola
hegeliana, como já demonstramos acima.

Passemos, entretanto, ainda que apenas brevemente, para a religião, já
que essa se encontra o mais afastada possível da vida material e parece
ser a mais alheia possível. A religião surgiu em uma época
originariamente bastante silvestre, a partir 
de igualmente silvestres, equivocadas, representações dos homens
sobre a sua própria natureza e a natureza exterior circundante. Toda a
ideologia, porém, desde que ela exista, desenvolve"-se em conexão com o
material da representação dado, dá a ele uma forma ulterior; caso
contrário, ela não seria ideologia, isto é, ocupação com pensamentos
como essencialidades autônomas, desenvolvendo"-se independentemente,
submetidas apenas às suas próprias leis. O fato de as condições
materiais de vida dos homens, em cuja cabeça esse processo de pensamento
avança, determinarem definitivamente o curso desse processo, permanece
necessariamente inconsciente para esses homens, afinal, caso contrário,
toda a ideologia chegaria ao fim. Essas representações religiosas originárias, comuns a todo grupo de povos aparentados na maior parte dos casos, desenvolvem"-se portanto após a separação do grupo, de modo
particular em cada povo, dependendo das condições de vida particulares,
e esse processo, para uma série de grupos de povos --- expressamente para
os arianos (chamados indo"-europeus) ---, está demonstrado detalhadamente
pela mitologia comparada. Os Deuses assim elaborados por cada povo eram
Deuses nacionais, cujo reino não ia além do território nacional a ser
protegido por eles, para além de cujas fronteiras outros Deuses detinham
incontestavelmente a última palavra. Eles somente podiam sobreviver na
representação enquanto a nação existisse; caíam com a sua decadência. O
império mundial romano, cujas condições econômicas de surgimento não
podemos investigar aqui, trouxe à tona a decadência das antigas
nacionalidades. Os antigos Deuses nacionais entraram em declínio, mesmo
os Deuses romanos que apenas estavam talhados para o estreito círculo da
cidade de Roma; a necessidade de completar o império mundial com uma
religião mundial apareceu claramente nas tentativas de
erguer altares, ao lado dos nativos de Roma, a todos e quaisquer Deuses
estrangeiros respeitáveis. Mas uma nova religião mundial não se faz
dessa maneira, por decretos imperiais. A nova religião mundial, o
cristianismo, já tinha surgido em silêncio, a partir de uma mistura de
teologia oriental generalizada, nomeadamente judaica, e de filosofia
grega vulgarizada, nomeadamente estoica. O quanto ela parecia
originária, temos ainda que pesquisar exaustivamente, pois a sua
cofiguração oficial que nos foi transmitida é apenas aquela em que se
tornou religião de Estado, e que para esse fim foi adaptada pelo Concílio
de Niceia.\endnote{Primeiro
  concílio ecumênico de bispos cristãos convocado pelo Imperador Romano
  Constantino \versal{I} em 325, na cidade de Nicéia (atualmente, província de
  Bursa na Turquia). Organizado na forma do senado romano, o concilio
  discutiu, entre outras questões, sobre a divindade da figura de Cristo
  e sua relação com o Deus"-Pai. Do fundo da disputa político"-religiosa
  entre Alexandre \versal{I} e Ário surge a questão da doutrina da revelação. Um
  dos resultados é a tentativa de estabelecimento de uma unidade do
  credo cristão em oposição às visões que deveriam ser consideradas
  heréticas. Na passagem, Engels se refere justamente a esse aspecto
  quando fala da ``mistura de teologia oriental generalizada,
  nomeadamente judaica, e de filosofia grega vulgarizada, nomeadamente
  estoica'', assim como remete à questão da estrutura de organização
  institucional da Igreja, tendo como resultado principal a promulgação
  da primeira forma da lei canônica, uma lei que, naquela época,
  organizou tanto a vida espiritual como material. {[}\versal{N.\,T.}{]}} É suficiente o fato
de que apenas 250 anos depois tenha se tornado religião de Estado para
demonstrar que era a religião adequada às circunstâncias da época. Na
Idade Média, na exata medida em que o feudalismo se desenvolvia,
o cristianismo transformou"-se na religião que correspondia a ele, com hierarquia feudal
correspondente. E quando a burguesia apareceu, desenvolveu"-se, em
oposição ao catolicismo feudal, a heresia protestante, primeiro, no sul
da França, entre os Albigenses,\endnote{Os Albigenses
  eram membros de um movimento religioso herético que se estabeleceu
  desde o fim do século \versal{XII} no sul da França, em torno da cidade de
  Albi. Composto em grande parte por artesãos e comerciantes das
  cidades, além de alguns nobres, todos contrários, entre outras coisas,
  ao controle de terra pela Igreja e a hierarquia eclesiástica. {[}\versal{N.\,T.}{]}} na
época de maior florescimento das cidades dessa região. A Idade Média
tinha anexado à teologia todas as restantes formas da ideologia:
filosofia, política, prática do 
direito.\endnote{Engels já havia indicado isso em um artigo publicado na \emph{Nova
  Gazeta Renana} de 1850: \versal{ENGELS}, F. \emph{Der Deutsche Bauernkrieg}
  {[}A guerra camponesa alemã{]}. \versal{IN}: \versal{MEGA}, \versal{I}.10. {[}\versal{N.\,T.}{]}} Tinha"-as
tornado subdivisões da teologia. Obrigou, portanto, todo o movimento
social e político a assumir uma forma teológica; os ânimos das massas,
alimentadas como animais exclusivamente com religião, tiveram que
mostrar seus próprios interesses em disfarces religiosos para criar uma
grande tempestade. \textbar{} E assim como a burguesia criou desde o início um
apêndice de plebeus urbanos não reconhecidos por nenhum estamento,
trabalhadores que recebiam por uma jornada diária e prestadores de
serviços de todos os tipos, precursores do proletariado tardio, \textbar{} a
heresia se dividiu, desde o início, em um herege moderado"-burguês e um
revolucionário"-plebeu, também abominado pelos hereges burgueses.

O caráter inexterminável da heresia protestante correspondia à
invencibilidade da burguesia ascendente; quando a burguesia era forte o
suficiente, a sua luta com a nobreza feudal, até então predominantemente local, começou a tomar dimensões nacionais. A primeira grande ação
aconteceu na Alemanha --- a chamada Reforma. A burguesia não era
suficientemente forte, tampouco estava suficientemente desenvolvida,
para conseguir unificar sob a sua bandeira os estamentos rebeldes
restantes --- os plebeus das cidades, a baixa nobreza e os camponeses, no %MANTIVE
campo. Primeiro, a nobreza foi abatida; os camponeses levantaram"-se em
uma insurreição que formou o ponto culminante de todo este movimento
revolucionário; as cidades os abandonaram e, assim, a revolução sucumbiu
aos exércitos dos príncipes da terra, que embolsaram todos os ganhos. A
partir de então, a Alemanha desaparece por três séculos da série de
países que intervêm na história de forma autônoma. Mas, ao lado do
alemão Lutero, surgiu o
francês Calvino;
com a fina precisão francesa, trouxe para primeiro plano o caráter
burguês da Reforma, republicanizou e democratizou a Igreja. Enquanto a
Reforma luterana estagnava e levava a Alemanha à ruína, a calvinista
servia de bandeira aos republicanos em Genebra, na Holanda, na Escócia,
libertava a Holanda da Espanha e do Império alemão\endnote{Período em que a Holanda fez parte do Sacro Império Romano"-Germânico
  entre 1477 e 1555. {[}\versal{N.\,T.}{]}} e fornecia o
disfarce ideológico ao segundo ato da revolução burguesa que estava em
processo na Inglaterra. O calvinismo comprovava"-se aqui como o autêntico
disfarce religioso dos interesses da burguesia daquela época e, por
isso, não foi plenamente reconhecido quando a revolução de 1689 chegou a
um fim por um compromisso de uma parte da nobreza com os burgueses.\endnote{Referência à revolução gloriosa de 1689. No
  capítulo sobre a \emph{Acumulação primitiva} (\emph{originária}), ao
  mencionar a revolução gloriosa como um dos impulsos da dominação
  violenta constitutiva da expropriação do trabalhador da propriedade
  rural, Marx indica justamente o caráter histórico fundamental desse
  compromisso entre aristocracia rural e capitalistas, ambos
  ``extratores de mais"-valor'': ``A \emph{Glorious Revolution}, com
  Guilherme \versal{III} de Orange, levou ao poder extratores do mais"-valor
  fundiários e capitalistas. Estes inauguraram, em escala colossal, a
  nova Era de roubo de domínios do Estado, até então realizado em
  proporções apenas modestas. Essas terras foram presenteadas, vendidas
  a preços irrisórios ou, mediante usurpação direta, anexadas a
  propriedades privadas. Tudo isso ocorreu sem nenhuma observância da
  etiqueta legal''. (\versal{MARX}, K. \emph{Das Kapital}. Erster Band. 39 Aufl.
  2008, p. 751). {[}\versal{N.\,T.}{]}} A Igreja de Estado
inglesa foi restabelecida, não em sua configuração anterior, enquanto
catolicismo com o rei como papa, mas fortemente calvinizada. A velha
Igreja de Estado tinha celebrado o alegre domingo católico e combatido o
maçante domingo calvinista; este foi introduzido pela nova Igreja de
Estado aburguesada, e ainda hoje ele embeleza a Inglaterra.

Na França, a minoria calvinista foi oprimida, catolicizada ou expulsa em
1685;\endnote{Fim do Édito de
  Nantes em 1685 que, desde 1598, garantia aos calvinistas franceses
  tolerância religiosa. {[}\versal{N.\,T.}{]}} mas, para que isso serviu? Já nessa época, o
livre"-pensador Pierre
Bayle estava no trabalho e, em 1694,
nascia Voltaire.
A medida violenta
de Luís \versal{XIV} apenas facilitou à burguesia francesa fazer sua
revolução sob a forma não religiosa, exclusivamente política, a única
apropriada à burguesia desenvolvida. Em vez de protestantes, foram
livres"-pensadores que se sentaram nas assembleias nacionais. O
cristianismo havia entrado por meio disso em seu último estágio. Tinha
se tornado incapaz de servir a qualquer classe progressiva como disfarce
ideológico das suas aspirações; tornou"-se cada vez mais posse exclusiva
das classes dominantes e essas o aplicavam como mero meio de governo
pelo qual as classes inferiores eram mantidas dentro das barreiras. Com
isso, então, cada uma das diversas classes utiliza a própria religião
que lhe corresponde: a aristocracia rural possuidora de 
terras utiliza o jesuitismo católico ou a ortodoxia protestante; o
burguês liberal e radical, o racionalismo; e não faz nenhuma diferença
se os próprios senhores acreditam nas respectivas religiões ou não.

Vemos, portanto, que a religião, uma vez formada, contém sempre uma
matéria tradicional, assim como que, em todos os âmbitos ideológicos, a
tradição é uma grande força conservadora. Mas as transformações que se
processam nessa matéria resultam das relações das classes, portanto,
das relações econômicas dos homens que empreendem essas transformações.
E isso é o suficiente aqui.

No exposto, somente é possível oferecer um esboço geral da concepção de
história de Marx, no máximo mais algumas ilustrações. A prova deve\est\ ser
fornecida na própria história, e posso dizer que já foi suficientemente
fornecida em outros escritos. Essa concepção põe fim, porém, à filosofia
no domínio da história, assim como a concepção dialética da natureza
torna tão desnecessária quanto impossível toda a filosofia da natureza.
Não se trata mais de conceber conexões na cabeça, mas descobri"-las nos
fatos. A única coisa que resta para a filosofia expulsa da natureza e da
história é o domínio do pensamento puro, na medida em que resta algo: a
doutrina das leis do próprio processo do pensamento, a lógica e a
dialética.

\asterisc

Com a revolução de 1848, a Alemanha ``culta'' recusou a teoria e, do
alto, desceu para o chão da \emph{práxis}. O pequeno ofício e a manufatura, que
se baseavam no trabalho manual, foram substituídos por uma grande
indústria efetiva; a Alemanha voltou a aparecer no mercado mundial; o
novo império pequeno"-alemão eliminou pelo
menos as mais gritantes inconveniências que a medíocre divisão em pequenos
Estados, os restos 
do feudalismo e a economia burocrática haviam deixado para o caminho
desse desenvolvimento. Porém, na mesma medida em que a especulação se mudava do gabinete de estudo filosófico para instituir o seu templo na bolsa de valores, perdia-se para a Alemanha culta aquele grande sentido teórico que havia sido a glória da Alemanha durante o tempo da sua mais profunda degradação política: o sentido de uma pesquisa puramente científica, independentemente se o resultado alcançado fosse aproveitável na prática ou não, ou contrário
às regras. De fato, a ciência da natureza oficial alemã, justamente no
âmbito da investigação singular, manteve"-se à altura da época, mas a
revista americana \emph{Science} já observa, com razão, que os
progressos decisivos\est\ no âmbito das grandes conexões entre fatos
singulares, da sua generalização em leis, são agora feitos muito mais na
Inglaterra do que, como anteriormente, na Alemanha. E, no âmbito das
ciências históricas, incluindo a filosofia, desapareceu, junto à
filosofia clássica, com maior razão, o velho 
espírito teórico"-brutal:\endnote{Aqui é possível remeter a uma conhecida passagem de Marx, onde defende a necessidade de uma crítica teórica sem restrições (\textit{rücksichtlos}), aquela que não olha para trás (\textit{rück"-sicht}) impulsionada pela investigação da estrutura de poder do presente. O contexto é uma carta de Marx a Ruge, de 1843, na qual reflete sobre a situação política na Alemanha do período e a função do crítico. A crítica deve ser ``brutal (\textit{rücksichtlos}) tanto no sentido de [\ldots{}] não pode temer os seus próprios resultados quando no sentido de que não pode temer os conflitos com os poderes estabelecidos''. \versal{MARX}, K. Deutsche Französische Jahrbücher 1.
  Doppellieferund, Februar, 1844. In: \versal{MARX}, K.; \versal{ENGELS}, F. \emph{Werke}. Band 1. Berlin/\versal{DDR}: Dietz Verlag, 1976, p. 344. Comparando a situação da filosofia alemã do período clássico com a nova função da filosofia pós"-1870, é que Engels pode afirmar, em um sentido próximo ao de Marx em 1843, o fim do ``velho espírito teórico"-brutal'' da crítica na Alemanha, completamente absorvido pela estrutura de poder do Estado cada vez mais burguês, e seu único caminho agora possível: a consciência crítica da classe trabalhadora.  {[}\versal{N.\,T.}{]}} tomam seu lugar o ecletismo desprovido de pensamento, a preocupação angustiada com carreiras e rendimentos descendo até ao arrivismo mais ordinário. Os representantes oficiais desta ciência tornaram"-se ideólogos
não encobertos da burguesia e do Estado existente --- mas em um tempo em
que ambos estão em oposição aberta à classe trabalhadora.

E é apenas na classe trabalhadora que continua a subsistir intacto o
sentido teórico alemão. Aqui ele não pode ser exterminado; aqui não têm
lugar as preocupações com a carreira, as pequenas atitudes sorrateiras
para tirar proveito, a benevolente proteção vinda de cima; pelo
contrário, quanto mais sem restrições e 
imparcialmente a ciência procede, tanto mais se encontra em consonância
com os interesses e as aspirações dos trabalhadores. A nova orientação,
que reconheceu na história do desenvolvimento do trabalho a chave para a
compreensão de toda história da sociedade, voltou"-se, antes de tudo,
preferencialmente à classe trabalhadora e encontrou aí a receptividade
que não procurou, tampouco esperava, na ciência oficial. O movimento
trabalhador alemão é o herdeiro da filosofia clássica alemã.

\quebra

