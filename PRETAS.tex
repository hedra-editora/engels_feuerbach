\textbf{Friedrich Engels} (Barmen, 1820--Londres, 1895), revolucionário alemão,
conhecido por sua profícua parceria com Karl Marx, com quem divide a autoria do célebre \textit{Manifesto comunista}. Filho mais velho
de um industrial da tecelagem, Engels foi um dos primeiros teóricos modernos a escrever sobre as condições de vida da classe trabalhadora, considerada o produto mais autêntico da modernidade capitalista. Em Bruxelas, auxiliou na formação da Liga
dos Comunistas, em 1847. Além de colaborar com Karl Marx na elaboração de \textit{O Capital}, algumas de suas obras próprias foram, posteriormente, consideradas fundantes do marxismo, como o \textit{Anti-Duhring} e \textit{Ludwig Feuerbach e o fim da filosofia clássica alemã}. Com a morte de Marx, trabalhou na preparação e na publicação dos dois últimos volumes de \textit{O Capital}.

\textbf{Ludwig Feuerbach e o fim da filosofia clássica alemã}, ou \emph{o
ponto de saída da filosofia clássica alemã}, é um dos textos mais
conhecidos do período tardio de Engels, com grande influência nas
discussões marxistas posteriores. Foi publicado
pela primeira vez nos volumes 4 e 5 da revista
\emph{Die Neue Zeit} (1886). Uma versão brevemente estendida sai em 1888 (Dietz, Stuttgart), juntamente com o anexo da primeira
aparição das \emph{Teses sobre Feuerbach} de 1845. O texto reelabora a crítica à filosofia alemã desenvolvida em seu período de juventude, junto com Marx, ao mesmo tempo em que defende a potência
crítica do \emph{materialismo-histórico marxista} em oposição aos outros
materialismos, idealismos e positivismos do final do século \textsc{xix}. Trata-se de uma discussão sobre materialismo e idealismo,
mediada pelo duplo sentido da dialética hegeliana,
revolucionária e reacionária; passa criticamente por Feuerbach e pelo
papel dos progressos da ciência diante da clássica relação alemã
metafísica/filosofia da natureza, para encerrar com Marx e a
consciência de classe revolucionária do trabalhador, essa o verdadeiro
\emph{ponto de saída}, não apenas da filosofia alemã.
        
\textbf{Vinicius Matteucci de Andrade Lopes} é graduado em Direito pela
\textsc{puc"-sp} (2010) e em História pela \textsc{fflch"-usp} (2013), com
período de intercâmbio em Filosofia na \emph{Universität Heidelberg} (2012--2013) e na
\emph{Albert-Ludwigs-Universität Freiburg} (2013). Atualmente é
doutorando em Filosofia (\textsc{fflch-usp}), como bolsista do \textsc{daad} na
\emph{Goethe Universität Frankfurt am Main} (2019--2021). Desenvolve
pesquisa sobre os efeitos ideológicos da expansão da relação"-capital nos
anos 1920 e 1930 na Alemanha, tendo como ponto de inflexão as discussões
filosóficas e políticas do período. Tradutor literário do alemão,
especializado em história, filosofia, sociologia e direito.


