\textbf{Friedrich Engels} (Barmen, 1820--Londres, 1895), revolucionário alemão,
desenvolveu, junto com Marx, o chamado ``socialismo científico''. Filho mais velho
de um industrial da tecelagem, viveu em Berlim e depois em Manchester, onde
conheceu Marx, em 1842, ao se envolver com o jornalismo radical e a política.
Ao voltar à Alemanha em 1844, passou por Paris, onde reencontrou Marx,
aproximando"-se dele definitivamente. Em Bruxelas, auxiliou na formação da Liga
dos Comunistas. Morou em Colônia e participou como fundador da \textit{Nova
Gazeta Renana}. Em 1849, tomou parte de um levante no sul da Alemanha e, com seu
fracasso, voltou à Inglaterra. Em Manchester, volta a trabalhar na empresa de
seu pai e passa a sustentar Marx. 
Com a morte de Marx, trabalhou na preparação e na publicação dos dois últimos
volumes de \textit{O Capital}. Investiu seu tempo em outras produções teóricas e
teve significativa influência na social"-democracia alemã.

\textbf{Ludwig Feuerbach e o fim da filosofia clássica alemã}, ou \emph{o
ponto de saída} da filosofia clássica alemã, é um dos textos mais
conhecidos do período tardio de Engels, com grande influência nas
discussões marxistas posteriores. Foi publicado
pela primeira vez nos volumes 4 e 5 da revista
\emph{Die Neue Zeit} (1886). Uma versão brevemente estendida sai em 1888 (Dietz, Stuttgart), juntamente com o anexo da primeira
aparição das \emph{Teses sobre Feuerbach} de 1845. O texto é tanto um aceno à crítica de Marx e Engels à
\emph{Ideologia alemã} dos anos 1840 --- lembrando que esse manuscrito
foi publicado somente
em 1932 --- como uma crítica de aspectos das formas de consciência
filosófica em 1886, pós"-revoluções de 1848, golpe de Napoleão \textsc{iii},
comuna de Paris e desfecho da \emph{via prussiana} com a unificação da
Alemanha. Passados 40 anos dos manuscritos da \emph{Ideologia alemã}, o
papel da filosofia e principalmente da ciência havia mudado. O
\emph{ponto de saída} da expansão da relação"-capital pós 1848 decreta o
fim do agora clássico período da filosofia alemã, que se encerraria com a
morte de Hegel. Diante disso, era preciso defender a potência
crítica do \emph{materialismo-histórico marxista} em oposição aos outros
materialismos, idealismos e positivismos do final do século \textsc{xix}. O texto
começa justamente com uma discussão sobre materialismo e idealismo,
mediada pelo duplo sentido da dialética hegeliana,
revolucionária e reacionária; passa criticamente por Feuerbach e pelo
papel dos progressos da ciência diante da clássica relação alemã
metafísica/filosofia da natureza; e se encerra com Marx e a
consciência de classe revolucionária do trabalhador, essa o verdadeiro
\emph{ponto de saída}, não apenas da filosofia alemã.
        
\textbf{Vinicius Matteucci de Andrade Lopes} é graduado em Direito pela
\textsc{puc"-sp} (2010) e em História pela \textsc{fflch"-usp} (2013), com
período de intercâmbio em Filosofia na \emph{Universität Heidelberg} (2012--2013) e na
\emph{Albert-Ludwigs-Universität Freiburg} (2013). Atualmente é
doutorando em Filosofia (\textsc{fflch-usp}), como bolsista do \textsc{daad} na
\emph{Goethe Universität Frankfurt am Main} (2019--2021). Desenvolve
pesquisa sobre os efeitos ideológicos da expansão da relação"-capital nos
anos 1920 e 1930 na Alemanha, tendo como ponto de inflexão as discussões
filosóficas e políticas do período. Tradutor literário do alemão,
especializado em história, filosofia, sociologia e direito.


