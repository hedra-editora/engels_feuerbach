\textbf{Friedrich Engels} (Barmen, 1820--Londres, 1895), revolucionário
alemão, conhecido por sua profícua parceria com Karl Marx, com quem
divide a autoria do célebre \emph{Manifesto comunista}. Filho mais velho
de um industrial da tecelagem, Engels foi um dos primeiros teóricos
modernos a escrever sobre as condições de vida da classe trabalhadora,
considerada o produto mais autêntico da modernidade capitalista. Em
Bruxelas, auxiliou na formação da Liga dos Comunistas, em 1847. Além de
colaborar com Karl Marx na elaboração de \emph{O Capital}, algumas de
suas obras próprias foram, posteriormente, consideradas fundantes do
marxismo, como o \emph{Anti-Duhring} e \emph{Ludwig Feuerbach e o fim da
filosofia clássica alemã}. Com a morte de Marx, trabalhou na preparação
e na publicação dos dois últimos volumes de \emph{O Capital}.

\textbf{Ludwig Feuerbach e o fim da filosofia clássica alemã}, ou
\emph{o ponto de saída da filosofia clássica alemã}, é um dos textos
mais conhecidos do período tardio de Engels, com grande influência nas
discussões marxistas posteriores. Foi publicado pela primeira vez nos
volumes 4 e 5 da revista \emph{Die Neue Zeit} (1886). Uma versão
brevemente estendida sai em 1888 (Dietz, Stuttgart), juntamente com o
anexo da primeira aparição das \emph{Teses sobre Feuerbach} de 1845. O
texto reelabora a crítica à filosofia alemã desenvolvida em seu período
de juventude, junto com Marx, ao mesmo tempo em que defende a potência
crítica do \emph{materialismo-histórico marxista} em oposição aos outros
materialismos, idealismos e positivismos no contexto histórico alemão
pós-revolução de 1848.
        
\textbf{Vinicius Matteucci de Andrade Lopes} é graduado em Direito pela
\textsc{puc"-sp} (2010) e em História pela \textsc{fflch"-usp} (2013),
com período de intercâmbio em Filosofia na \emph{Universität Heidelberg}
(2012--2013) e na \emph{Albert-Ludwigs-Universität Freiburg} (2013).
Atualmente é doutorando em Filosofia (\textsc{fflch-usp}), como bolsista
do \textsc{daad} na \emph{Goethe Universität Frankfurt am Main}
(2019--2021). Desenvolve pesquisa sobre os efeitos ideológicos da
expansão da relação"-capital nos anos 1920 e 1930 na Alemanha, tendo
como ponto de inflexão as discussões filosóficas e políticas do período.
Tradutor literário do alemão, especializado em história, filosofia,
sociologia e direito.


