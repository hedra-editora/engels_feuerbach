\pagebreak
\thispagestyle{empty}
\movetooddpage
\chapter{Apresentação}

%\begin{flushright}
%\textsc{vinicius matteucci de andrade lopes}
%\end{flushright}

\noindent{}\emph{Ludwig Feuerbach e o fim da filosofia clássica alemã} (1886) teve
certamente uma influência fundamental na construção do marxismo do
século \textsc{xx}. Basta dizer que é um texto com o qual Lukács e Lenin
estabeleceram mais de uma vez diálogo. Considerada pelo próprio autor um
aprofundamento da crítica que desenvolveu com Marx, desde a juventude, à
filosofia pós"-hegeliana, a obra tornou"-se um exemplo de como realizar
uma investigação marxista de importantes vertentes do pensamento
filosófico ocidental. %investigação marxista do pensamento filosófico ocidental

Por intermédio de Bernstein e Kautsky, o texto aparece para o público
pela primeira vez nos volumes 4 e 5 da revista \emph{Die Neue Zeit}, em
1886. Uma versão estendida é publicada em 1888 (Dietz, Stuttgart),
juntamente com o anexo da primeira aparição das \emph{Teses sobre %confuso
Feuerbach}, de 1845. O contexto de redação remete à publicação da tese de doutorado do
dinamarquês Carl Nicolaj Starcke acerca da obra de Ludwig Feuerbach,
filósofo que impactou profundamente a formação intelectual de Marx e
Engels, sobretudo no que se refere à crítica de ambos à filosofia
hegeliana. Mas o leitor atento observará, rapidamente, que o texto
engelsiano não se limita a apresentar uma resenha crítica aos aportes de
Starcke sobre Feuerbach. Ao contrário, Engels aproveita a resposta a
Starcke para elaborar um amplo balanço de toda a experiência filosófica na
Alemanha, sem perder de vista as imbricações entre a particularidade do
desenvolvimento capitalista alemão e suas repercussões na formação do
pensamento filosófico após o fracasso das revoluções de 1848.

Dividido em quatro partes, o texto é um aceno à
crítica de Marx e Engels à \emph{ideologia alemã} dos anos 1840 ---
lembrando que os manuscritos desse período (\emph{A ideologia
alemã}) somente serão publicados em 1932 --- e, ao mesmo tempo, uma crítica aos aspectos
fundamentais da consciência filosófica dos anos 1886, cujas raízes remetem ao contexto histórico que se segue às
revoluções de 1848, ao golpe de Napoleão \textsc{iii}, à Comuna de Paris e
ao desfecho da \emph{via prussiana} com a unificação alemã, em 1871. Mais especificamente, Engels observa que, passados quarenta anos dos manuscritos da \emph{Ideologia alemã}, o
papel da filosofia e, principalmente, da ciência havia mudado. O
\emph{ponto de saída} da expansão da relação"-capital pós"-1848 decreta o
fim do agora ``clássico'' período da filosofia alemã, abrindo espaço para
o primeiro grande momento de autorreflexividade do mundo burguês, marcado pela contradição entre a 
\emph{aparência da razão} burguesa e seu \emph{conteúdo irracional}. Ao
conectar, ainda que indiretamente, o \emph{ponto de
expansão da relação"-capital} na Alemanha pós"-1848 e o \emph{ponto de
transição do período clássico da filosofia burguesa}, Engels sugere,
pela primeira vez, um novo momento de consolidação da \emph{consciência
histórica burguesa} que iria desaguar --- seguindo um dos principais
marxistas influenciados por esse texto, Georg Lukács --- no irracionalismo
do período do \textsc{ii} Reich alemão.

Diante de tal conjuntura era preciso defender a potência crítica do
\emph{materialismo"-histórico marxista}, em direta oposição aos outros
materialismos --- principalmente ao anglo"-francês, bem como o de Feuerbach ---, ao
idealismo ``clássico'' --- na figura central de Hegel --- e às metafísicas
positivistas do final do século \textsc{xix}, posteriores ao momento clássico,
como o \emph{neokantismo} na Alemanha. A reflexão de Engels em \emph{Ludwig Feuerbach e o fim da filosofia clássica alemã} pode ser
considerada, nesse sentido, como a primeira aproximação materialista
\emph{crítica do neokantismo}.

Em termos estruturais, é possível afirmar que o texto começa, justamente,
com uma discussão conceitual sobre materialismo e idealismo mediada
pelo duplo sentido da dialética hegeliana (seu lado revolucionário
e reacionário); passa criticamente, então, por Feuerbach e pelo papel dos
progressos científicos diante da clássica relação alemã
metafísica/filosofia da natureza; e encerra com a crítica de Marx
e a consciência de classe revolucionária do trabalhador, essa o
verdadeiro \emph{ponto de saída}, e não apenas da filosofia alemã.

O leitor que conhece as obras de Engels certamente poderá,
também, estabelecer uma aproximação com outros textos escritos ao longo
da década de 1870 e 1880, como o \emph{Anti"-Dühring} (1878) e a
\emph{Dialética da natureza} (1886), principalmente no que se refere ao
problema do método dialético. Uma inflexão, porém, que poderia limitar a
intenção fundamental do próprio texto: ser, ao mesmo tempo,
um acerto de contas com o materialismo histórico iniciado por Engels e
Marx em 1845, e um enfrentamento crítico à conjuntura alemã de 1886. O
que costura esse duplo aspecto, mesmo ao levar"-se em conta as discussões
conceituais esboçadas, é o passo e descompasso específico do
desenvolvimento histórico alemão da década de 1840 até o início da década de 1880.

Engels é um autor que nunca se esquivou da tentativa de desdobrar temas
abstratos ou conceituais na sua vinculação com a concretude histórica.
Uma vinculação que se torna, a partir de então, pressuposto de qualquer
leitura materialista, muito mais fácil de ser prometida do que cumprida
enquanto investigação.

\pagebreak
\section*{Nota da tradução}

Como mencionado na Apresentação, o texto de Engels tem duas versões. A primeira, de 1886, publicada na revista \emph{Die Neue Zeit} em dois momentos (abril e maio), e uma
versão brevemente estendida publicada em 1888 pela \emph{Dietz Verlag}
como livro, juntamente com a primeira aparição das 11 teses de Marx
sobre Feuerbach de 1845 (\emph{Karl Marx über Feuerbach vom Jahre
1845}). A presente tradução, embora baseada na versão da
\textsc{mega},\footnote[*]{\textsc{engels}, F. \emph{Werke. Artikel. Entwürfe.
Oktober 1886 bis Februar} 1891, Band 31, \textsc{i}. \emph{bearbeitet
von Renate Merkel"-Melis.} Akademie Verlag GmbH, Berlin, 2002.} que se
vale da primeira edição em revista, apresenta as partes estendidas que compõe o
livro de 1888, assim como a \emph{nota prévia} não presente em 1886. Ao
longo do texto indicamos com barras verticais {[}\textbar\textbar{]}, tanto na versão alemã como na
tradução, as partes acrescentadas por Engels em 1888. As notas explicativas foram reunidas ao final do livro, quando indicadas por {[}\versal{N.\,A.}{]} referem"-se a comentários do próprio Engels, quando assinaladas por  {[}\versal{N.\,T.}{]} são de autoria do tradutor. Mantivemos o título original das traduções latinas que
seguem a versão francesa de 1894 (\emph{Ludwig Feuerbach et la fin}
{[}\emph{Ausgang}{]} \emph{de la philosophie classique allemande}). A
rigor, o termo exato, mais literal e que abrange o múltiplo sentido da
leitura de Engels seria \emph{o ponto de saída} (\emph{Ausgang})
\emph{da filosofia clássica alemã}. A tradução italiana de Palmiro
Togliatti --- \emph{Ludwig Feuerbach e il punto d'approdo della filosofia
classica tedesca} (1976) --- também questiona, de certa forma, essa
limitação do \emph{Ausgang} como um mero \emph{fim}, em que um movimento
se encerraria e deixaria de existir em si mesmo. Trata"-se, antes, de um
\emph{ponto de chegada} (\emph{punto d'approdo}), que acumula todo um
processo precedente e abre um novo caminho \emph{a partir} \emph{desse}
\emph{acumulo}. Importante aqui é considerar o sentido de movimento, de
transição, que o ano de 1848 representava como ponto, ao mesmo tempo,
de chegada e saída, pelo qual as portas para a outra época da
racionalidade burguesa se abrem.
